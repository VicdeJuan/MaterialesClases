
\newcommand{\fullAnalisis}[1]{
\textit{Esta solución ha sido generada automáticamente. Puede contener errores}
	
\input{#1}
\mydominio
\mysimetria
\myptscorte
\myasintotas
\mygrafo
\newpage
}

\newcommand{\fullAnalisisBach}[1]{
\input{#1}
\mydominio
\mysimetria
\myptscorte
\myasintotas
\mymonotonia
\mygrafo
\newpage
}




\newcommand{\mydominio}{
\paragraph{Dominio}
%
%\[ D(f) = \{x\in\real \tq \sage{f.denominator()(x)} \neq 0\} = \sagestr{dominion(f=f,hasDen=True,hasLog=false,hasRt=false)}\]
%
\[
\sagestr{dominion(f=f,hasDen=True,hasLog=false,hasRt=false)}
\]
}

\newcommand{\mysimetria}{
\subparagraph{Simetría}

\ifverbose
    Para estudiar la simetría de una función calculamos $f(-x)$ y comparamos con $f(x)$. 
    %
    En este caso:
\fi

\[f(-x) = \sage{latex(f(-x))}\]

¿Es igual a f(x) o a -f(x)? \sagestr{simetria(f)}

}

\newcommand{\myptscorte}{
    
\subparagraph{Puntos de corte con los ejes}

\textit{\textbf{Eje X}
}\ifverbose
Para calcular los puntos de corte de la función con el eje x resolvemos la ecuación $f(x) = 0$ cuya solución es: 
\\

\else 
$f(x) = 0\rightarrow $
\fi
\sagestr{puntosEjeX(func = f)} 

\textit{\textbf{Eje Y}}
\ifverbose
Para calcular los puntos de corte de la función con el eje y calculamos $f(0)$. 
%
En este caso:
\fi
\sagestr{puntosEjeY(func = f)}

}

\newcommand{\myasintotas}{
\paragraph{Asíntotas}
\subparagraph{Asíntotas verticales}


\ifverbose
Los posibles puntos en los que la función puede tener una asíntota vertical son aquellos en los que se anula el denominador. 
%
Por ello calculamos, si no lo hemos hecho ya:
%
\[\sagestr{latex(f.denominator(normalize=False))} = 0 \]
\fi

\sagestr{asintotesV(f = f,den = f.denominator(normalize=False), isLog = TieneLog,AV = AV)}



\subparagraph{Asíntotas horizontales u oblícuas}

\ifverbose
Las asíntotas horizontales y oblicuas nos dan la información acerca de la tendencia de la función en $-\infty$ y en $+\infty$.

Para calcular las asíntotas, necesitamos calcular el límite de la función tanto en $+\infty$ como en $-\infty$:
\fi

\[\lim_{x\mapsto \pm\infty} \sagestr{latex(f(x))} \]

\sagestr{asintotesHO(f = f,AH=AH,AO = AO)}


}


\newcommand{\mymonotonia}{
\paragraph{Monotonía}
	
$f'(x) = \sagestr{latex(diff(f(x),x,1).full_simplify())}$

Calculamos los puntos críticos, aquellos en los que $f'(x) = 0$

\sagestr{solveDerivadaNula(f=f,sols=ptsCrit,onlyReal=True)}


Incluimos a estos puntos aquellos en los que la función primitiva no exista, si hay alguno.
%
\sagestr{ptsFrontera(f=f,ptscrit=ptsCrit,recta=recta)}

\sagestr{estudiarSignoDiff(f)}
}


\newcommand{\mygrafo}{
\paragraph{Gráfica de la función}
\ifverbose
En azul se marcan las asíntotas verticales, en rojo las horizontales y en morado las oblícuas.
\fi
\begin{center}
\sageplot{_myplot(f,AH=uniq(AH),AV=uniq(AV),AO=AO,extremos=ptsCrit)}
\end{center}
}




\begin{sagesilent}

#Variables auxiliares necesarias
AV=[]
AH=[]
AO=[]
ptsCrit=[]
recta=[]
min=[]
max=[]

# Definición de las funciones

## Función para devolver un número sencillo entre 2 valores, que pueden ser infinitos.
# Input: j,k los 2 valores.
def random_between(j,k):
 if ((j==-Infinity) and (k == Infinity)):
  return 0

 if (j==-Infinity):
  return int(k-2)

 if (k==Infinity):
  return int(j+2)

 if (j < 0 and k>0):
  return 0

 if (j < 1 and k>1):
  return 1


 if (k-j>=2):
  return int(random()*(k-j-1))+j+1
 else:
  return round((k+j)/2.0,2)

#Resuelve los puntos de discontinuidad de la función.
def ptsDiscontinuidad(f,isLog):
    _den0=solve(f.denominator(normalize=False) == 0,x,solution_dict=True)
    den0 = [a[x] for a in _den0 if imag(a[x])==0]
    return den0

def _emptylist(aux):
    [aux.pop() for a in xrange(len(aux))]
    
## Función para calcular las asíntotas verticales de la función.
#Input:
# f: función
# den: denominador de la función en caso de haberlo.
# boolean log: tieneLog 
# AV: array vacío para rellenar
def asintotesV(f,den,isLog,AV):
    _emptylist(AV)
    _den0=solve(f.denominator(normalize=False) == 0,x,solution_dict=True)
    den0 = [a[x] for a in _den0 if imag(a[x])==0]

    r=""
    k=0

    if (Verbose == 1):
        r+= "Soluciones: "
        for i in den0:
            r+="$x_"+str(k)+"="+latex(i)+" $;\\quad"
            k+=1
        if (den0 == []):
            r+= "No tiene soluciones reales"

    if (isLog == 1):
        _log0 = solve(exp(f)==0,x,solution_dict=True)
        log0 = [a[x] for a in _log0 if imag(a[x])==0]  

        if (Verbose == 1):
            r+= "\\paragraph{Atención: } Como la función es un logaritmo, podríamos tener una asíntota vertical (ya que $\\displaystyle\\lim_{x\\mapsto 0^+} \\log(x) = -\infty$). Vamos a comprobarlo.\\\\"

            r+= "Para ello, calculamos los puntos en los que se haga 0 el interior del logaritmo .\\\\"

        
            r+= "En este caso:"
        r+= "\\[ "+latex(exp(f))+"= 0 \\]"

        r+= "Soluciones: "
        for i in log0:
            r+="$x_"+str(k)+"="+latex(i)+" $;\\quad"
            k+=1
            den0.append(i)
        if (log0 == []):
            r+= "No tiene soluciones reales"

    if (den0 == []):
        r+= "\\paragraph{La función no tiene A.V.}"

    for x0 in den0:
        r+= "\\paragraph{Asintota en $x ="
        r+= latex(x0)+"$}"
        ld=limit(f(x),x=x0,dir="plus")
        li=limit(f(x),x=x0,dir="minus")
        if (Verbose==1): 
            r+="Calculamos"
        r+= "\[\\lim_{x\\mapsto " + latex(x0) + "} " + latex(f(x)) + " = \\left\\{\\begin{array}{l} \\displaystyle\\lim_{x\\mapsto " + latex(x0) + "^{+}} " + latex(f(x)) + " = " + latex(ld) + " \\\\\\\\\\displaystyle\\lim_{x\\mapsto " + latex(x0) + "^{-}} " + latex(f(x)) + " = " + latex(li) + " \\\\\\end{array}\\right.\] "
        if (abs(li) == Infinity and abs(ld) == Infinity and sign(li) == sign(ld)):
            r+= "\n La recta x =" + latex(x0) + " es una \\textbf{asíntota vertical} de f(x)"
            AV.append(x0)
        elif (abs(li) == Infinity and abs(ld) == Infinity and sign(li) != sign(ld)):
            AV.append(x0)
            if (Verbose == 1):
                r+= "\n Aunque el límite no exista (porque los límites laterales son diferentes), su magnitud sigue siendo infinita por ambos lados, por lo que la recta x = $" + latex(x0) + "$ es una \\textbf{asíntota vertical} de f(x)"
            else:
                r+= "\n $x = " + latex(x0) + "$ es \\textbf{A.V.} de f(x)"
        else:
            r+= "\n \\textbf{No hay asíntota vertical} en $x="+latex(x0)+"$."
    AV=uniq(AV)
    return r


## Función para calcular las asíntotas horizontales u oblicuas.
#Input:
# f: función

def asintotesHO(f,AH,AO):
    _emptylist(AH)
    _emptylist(AO)
    r=""
    for infty in [Infinity, -Infinity]:
        if (Verbose == 1):
            r+= "\\paragraph{Límite de la función en $"+latex(infty)+"$:}"
        else:
            r+= "\\paragraph{$"+latex(infty)+"$:}"
        ld=limit(f, x=infty)  

        r+= "\[\\lim_{x\\mapsto "+latex(infty)+"}" + latex(f(x)) + "= ... = "+latex(ld)+"\]"    

        if (not (abs(ld) == Infinity)): # HORIZONTAL
            AH.append(ld)
            if (Verbose == 1):
                r+= "En $"+latex(infty)
                r+= "$ f(x) tiene una \\textbf{asíntota horizontal} en $y = "+latex(ld)+"$."
            else:
                r+= "$y = "+latex(ld) +" $ es una \\textbf{A.H.} de f(x)"

        else: # OBLICUA
            if (Verbose == 1):
                r+= "Como el límite obtenido es de magnitud infinita la función no tiene asíntota horizontal, pero puede tener una asíntota oblícua. Para saber si tiene una asíntota oblícua calculamos: "
            else:
                r+= "No tiene horizontal. ¿Oblícua?"

            m=limit(f/x,x=infty)
            r+= "\[m=\\lim_{x\\mapsto "+latex(infty)+"} \\frac{f(x)}{x} = \\lim_{x\\mapsto "+latex(infty)+"} \\frac{"+latex(f(x))+"}{x}=\\lim_{x\\mapsto "+latex(infty)+"} "+latex(f(x)/x)+" = ... = "+latex(m)+"\]"
            if (abs(m) != Infinity):
                if (Verbose == 1):
                    r+= "        En este caso tenemos m=$"+latex(m)+"$ por lo que sí hay asíntota oblícua. Calculamos n:"
                r+= "    \[n= \\lim_{x\\mapsto "+latex(infty)+"} f(x) - m·x = \\lim_{x\\mapsto "+latex(infty)+"}\\left("+latex(f(x))+"-"+latex(m)+"x \\right) = \\lim_{x\\mapsto "+latex(infty)+"}"+latex(f(x)-m*x)+" = "+ limit(f-m*x,x=infty)+"\]"
                _n = limit(f-m*x,x=infty)
                n=_n
                y=m*x+n
                r+= "En $"+latex(infty)+"$, f(x) tiene una \\textbf{asíntota oblícua} en $y="+latex(y)+"$."
                AO.append(m*x+n)
            else:
                if (Verbose == 1):
                    r+= "En este caso, ese límite tenemos $m="+latex(m)+"$ por lo que \\textbf{no hay asíntota} oblícua (ni horizontal)."
                else:
                    r+= "\\textbf{No hay A.O. ni A.H.}"
    AH=uniq(AH)
    return r

# Estudio de la simetría de la función.
def simetria(f):
    if (f(x) == f(-x)):
        if (Verbose == 1):
            return "Sí, a f(x), entonces la función es par."
        else:
            return "Sí, $\\Rightarrow$ par"
    elif (f(x) == - f(-x)):
        if (Verbose == 1):
            return "Sí, a -f(x) entonces la función es impar."
        else:
            return "Sí, $\\Rightarrow$ impar"
    else:
        if (Verbose == 1):
            return "No, entonces la función no tiene simetría respecto del eje Y."
        else:
            return "No, $\\Rightarrow$ ninguna simetría"

###### Estudio de los puntos de corte
def puntosEjeX(func):
    r=""
    __roots = solve(func(x)==0,x,solution_dict=True)
    _roots = [a[x] for a in __roots if imag(a[x])==0]
    for i in xrange(len(_roots)):
        r+="$x_"+str(i)+" = "+latex(_roots[i])+"\\to \\left("+latex(_roots[i])+",0\\right)$;\\quad"
    if (_roots == []):
        r += "No tiene solución."
    if Verbose:
        r+= "\\\\\\textit{Obs: Sólo puede haber puntos de corte en puntos del dominio.}"
    return r

def puntosEjeY(func):
    r=""
    if (func.denominator() in ZZ):
        r += "$" + latex(func(0)) +"$"
    else:
        if (func.denominator()(0) != 0):
            r += "$f(0) = " + latex(func(0)) + "$, con lo que la gráfica corta en $\\left(0,"+latex(func(0)) + "\\right)$ al eje Y."   
        else:
            r +=  "No existe f(0)."    
    return r


############### Monotonía


# Función para calcular los puntos que anulan la derivada.
# Recibe:
# f: función
# boolean onlyReal: para calcular sólo soluciones reales.
def solveDerivadaNula(f,sols,onlyReal):
    _emptylist(sols)
    __sols = solve(diff(f(x))==0,x,solution_dict=True)
    if onlyReal:
        _sols = [a[x] for a in __sols if imag(a[x])==0]
    else: 
        _sols=__sols
    for b in _sols:
        sols.append(b) 
    i=0
    retval=""
    for r in sols:
        retval += "$x_"+str(i)+" = "+latex(r.full_simplify())+"$;\\quad"
        i+=1
    if (sols == []):
        retval += "No tiene soluciones reales."
    return retval


def ptsFrontera(f,ptscrit,recta):
    puntosFrontera = flatten([ptsDiscontinuidad(f=f,isLog=false),ptscrit])
    recta=[-Infinity]

    i=0
    r=""
    for p in puntosFrontera:
        r+= "$x_"+str(i)+"="+latex(p)+"$;\\quad"
        i+=1

    for p in puntosFrontera:
        recta.append(p)
    recta.append(+Infinity)
    recta=sorted(recta)
    return r

# Construye una cadena de texto los intervalos formados por una lista de puntos frontera.
def intervalos(recta):
    r = ""
    for i in xrange(len(recta)-1):
        r += "$\\left("+latex(recta[i])+","+latex(recta[i+1])+"\\right)$;\\quad"  
    return r

# Estudia el signo de la función dada la función y los puntos frontera.
def diffsign(f,recta,min,max):
    _emptylist(min)
    _emptylist(max)
    g=diff(f,x,1)
    prev=0
    actual=0
    retval = "\\begin{itemize}"
    for i in xrange(len(recta)-1):
        rval=random_between(recta[i],recta[i+1])
        retval += "\\item $\\left("+latex(recta[i])+","+latex(recta[i+1])+"\\right)$:"
        if (Verbose == 1):
            retval += " Tomamos, por ejemplo, $f'("+latex(rval)+") = "+latex(round(g(rval),3))+"...$  y miramos su signo:" 
        else:
            retval += "$f'("+latex(rval)+") = "+latex(round(g(rval),3)) +"$"
        prev=actual
        actual=sign(g(rval))
        if actual==1: 
            if (Verbose == 1):
                retval += "Positivo, por lo que la función \\textbf{crece} en este intervalo"  
            else:
                retval += "Positivo $\\Rightarrow$ crece"
        else:
            if (Verbose == 1):
                retval += "Negativo, por lo que la función \\textbf{decrece} en este intervalo"
            else:
                retval += "Negativo $\\Rightarrow$ decrece"

        if (prev == -1 and actual == 1):
            min.append(recta[i])
        if (prev == 1 and actual == -1):
            max.append(recta[i])
    retval += "\\end{itemize}"
    return retval

# Enumera una lista de valores entre llaves.
def printlist(m):
    retval = "\\{"
    for _m in m:
    	retval += "x = " + latex(_m) + ";\\quad"
    return retval+"\\}"



def _getRealroots(f):
 __roots = solve(f==0,x,solution_dict=True)
 _roots = [a[x].full_simplify() for a in __roots if imag(a[x])==0]
 return _roots



def _f_sign_monot(recta,f,min,max):
 _emptylist(min)
 _emptylist(max)
 g=diff(f,x,1)
 recta=[-Infinity,Infinity]
 for a in _getRealroots(g):
  recta.append(a)
 for a in ptsDiscontinuidad(f=f,isLog=false):
  recta.append(a)
 recta=sorted(recta) 
 retval = "Los intervalos a estudiar son: "+latex(intervalos(recta))
 
 prev=0
 actual=0
 retval += "\\begin{itemize}"
 for i in xrange(len(recta)-1):
  rval=random_between(recta[i],recta[i+1])
  retval += "\\item $\\left("+latex(recta[i])+","+latex(recta[i+1])+"\\right)$:"
  if (Verbose == 1):
   retval += " Tomamos, por ejemplo, f'($"+latex(rval)+") = "+latex(round(g(rval),3))+"...$  y miramos su signo:" 
  else:
   retval += "f'($"+latex(rval)+") = "+latex(round(g(rval),3)) +"$"
  prev=actual
  actual=sign(g(rval))
  if actual==1: 
   if (Verbose == 1):
    retval += "Positivo, por lo que la función \\textbf{crece} en este intervalo."  
   else:
    retval += "Positivo $\\Rightarrow$ crece."
  else:
   if (Verbose == 1):
    retval += "Negativo, por lo que la función \\textbf{decrece} en este intervalo."
   else:
    retval += "Negativo $\\Rightarrow$ decrece."


 return retval+"\\end{itemize}"


def estudiarSignoDiff(f):
 _retval = _f_sign_monot(recta,f,min,max)
 return _retval

def estudiarSigno(f): 
 _retval = _f_sign_monot(recta,f,min,max)
 return _retval


## Function str ismin(f,x0).
## f: second derivate of the function to study.
## x0: value to check if it is maximum.
## return 	string with the answer.
def ismin(f,x0):
 val = f(x0)
 retval = "f("+str(x0)+") = "+str(f(x0))
 if val<0:
  retval += "<0 por lo que hay un máximo relativo en "+str(x0)
 elif val>0:
  retval += ">0 por lo que hay un mínimo relativo en "+str(x0)
 else:
  retval += "No podemos determinar si se trata de un máximo o un mínimo de esta manera" 
 return retval
 

## Devuelve en texto el dominio de la función. 
# f: función
# boolean hasDen: una función racional
# boolean hasLog: una función logarítmica
# boolean hasRt: una función irracional

def dominion(f,hasDen,hasLog,hasRt):
 s0=latex(f.denominator(normalize=False)(x))
 s1=",".join(str(a[x]) for a in solve(f.denominator(normalize=False) == 0,x,solution_dict=True))
 return "D(f) = \\{x\\in\\real \\tq "+s0+" \\neq 0 \\} = \\real \\setminus \{"+ s1 + "\}"
 
### Función para pintar las gráficas.
# f: función
# int list AV: lista de puntos en los que hay una asíntota vertical.
# int list AH: lista de puntos en los que hay una asintota horizontal.
# EXPR AO: ecuación de la asíntota oblícua y=mx+n
def _myplot(f,AV,AH,AO,extremos):
 
 discont=list(set(ptsDiscontinuidad(f=f,isLog=false)).difference(AV))
 xmin=-8
 xmax=8
 ymin=-10
 ymax=10
 c1=['blue','red','green','black','purple',colors]
 P=plot(f,xmin=xmin,xmax=xmax,ymin=ymin,ymax=ymax,figsize=6,plot_points=2500,legend_label="f(x)", axes_labels=['$x$','$y$'],exclude=f.denominator()==0,color='green')
 j=0
 for i in xrange(len(AV)):
  lab="x="+str(AV[i])
  v=AV[i]
  P+=line([(v,ymin),(v,ymax)],legend_label=lab,color='blue',linestyle='-')
  j=i
 for i in xrange(len(AH)):
  lab="y="+str(AH[i])
  h=AH[i]
  P+=line([(xmin,h),(xmax,h)],legend_label=lab,color='red',linestyle='-')
 for i in xrange(len(discont)):
  x0=discont[i]
  ld=limit(f(x),x=x0,dir="plus")
  P+=circle((x0,ld),0.2,facecolor='white',fill=True)
 for i in xrange(len(extremos)):
  x0=extremos[i]
  P+=circle((x0,f(x0)),0.15,facecolor='orange',fill=True)
  
 if AO != []:
  P+=plot(AO,color='purple',xmin=xmin,xmax=xmax,ymin=ymin,ymax=ymax)
 _emptylist(extremos)
 _emptylist(AH)
 _emptylist(AV)
 _emptylist(AO)
 
 return P
\end{sagesilent}


%%%%%%% Plot
