\documentclass[palatino,nochap]{Docencia}


\title{Ejercicios de polinomios}
\author{Víctor de Juan}
\date{17/18}



% Paquetes adicionales

\usepackage[author={Víctor de Juan, 2017}]{pdfcomment}

\makeatletter
\newcommand{\annotate}[2][]{%
\pdfstringdef\x@title{#1}%
\edef\r{\string\r}%
\pdfstringdef\x@contents{#2}%
\pdfannot
width 2\baselineskip
height 2\baselineskip
depth 0pt
{
/Subtype /Text
/T (\x@title)
/Contents (\x@contents)
}%
}
\makeatother

% --------------------

\begin{document}

\section{Ejercicios  de polinomios}

\begin{enumerate}

\item Sea $P(x) = 3x^3-3x^2-3x+3 ]$
\begin{itemize}
	\item ¿Es divisible por $(x-1)$?
\end{itemize}

\vspace{1cm}\hrule

\item Factoriza: $P(x) = 3x^3-x^2+9x-3 $
\vspace{1cm}\hrule



\item Sea $P(x) = 6x^3-10x^2+4x \;\;\; $

\begin{itemize}
	\item Factoriza.
\end{itemize}

\vspace{1cm}\hrule

\item Sea $P(x) = 2x^3-2x^2+kx+4$.
\begin{itemize}
	\item Halla el valor de $k$ para que $P(x)$ sea divisible por $x-2$.
\end{itemize}


\vspace{1cm}\hrule

\item Sea $P(x) = x^4+4x^3+6x^2+4x+1$
\begin{itemize}
	\item Factoriza.
\end{itemize}

\vspace{1cm}\hrule

\item Sea $P(x) = 21x^2+10x-2 $.

\vspace{1cm}\hrule

\item \textbf{Ampliación realmente muy interesante: } Sea $P(x) = 6x^3+ax^2+bx-1$, con $a,b\inℤ$
\begin{itemize}
	\item Halla el valor de $a,b$ para que $P(x)$ sea divisible por $(x-\frac{1}{3})$ y por $(x-\frac{1}{5})$.
	\item Halla el valor de $a,b$ para que $P(x)$ sea divisible por $(x-\frac{1}{3})$ y por $(x-\frac{1}{2})$.
\end{itemize}

\vspace{1cm}\hrule

\item \textbf{Ampliación menos interesante por demasiada dificultad: } Sea $P(x) = 4x^2+bx+1$, $b\inℤ$. 
\begin{itemize}
	\item Sabemos que sus raíces $α_1,α_2$ son fraccionarias y positivas. ¿Cuáles son? ¿Cuánto vale $b$?
\end{itemize}

\end{enumerate}

\end{document}
\grid