\documentclass[palatino,nosec]{Docencia}


\title{Ejercicios funciones}
\author{Departamento de Matemáticas}
\date{17/18}

\begin{abstract}
\end{abstract}

% Paquetes adicionales

\usepackage[author={Víctor de Juan, 2017}]{pdfcomment}

\makeatletter
\newcommand{\annotate}[2][]{%
	\pdfstringdef\x@title{#1}%
	\edef\r{\string\r}%
	\pdfstringdef\x@contents{#2}%
	\pdfannot
	width 2\baselineskip
	height 2\baselineskip
	depth 0pt
	{
		/Subtype /Text
		/T (\x@title)
		/Contents (\x@contents)
	}%
}
\makeatother

% --------------------
\newcommand{\cimplies}{\text{\hl{$\implies$}}}
\renewcommand{\vx}{\overset{\rightarrow}{x}}
\renewcommand{\vy}{\overset{\rightarrow}{y}}
\renewcommand{\vz}{\overset{\rightarrow}{z}}
\newcommand{\vi}{\overset{\rightarrow}{i}}
\newcommand{\vj}{\overset{\rightarrow}{j}}
\renewcommand{\vec}[1]{\overset{\rightarrow}{#1}}

\usepackage{pgf,tikz}
\usetikzlibrary{arrows}

\begin{document}
\pagestyle{plain}

	
\begin{problem} Demuestra que el vértice de la parábola $f(x) = ax^2+bx+c$ tiene de coordenada horizontal $x=\rfrac{-b}{2a}$
\solution
\[
	f'(x) = 2ax+b
\]
\end{problem}


\begin{problem} Halla las rectas tangentes a la función $f(x)$ en los máximos (no es necesario intervalos de crecimiento y decrecimiento).
\[
f(x) = 3x^4-4x^3-36x^2
\]

¿Hay algún máximo absoluto?

\solution

\end{problem}

\begin{problem} Estudia la monotonía (intervalos de crecimiento, decrecimiento y extremos relativos y absolutos) de $f(x) = x^4-2x^2$
\solution
\end{problem}

\begin{problem} Estudia la monotonía (intervalos de crecimiento, decrecimiento y extremos relativos y absolutos) de $f(x) = x^3$
\solution
\end{problem}

\begin{problem} Estudia sistemáticamente la función: \[f(x) = \frac{2x-1}{x^2-4x+4}\]

\solution
\end{problem}

\begin{problem} Estudia las asíntotas de \[f(x) =\frac{x^3-4x^2+x+6}{2x^3-14x^2+32x-24}\]
\solution
\end{problem}

\begin{problem} Deriva $f(x) = \tan(x^2-3x)$
\solution
\end{problem}

\end{document}