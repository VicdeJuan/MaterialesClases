
\subparagraph{Estudio de la monotonía}


\ifverbose
La monotonía responde a los intervalos de crecimiento y decrecimiento de la función, así como a la existencia de extremos relativos.
%
Para ello, la mejor herramienta es utilizar la derivada de la función.

Calculamos la derivada de la función:
\fi

\[f'(x) = \sagestr{latex(diff(f,x,1).full_simplify()(x))} \]

\ifverbose
    Para calcular los puntos críticos\footnote{Aquellos puntos que pueden marcar un cambio en la tendencia de la función.}, resolvemos f'(x) = 0 cuyas soluciones son:
    \\

\else
    Resolvemos $f'(x) = 0\rightarrow $. 
\fi
\begin{sagesilent}
[ptscrit,ptscritStr] = solveDerivadaNula(f,true)
[puntosFrontera,ptsFronteraStr] = ptsFrontera(f,ptscrit,denIs0)
\end{sagesilent}

\ifverbose
\hl{E incluimos los puntos de discontinuidad de la función primitiva} (en caso de haberlos) ya que se puede producir un cambio en la tendencia de la función. 
%
Por lo tanto, los puntos críticos son : 
\sagestr{ptsFronteraStr}

\else
Añadidmos discontinuidades. 

Puntos críticos = \{\sagestr{ptsFronteraStr}\}
\fi



\ifverbose
Con estos resultados ya podemos estudiar los intervalos de crecimiento y decrecimiento. 
%
Para ello tomamos los siguientes intervalos en los que los puntos críticos dividen la recta real:
\fi

\sagestr{intervalos(puntosFrontera)}

\ifverbose
Y ahora, estudiamos la monotonía el signo de la derivada en esos intervalos:
\fi
\begin{sagesilent}
[max,min,diffsign_str] = diffsign(f,puntosFrontera)

\end{sagesilent}

\sagestr{diffsign_str}

\textbf{\textit{Máximos relativos}}
\ifverbose
Los puntos en los que a la izquiera la función crece (derivada positiva) y a la derecha decrece (derivada negativa) son los máximos relativos. En este caso: 
\fi

\[\text{Máximos relativos: } \sagestr{printlist(max)}\]

\textbf{\textit{Mínimos relativos}}
\ifverbose
Los puntos en los que a la izquierda la función crece (derivada positiva) y a la derecha decrece (derivada negativa) son los máximos relativos. En este caso son: 
\fi

\[\text{Mínimos relativos: }\sagestr{printlist(min)}\]


