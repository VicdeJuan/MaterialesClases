\documentclass[palatino,nosec]{apuntes}

\title{Derivadas}
\author{}
\date{16/17 C2}

\newcommand{\sen}{\operatorname{\sen}}


% Paquetes adicionales

% --------------------

\begin{document}
%\pagestyle{plain}
%\maketitle

%\tableofcontents
%% Contenido.

\section*{Ejercicio 281.15}

\begin{itemize}


\item[\textbf{a)}] $\displaystyle f(x) = x\sqrt[3]{x^2}$
	\paragraph{Opción 1: Introducir la $x$ en el radical}

	\[
		f(x) = x\sqrt[3]{x^2} = \sqrt[3]{x^5}
	\]
	Y ahora derivamos:

	\[
		f(x) = \sqrt[3]{x^5} = x^{\frac{5}{3}} \to f'(x) = \frac{5}{3}x^{\rfrac{5}{3}-1} = \frac{5x^{\rfrac{2}{3}}}{3} = \frac{5\sqrt[3]{x^2}}{3}
	\]

	\paragraph{Opción 2: Derivar como un producto}

	\[
		f(x) = \underbrace{x}_{u}\underbrace{\sqrt[3]{x^2}}_{v} 
	\]
	
	Utilizamos la regla del producto: $f(x) = u·v \to f'(x) = u'·v + u·v'$

	\[
		f'(x) = \underbrace{1}_{u'}·\underbrace{\sqrt[3]{x^2}}_{v} + \underbrace{x}_{u}·\underbrace{\left(x^{\rfrac{2}{3}}\right)'}_{v'} = \sqrt[3]{x^2} + x·\left(\frac{2}{3}·x^{\rfrac{2}{3}-1}\right) = \sqrt[3]{x^2} + \frac{2}{3}x·x^{\rfrac{-1}{3}} 
	\]

	Utilizamos: $x^a·x^b = x^{a+b}$

	\[
		f'(x) = \sqrt[3]{x^2} + x^{1-\rfrac{1}{3}} = \sqrt[3]{x^2} + \frac{2}{3}x^{\rfrac{2}{3}} = \sqrt[3]{x^2} + \frac{2}{3}\sqrt[3]{x^2} = \frac{5}{3}\sqrt[3]{x^2}
	\]

\textbf{Posibles errores:}

	\begin{enumerate}
		\item Ignorar que es un producto:
		\[f(x) = x\sqrt[3]{x^2} \to f'(x) = 1·\frac{2}{3}x^{\rfrac{-1}{3}} = ...\]

		\item Igualar la función a la derivada
		\[f(x) = x·\sqrt[3]{x^2} \textcolor{red}{\neq} \underbrace{1}_{u'}·\underbrace{\sqrt[3]{x^2}}_{v} + \underbrace{x}_{u}·\underbrace{\left(x^{\rfrac{2}{3}}\right)'}_{v'}\]
	\end{enumerate}


\item[\textbf{b)}] $\displaystyle f(x) = x·e^x$

	Derivamos como un producto:

	\[
		f(x) = \underbrace{x}_{u}·\underbrace{e^x}_{v}
	\]

	Utilizamos la regla del producto: $f(x) = u·v \to f'(x) = u'·v + u·v'$ y la tabla de derivadas: $f(x) = x \to f'(x) = 1$ y $f(x) = e^x \to f'(x) = e^x$

	\[
		f'(x) = 1·e^x + x·e^x = e^x(x+1)
	\]




\item[\textbf{c)}] $\displaystyle f(x) = x·\sen(x)$
	
	Derivamos como un producto:

	\[
		f(x) = \underbrace{x}_{u}·\underbrace{\sen(x)}_{v}
	\]

	Utilizamos la regla del producto: $f(x) = u·v \to f'(x) = u'·v + u·v'$ y la tabla de derivadas: $f(x) = x \to f'(x) = 1$ y $f(x) = \sen(x) \to f'(x) = \cos(x)$

	\[
		f'(x) = 1·\sen(x) + x·\cos(x)
	\]


	


\item[\textbf{d)}] $\displaystyle f(x) = x·\ln(x)$

	Derivamos como un producto:

	\[
		f(x) = \underbrace{x}_{u}·\underbrace{\ln(x)}_{v}
	\]

	Utilizamos la regla del producto: $f(x) = u·v \to f'(x) = u'·v + u·v'$ y la tabla de derivadas: $f(x) = x \to f'(x) = 1$ y $f(x) = \ln(x) \to f'(x) = \frac{1}{x}$

	\[
		f'(x) = 1·\ln(x) + x·\frac{1}{x} = \ln(x) + 1
	\]




\item[\textbf{e)}] $\displaystyle f(x) = \sen(x)·\cos(x)$

	Derivamos como un producto:

	\[
		f(x) = \underbrace{\sen(x)}_{u}·\underbrace{\cos(x)}_{v}
	\]

	Utilizamos la regla del producto: $f(x) = u·v \to f'(x) = u'·v + u·v'$ y la tabla de derivadas: $f(x) = \sen(x) \to f'(x) = \cos(x)$ y $f(x) = \cos(x) \to f(x) = -\sen(x)$

	\[
		f'(x) = \cos(x)·\cos(x) + \sen(x)·(-\sen(x)) = \cos^2(x) - \sen^2(x)
	\]




\item[\textbf{f)}] $\displaystyle f(x) = \frac{x^2+1}{x}$

	\paragraph{Opción 1}
	Recomendada en funciones racionales con un monomio en el denominador

	\[
		f(x) = \frac{x^2+1}{x} = \frac{x^2}{x} + \frac{1}{x} = x+\frac{1}{x}
	\]

	Derivamos utilizando la regla de la suma: $f(x) = u+v \to f'(x) = u'+v'$ y utilizando las tablas: $f(x) = x^n \to f'(x) = n·x^{n-1}$ 

	\[
		f(x) = x+\frac{1}{x} \to f'(x) = 1+\left(x^{-1}\right)' = 1+(-1)·x^{-1-1} = 1-x^{-2} = 1-\frac{1}{x^2}
	\]

	\paragraph*{Opción 2} 

	Utilizamos la regla del cociente: $\displaystyle f(x) = \frac{u}{v} \to f'(x) = \frac{u'·v - u·v'}{v^2}$

	En este caso: $u = x^2+1$ y $v = x$.

	\[
		f(x) = \frac{x^2+1}{x} \to f'(x) = \frac{(x^2+1)'·x - \textcolor{red}{(}x^2+1\textcolor{red}{)}·1}{x^2} = \frac{2x·x - \textcolor{red}{(}x^2+1\textcolor{red}{)}}{x^2} =
	\]
	\[
		= \frac{2x^2-x^2-1}{x^2} = \frac{x^2-1}{x^2} = 1-\frac{1}{x^2}
	\]

	\textit{El paréntesis está en rojo porque es un error común olvidárselo}



\item[\textbf{g)}] $\displaystyle f(x) = (x^2+2x)·\sen(x)$

	Tenemos un producto, con $u=(x^2+2x)$, que a su vez, es una suma.

	Utilizando las reglas de la suma y del producto, consultando la tabla de las derivadas:

	\[
		f(x) = \underbrace{(x^2+2x)}_{u}·\underbrace{\sen(x)}_{v} \to f'(x) = \underbrace{(2x+2)}_{u'}·\underbrace{\sen(x)}_{v} + \underbrace{(x^2+2x)}_{u}·\underbrace{\cos(x)}_{v'}
	\]

	Si se prefiere:
	\[
		f'(x) = x^2\cos(x) + 2x(\sen(x) + \cos(x)) + 2 \sen(x)
	\]
	
	\textbf{¿Y se puede...} deshacer el paréntesis antes de derivar? Por poder se puede, pero es más largo:

	\[
		f(x) = \underbrace{x^2}_{u_1}·\underbrace{\sen(x)}_{v_1} + \underbrace{2x}_{u_2}\underbrace{\sen(x)}_{v_2} \to f'(x) = u_1'v_1 + u_1·v_1' + u_2'·v_2+u_2·v_2' = ...
	\]

	Pudiendo ahorrar tiempo y cálculos...


\item[\textbf{h)}] $\displaystyle f(x) = (e^x - x)·\ln{x}$

	Misma estructura que el apartado anterior.

	\[
		f(x) = \underbrace{(e^x - x)}_{u}·\underbrace{\ln{x}}_{v} \to f'(x) = \underbrace{(e^x-1)}_{u'}·\underbrace{\ln(x)}_v + \underbrace{(e^x-x)}_{u}·\underbrace{\frac{1}{x}}_{v'}
	\]

	\[
		f'(x) = (e^x-1)·\ln(x) + \frac{e^x}{x} - 1
	\]


\end{itemize}
%% Apendices (ejercicios, examenes)
%\appendix\chapter{---}% -*- root: ../Derivadas.tex -*-




%\section*{Ejercicio 281.16}

%\begin{itemize}
%	\item $3x^2·\log_2(x)$
%	\item $f(x) = e^x\sen(x)$
%	\item $f(x) = \sqrt[3]{x}·\cos(x)$
%	\item $f(x) = \cos(x)·\tg(x)$
%	\item $f(x) = 4x·\sen(x) + x^3·\cos(x)$
%	\item $f(x) = \ln(x)\frac{1}{x^4} + x^2·e^x$
%	\item $f(x)  = (\sen(x) - \cos(x)) ·\tg(x)$
%	\item $f(x) = 4x\sqrt{x} + \frac{\sen(x)}{x}$

%\end{itemize}

\printindex
\end{document}
\grid
