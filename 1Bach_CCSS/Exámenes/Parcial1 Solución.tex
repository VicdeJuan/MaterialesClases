\documentclass[palatino,nosec,nochap]{Docencia}


\title{Solucionario primer parcial}
\author{}
\date{18/19}


% Paquetes adicionales

\usepackage[author={Víctor de Juan, 2017}]{pdfcomment}

\makeatletter
\newcommand{\annotate}[2][]{%
\pdfstringdef\x@title{#1}%
\edef\r{\string\r}%
\pdfstringdef\x@contents{#2}%
\pdfannot
width 2\baselineskip
height 2\baselineskip
depth 0pt
{
/Subtype /Text
/T (\x@title)
/Contents (\x@contents)
}%
}
\makeatother



\usepackage{eso-pic}
\newcommand\BackgroundPic{%
\put(0,0){%
\parbox[b][\paperheight]{\paperwidth}{%
\vfill
\centering
%\includegraphics[width=\paperwidth,height=\paperheight,%keepaspectratio]{../../../BWLogo.jpeg}%
\vfill
}}}





\begin{abstract}
Solucionario del primer examen parcial de Matemáticas Aplciadas a las Ciencias Sociales de 1º de Bachillerato.

\nota{Estos ejemplos no están exentos de erratas. En caso de descubrir alguna, por favor, comunicarlas al autor.}
\end{abstract}

% --------------------
\newcommand{\cimplies}{\text{\hl{$\implies$}}}

\begin{document}
\pagestyle{plain}
\maketitle

\AddToShipoutPicture{\BackgroundPic}

\newpage

\begin{problem}

Resuelve los siguientes sistemas de ecuaciones:

\ppart
\[
\left\{\begin{array}{lccccc}
e_1: &	2x &	-	y &		+3z &	= & 6 \\
e_2: &	4x &	+	y &		-z &	= & 1 \\
e_3: &	-2x &	-  3y &		 &	= & -4 
\end{array}\right\}
\]


\ppart
\[
\left\{\begin{array}{lccccc}
e_1: &	2x &	-	y &		-z &	= & 1 \\
e_2: &	-3x &	+	y &		   &	= & 7 \\
e_3: &	-2x &	+   y &		+z&	= & -8 
\end{array}\right\}
\]

\solution

\spart

\[
\left\{\begin{array}{lccccc}
e_1: &	2x &	-	y &		+3z &	= & 6 \\
e_2: &	4x &	+	y &		-z &	= & 1 \\
e_3: &	-2x &	-  3y &		 &	= & -4 
\end{array}\right\}
%%
%%
\overset{(1)}{\dimplies}
%%
%%
\left\{\begin{array}{lccccc}
e_1: &	2x &	-	y &		+3z &	= & 6 \\
e_2: &	   &	+	3y &	-7z &	= & -11\\
e_3: &  &	 	-4y &	+3z &	= & 2\\
\end{array}\right\}
%%
%%
\dimplies\]
\[
%%
%%
\left\{\begin{array}{lccccc}
e_1: &	2x &	-	y &		+3z &	= & 6 \\
e_2: &	   &	+	3y &	-7z &	= & -11\\
e_3: &  &	 	-4y &	+3z &	= & 2\\
\end{array}\right\}
\overset{(2)}{\dimplies}
\left\{\begin{array}{lccccc}
e_1: &	2x &	-	y &		+3z &	= & 6 \\
e_2: &	   &	+	3y &	-7z &	= & -11\\
e_3: &	   &		   &	- 19z & = -38
\\
\end{array}\right\}
\]


\paragraph{1)} $e_2 = e_2-2e_1$ y $e_3 = e_3+e_1$

\[
\begin{array}{rccccc}
e_2: &	4x &	+	y &		-z &	= & 1 \\
e_1: &	4x &	-	2y &	+6z &	= & 12 \\
\hline
e_2: &	   &	+	3y &	-7z &	= & -11\\
\end{array}
\]	


\[
\begin{array}{rccccc}
e_3: &	-2x &	-  3y &		 &	= & -4 \\
e_1: &	2x &	-	y &		+3z &	= & 6 \\
\hline
e_3: &  &	 	-4y &	+3z &	= & 2\\
\end{array}
\]	


\paragraph{2)} $e_3 = 3·e_3-4·e_2$

\[
\begin{array}{rccccc}
e_2: &	   &	+	12y &	-28z &	= & -44\\
e_3: &  &	 	-12y &	+9z &	= & 6\\
\hline
e_3: &	   &		   &	- 19z & = & -38
\end{array}
\]	


\paragraph{Discusión y resolución}

\[e_3: -19z=-38 \dimplies z=2\]


Utilizando $z=2$ en $e_2$:
\[
e_2: -12y + 9z = 6 \dimplies -12y = 6-9·2 \dimplies -12y = -12 \dimplies y=1\]

Utilizando $z=2$, $y=1$ en la ecuación que nos falta ($e_1$):

\[
e_1: 2x	- y +3z = 6 \dimplies 2x - 1 + 3·2 = 6 \dimplies 2x = 1 \dimplies x=\rfrac{1}{2}
\]


Solución: $(x,y,z) = \left(\rfrac{1}{2},1,2\right)$.

\paragraph{Comprobación}

Sustituimos los valores obtenidos en el sistema inicial:

\[
\left\{\begin{array}{lccccll}
e_1: &	2x &	-	y &		+3z &	= & 6 &\to 2·\rfrac{1}{2} - 1 + 3·2 = 1-2+6=6\\
e_2: &	4x &	+	y &		-z &	= & 1 &\to 4·\rfrac{1}{2} + 1 - 2 = 2+1-2 = 2\\
e_3: &	-2x &	-  3y &		 &	= & -4 &\to -2·\rfrac{1}{2} - 3·1 = -1-3 = -4
\end{array}\right\}\begin{array}{c}\\\\\\\\\text{cqc}\end{array}
\]

\textit{Si estás pensando que se deberían haber reordenado las columnas para aprovechar el hueco de la $z$, tienes razón.}



\spart 

\[
\left\{\begin{array}{lccccc}
e_1: &	2x &	-	y &		-z &	= & 1 \\
e_2: &	-3x &	+	y &		   &	= & 7 \\
e_3: &	-2x &	+   y &		+z&	= & -8 
\end{array}\right\}
%%
%%
\dimplies
%%
%%
\left\{\begin{array}{lccccc}
e_1: & z & +  y & -2x & = & -8 \\	
e_2: &   & +  y & -3x & = & 7  \\
e_3: &-z & -  y & +2x & = & 1
\end{array}\right\}
%%
%%
\overset{(1)}{\dimplies}
%%
%%
\]\[
\left\{\begin{array}{lccccc}
e_1: & z & +  y & -2x & = & -8 \\	
e_2: &   & +  y & -3x & = & 7  \\
e_3: &	 &		&    0&	= & -7\\
\end{array}\right\}
\]


\paragraph{1)} $e_3 = e_3+e_1$

\[
\begin{array}{rccccc}
e_3: &-z & -  y & +2x & = & 1\\
e_1: & z & +  y & -2x & = & -8 \\	
\hline
e_3: &	 &		&    0&	= & -7\\
\end{array}
\]	


\textbf{Conlcusión:} Es un sistema incompatible porque tiene una ecuación incompatible.

\end{problem}




\newpage
\begin{problem}

Resolver la siguiente ecuación: 

\[
	\frac{1+\displaystyle\frac{2x-6}{2x}}{x-1} = \frac{2x-3}{x^2-x}
\]

\solution

\[
	\frac{1+\displaystyle\frac{2x-6}{2x}}{x-1} = \frac{2x-3}{x^2-x} 
	\dimplies 
	\frac{1+\displaystyle\frac{2(x-3}{2x}}{x-1} = \frac{2x-3}{x^2-x} 
	\dimplies
	\frac{1+\displaystyle\frac{(x-3}{x}}{x-1} = \frac{2x-3}{x^2-x} 
	\dimplies
	\]\[
	\frac{\displaystyle\frac{x+x-3}{x}}{x-1} = \frac{2x-3}{x^2-x} 
	{\dimplies}
	\frac{2x-3}{x(x-1)} = \frac{2x-3}{x^2-x} 
	\overset{(x\neq0\;;\;x\neq1)}{\dimplies}
	1=1
\]

Es una ecuación compatible indeterminada\footnote{Porque "se van" todas las incógnitas}, cuya solución son $\real\setminus\{0,1\}$ ya que anulan los denominadores de la fracción inicial (y son los valores con los que hemos perdido alguna equivalencia).

\end{problem}

\begin{problem}

Resolver la siguiente ecuación irracional:

\[
	1+\sqrt{x-1} = \frac{2}{\sqrt{x-1}}
\]

\solution


\[
	1+\sqrt{x-1} = \frac{2}{\sqrt{x-1}} \overset{x\neq1}{\dimplies} \sqrt{x-1}\left(1+\sqrt{x-1}\right) = 2 \dimplies \sqrt{x-1}+x-1=2\dimplies	\]
	\[ \sqrt{x-1} = 3-x \implies x-1 = (3-x)^2 
	\dimplies x-1 = 9+x^2-6x \dimplies x^2-7x+10 = 0 \dimplies \left\{\begin{array}{c}
	x=2\\
	x=5
	\end{array}\right.
\]

\textbf{Comprobación}
\begin{itemize}
	\item $x=2 \to 1+\sqrt{2-1} = \frac{2}{\sqrt{2-1}} \dimplies 1+1=2 \implies $ Sí es solución.
	\item $x=5 \to 1+\sqrt{5-1} = \frac{2}{\sqrt{5-1}} \dimplies 1+2\neq1$ No es solución.
\end{itemize}

\end{problem}


\newpage

\begin{problem}

¿Qué números reales cumplen la siguiente igualdad?

\[\log_3x·\log_x5 = \log_35\]

\solution

Utilizando el cambio de base: \[ \log_aA =\frac{\log_bA}{\log_ba}\] 

Teniendo 2 logaritmos en base 3, lo normal sería pensar en cambiar el tercer logaritmo a base 3 también. Veamos qué ocurre:

\[
	\log_3x·\log_x5 = \log_35 \dimplies \log_3x·\frac{\log_35}{\log_3x} = \log_35 \implies \log_35 = \log_35 \dimplies 1=1
\]

Es una ecuacion compatible indeterminada, por lo que parece que todos los números reales cumplen esta igualdad. 
%
Sin embargo, dado que $x$ está tanto como argumento como base del logaritmo, no tendría sentido tomar, por ejemplo, $x=0$ ya que $\log_05$ no está definido. 

\textbf{Conclusión: } $x$ puede tomar cualquier valor real que valga tanto como base como argumento del logaritmo, es decir, cualquier número real positivo distinto de 1. 

\textit{ En lenguaje algebraico escribiríamos: $\{x\in\real\tq x>0 \wedge x neq 1\}$, donde $\wedge$ es la conjunción copulativa "y"}

\end{problem}

\begin{problem} Sea $P(x) = -2x^2+3mx-12$. Sabiendo que $x=2$ es una raíz de $P(x)$, factorízalo.

\solution

Si $x=2$ es raíz del polinomio, por la definición de raíz, se cumple $P(2) = 0$.

$P(2) = 0 \dimplies -2·2^2 + 3·m·2 - 12 = 0 \dimplies -8 + 6m -12 = 0 \dimplies m=\frac{20}{6} = \frac{10}{3}$

Si $m=\rfrac{10}{3}$, entonces $P(x) = -2x^2+3\frac{10}{3}x-12 = -2x^2+10x-12$

Para factorizar, calculo sus raíces resolviendo $P(x) = 0 \dimplies -2x^2+10x-12 \dimplies x_1=2 \;\;\; x_2=3$

La factorización por tanto será: $P(x) = -2(x-2)(x-3)$

\end{problem}


\end{document}