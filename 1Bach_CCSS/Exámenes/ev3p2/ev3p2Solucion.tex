\documentclass[palatino,nosec,nochap]{Docencia}


\title{Solucionario del examen final 3ª evaluación}
\author{}
\date{18/19}


% Paquetes adicionales

\usepackage[author={Víctor de Juan, 2019}]{pdfcomment}

\makeatletter
\newcommand{\annotate}[2][]{%
\pdfstringdef\x@title{#1}%
\edef\r{\string\r}%
\pdfstringdef\x@contents{#2}%
\pdfannot
width 2\baselineskip
height 2\baselineskip
depth 0pt
{
/Subtype /Text
/T (\x@title)
/Contents (\x@contents)
}%
}
\makeatother


\usepackage{sagetex}

\newif\ifverbose
\verbosetrue


\usepackage{eso-pic}
\newcommand\BackgroundPic{%
\put(0,0){%
\parbox[b][\paperheight]{\paperwidth}{%
\vfill
\centering
%\includegraphics[width=\paperwidth,height=\paperheight,%keepaspectratio]{../../../BWLogo.jpeg}%
\vfill
}}}




\newcommand{\fullAnalisisBachDiecinueve}[1]{
\input{#1}
\mydominio
\myptscorte
\myasintotas
\mygrafo
\newpage
}




\begin{abstract}
Solucionario del examen final de la primera evaluación de Matemáticas Aplciadas a las Ciencias Sociales de 1º de Bachillerato.

\nota{Estos ejemplos no están exentos de erratas. En caso de descubrir alguna, por favor, comunicarlas al autor.}
\end{abstract}

% --------------------
\newcommand{\cimplies}{\text{\hl{$\implies$}}}

\begin{document}
\input{__sageUtilities}
\ifverbose
	\begin{sagesilent}
		Verbose=1
	\end{sagesilent}
\else
	\begin{sagesilent}
		Verbose=0
	\end{sagesilent}
\fi

\pagestyle{plain}
\maketitle

\AddToShipoutPicture{\BackgroundPic}

\newpage


\begin{problem}[1]
Estudia la continuidad de la siguiente función:

\[
f(x) = \left\{\begin{array}{ccc}
\frac{1}{x-2}&\text{ si }& x<2\\
x+1 & \text{ si } & 2\leq x\leq5\\
\frac{x^2-4x+7}{x-3} & \text{ si } & x>5
\end{array}\right.
\]
\solution

Cada trozo es continuo en su dominio porque son funciones racionales, salvo el trozo intermedio que es una función lineal.
%
Es decir, $f(x)$ es continua en $(-\infty,2)\cup(2,5)\cup(5,\infty)$

Condición de continuidad: $f(x)$ es continua en $x=a$ si existe $f(a)$, existe $\lim_{x\to a}f(x)$ y son iguales.

\textbf{Estudiamos $x=2$:}
\begin{itemize}
	\item 	$f(2)  = 2+1 = 3$
	\item Necesitamos hacer los límites laterales, ya que la definición de la función cambia según el lado desde el que nos acerquemos al 2. 
	\[ \lim_{x\to2^+} f(x) =  \lim_{x\to2^+} x+1=2+1=3\]
	\[ \lim_{x\to2^-} f(x) =  \lim_{x\to2^-} \frac{1}{x-2}=-\infty\]
	Los límites laterales no coinciden, por lo que la función no es continua.
\end{itemize}


\textbf{Estudiamos $x=5$:}
\begin{itemize}
	\item 	$f(5)  = 5+1 = 3$
	\item Necesitamos hacer los límites laterales, ya que la definición de la función cambia según el lado desde el que nos acerquemos al 2. 
	\[
	\left.\begin{array}{c}
			\lim_{x\to5^+} f(x) =  \lim_{x\to5^+} \frac{x^2-4x+7}{x-3} = \frac{12}{2} = 6\\
			\lim_{x\to5^-} f(x) =  \lim_{x\to5^-} x+1 = 5+1 = 6
		\end{array}\right\} \implies \lim_{x\to 5} f(x) = 6\]
	\item Podemos concluir que $f(x)$ es continua en $x=5$ ya que $f(5) = \lim_{x\to 5}f(x)$
\end{itemize}
\textbf{Conclusión:} $f(x)$ es continua en $\real-\{2\}$
\end{problem}

\begin{problem}

Halla la función derivada de las siguientes funciones:
\ppart \[f(x) = \sqrt[5]{\ln\left(\rfrac{1}{x}\right)}\]

\ppart \[g(x) = \frac{x\ln(x)}{e^x}\]

\solution

\textit{Este ejercicio no obtendría la puntuación máxima ya que faltan todas las fórmulas de derivación utilizadas.}

\spart 
\[
f'(x) = \frac{1}{5\sqrt[5]{\ln\left(\frac{1}{x}\right)^4}}·\left(\ln\left(\frac{1}{x}\right)\right)' \overset{(*)}{=}\frac{1}{5\sqrt[5]{\ln\left(\frac{1}{x}\right)^4}}·\frac{-1}{x} = \frac{-1}{5x\sqrt[5]{\ln\left(\frac{1}{x}\right)^4}}
\]

\[
\left(\ln\left(\frac{1}{x}\right)\right)' = \frac{1}{\rfrac{1}{x}}·\left(\rfrac{1}{x}\right)' = x·(-x^{-2}) = \frac{-1}{x}
\]

\spart 
\[
g'(x) = \frac{(x\ln(x))'·e^x - x\ln(x)·(e^x)'}{(e^x)^2} = \frac{(\ln(x) +1)·e^x - x\ln(x)·e^x}{e^{2x}} = \frac{\ln(x) +1 - x\ln(x)}{e^{x}}
\]

\[
(x\ln(x))' = x'·\ln(x) + x·\rfrac{1}{x} = \ln(x) + 1
\]

\end{problem}




\begin{problem}
Calcula el dominio de la siguiente función: $f(x) = \log_4(2x^2-16)$
\solution

\[D(f) = \{x\in\real \tq 2x^2-16 > 0\} = (-\infty,-2\sqrt{2}) \cup (2\sqrt{2},\infty)\]

\end{problem}

\begin{problem}
Realiza el estudio completo (dominio, puntos de corte con los ejes, asíntotas y esboza la gráfica de la siguiente función).

\[
f(x) = \frac{-2x^2+10x-12}{x^2-9}
\]
\solution

\textit{Esta solución ha sido generada automáticamente. Puede contener errores.}
\fullAnalisisBach{func}

\end{problem}

\begin{problem}
Dada la función $f(x) = 2x^2+5$ halla las coordenadas del máximo o mínimo de esta parábola.
\solution

El máximo o mínimo tiene como recta tangente una recta horizontal cuya pendiente es 0. Por lo tanto buscamos resolver $f'(x) = 0$.

$f'(x) = 4x \implies f'(x) = 0 \dimplies x=0$. El máximo o mínimo se encuentra en $(0,f(0)) = (0,5)$


\end{problem}


\begin{problem}[5]

Halla la ecuación de la recta tangente a $f(x) = 7^{3x}$ en el punto $x=10$.

\solution

La fórmula a utilizar es $y-f(a) = f'(a)·(x-a)$

En este caso:
\begin{itemize}
	\item $f(a) = 7^{30}$
	\item $f'(x) = 7^{3x}·\ln(7)·3$, sustituyendo en x=10 tenemos $f'(10) = 7^{30}·\ln(7)·3$
\end{itemize}
Por lo tanto la recta tengente será:

\[
y-7^{30} = 7^{30}·\ln(7)·3·(x-10) \dimplies y = 7^{30}·\ln(7)·3 + 7^{30} \dimplies y = 7^{30}(\ln(7)·1+1)
\]

\end{problem}

\end{document}