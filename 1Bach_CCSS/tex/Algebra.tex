% -*- root: ../Cuaderno.tex -*-

\chapter{Álgebra}

\section{Repaso de 4º}

\subsection{Logaritmos}

\paragraph{Introducción}

Vamos a aprender una nueva manera de multiplicar. En realidad ya sabéis, aunque no seáis conscientes.\footnote{Fuente: \href{https://www.youtube.com/watch?v=FB3\_BeukBBk\&t=99s}{Mark Foskey, youtube.com}}


\begin{itemize}
\item Caso 1: $1000000·10000000 = 10^6·10^7 = 10^{13}$. ¿Y podremos hacer esto con otros números que no sean el 10?

\item Caso 2: $64·128 = 2^6·2^7 = 2^{13} = 8192$
\item ¿Caso 3?: Me construyo la tabla del 3.
\begin{center}
\begin{tabular}{cccccccccccc}
1& 3& 9& 27& 81& 243& 729& 2187& 6561& 19683& 59049& 177147\\
\textcolor{red}{0} & \textcolor{red}{1} & \textcolor{red}{2} & \textcolor{red}{3} & \textcolor{red}{4} & \textcolor{red}{5} & \textcolor{red}{6} & \textcolor{red}{7} & \textcolor{red}{8} & \textcolor{red}{9} & \textcolor{red}{10} & \textcolor{red}{11}
\end{tabular}
\end{center}
\item Caso 4: ¿Y para números que no son potencias enteras? Por ejemplo, $64*40$. Pues si $32=2^5$ y $64=2^6$, $40=2^{5,...}$ ¿Tiene sentido?

\item Caso 5: Lo que hicieron, Yost y Napier, fue coger la tabla del 1,0001 en lugar de la tabla del 3 y dividir por mil los números rojos, dando lugar a la tabla de logaritmos:

\begin{center}
	\begin{tabular}{cl}
		0.0 & 1.0\\
		0.001 & 1.001\\
		0.002 & 1.002\\
		0.003 & 1.003\\
		0.004 & 1.00401\\
		0.005 & 1.00501\\
		0.006 & 1.00602\\
		0.007 & 1.00702\\
		0.008 & 1.00803\\
		0.009 & 1.00904\\
		$\vdots$ & \quad\quad$\vdots$\\
		0.991 & 2.69259\\
		0.992 & 2.69529\\
		0.993 & 2.69798\\
		0.994 & 2.70068\\
		0.995 & 2.70338\\
		0.996 & 2.70608\\
		0.997 & 2.70879\\
		0.998 & 2.7115\\
		0.999 & 2.71421\\
		1.0 & \hl{2.71692} (Una aproximación de $e$)\\
	\end{tabular}
\end{center}

\end{itemize}

$40=2^{5...}$. Ese $5...$ es lo que llamamos "logaritmo" en base 2 de 40. ¡Los logaritmos son exponentes! Es el \textit{exponente al que hay que elevar} ...


\begin{defn}[Logaritmo]
Sean $a\in ℝ>0,a≠1$ y $N\in\real$.

Se llama logaritmo en base $a$ de $N$ al exponente $x$ que cumple: $a^x = N$ y se escribe:
\[
	\log_aN=x\dimplies a^x=N
\]
\end{defn}

\textbf{Conectando con otras operaciones matemáticas:} 
\[
	\begin{array}{c}
		2^3=8\\
		\sqrt[3]{8}=2\\
		\text{??? }= 3
	\end{array}
\]

La relación $2^3=8$ se puede expresar de otras maneras, dando como resultado el 2 (raíz cúbica) y dando como resultado el 3 (logaritmo).

\nota{De la propia definición se entiende:}
\begin{enumerate}
	\item $y\in\real, y<0 \implies ∀a,\nexists\log_a y$
	\item $\log_a 1 = 0 \dimplies a^0 = 1$
	\item $\log_a a = 1 \dimplies a^1 = a$
	\item $\log_a a^q = q \dimplies a^q = a^q$ por definición.
	\item $a^{\log_a N} = N$
\end{enumerate}

\paragraph{Propiedades}

Vamos a razonar las propiedades de los logaritmos teniendo en cuenta que son exponentes. De hecho, las propiedades de los logaritmos no son otra cosa que las propiedades de las potencias escritas de otra manera.

\begin{itemize}
	\item $\log_a(AB) = \log_a(A) + \log_a(B)$
	\subitem Por ejemplo:
	\[9·8 = 2^x \dimplies x = \log_2 (9·8) = \log_2 8 + \log_2 9\]
	\subitem Como los logaritmos son exponentes, esta propiedad se podría leer: \textit{el exponente de un producto es la suma de los exponentes}.
	\item $\log_a\left(\rfrac{A}{B}\right) = \log_a(A) - \log_a(B)$
	\item Como los logaritmos son exponentes, ¿cómo se leería esta propiedad?
	\item $\log_a(A)^n = n·\log_a(A)$ 

	\item $\displaystyle\log_aA = \frac{\log_bA}{\log_ba}$ \textbf{(Cambio de base})
\end{itemize}

\paragraph{Tomar logaritmos:}

\[
A = B \overset{A>0; a>0; a\neq 1}{\dimplies} a^{\log_a A} = a^{\log_a B} \dimplies \log_aA=\log_aB
\]


\subsubsection{Ejemplos con logaritmos}

Calcula:
\begin{itemize}
	\item $\log \sqrt{0,0001}$
	\item \textit{[Ejercicio de examen de años anteriores]} Demuestra $\log_a b · \log_b a = 1$ 
\end{itemize}


\subsection{Factorización}

\begin{itemize}
	\item Alguien sale a la pizarra a factorizar. Utilizando 
https://www.wolframalpha.com/problem-generator/quiz/?category=Algebra\&topic=FactorPolynomial 
	\item Diferencia raíz y factor.
	\item Verdadero o falso.
	\subitem Un polinomio de grado 2 tiene 2 raíces reales. (Falso: $x^2+1$)
	\subitem Un polinomio con 2 raíces tiene 2 factores. (Discutible: $(x+1)^2(x-1)$)
	\subitem Un polinomio con 2 factores tiene 2 raíces.
	\subitem Un polinomio es irreducible si no tiene raíces reales. (Falso: $P(x) = x^4+2x^2+1$)
	\subitem 2 polinomios con las mismas raíces son iguales: (Falso: $(x^2+1)(x-2)$ y $(x-2)$)
	\subitem 2 polinomios con los mismos factores son iguales: (Falso: $2(x-1)$ y $(x-1)$)

	\item Halla “m” para que $2x^3-2x^2+m·x+4$ sea divisible por $(x-2)$

	\item Deberes: ejercicios 6-9
\end{itemize}


\subsection{Teoremas de factorización}

\paragraph{Ejemplo}
Factoriza: $P(x) = 3x^3-x^2+9x-3 = 3(x^2+3)\left(x-\rfrac{1}{3}\right)$

La factorización del polinomio. ¿Qué raíces puede tener? Ni $\pm1,\pm3$, ¿entonces? Teorema de las raíces racionales.


\hl{\textit{Entregar en hoja aparte: Teoremas de Polinomios}}

\begin{theorem}[Teorema\IS del factor]
Sea $P(x) = a_nx^n+a_{n-1}x^{n-1}+...+a_1x+a_0$ con $a_n≠0$ y $a_n,a_{n-1},...,a_1,a_0\in\real$. 
Sea $α\in\real$.

\[
	P(α) = 0 \dimplies \frac{P(x)}{(x-α)} = Q(x)
\]
\end{theorem}

De hecho este teorema es un caso particular del teorema del resto:
\begin{theorem}[Teorema\IS del resto]
Sea $P(x) = a_nx^n+a_{n-1}x^{n-1}+...+a_1x+a_0$ con $a_n≠0$ y $a_n,a_{n-1},...,a_1,a_0\in\real$.

Entonces, el resto de $\frac{P(x)}{x-α} = P(α)$
\end{theorem}


\begin{theorem}[Teorema\IS de la factorización]
Sea $P(x) = a_nx^n+a_{n-1}x^{n-1}+...+a_1x+a_0$ con $a_n≠0$ y $a_n,a_{n-1},...,a_1,a_0\in\real$ y $α_1,α_2,...,α_n\in\real$ las raíces o ceros de $P(x)$. 

Entonces,\[P(x) = a_n(x-α_1)(x-α_2)...(x-α_n)\]
\end{theorem}


\begin{theorem}[Teorema\IS de las raíces enteras]
Sea $P(x) = a_nx^n+a_{n-1}x^{n-1}+...+a_1x+a_0$, con $a_n≠0$, una raíz entera $r$ de $P(x)$ tiene que ser divisor del término independiente.
\end{theorem}



\begin{theorem}[Teorema\IS de las raíces racionales]
Sea $P(x) = a_nx^n+a_{n-1}x^{n-1}+...+a_1x+a_0$, con $a_n≠0$,$a_i\inℤ$ una raíz fraccionaria $\rfrac{n}{m}$ del polinomio $P(x)$ tiene que cumplir $n|a_0$ y $m|a_n$.
\end{theorem}


\subsubsection{Ejercicios:}

\begin{enumerate}
\item Alguien en la pizarra. Corrijo lo que se haya dejado, escribiendo los teoremas, etc.

Sea $P(x) = 3x^3-3x^2-3x+3$ .¡Factoriza! $P(x) = 3(x-1)(x+1)^2$

\begin{itemize}
	\item ¿Es divisible por $(x-1)$? Comprobamos $P(1) = 3-3-3+3 = 0 \overset{T.F}{\implies}$ Sí.
\end{itemize}

\textit{¡Mira que tontería dice el teorema del factor si miras el polinomio factorizado!}

\item (Ellos) Sea $P(x) = 6x^3-10x^2+4x = 6x(x-1)(x-\rfrac{2}{3})$ 
\begin{itemize}
	\item Factoriza.
	\subitem $P(0) = 0$. Por el teorema del factor sabemos que $x-0$ es un factor.
	\subitem Posibles raíces: $n=\pm1,\pm2,\pm4$ y $m=\pm1,\pm2,\pm3,\pm6$	
	\subitem Por el teorema de la factorización, $Q(x) = 3x^3-5x+2x$ tendrá las mismas raíces que $P(x) = 6x^3-10x^2+4x$. \hl{(Ojo, no podemos simplificar, pero las raíces son las mismas)}. Ahora las posibles raíces son $\rfrac{n}{m}$ donde $n\in\{\pm1,\pm2\}$ y $m\in\{\pm1,\pm3\}$
	\subitem $P(1) = 0$. Por el teorema del factor sabemos que $0$ es una raíz. ¿Es esto más fácil que Ruffini? ¿Y ahora?
\end{itemize}

\item Sea $P(x) = 2x^3-2x^2+kx+4$.
\begin{itemize}
	\item Halla el valor de $k$ para que $P(x)$ sea divisible por $x-2$.
	\subitem Por el teorema del factor, buscamos $P(2) = 0$. Entonces:
	\[
		P(2) = 0 \dimplies 2^4-2^3+2k+4 = 0 \dimplies 16-12+2k = 0 \dimplies k = -2
	\]
\end{itemize}




\item Sea $P(x) = 4x^2+kx+1$.
\begin{itemize}
	\item Halla el valor de $k$ para que sea divisible por $\left(x-\rfrac{1}{3}\right)$. $k=\frac{13}{3}$.
	\item Pero, $3$ no divide a $4$. ¿Cómo podría ser una raíz $\rfrac{1}{3}$?
\end{itemize}


\item Sea $P(x) = 6x^3+ax^2+bx-1$, con $a,b\inℤ$
\begin{itemize}
	\item Halla el valor de $a,b$ para que $P(x)$ sea divisible por $(x-\rfrac{1}{3})$ y por $(x-\rfrac{1}{5})$.
	\subitem Por el teorema de las raíces racionales, $5$ no divide al coeficiente principal, por lo que $P(x)$ no puede ser divisible por $(x-\rfrac{1}{5})$.
	\item Halla el valor de $a,b$ para que $P(x)$ sea divisible por $(x-\rfrac{1}{3})$ y por $(x-\rfrac{1}{2})$.
	\subitem Por el teorema del factor, buscamos:
	\[
	\left\{
		\begin{array}{c}
			P(\rfrac{1}{2}) = 0 \dimplies \frac{6}{8} + \frac{a}{4} + \frac{b}{2} - 1 = 0\\
			P(\rfrac{1}{3}) = 0 \dimplies \frac{6}{27} + \frac{a}{9} + \frac{b}{3} - 1 = 0
		\end{array}\right\}\dimplies ... \quad (a,b) = (-1,-4)
	\]
\end{itemize}

\item\textbf{Ampliación, puesto pero sin corregir} Sea $P(x) = 4x^2+bx+1$, con $b∈ℤ$. 
\begin{itemize}
	\item Sabemos que sus raíces $α_1,α_2$ son fraccionarias y negativas. ¿Cuáles son? ¿Cuánto vale $b$?
	\subitem Por el teorema de las raíces racionales, $α_1 = \rfrac{n_1}{m_1}$, sabemos que $n_1$ divide a $1$. Análogo para $α_2$.

	Por otro lado, sabemos que $m_2$ divide a 4. Las posibilidades son $2,4$, con lo que $α_1,α_2 \in \{\rfrac{1}{2},\rfrac{1}{4}\}$

	Por el teorema del factor, $P(\rfrac{1}{2}) = 1+b\rfrac{1}{2}+1 = 0 \implies b=-4$. 

	Por el teorema del factor, $P(\rfrac{1}{4}) = \rfrac{1}{4}+b\rfrac{1}{4}+1 = 0 \implies b=-2$.

	Si queremos que sea divisible por los 2 factores, b tiene que valer a la vez $4$ y $-2$. Entonces, necesariamente $P(x) = 4(x-\rfrac{1}{2})^2$ o $P(x) = 4(x-\rfrac{1}{4})^2$. 

	Desarrollando la segunda opción, obtenemos como término independiente $\rfrac{1}{4}≠1$, por lo que no es posible. 
	%
	Por otro lado, desarrollando la primera opción obtenemos algo con sentido.

	\[
		4\left(x+\rfrac{1}{2}\right)^2 = 4\left(x^2+x+\rfrac{1}{4}\right) = 4x^2+4x+1 \implies b=4
	\]

\end{itemize}

	\item Factorizar $P(x) = 9x^3-\frac{27}{2}x^2+\frac{13}{2}x-1 = 9·(x-1/2)(x-2/3)(x-1/3)$. Pista (para ahorraros pruebas innecesarias con Ruffini), todas las raíces son fraccionarias y positivas.

	\item Factorizar $P(x) = x^7+2x^4+x = x(x^3+1)^2$

	

\item Sea $P(x) = 21x^2+10x-2$. $P(x) + 3 = 21(x+1/3)(x+1/7)$.

\end{enumerate}




\section{Tema 2: Ecuaciones}

\subsection{Teoría sobre ecuaciones}


\begin{example}
\[
	-20 = -20 \dimplies 25-45 = 16-36 \dimplies 5^2-5·9 = 4^2-4·9 \dimplies 5^2-5·9+\left(\rfrac{9}{2}\right)^2 = 4^2-4·9+\left(\rfrac{9}{2}\right)^2 \dimplies
\]
\[
	\left(5-\rfrac{9}{2}\right)^2 = \left(4-\rfrac{9}{2}\right)^2 \text{\hl{\;\;;\;\;}} 5-\rfrac{9}{2} = 4-\rfrac{9}{2} \dimplies 5=4
\]
\end{example}


\obs Dividir por 0 no mantiene la equivalencia.
%
En general, tomar una raíz no mantiene equivalencia entre ecuaciones (tampoco elevar a una potencia).

\begin{defn}[Ecuaciones equivalentes]
Dos ecuaciones son equivalentes si tienen las mismas incógnitas y las mismas soluciones.
\end{defn}

\begin{example}

\[
	\frac{(x+1)(x-1)}{x+1} = 5 \implies x-1 = 5
\]

\end{example}

\paragraph{Clasificación de ecuaciones}

Las ecuaciones según sus soluciones pueden ser:
\begin{itemize}
	\item Incompatible: no tiene ninguna solución. Ejemplo: $5x=5x+2$
	\item Compatible determinada: tiene un número finito de soluciones. Ejemplo: $3x=6$.
	\item Compatible indeterminada: tiene infinitas soluciones. Ejemplo $2x-\frac{3x-1}{3} = x+\frac{1}{3}$. Solución: $x=λ, ∀λ∈ℝ$.
\end{itemize}


\begin{example}
	¿Son equivalentes?
	\begin{itemize}
		\item $9x=3x^2 \dimplies 9=3x$ [CD]
		\item $9=3x \dimplies x=3$ [CD]
		\item $4=5 \dimplies 1=0$ [IN]
		\item $9x=3^2x \dimplies 0x=0$ [CI]
		\item[difícil] $9x=\frac{(3x)^2}{x} \dimplies 9x=9x$ [CI]

\paragraph{Conclusiones:} \textbf{¡Ojo con simplificar ecuaciones!} Cuando "desaparezcan" incógnitas mirar con cuidado, porque estaremos perdiendo soluciones en la inmensa mayoría de los casos.

¿Cuándo no?

	\item $\frac{\sqrt{9x^2}}{x^2}=\frac{3x}{x^2} \dimplies \frac{9}{x} = \frac{9}{x} \dimplies 0=0$
	\end{itemize}
\end{example}



\subsection{Racionales}

Ecuaciones racionales.

\paragraph{Ejemplo}
\[
	\frac{2x}{x-2} + \frac{3x}{x+2} = \frac{7x^2}{x^2-4} \dimplies \frac{2x(x+2)}{(x-2)(x+2)} + \frac{3x(x-2)}{(x+2)(x-2)} = \frac{7x^2}{x^2-4} \dimplies 
\]
\[
	\frac{2x(x+2)+3x(x-2)}{x^2-4} = \frac{7x^2}{x^2-4} \text{\hl{$\implies$}} 2x^2+4x+3x^2-6x=7x^2 \dimplies 5x^2-7x^2-2x = 0 \dimplies 
\]
\[
	x(-x-1) = 0 \dimplies x_1 = 0 \wedge x_2 = -1
\]

\hl{¿Son soluciones las 2?}

La equivalencia la hemos perdido si $x\neq \pm2$, por lo que las soluciones $x_1 = 0 \wedge x_2 = -1$ son válidas (\ul{y no es necesario hacer la comprobación}).



\hl{Ejercicios: 84 ad + propios con trampa de equivalencias}


\paragraph{Cuidado:} casuística nueva de equivalencias e implicaciones. Hasta ahora, los valores peculiares de las implicaciones que no son equivalencias sólo nos ahorraban alguna comprobación\footnote{que tampoco está demás hacer para asegurarnos que hemos operado bien}.
%
Pero puede darse el caso de que pasen otras cosas. Por ejemplo, ¿qué pasa en esta simplificación?

\[(x-1)·\frac{x}{x+1} = (x-1)·(x^2-9) \text{\hl{$\implies$}} \frac{x}{x+1} = x^2-9\]

A la izquierda el 1 es solución y a la derecha no, luego las ecuaciones no son equivalentes. Pero, ¿qué pasa con el $x=1$?

En esta implicación he \ul{perdido una solución}. Es una equivalencia si $x\neq 1$, y en el caso $x=1$, tengo una solución. 

\paragraph{Más ejemplos}
\begin{itemize}
	\item
	\[
		\frac{1}{1-\frac{1}{x+1}} = \frac{x+1}{x} \dimplies \frac{1}{\frac{x+1-1}{x+1}}=\frac{x+1}{x} \dimplies \frac{x+1}{x} = \frac{x+1}{x} \text{\hl{$\implies$}} x=λ, ∀λ∈ℝ\setminus\{0,-1\}
	\]

	\item
	\[
		\frac{1+\displaystyle\frac{x+1}{x-1}}{2-\displaystyle\frac{x-1}{x+1}}=2 \dimplies \frac{\displaystyle\frac{x-1+x+1}{x-1}}{\displaystyle\frac{2x+2-x+1}{x+1}} = 2 \dimplies
	\]
	\[	
		\frac{\displaystyle\frac{2x}{x-1}}{\displaystyle\frac{x+3}{x+1}}=2 \text{\hl{$\implies$}} 2x^2+2x=x^2+4x-6 \dimplies 2x=6 \dimplies x=3
	\]

	\item

	\[
		\frac{3}{x} - \frac{x}{x+2} = \frac{5x-1}{x^2+x-2}
	\]

	\item Ejercicios 83 y siguientes del libro.
\end{itemize}

\subsection{Ecuaciones irrracionales}

\paragraph{Ejemplo:}
\[
	\sqrt{x+1} - \sqrt{x^2-5}=0 \text{\hl{$\implies$}} x+1 = x^2-5 \dimplies (x_0,x_1) = (3,-2)
\]

\textbf{Comprobamos} porque hemos perdido la equivalencia: 

$\sqrt{-2+1} = \sqrt{(-2)^2-5} \dimplies \sqrt{-1} = \sqrt{-1}$; -2 no es una solución en los reales.

Por otro lado: $\sqrt{3+1} = \sqrt{3^2-5} \dimplies \sqrt{2}=\sqrt{2}$

\textit{La comprobación no sería necesaria si hubiéramos reparado en que la equivalencia se mantendría siempre que el interior de las raíces fuera positivo, es decir $x\geq -1 \wedge x\geq \sqrt{5}$}

\paragraph{Ejercicio:} 
\[
	\sqrt{x+4}+\sqrt{x-1} = 5 \text{\hl{$\implies$}} (x+4)+(x-1) + 2\sqrt{(x+4)(x-1)} = 25 \text{\hl{$\implies$}} (22-2x)^2 = 4(x^2+3x-4) \dimplies 
\]
\[
	4x^2-88x + 484 = 4x^2+12x-16 \dimplies -100x + 500 = 0 \dimplies x=5
\]

Comprobamos:
\[
	\sqrt{5+4}+\sqrt{5-1} = 3+2 = 5
\]

\textit{Aquí la comprobación si resulta mucho más útil porque nos ahorra resolver la inecuación $(x+4)(x-1) \geq 0$}

\begin{itemize}
	\item Dudas.
	\item Corregir irracional. 89b.
	\item 85e,90a no lo hagáis.
	\item Empezamos 
\end{itemize}



\subsection{Exponenciales y logarítmicas}

Pregunta: $a^x=a^y \overset{?}{\dimplies} x=y$

$x=y\implies a^x=a^y$ Sí.

$a^x=a^y\implies x=y$ No. Contraejemplo: $1^2=1^3$.  Basicamente, si los logaritmos no mantenían la equivalencia, tampoco lo iban a hacer estas.

Siempre que la base no sea $0,±1$ sí serán equivalentes. ¿Y si tenemos un polinomio como base? Pues como puede ser uno de esos valores, no mantenemos la equivalencia o calculamos para qué valores sí sería equivalente o no.

Trabajamos 108abc, 109ac,111c (el más difícil de todos)

\subsection{Logarítmicas}

Los logaritmos tampoco conservan las equivalencias:

Versión innecesariamente larga:
\[
	-5 = -5 \dimplies -30+25 = 1-6 \dimplies -30+25+9 = 9+1-6 \dimplies (3-5)^2 = (3-1)^2 \dimplies 
\]
\[
	\log(3-5)^2 = \log(3-1)^2 \dimplies 2\log(3-5) = 2\log(3-1) \dimplies \log(3-5)=\log(3-1) \dimplies \log2=\log-2
\]

Versión corta:
\[
	(-2)^2 = (2)^2 \text{\hl{$\implies$}} 2\log(-2) = 2\log(2) \dimplies \log(-2) = \log(2) \dimplies -2=2
\]

Ecuación de ejemplo:


\[
	\log x=t \implies 5t=3t+\log6^2 \dimplies 2t=2\log6 \dimplies t=\log6 \dimplies \log x=\log6 \text{\hl{$\implies$}} x=6
\]

Si estás más versado en la abstracción algebraica:

\[
	5\log x=3\log x+2\log 6 \dimplies 2\log x=2\log6 \text{\hl{$\implies$}} x=6
\]


Incluso, habría una tercera manera:
 
\[
	5\log x=3\log x+2\log 6 \dimplies \log x^5=\log36x^3 \dimplies x^5=36x^3 \dimplies
\]
\[
	x^5-36x^3 = 0 \dimplies x^3(x^2-36) = 0 \dimplies x^3(x+6)(x-6) = 0
\]




\paragraph{Ejercicio}
\[
	\log\frac{2x-2}{x} = 2\log(x-1)-\log x \dimplies \log \frac{2x-2}{x}=\log\frac{(x-1)^2}{x} \cimplies \frac{2(x-1)}{x} = \frac{(x-1)^2}{x} \overset{1}{\cimplies}
	\]
	\[ 
	x-1=2 \dimplies x=3
\]
En 1 hemos simplificado 2 factores. $x$ y $(x-1)$. En esta simplificación podríamos haber perdido soluciones, en concreto, si $0,1$ fueran soluciones no lo obtendríamos. 

En este caso no son solución porque $\log 0$ no existe.

\paragraph{Ejercicio}
\[
\frac{\log (4-x)}{\log(x+2)}=2 \cimplies \log(4-x) = \log(x+2)^2 \cimplies 4-x=(x+2)^2 \dimplies 4-x=x^2+4x+4 \dimplies
	\]
	\[ x^2+5x=0 \dimplies x_1=0 \wedge x_2=-5
\]
$x_2=-5$ no es solución porque $\nexists\log(-5+2)=\log(-3)$. Por otro lado, $\frac{\log4}{\log2} = \log_24=2$ cqc.



\paragraph{Ejercicio} mientras corrigen

\[
	\log_x 3 = \ln \sqrt{3} \dimplies \frac{\ln3}{\ln x}=\ln\sqrt{3} \dimplies \frac{\ln3}{\ln x}=\frac{\ln3}{2} \dimplies \frac{1}{\ln x}=\frac{1}{2} \implies \ln x = 2 \implies e^2=x
\]

\paragraph{Ejercicio}
\[
	\log_332+\log_{\rfrac{1}{3}}(6-x) = \log_{\sqrt{3}}x \dimplies \log_332+\frac{\log_3(6-x)}{\log_3\rfrac{1}{3}} = \frac{\log_3x}{\log_3{\sqrt{3}}} \dimplies 
\]
\[
	\log_332-\log(6-x)=2\log_3x\dimplies \log_3\left(\frac{32}{6-x}\right)=\log_3x^2 \cimplies 32=x^2(6-x) \dimplies -x^3+6x^2-32 = 0
\]
\[
	-(-2)^3 + 6(-2)^2-32 = 8+24-32 = 0\implies x_1=-2 \wedge x_2=x_3=4
\]
 
Comprobamos:

\[
	\log_332+\log_{\rfrac{1}{3}}(6-4) = \log_{\sqrt{3}}4 \dimplies \log_332-\log_32=\log_34^2 \dimplies \log_3\frac{32}{2}=\log_316 \;\;\text{   cqc.}
\]



\section{Sistemas de ecuaciones}

Minimísimo repaso a la reducción como método para resolver sistemas de ecuaciones.

\subsection{Sistemas lineales: Gauss}

\begin{enumerate}
	\item Sistema triangulado.
	\item Triangular yo un SCD.
	\item Triangular ellos un SCD.
	\item Deberes: Triangular ellos un SCD.
	\item Entregar la teoría. Yo 2 ejemplos: SCI, Incomp.
	\item Ellos SCI, incomp.
	\item Problemas.
\end{enumerate}

Por grupos, resolver:
\[
\left\{\begin{array}{lccccc}
e_1: &2x&+y&-2z&=&7\\
e_2: &x&+y&+z&=&0\\
e_3: &3x&+2y&+2z&=&1
\end{array}\right\} \dimplies (x,y,z) = (1,1,-2)
\]



\begin{itemize}
	\item Corregimos (si quieren) 113 incompatible y CI.
	\item ¿En qué consiste discutir un sistema? Escribir completo los 3 casos
	\item Sistemas con parámetros
\end{itemize}

\paragraph{Discusión de un sistema}

Una vez llegado al \hl{sistema escalonado} pueden darse 3 situaciones:

\begin{itemize}
	\item La ecuación con una única incógnita es incompatible $\implies$ Sistema Incompatible.
	\item Número de incógnitas > número de ecuaciones $\implies$ Compatible indeterminado.
	\item Número de incógnitas = número de ecuaciones, siendo la última ecuación compatible determinada $\implies$ Compatible determinado.
\end{itemize}


\subsection{Sistemas no lineales}

Ejercicios: 114cf (f es interesante),115acdf

\[
\left\{
	\begin{array}{c}
		x^2-2xy+y^2 = 1\\
		x^2-y^2 = 12\\
	\end{array}
\right\}
\]