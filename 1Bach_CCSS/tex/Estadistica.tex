
\section{Estadística unidimensional descriptiva}

\paragraph{¿Para qué sirve la estadística?} Sirve para hacer predicciones. Estudiar los datos permite obtener información para poder hacer predicciones. 

Por ejemplo, si la nota media en una clase es de $\overline{x}=5$ y sabemos que los datos varían muy poco, es probable que si he perdido un examen, su nota fuera a ser un 5. En cambio, si la nota media es $\overline{x} = 5$ pero los datos están muy dispersos y hay desde $0$ hasta $10$, no tengo mucha posibilidad de estimar su nota sin equivocarme.

\subsection{Vocabulario estadístico}

Rellenar de ejemplos el margen del libro a lápiz.

\begin{itemize}
	\item Población
	\item Muestra
	\item Individuo
	\item Variables:
		\subitem Cualitativas o cuantitativas.
		\subitem Discretas o continuas.
\end{itemize}

\subsection{Tabla de frecuencias}
Ejercicio 4.

\subsection{Medidas}

\subsubsection{Centralización}
Ejercicios 10 y 12.

\subsubsection{Posición}

\subsubsection{Dispersión}

\section{Probabilidad, binomial y normal}

\subsection{Probabilidad}

\subsection{Binomial}

\subsection{Normal}


\section{Estadística bidimensional}

