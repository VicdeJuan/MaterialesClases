
\section{Estadística unidimensional descriptiva}

\paragraph{¿Para qué sirve la estadística?} Sirve para hacer predicciones. Estudiar los datos permite obtener información para poder hacer predicciones. 

Por ejemplo, si la nota media en una clase es de $\overline{x}=5$ y sabemos que los datos varían muy poco, es probable que si he perdido un examen, su nota fuera a ser un 5. En cambio, si la nota media es $\overline{x} = 5$ pero los datos están muy dispersos y hay desde $0$ hasta $10$, no tengo mucha posibilidad de estimar su nota sin equivocarme.

\subsection{Vocabulario estadístico}

Rellenar de ejemplos el margen del libro a lápiz.

\begin{itemize}
	\item Población
	\item Muestra
	\item Individuo
	\item Variables:
		\subitem Cualitativas o cuantitativas.
		\subitem Discretas o continuas.
\end{itemize}

\subsection{Tabla de frecuencias}
Comentar la tabla resuelta. 

\begin{itemize}
	\item ¿Por qué $\sqrt{n}$? Por simetría. Para tener el mismo número de intervalos que de datos por intervalo.
	\item Las $h_i$ y $H_i$ son porcentajes. ¿Cuánta gente está por debajo de $x$ kilos? Miramos la $H_i$ que corresponda.
	\item Interpretación de $x_i$. ¿Para qué sirven entonces los intervalos? Porque en realidad la variable con la que trabajo es una variable continua.
\end{itemize}

\subsection{Medidas}

\subsubsection{Centralización y posición}


\paragraph{Medidas de posición}

Las medidas de posición son los percentiles, aunque cuando dividimos la muestra en 4 partes hablamos de cuartiles. Al dividir la muestra en 10, hablamos de deciles. Al dividir la muestra en 2, hablamos de la mediana.



\paragraph{Utilidad de la mediana:} comparemos 2 muestras:
\[A=\{0,0,0,0,0,9,9,9,9\} \to \overline{x}_A = \frac{9*4}{9}=4\;\;\; Me_A = 0\]
\[B=\{4,4,4,4,4,4,4,4,4\} \to \overline{x}_A = \frac{9*4}{9}=4\;\;\; Me_A = 4\]

Estas 2 muestras tienen la misma media. La mediana nos ayuda a distinguir. 

\subparagraph{Sensibilidad a datos atípicos}
Supongamos una variante de la muestra $A$ anterior:

\[A'=\{0,0,0,0,0,9,9,9,90\} \to \overline{x}_A = \frac{9*3+90}{9}=29,25\;\;\; Me_A = 0\]

Por un dato atípico, la media sale realmente disparada. La mediana, en cambio, no se ve afectada.

\paragraph{Ejercicio 54 del libro}

\paragraph{Ejercicios prácticos de calcular percentiles} sobre la tabla resuelta de la página 223. 
%
Calculamos los cuartiles, deciles y percentiles.

\begin{defn}[Percentil n]
Es el dato que tiene a su izquerda el $n\%$ de los datos de la muestra.
\end{defn} 

\[
\begin{array}{c|c|c}
x_i & f_i&F_i\\\hline
1&15&15\\
2&7&22\\
3&33&56\\
4&10&6
\end{array}
\]

Vamos a calcular $P_{47}$

\[
P_{47} \to 47\% · 66 = 22,82 \to P_{47} = d_{23} = 3
\]

El $P_{47}$ debería ser el dato que está en la posición $22,82$. Como esa posición no existe, podríamos pensar en hacer la media entre el $d_{22}$ y el $d_{23}$. Incluso, una media ponderada. Pero no vamos a meternos tanto. 
%
Para calcular percentiles nos quedaremos con la primera frecuencia acumulada que supera, es decir, $\displaystyle P_k = d_{i}$ donde $i$ es el mínimo que cumple $F_i\geq k\%·N$

Cuando trabajamos con muestras de datos pequeños, los percentiles no funcionan bien. Cuando trabajamos con muestras grandes sí. ¿Qué sentido tiene esto? 
%
Si queremos medir la dispersión de 4 datos, no tiene mucho sentido, porque ya, simplemente viendo los datos nos lo imaginamos bien. 
%
¿Cuándo tienen sentido los percentiles? 
%
Cuando un bebé nace, querríamos saber si su peso está dentro de unos márgenes normales. 
%
Si el peso del hijo es $P_{99}$, es de los bebés más gordos que han nacido, ya que el $99\%$ de los bebés nacidos han pesado menos que él. 

\textit{En el ejemplo de los bebés, al tener tantísimos datos, los percentiles se comportan de una manera razonable. ¿Cómo corregir los percentiles para que funcionen en muestras de datos pequeñas? Estudia Matemáticas.}

\obs Con la mediana sí nos podemos poner exquisitos y calcular su valor exacto.

\subsubsection{Dispersión}

\textit{Que todo el mundo copie esto. Por si lo necesitas para tu vida futura. Yo no me sé las fórmulas de memoria. Sé de dónde salen y las razono y eso voy a intentar con vosotros.}

Una medida de la dispersión bastante intuitiva sería: ¿Cuánto se separa cada dato de la media? Y si sumo todas esas desviaciones, debería obtener una medida general. ¿Y por qué no cuánto se separa de la mediana? Total, la mediana también es una medida de centralidad. \footnote{Por esto necesitamos que la mediana sea un valor concreto y no puede ser un intervalo.}
%
Por motivos más complejos (de los que hablaremos al llegar a regresión), la que se utiliza habitualmente es la media, aunque en ocasiones sí se utiliza la mediana. ¿Por qué? Porque es menos sensible a datos atípicos.

Supongamos una muestra $A=\{5,5,5,5,5,5,5,5\}$ ¿Cuál sería su desviación? Sea cual sea lo que entendamos por desviación, debería ser 0.

Supongamos otra muestra, algo más compleja: $B = \{ 2,2,2,3,3,3,4,4,4,5,5,5 \} \to \overline{x}_B=3,5$
\paragraph{Opción 1:} Suma de las desviaciones

\[ \sum_{i=1}^{12} (x_i-\overline{x}) = 0\]

¿Porqué ocurre esto? Claramente los datos no datos uniformes. Tomando $i=1\to x_1-\overline{x} = -1,5$. Por otro lado, $i=12 \to x_{12}-\overline{x} = 1,5$, los datos se cancelan al sumarlos.

\paragraph{Opción 2:} Para solucionarlo podríamos pensar en el valor absoluto, para que esto no ocurra.

\[ \sum_{i=1}^{12} \abs{x_i-\overline{x}} = 12\]

Esto ya parece algo más razonable, pero el valor absoluto es una operación matemática con la que resulta muy complicado trabajar, porque aparecen ramas... 

\subparagraph{Opción 2.2: } Media de las desviaciones ¿Y si tengo el doble de datos pero igual de distribuidos?  Esta manera de calcular la desviación aumenta. ¿Cómo corregirlo? Dividiendo por el número de datos.

\[ \frac{\sum_{i=1}^{12} \abs{x_i-\overline{x}}}{12} = 1\]

\paragraph{Opción 3:} Otra manera de poner positivo algo que es negativo sería elevar a una potencia de índice par. Por ejemplo, 2.

\[ \frac{\sum_{i=1}^{12} \left(x_i-\overline{x}\right)^2}{12} = \frac{5}{4} = 1.25\]

\subparagraph{Opción 3.2} ¿Y podría elevar a orden 4? Hombre, por poder sí, pero entonces las diferencias pequeñas se hacen demasiado pequeñas y las grandes demasiado grandes.

\[ \frac{\sum_{i=1}^{12} \left(x_i-\overline{x}\right)^4}{12} = \frac{41}{16} > 2·\frac{5}{4}\]


\paragraph{Opción 4:} A ver, si hemos elevado al cuadrado, hemos introducido artificialmente un mucho ruido en la medición. ¿Cómo podríamos contrarrestar eso? Haciendo la raíz cuadrada.

\[ \sqrt{\frac{\displaystyle\sum_{i=1}^{12} \left(x_i-\overline{x}\right)^2}{12}} = 1.12\]

\begin{table}[hbtp]
\centering
\label{SD:medidasDispersion}
\caption{Distintas opciones para medir la dispersión con respecto a la media.}
\begin{tabular}{c|ccc}
\textbf{Nombre} & \textbf{Medidas} & \textbf{Medida razonable} & \textbf{Contras}\\\hline\hline
 & $\displaystyle\sum_{i=1}^{N} (x_i-\overline{x})$ & No & Puede resultar 0 en datos sí dispersos\\\\
Desv. media  & $\displaystyle\frac{\displaystyle\sum_{i=1}^{N} \abs{x_i-\overline{x}}}{N}$ & Sí & Difícil de trabajar con el valor absoluto (derivadas)\\\\
Varianza $\sigma^2$ & $\displaystyle \frac{\displaystyle\sum_{i=1}^{N} \left(x_i-\overline{x}\right)^2}{N}$ & Sí. & Medida algo falseada por elevar al cuadrado\\\\
Desv. típica $\sigma$ & $\displaystyle \sqrt{\frac{\displaystyle\sum_{i=1}^{N} \left(x_i-\overline{x}\right)^2}{N}}$ &Sí. & Corrección artificial de elevar al cuadrado\\\\\hline
\end{tabular}
\\
\textit{\textbf{Obs:} Todas estas mismas medidas se pueden calcular respecto de la mediana. Cambiarían los nombres}
\end{table}


\textbf{Repasamos medidas de dispersión: } fórmula y ya está. Tizas de colores para las letras.

\begin{defn}[Varianza]
\[
	\sigma^2 = \frac{\displaystyle\sum_{i=1}^{N} \left(x_i-\overline{x}\right)^2}{N}
\]
Donde $N$ es el número total de datos, $x_i$ cada uno de los datos y $\overline{x}$ la media.

\obs En caso de haber datos $x_i$ repetidos, existirá un número de veces que cada dato se repite $f_i$. 
%
Podemos calcular la varianza de una manera más sencilla:
\[
	\sigma^2 = \frac{\displaystyle\sum_{i=1}^{n} f_i·\left(x_i-\overline{x}\right)^2}{N}
\]
Donde $n$ el número de clases (filas de la tabla de frecuencias) y lo demás se mantiene.
\end{defn}

\begin{prop}
	\[\sigma^2 = \frac{\displaystyle\sum_{i=1}^{N} \left(x_i-\overline{x}\right)^2}{N} = \frac{\displaystyle\sum_{i=1}^N x_i^2}{N} - \overline{x}^2\]
\obs Por los datos repetidos, se cumple:
	\[\sigma^2 = \frac{\displaystyle\sum_{i=1}^{n} f_i\left(x_i-\overline{x}\right)^2}{N} = \frac{\displaystyle\sum_{i=1}^N f_ix_i^2}{N} - \overline{x}^2\]
\end{prop}

Si alguien me demuestra esta proposición fuera de clase +0,25 en el examen (que te recuerdo que vale 2 tercios).

\begin{proof}
	\begin{dmath}
	\sigma^2 = \frac{\displaystyle\sum_{i=1}^{N} \left(x_i-\overline{x}\right)^2}{N} =
	  \frac{\displaystyle\sum_{i=1}^{N} \left(x_i^2+\overline{x}^2-2x_i\overline{x}\right)}{N} = 
	  \frac{\displaystyle\sum_{i=1}^{N} x_i^2}{N} + \underbrace{\frac{\displaystyle\sum_{i=1}^{N}\overline{x}^2}{N}}_{\small\displaystyle\sum_{i=1}^9 z = 9·z}-\frac{\displaystyle\sum_{i=1}^{N} 2x_i\overline{x}}{N} =
	  \frac{\displaystyle\sum_{i=1}^{N} x_i^2}{N} + \frac{N·\overline{x}^2}{N}-2\overline{x}\underbrace{\frac{\displaystyle\sum_{i=1}^N x_i}{N}}_{\text{ def. }\overline{x}} = 
	  \frac{\displaystyle\sum_{i=1}^{N} x_i^2}{N} + \overline{x}^2-2\overline{x}^2 = 
	  \frac{\displaystyle\sum_{i=1}^{N} x_i^2}{N} - \overline{x}^2
	\end{dmath}
	En caso de datos repetidos, donde $N$ no es el número total de datos, sino de filas de la tabla. Llamamos $n$ al número total de datos
	\begin{dmath}
		\sigma^2 = \frac{\displaystyle\sum_{i=1}^{n} \left(x_i-\overline{x}\right)^2}{N} =
	  \frac{\displaystyle\sum_{i=1}^{n} f_i\left(x_i^2+\overline{x}^2-2x_i\overline{x}\right)}{N} = 
	  \frac{\displaystyle\sum_{i=1}^{n} f_ix_i^2}{N} + \underbrace{\frac{\displaystyle\sum_{i=1}^{n}f_i\overline{x}^2}{N}}_{\small\displaystyle\sum_{i=1}^9 z = 9·z}-\frac{f_i\displaystyle\sum_{i=1}^{n} 2x_i\overline{x}}{N} =
	  \frac{\displaystyle\sum_{i=1}^{n} x_i^2}{N} + \frac{N·\overline{x}^2}{N}-2\overline{x}\underbrace{\frac{\displaystyle\sum_{i=1}^N x_i}{N}}_{\text{ def. }\overline{x}} = 
	  \frac{\displaystyle\sum_{i=1}^{n} x_i^2}{N} + \overline{x}^2-2\overline{x}^2 = 
	  \frac{\displaystyle\sum_{i=1}^{n} x_i^2}{N} - \overline{x}^2
	\end{dmath}
\end{proof}

En la varianza se eleva al cuadrado cada diferencia de una manera algo arbitraria. 
%
Para corregir ese falseamiento se podría utilizar la raíz cuadrada.

\begin{defn}[Desviación típica]
	\[ 
		\sigma = \sqrt{\sigma^2} = \sqrt{\displaystyle\frac{\displaystyle\sum_{i=1}^{N} \left(x_i-\overline{x}\right)^2}{N}}
	\]
\end{defn}

\obs Para la varianza se eleva al cuadrado, pero se podría elevar a cualquier número par. Lo único que buscamos es que la suma se vuelva positiva.

\begin{defn}[Mi varianza personal]
\[
	\sigma^{2a} = \frac{\displaystyle\sum_{i=1}^{N} \left(x_i-\overline{x}\right)^{2a}}{N}
\]
Donde $N$ es el número total de datos, $x_i$ cada uno de los datos, $\overline{x}$ la media y $a\in\mathbb{N}$
\end{defn}


\begin{defn}[Mi desviación típica personal]
\[
	\sigma^{a} = \sqrt{\displaystyle\frac{\displaystyle\sum_{i=1}^{N} \left(x_i-\overline{x}\right)^{2a}}{N}}
\]
Donde $N$ es el número total de datos, $x_i$ cada uno de los datos, $\overline{x}$ la media y $a\in\mathbb{N}$
\end{defn}

\newcommand{\coefdemierda}{Coeficiente de variación\xspace}
\begin{defn}[\coefdemierda]
Se define como $\frac{\sigma}{\overline{x}}$
\end{defn}

\obs ¿Qué ocurre con el \coefdemierda si $\overline{x} = 0$? 
Por eso no se utiliza este coeficiente.



\subsection{Problema práctico}

27,28,39,55,50 voluntario para entregar a limpio en menos de 7 días.

Problema 21 

Trabajar en 29, 30 (puede ser esquemático),32 ,35 ,43 ,51 ,53 ,54 

\section{Estadística bidimensional}
Preparar PPT con los ejemplos del libro:

Clase 1) Tabla de datos bidimensionales. Construir una, calcular distribuciones marginales y condicionadas para interiorizar bien las tablas, lo que significan, etc.

Algún ejercicio de marginales + diagramas de dispersión. Covarianza vs correlación

Covarianza: cuánto varía cada dato respecto de la media. $\sigma_{xy}$. No hace falta que te sepas las fórmula porque vamos a utilizar la calculadora a mansalva. 

Esperadme en clase mañana sentados por grupos de calculadoras iguales.

Rectas de regresión e intuición gráfica.

Cálculo de todo con calculadora.

Recta de regresión de Y sobre X (PPT)

Tablita resumen del coeficiente de correlación (PPT)

La recta de regresión siempre pasa por el punto $(\gor{x},\gor{y})$

¿Qué recta calcular? ¿Y sobre X o X sobre Y? Si quiero predecir Y, hago Y sobre X. Si quiero predecir X, hacemos X sobre Y.

Recomendación: https://www.vitutor.com/estadistica/bi/ejercicios\_correlacion.html

\subsubsection{Causalidad vs Correlación}

\begin{itemize}
	\item https://www.telecinco.es/informativos/ciencia/beneficios-chupar-chupete-bebe_0_2661750040.html
	\item https://www.infosalus.com/nutricion/noticia-bebidas-endulzadas-artificialmente-reducen-riesgo-recurrencia-cancer-colon-muerte-20180723074434.html
	\item https://www.infobae.com/america/tendencias-america/2018/10/23/un-estudio-afirma-que-la-comida-organica-reduce-un-25-el-riesgo-de-cancer/
\end{itemize}

Ejercicios para clase: 64 b,d,e ; 65
Ejercicios para practicar: 62, 63

Ejercicios de exámenes de años anteriores.

\section{Probabilidad, binomial y normal}

\subsection{Combinatoria}

Principio multiplicativo:

Caminos para ir de $A$ a $C$, pasando por $B$. ¿Cuántos caminos posibes hay? 

% https://www.fing.edu.uy/tecnoinf/mvd/cursos/pye/materiales/practico/pye-pr02.pdf
\begin{enumerate}
	\item En un centro escolar hay 40 en 1º de ESO, 35 en 2º, 32 en 3º y 28 en 4º. Para hablar con la dirección se quiere formar una comisión que esté integrada por un alumno de cada curso. ¿Cuántas comisiones se pueden formar?
	{$40·35·32·28=54400$}
	\subitem Si el alumno de 1º de la ESO ya ha sido elegido. ¿Cuántas comisiones distintas se pueden formar?
	{$35·32·28=360$}
	\item ¿Cuántos números de tres cifras se pueden formar con las cifras 1, 2, 3 y 4 sin que se repita ninguna? 
	{$4·3·2·1 = 4! = 24$}
		\subitem b) ¿Cuántos terminan en 34? 
		{$2$, el $1234$ y el $2134$}
		\subitem c) ¿Cuántos habrá que sean mayores que 300?
		{24, porque el más pequeño, $1234$ es mayor que 300}
	\item ¿De cuántas maneras 4 personas en 10 sitios?
	{$10·9·8·7 = 5040$}
	\subitem ¿Y si pueden sentarse unos encima de otros? 
	{$10·10·10·10 = 10000$}
	\item ¿De cuántas maneras puedo ordenar 4 bolas?
	\item En una urna hay 6 bolas distintas, ¿De cuántas maneras distintas pueden sacarse? 
	\item En una urna hay 3 bolas rojas, 1 azul, 1 amarilla y 1 naranja ¿De cuántas maneras distintas pueden sacarse? 
	\item En una urna hay 8 bolas: 4 verdes y las otras 4 distintas. ¿De cuántas maneras distintas pueden sacarse?
	\item En una urna hay 10 bolas: 2 rojas, 4 verdes y las otras 4 distintas. ¿De cuántas maneras pueden sacarse?
	\item En una urna hay tres bolas rojas, tres verdes, cuatro negras y dos azules. ¿De cuántas maneras distintas pueden sacarse, bola a bola, de la urna?

%%%%%%%%%%%%%%%%%%%%%%%%%%%%%%%
	\item 10 alumnos. Se reparten 3 premios. Se dan varios casos: Premios iguales o distintos. Una persona puede recibir varios premios o no.
	\subitem ¿En qué caso deberían salir más posibilidades?
	\subitem Comprueba tu intuición calculando las posibilidades de cada caso.
	\item Número de diagonales de un cuadrado, de un hexágono, de un heptágono, de un polígono de $k$ lados.
	\item 5 hombres y 4 mujeres. Hombres en lugares impares y mujeres en lugares pares.
	\item Número de posibles matrículas.
	\item En un grupo de 10 amigos, ¿cuántas distribuciones de sus fechas de cumpleaños pueden darse al año?
	\item Lanzo 4 monedas al aire. ¿Cuántos resultados posibles puedo obtener?
	\item Un alumno tiene que elegir 7 de las 10 preguntas de un examen. ¿De cuántas maneras puede elegirlas? ¿Y si las 4 primeras son obligatorias?
	\item Una línea de tren tiene 25 estaciones. ¿Cuántos billetes distintos habrá que imprimir si cada billete lleva impresas las estaciones de origen y destino?
	\item ¿De cuántas maneras distintas pueden llegar 3 atletas a la meta? (Pueden llegar exactamente al mismo tiempo).
	\item ¿Cuántos números de DNI distintos existen? ¿Y números de teléfono?
	\item Una lista de Spotify tiene 30 canciones. ¿Cuántas posibles reproducciones de la lista completa, si las canciones se reproducen aleatoriamente, puede haber? ¿Y si la primera canción siempre debe ser la misma? ¿Y si no se repiten aleatoriamente, sino que siguen el orden alfabético?
	\subitem ¿Cuál es la probabilidad de que una lista de reproducción de 30 canciones reproducidas al azar siga el orden alfabético?
	\item Una persona tiene 6 chaquetas y 10 pantalones. ¿De cuántas formas distintas puede combinar estas prendas?
	\item Una familia, formada por los padres y tres hijos, van al cine. Se sientan en butacas consecutivas.
		\subitem a) ¿De cuántas maneras distintas pueden sentarse?
		\subitem b) ¿Y si los padres se sientan en los extremos?
	\item Con los números 3, 5, 6, 7 y 9 
		\subitem ¿cuántos productos distintos se pueden obtener multiplicando dos de estos números? 
		\subitem ¿Cuántos de ellos son múltiplos de 2? 
		\subitem ¿Cuántos cocientes distintos se pueden obtener dividiendo dos de estos números?
	\item ¿Cuántos números hay entre 2000 y 3000 que tengan sus cifras diferentes?
	\item ¿Cuántos resultados distintos pueden aparecer al lanzar un dado 4 veces?
	\item a) ¿Cuántos números de 6 cifras puedes escribir con los dígitos 1, 2 y 3?. b) ¿Cuántos de ellos contienen todos los dígitos 1, 2 y 3 al menos una vez? 
	\item ¿Cuántos números de 4 cifras puedes escribir con los dígitos 1,2,3 y 4? b) ¿Cuántos de ellos contienen los 4 dígitos? c) ¿Cuántos de ellos tienen alguna cifra repetida?
	\item Todas las personas que asisten a una reunión se estrechan la mano. Si hubo 105 apretones, ¿cuántas personas asistieron?
	\item ¿Cuántas columnas tenemos que cubrir para acertar seguro una quiniela?. Cada columna tiene 15 resultados a elegir entre 1, X, 2. 
	\item Ocho amigos van de viaje llevando para ello dos coches. Si deciden ir 4 en cada coche.
		\subitem a) ¿De cuántas formas pueden ir si todos tienen carnet de conducir?
		\subitem b) ¿De cuántas formas pueden ir si sólo tres tienen carnet de conducir?
	\item  ¿De cuántas maneras pueden ordenarse 6 libros en un estante si:
		\subitem a) es posible cualquier ordenación?
		\subitem b) 3 libros determinados deben estar juntos?
		\subitem c) dos libros determinados deben ocupar los extremos?
		\subitem d) tres libros son iguales entre sí?
	\item ¿Cuántas palabras distintas se pueden formar con las letras de la palabra SOBRE?
	\item ¿Cuántas palabras distintas se pueden formar con las letras de la palabra MATEMATICAS?
\end{enumerate}

\subsection{Probabilidad}

\subsubsection{Operaciones con conjuntos}


\begin{itemize}
	\item Union.
	\item Intersección..
	\item Diferencia.
	\item Complementario.
\end{itemize}


\begin{prop}[Propiedades de conjuntos]
Algunas propiedades de conjuntos, necesarias para probabilidad:

	\begin{itemize}
		\item $A\cup A=A$ y $A\cap A = A$
		\item $A\cup B=B\cup A$ y $A\cap B = B\cap A$
		\item $A\cup \emptyset=A$ y $A\cap \emptyset = \emptyset$
		\item $A\cup (B\cap C) = (A\cup B)\cap (A\cup C)$
		\item $A\cap (B\cup C) = (A\cap B)\cup (A\cap C)$
		\item $\overline{\overline{B}} = B$
		\item $\overline{A}\cap A = \emptyset$
		\item $A- B = A\cap \overline{B}$
		\item[DeMorgan] $\overline{A\cup B} = \overline{A}\cap\overline{B}$
		\item[DeMorgan] $\overline{A\cap B} = \overline{A}\cup\overline{B}$
	\end{itemize}
\end{prop}
\begin{proof}
Dibujaremos gráficamente cada propiedad.
\end{proof}







\subsubsection{Introducción a la Probabilidad}

{¿Qu\'e es una probabilidad? La respuesta intuitiva será un número entre $0$ y $1$. Primera en la frente: \hl{es una función}. Siempre es la "probabilidad de un suceso", ¿no? Necesito saber de qué suceso. 

Una probabilidad es una función. ¿Cuántas cosas recibe una función?

Todas las funciones tienen \hl{un origen y un destino}. Una probabilidad llega a un número real. ¿De qué parte? (Ejemplo de raíces: parte de números positivos para dar un número real).

Dado un dado, ¿cuál es su espacio muestral? Razonado con el mítico dibujito de funciones 
$E=\{1,2,3,4,5,6\}$. Ahora puedo calcular $P(\{3\}) = ...$, ¿no? Perfecto. ¿La probabilidad, entendida como función, \hl{¿parte del espacio muestral?} No, porque ¿$P(\{1\}\cup\{2\})$? 

Vamos a ver una \hl{definición axiomática} de la probabilidad. ¿Axioma? ¿A alguien le dice algo? Os lo cuento, fundamentalmente, para que tengáis una \textbf{visi\'on real de lo que son las Matemáticas.}}

Un axioma es una propiedad que se acepta sin demostración. A partir de ahí construimos la teoría al completo.

{¿Cuántas propiedades crees que hacen falta aceptar sin demostrar para la teoría de la probabilidad?}


\begin{defn}[Axiomática de Kolmogorov]
Dado un conjunto de sucesos elementales $E$ (llamado espacio muestral), $\mathcal{P}(E)$ al conjunto de todos los subconjutos de $E$ y una función $P$ que asigna un valor real a cualquier suceso $A$.

Decimos que $P$ es una probabilidad del espacio $E$ si cumple las siguientes propiedades:

\begin{itemize}
	\item $\forall A\in \mathcal{P}(E), P(A)\geq 0$
	\item $P(E) = 1$
	\item $\forall A\in \mathcal{P}(E)$ con $\{A_1,A_2,..., A_k\}$ sucesos incompatibles dos a dos. Entonces \[ P\left(\bigcup_{j=1}^k A_j\right) = \sum_{j=1}^k P(A_j) \]
\end{itemize}

\obs Llamamos \concept{suceso} a los elementos de $\mathcal{P}(E)$.

\end{defn}

\textit{¿Y esta definición por qué? ¿Dónde ha quedado eso de casos favorables entre posibles? Lo seguiremos utilizando, pero es una consecuencia lógica de estas propiedades.}

\textit{Sólo con estas 3 propiedades vamos a poder construir todas las demás. ¿Te acuerdas cuando el año pasado decía que las Matem\'aticas de 4º son un poco "de andar por casa"?}

\begin{example} Un dado. 

$E = \{1,2,3,4,5,6\}$

$\mathcal{P}(E) = \{1,2,3,4,5,6,\{1,2\},\{1,3\},\{1,4\} ... , \{6,6\},\{1,2,3\} ..., \emptyset \}$

$\emptyset \equiv$ "no sale nada"

Por cierto, ¿cuántos elementos tiene? $6+6·5+6·5·4+6·5·4·3+... + 6!+1$

Definiendo $P(e) = \rfrac{1}{6} \forall e\in E$, con la tercera propiedad podemos calcular:

$P(\{1\} \cup \{2\} = P(\{1\}) + P(\{2\}) = \rfrac{2}{6} = \rfrac{1}{3}$

\textit{Hacer énfasis en que la probabilidad recibe conjuntos, no números}

Calcular la probabilidad de sacar un número par y mayor que $3$ (Razonamiento con dibujos aplicando el tercer axioma).

\end{example}

\begin{example}
Sea $E=\{e_1,e_2,e_3\}$. Estudia si las siguientes funciones son espacios de probabilidad:

\begin{itemize}
	\item $P_1(e_1) = 1/2 ; P_1(e_2) = 1/2 ; P_1(e_3) = 1/3$
	\item $P_2(e_1) = 1/3 ; P_2(e_2) = 0 ; P_2(e_3) = 2/3$
\end{itemize}

\obs Fíjate lo que nos hemos abstraído, que no necesitamos nada de nada. 
\end{example}

\begin{prop}[Propiedades de una función de Probabilidad]
$\forall A,B \in \mathcal{P}(E)$
\begin{itemize}
	\item $P(\overline{A}) = 1 - P(A)$
	\begin{proof}
		\[1=P(E) = P(A\cup\overline{A} = P(A) + P(\overline{A})\]
	\end{proof}
	\item $P(\emptyset) = 0$ porque $\overline{\emptyset} = E$
	\item $A\subset B \dimplies P(A) \leq P(B)$
	\item $0\leq P(A)\leq 1$
	\item $P(A\cup B) = P(A) + P(B) - P(A\cap B)$
\end{itemize}
\end{prop}

\begin{prop}[Regla de Laplace]
En un espacio muestral equiprobable $E$ con $n$ elementos, podemos calcular:

\[\forall A\in \mathcal{P}(E)\;\;\; P(A) = \frac{\text{fav.}}{\text{pos}}\]
\end{prop}

Problemas 11,12,13,14


\subsection{Empezando probabilidad de 0}



Problemas 11 y 12.

\begin{problem}[2]
En una urna de lotería hay bolas 4 bolas amarillas, 3 rojas, 5 naranjas y 1 morada.

Calcula las siguientes probabilidades:

(Sacando de una en una)
\ppart No morada.

\textbf{Regla de LaPlace}

\ppart Ni roja ni amarilla.

(Sacando de varias en varias)

\ppart Las 3 bolas rojas.

\ppart 2 bolas rojas y una amarilla.

\solution
\end{problem}

\textbf{Tablas de contingencia:} explicar probabilidad condicionada desde tablas de contingencia.

En una tabla, traducir frases a probabilidades y sucesos con condicionadas.

\textit{Para resolver este problema en uno de los tercios de la pizarra hacemos una raya vertical discontinua a la derecha para ir escribiendo el resumen teórico con las fórmulas y demás. Sé que no les gusta y prefieren primero la teoría y luego ejemplos, pero considero importante que vean de dónde salen las fórmulas y que las razonen desde el ejemplo.}


\begin{problem}[G1]
En una clase hay 140 mujeres, de las cuales 60 llevan gafas. Hay también 60 hombres con gafas y 40 hombres sin ellas.

Si elegimos una persona al azar,
\ppart ¿Cuál es la probabilidad de que lleve gafas?
\ppart ¿Cuál es la probabilidad de que lleve gafas, sabiendo que es mujer?
\ppart ¿Cuál es la probabilidad de que lleve gafas, sabiendo que es hombre?

\solution



\spart Hay 120 (60+60) personas que llevan gafas. En total hay 240 personas (140+60+40), luego, utilizando la regla de Laplace:  $P(G) = \frac{90}{210} = \frac{3}{7}$

\obs Errores cometidos. ¿Qué es "G"?

\obs Enseñar a simplificar fracciones con la calculadora.

\obs Construimos la tabla de "contingencia", que es una buena manera de agrupar la información.

\hl{Construir tabla}

\spart Ahora sólo me importan mujeres, ¿no? Es como si no hubiera hombres. Para la regla de Laplace, mis casos posibles son 140 (las mujeres = $M$) y los favorables son 60 (las mujeres que llevan gafas = $M\cap G$)

Es decir: $P(\text{ gafas, sabiendo que son mujeres} ) = \frac{P(G\cap M)}{P(G)} = \frac{60}{140}$

\spart ...

\end{problem}


\begin{problem}[G2]
En una clase hay 140 mujeres, de las cuales 60 llevan gafas. Hay también 30 hombres con gafas y 40 hombres sin ellas.

Si elegimos una persona al azar,
\ppart ¿Cuál es la probabilidad de que lleve gafas?
\ppart ¿Cuál es la probabilidad de que lleve gafas, sabiendo que es mujer?
\ppart ¿Cuál es la probabilidad de que lleve gafas, sabiendo que es hombre?

\solution

\spart Construimos la tabla y contestamos a las preguntas.

\spart ...

\spart ...

\end{problem}

\subsubsection{Independencia}

\begin{defn}[Independencia]
2 sucesos son independientes si $P(A \tq B) = P(A)$
\end{defn}

\textbf{Ejemplos}
\begin{itemize}
	\item En el problema $G1$, ¿género y gafas son independientes?
	\item En el problema $G2$, ¿género y gafas son independientes?
\end{itemize}

Vamos a darle a la maquinaria de la lógica.


\hl{Corregir 17 y 18. 
Explicar independencia desde la intersección.
}
\begin{corol}
Si $A$ y $B$ son sucesos independientes ($P(A) = P(B\tq A)$)

$P(B\tq A) = \frac{P(B\cap A)}{P(A)} \to P(A\cap B) = P(A) · P(B\tq A) = P(A) · P(B)$
\end{corol}



\hl{Resolver el 17 como árbol}

\textbf{Problemas de Árbol}
Problema 76,75,77

\textbf{Probabilidad total}
Problema 82,83,84...

Resolviendo un problema de diagrama de árbol, razonar el teorema.
\begin{theorem}[Probabilidad Total]
Sea $\{A_i\}_0^n$ una partición del espacio muestral $E$.

Sea $B$ un proceso del espacio muestral.

$P(B) = \sum_{i=0}^n P(B\tq A_i)$
\end{theorem}


\textbf{Bayes}
Problema 87 (o alguno de los del libro de esa página)

\begin{theorem}[Bayes]
\begin{gather*}
\left\{
\begin{array}{c}
	P(A\tq B) = \frac{P(A\cap B)}{P(B)} \to P(A\cap B) = P(A\tq B)· P(B)\\
	P(B\tq A) = \frac{P(B\cap A)}{P(A)} \to P(B\cap A) = P(B\tq A)· P(A)
\end{array} \right\} 
\\
\implies P(A\tq B)· P(B) = P(B\tq A)· P(A) \dimplies 
\\
P(B\tq A) = \frac{P(A\tq B)· P(B)}{P(A)}
\end{gather*}
\end{theorem}

Problemas 88 y 89. \hl{¿Entran así en selectividad?}


\textbf{Cuestiones teóricas}


\section{Distribuciones de probabilidad}

\begin{problem}
Se tira un dado de 6 caras 100 veces. Calcula la probabilidad de:

\ppart A="sacar 100 veces un 6"

\ppart B="sacar exactamente una vez un 5 y todo lo demás 6"

\ppart C="sacar exactamente una vez un 5"

\ppart D="sacar exactamente 2 veces un 5"

\ppart E="sacar exactamente 3 veces un 5"

\ppart F="sacar exactamente $k$ veces un 5"

\solution

\end{problem}

\textit{Fijaos, hemos llegado a una fórmula general de cálculo de probabilidades para resolver este caso específico sin necesidad de hacer el árbol ni nada de nada. Ya sabéis que me gusta partir de ejemplos para deducir las fórmulas generales.
\\
Esto, tiene un nombre propio y es con lo que empezamos tema nuevo: Distribuciones de probabilidad.
\\
En el libro viene bastante bien, asíque lo vamos a seguir. Página 294.
}

4 conceptos a distinguir, leyendo las definiciones del libro.
\begin{itemize}
	\item Distribución de probabilidad: binomial.
	\subitem \textit{La manera que tiene la probabilidad de distribuirse entre los sucesos}.
	\item Variable aleatoria: en los apartados C-F, sería el número de cincos.
	\subitem $P(C) = P(X=1)$.
	\subitem $P(E) = P(X=3)$.
	\item Función de probabilidad: función que devuelve la probabilidad.
	\item Función de distribución [para dentro de unos días. Básicamente es la probabilidad acumulada.].
\end{itemize}

Ejemplo página 292.
\begin{itemize}
	\item Podría definir cualquier variable aleatoria.
\end{itemize}


\textit{Cuando estén claros:} 

¿Podemos calcular la media de una distribución? En el ejemplo de la 292 y después el ejemplo del número de cincos. ¿Se puede calcular? Y, calculando la media, ¿se puede calcular la desviación típica?

Ya hemos visto las páginas 292-294. Lo interesante de este tema son 2 distribuciones especiales y sólo haremos ejercicios relativos a ellas. Lo demás, solamente es útil para pasar a las distribuciones que verdaderamente se utilizan.

Página 295: distribución binomial. Ejemplo resuelto.

Reconocer distribuciones binomiales, ejercicio 43.

Deberes: 46


______

Ejercicio resuelto página 296. Ejercicios tipo para hacer con esto.

\subsection{Binomial}

Problemas. Media y varianza - desviación típica.

\subsection{Normal}

Desde la binomial a la normal.

% Código en Sage para dibujar.
%import scipy.stats
%n=200
%p=0.5
%binom_dist = scipy.stats.binom(n,p)
%bar_chart([binom_dist.pmf(x) for x in range(n*p*2)])

Vamos dibujando binomiales cada vez con más datos y probabilidades cambiantes (primero de los extremos de la probabilidad (0,0009) hacia los no extremos (0,4)). ¿Hay alguna aproximación? Tal vez haya una curva suave que se parezca...

Sí la hay.  Te presento a distribución normal. En estadística se denomina normal a todo lo que sigue esta distribución. Centrado, y más pegado o menos pegado.

Casi todo está cerca de la media. Esta distribución sólo necesita de la media y la desviación típica (que no varianza). Según la media y la desviación típica que le demos sean, la gráfica será de una u otra manera.

Podemos ver que todas tienen la misma forma y conociendo cómo funciona una de ellas, podríamos conocer todas las demás. Solamente hay que mover el centro y la reajustar la tripa. A este "mover el centro y reajustar la tripa" se le llama "tipificar", volverlo típico, de la distribución normal típica, la 0,1. 

[Utilizar el PPT de apoyo]


De la misma manera: ejemplo resuelto 3. Leo yo el enunciado, antes de decirles que es del libro. Problemas 15 y 16. Para comparar datos y poder tomar decisiones. 


\hl{Clase del viernes 1/02}
Corregimos el 16.

Una vez entendida la tipificación, vamos con el cálculo de probabilidades. ¿Por qué no se calcula la probabilidad de "medir exactamente 1,75"? Porque... en realidad, 1.75 es entre 1.745 y 1.754, luego ya tengo un intervalo. 
Además, lo habitual será preguntarme por "¿Cuánta gente es más alta de 1.90?". En general, me interesarán intervalos, por ello la normal va por intervalos.

Tipificar el volver típico. Vamos a dedicarnos hoy a familiarizarnos con la distribución normal típica y con la tabla de la normal

Supongamos que tengo una normal de media 0 y de desviación típica 1. La tabla lo que me da es lo que dice el dibujo, el área, la suma de las barras, la probabilidad de estar por debajo. Como la distribución es simétrica, la media está en el medio, coincide con la mediana (esto no siempre es así). La mitad de la gente está por debajo, la mitad por encima.

\begin{itemize}
	\item $P(Z<0)$
	\item $P(Z<1)$
	\item $P(Z<1,13)$
	\item $P(Z<1,1234)$
	\item $P(Z > 1) = 1 - P(Z\leq 1) = ...$ [contrario]
	\item $P(Z > 3) = $ [contrario]
	\item $P(Z  > -1) = P(Z < 1)$ [simetría]
	\item $P(Z  > -1,21) = P(Z < 1)$ [simetría]
	\item $P(Z  < -1) = P(Z > 1) = 1 - P(Z<1)$ [simetría y suceso contrario.]
	\item $P(Z  < -4) = P(Z > 1) = 1 - P(Z<1)$ [simetría y suceso contrario.]
\end{itemize}

Ejercicios 66,67,68,69[abc]; 72; 74 ; Tipificar + De deberes 66-69[de]

Supongamos que la altura media de una mujer es 1,70cm y la desviación típica 7cm. ¿Cuántas personas podríamos esperar que midieran más menos de 1.75?

Necesitaríamos la tabla de la normal de media 1,70cm y de desviación típica 7cm, ¿no? Otra opción, más sencilla porque no podemos hacer estadística con Excel (todavía) sería \textit{tipificar} los datos, volverlos típicos, llevárnoslos a nuestra escala \textit{típica}, que sería la normal de media 0 y desviación típica 1. Tipificamos este valor: $\frac{1,85-1,70}{7} = $

Aprendemos a utilizar la tabla de la normal. 

A calcular los que nos faltan. Simetría y susceso contrario.

Una vez hayamos aprendido, empezaremos a resolver $P(Z<k) = 0,2325$ y $P(X<k) = 0,2325$ y derivados.


\begin{itemize}
	\item Procedimiento general: conseguimos un número mayor que 0,5 a la derecha y después arreglamos el interior del paréntesis jugando con la simetría.
	\item $P(X>k) = 0,8413$ [-1,00]
	\[P(X>k) = P(X<-k) = 0,8413 \implies -k = 1 \implies k=-1\]
	\item $P(X>k) = 0,9920$ [-2,41]
	\[P(X>k) = P(X<-k) = 0,9920 \implies -k = 2,41 \implies k=-2,41\]
	\hrule{}
	\item $P(X<k) = 0,1587$ [-1,00]
	\[P(X<k) = 0,1587 \dimplies P(X>k) = 0,8413 \dimplies P(X<-k) = 0,8413 \implies -k = 1 \dimplies k=-1\]
	\item $P(X<k) = 0,2877$ [-0,56]
	\[P(X<k) = 0,2877 \dimplies P(X>k) = 0,7123 \dimplies P(X<-k) = 0,7123 \implies -k = 0,56\]
	\hrule{}
	\item $P(X>k) = 0,1492$ [1,04]
	\[ P(X>k) = 0,1492 \dimplies P(X<k) = 1-0,1492 = 0,8508 \implies k=1.04\]
	Comprobación \[P(X>1.04) = 1-P(X<1.04) = 1- 0.8508 = 0,1492\]
	\item $P(X>k) = 0,0078$ [2,42]
	\[ P(X>k) = 0,0078 \dimplies P(X<k) = 1-0,0078 = 0,9922 \implies k=2.42\]
	Comprobación \[P(X>2.42) = 1-P(X<2.42) = 1- 0.9922 = 0,0078\]
	\subitem $P(X>2k+1) = 0,1736$ [0,93]
	\[ P(X>2k+1) = 0,1736 \dimplies P(X<2k+1) = 1-0,1736 = 0,8264 \implies 2k+1=0.94 \implies k=-0,03\]
	\subitem $P\left(X<\frac{k-2}{3}\right) = 0,2877$ [0,56]
	\[ P\left(X<\frac{k-2}{3}\right) = 0,2877 \dimplies PP\left(X>\frac{k-2}{3}\right) = P\left(X<-\frac{k-2}{3}\right) = 0,7123\]
	\[ \implies -\frac{k-2}{3} = 0,56 \implies k=0,56·3+2 = k=3,68\]

	\item Ejercicios 87 y 85

	Deberes: ejercicio 75

\end{itemize}

\hl{Clase del lunes 11: } Corregimos el 75, el 85 y hacemos más de calcular media y desviación típica.

¿Qué pasa si el valor buscado no coincide en la tabla? Que cogemos el más cercano.

Después, problemas de enunciado natural.

Hallar media y desviación típica desde unas pocas probabilidades.

\hl{Clase martes 12/02}
Corregir 85 cd. Hacerlos bien [Del solucionario si hace falta].
Resultados del 75 ce, que cometimos el mismo error.


\subsection{Aproximación de la binomial a la normal}

Si $X\equiv B(n,p)$, se puede aproximar por $X\equiv N(\mu=n·p, \sigma=\sqrt{n·p·(1-p)}$. ¿Cuándo se puede hacer esto? Cuando $np>5$ y $n(1-p)>5$

Problema 19.

Problema 91. Antes de nada, ¿es una binomial? 
\begin{itemize}
	\item Independencia.
	\item Cuento número de éxitos.
	\item Tengo una probabilidad y un número de intentos.
\end{itemize}

Deberes: Problema 92. También se puede hacer.

Comentamos problema 94. Es una binomial, pero si quieres aproximar con la normal, es perfectamente válido, si lo argumentas.

\subsubsection{Cálculo de media y desviación típica}

\begin{itemize}
	\item $X\equiv N(\mu,2); P(X<5) = 0,9332$. $[\mu = 2]$
	\[
	P\left(X<5\right) = P\left(\frac{X-\mu}{\sigma}<\frac{5-\mu}{\sigma}\right) = P\left(Z<\frac{5-\mu}{2}\right) = 0,9332
	\]
	Buscamos en la tabla 0,9332 y tenemos que:
	$\frac{5-\mu}{2} = 1,5 \implies \mu=5-3=2$
	\item $X\equiv N(3,\sigma); P(X<10) = 0,6368$. $[\sigma = 20]$
	\[
		P\left(X<10\right) = P\left(\frac{X-3}{\sigma}>\frac{10-3}{\sigma}\right) =0,6368 \implies \frac{10-3}{\sigma}=0,35 \implies 
	\]
	\item $X\equiv N(\mu,\sigma); P(X<12) = 0,5; P(X>6) = 0,8413$. $[\mu = 12, \sigma = 6]$
	\[
		P(X<12) = 0,5 \implies \mu = 12
	\]
	\[
		P\left(X>6\right) = \left(\frac{X-12}{\sigma}>\frac{6-12}{\sigma}\right) = \left(Z<-\frac{-1}{2\sigma}\right) = 0,8413
	\]
	Consultando en la tabla: $-\frac{-1}{2\sigma} = 0,8413 \implies \sigma = \frac{1}{2·0,8413} = 1,79$
\end{itemize}




