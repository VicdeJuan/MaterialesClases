
\section{Estadística unidimensional descriptiva}

\paragraph{¿Para qué sirve la estadística?} Sirve para hacer predicciones. Estudiar los datos permite obtener información para poder hacer predicciones. 

Por ejemplo, si la nota media en una clase es de $\overline{x}=5$ y sabemos que los datos varían muy poco, es probable que si he perdido un examen, su nota fuera a ser un 5. En cambio, si la nota media es $\overline{x} = 5$ pero los datos están muy dispersos y hay desde $0$ hasta $10$, no tengo mucha posibilidad de estimar su nota sin equivocarme.

\subsection{Vocabulario estadístico}

Rellenar de ejemplos el margen del libro a lápiz.

\begin{itemize}
	\item Población
	\item Muestra
	\item Individuo
	\item Variables:
		\subitem Cualitativas o cuantitativas.
		\subitem Discretas o continuas.
\end{itemize}

\subsection{Tabla de frecuencias}
Comentar la tabla resuelta. 

\begin{itemize}
	\item ¿Por qué $\sqrt{n}$? Por simetría. Para tener el mismo número de intervalos que de datos por intervalo.
	\item Las $h_i$ y $H_i$ son porcentajes. ¿Cuánta gente está por debajo de $x$ kilos? Miramos la $H_i$ que corresponda.
	\item Interpretación de $x_i$. ¿Para qué sirven entonces los intervalos? Porque en realidad la variable con la que trabajo es una variable continua.
\end{itemize}

\subsection{Medidas}

\subsubsection{Centralización y posición}


\paragraph{Medidas de posición}

Las medidas de posición son los percentiles, aunque cuando dividimos la muestra en 4 partes hablamos de cuartiles. Al dividir la muestra en 10, hablamos de deciles. Al dividir la muestra en 2, hablamos de la mediana.



\paragraph{Utilidad de la mediana:} comparemos 2 muestras:
\[A=\{0,0,0,0,0,9,9,9,9\} \to \overline{x}_A = \frac{9*4}{9}=4\;\;\; Me_A = 0\]
\[B=\{4,4,4,4,4,4,4,4,4\} \to \overline{x}_A = \frac{9*4}{9}=4\;\;\; Me_A = 4\]

Estas 2 muestras tienen la misma media. La mediana nos ayuda a distinguir. 

\subparagraph{Sensibilidad a datos atípicos}
Supongamos una variante de la muestra $A$ anterior:

\[A'=\{0,0,0,0,0,9,9,9,90\} \to \overline{x}_A = \frac{9*3+90}{9}=29,25\;\;\; Me_A = 0\]

Por un dato atípico, la media sale realmente disparada. La mediana, en cambio, no se ve afectada.

\paragraph{Ejercicio 54 del libro}

\paragraph{Ejercicios prácticos de calcular percentiles} sobre la tabla resuelta de la página 223. 
%
Calculamos los cuartiles, deciles y percentiles.

\begin{defn}[Percentil n]
Es el dato que tiene a su izquerda el $n\%$ de los datos de la muestra.
\end{defn} 

\[
\begin{array}{c|c|c}
x_i & f_i&F_i\\\hline
1&15&15\\
2&7&22\\
3&33&56\\
4&10&6
\end{array}
\]

Vamos a calcular $P_{47}$

\[
P_{47} \to 47\% · 66 = 22,82 \to P_{47} = d_{23} = 3
\]

El $P_{47}$ debería ser el dato que está en la posición $22,82$. Como esa posición no existe, podríamos pensar en hacer la media entre el $d_{22}$ y el $d_{23}$. Incluso, una media ponderada. Pero no vamos a meternos tanto. 
%
Para calcular percentiles nos quedaremos con la primera frecuencia acumulada que supera, es decir, $\displaystyle P_k = d_{i}$ donde $i$ es el mínimo que cumple $F_i\geq k\%·N$

Cuando trabajamos con muestras de datos pequeños, los percentiles no funcionan bien. Cuando trabajamos con muestras grandes sí. ¿Qué sentido tiene esto? 
%
Si queremos medir la dispersión de 4 datos, no tiene mucho sentido, porque ya, simplemente viendo los datos nos lo imaginamos bien. 
%
¿Cuándo tienen sentido los percentiles? 
%
Cuando un bebé nace, querríamos saber si su peso está dentro de unos márgenes normales. 
%
Si el peso del hijo es $P_{99}$, es de los bebés más gordos que han nacido, ya que el $99\%$ de los bebés nacidos han pesado menos que él. 

\textit{En el ejemplo de los bebés, al tener tantísimos datos, los percentiles se comportan de una manera razonable. ¿Cómo corregir los percentiles para que funcionen en muestras de datos pequeñas? Estudia Matemáticas.}

\obs Con la mediana sí nos podemos poner exquisitos y calcular su valor exacto.

\subsubsection{Dispersión}

\textit{Que todo el mundo copie esto. Por si lo necesitas para tu vida futura. Yo no me sé las fórmulas de memoria. Sé de dónde salen y las razono y eso voy a intentar con vosotros.}

\begin{defn}[Rango]
\end{defn}

Una medida de la dispersión bastante intuitiva sería: ¿Cuánto se separa cada dato de la media? Y si sumo todas esas desviaciones, debería obtener una medida general. ¿Y por qué no cuánto se separa de la mediana? Total, la mediana también es una medida de centralidad. \footnote{Por esto necesitamos que la mediana sea un valor concreto y no puede ser un intervalo.}
%
Por motivos más complejos (de los que hablaremos al llegar a regresión), la que se utiliza habitualmente es la media, aunque en ocasiones sí se utiliza la mediana. ¿Por qué? Porque es menos sensible a datos atípicos.

Supongamos una muestra $A=\{5,5,5,5,5,5,5,5\}$ ¿Cuál sería su desviación? Sea cual sea lo que entendamos por desviación, debería ser 0.

Supongamos otra muestra, algo más compleja: $B = \{ 2,2,2,3,3,3,4,4,4,5,5,5 \} \to \overline{x}_B=3,5$
\paragraph{Opción 1:} Suma de las desviaciones

\[ \sum_{i=1}^{12} (x_i-\overline{x}) = 0\]

¿Porqué ocurre esto? Claramente los datos no datos uniformes. Tomando $i=1\to x_1-\overline{x} = -1,5$. Por otro lado, $i=12 \to x_{12}-\overline{x} = 1,5$, los datos se cancelan al sumarlos.

\paragraph{Opción 2:} Para solucionarlo podríamos pensar en el valor absoluto, para que esto no ocurra.

\[ \sum_{i=1}^{12} \abs{x_i-\overline{x}} = 12\]

Esto ya parece algo más razonable, pero el valor absoluto es una operación matemática con la que resulta muy complicado trabajar, porque aparecen ramas... 

\subparagraph{Opción 2.2: } Media de las desviaciones ¿Y si tengo el doble de datos pero igual de distribuidos?  Esta manera de calcular la desviación aumenta. ¿Cómo corregirlo? Dividiendo por el número de datos.

\[ \frac{\sum_{i=1}^{12} \abs{x_i-\overline{x}}}{12} = 1\]

\paragraph{Opción 3:} Otra manera de poner positivo algo que es negativo sería elevar a una potencia de índice par. Por ejemplo, 2.

\[ \frac{\sum_{i=1}^{12} \left(x_i-\overline{x}\right)^2}{12} = \frac{5}{4} = 1.25\]

\subparagraph{Opción 3.2} ¿Y podría elevar a orden 4? Hombre, por poder sí, pero entonces las diferencias pequeñas se hacen demasiado pequeñas y las grandes demasiado grandes.

\[ \frac{\sum_{i=1}^{12} \left(x_i-\overline{x}\right)^4}{12} = \frac{41}{16} > 2·\frac{5}{4}\]


\paragraph{Opción 4:} A ver, si hemos elevado al cuadrado, hemos introducido artificialmente un mucho ruido en la medición. ¿Cómo podríamos contrarrestar eso? Haciendo la raíz cuadrada.

\[ \sqrt{\frac{\displaystyle\sum_{i=1}^{12} \left(x_i-\overline{x}\right)^2}{12}} = 1.12\]

\begin{table}[hbtp]
\centering
\label{SD:medidasDispersion}
\caption{Distintas opciones para medir la dispersión con respecto a la media.}
\begin{tabular}{c|ccc}
\textbf{Nombre} & \textbf{Medidas} & \textbf{Medida razonable} & \textbf{Contras}\\\hline\hline
 & $\displaystyle\sum_{i=1}^{N} (x_i-\overline{x})$ & No & Puede resultar 0 en datos sí dispersos\\\\
Desv. media  & $\displaystyle\sum_{i=1}^{N} \abs{x_i-\overline{x}}$ & Sí & Difícil de trabajar con el valor absoluto\\\\
Varianza $\sigma^2$ & $\displaystyle \frac{\displaystyle\sum_{i=1}^{N} \left(x_i-\overline{x}\right)^2}{N}$ & Sí. & Medida algo falseada por elevar al cuadrado\\\\
Desv. típica $\sigma$ & $\displaystyle \sqrt{\frac{\displaystyle\sum_{i=1}^{N} \left(x_i-\overline{x}\right)^2}{N}}$ &Sí. & Corrección artificial de elevar al cuadrado\\\\\hline
\end{tabular}
\\
\textit{\textbf{Obs:} Todas estas mismas medidas se pueden calcular respecto de la mediana. Cambiarían los nombres}
\end{table}

\subsection{Problema práctico}

Problema 50 del libro. 


\todo{Hacer el ejercicio 239.55 durante el puente}

\section{Probabilidad, binomial y normal}

\subsection{Probabilidad}

\subsection{Binomial}

\subsection{Normal}


\section{Estadística bidimensional}

