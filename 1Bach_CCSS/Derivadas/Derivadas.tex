\documentclass[palatino,nosec]{apuntes}

\title{Derivadas}
\author{}
\date{16/17 C2}

\newcommand{\sen}{\operatorname{\sen}}


% Paquetes adicionales

% --------------------

\begin{document}
%\pagestyle{plain}
%\maketitle

%\tableofcontents
%% Contenido.

1 BACH:

\begin{itemize}
\item Marc preguntaba por la utilidad de las derivadas. Hoy vamos a ver una utilidad muy concreta.

\item Señal de tráfico de pendiente.
\item Imagen global del dibujo de geogebra por si alguien lo quiere copiar, que sepa qué hueco dejar.
\item Explicar qué vemos en el geogebra. 
\item Pendiente de la recta.
\item Gráfica $\to$ Otra manera de calcular la pendiente. ¿A alguien le suena esto?
\item $h \to 0$.
\item Como el ordenador no tiene suficiente precisión y se hace un jaleo, vamos a utilizar las herramientas que tenemos: límites
\item [Pizarra] Pendiente $m = \lim_{x\to x_1} \frac{f(x) - f(x_1)}{x-x_1} = \lim_{h\to 0} \frac{f(x+h)-f(x)}{h}$
\item [Pizarra] Ejemplo: \[\lim_{h\to 0}\frac{f(x+h)-f(x)}{h} = \lim_{h\to 0} \frac{0.2·(3.5+h)^2 - 0.2·3.5^2}{h} = \lim_{h\to 0} \frac{0.2·(3.5^2+2·3.5·h+h^2) - 0.2·3.5^2}{h} = 
\]
\[
= \lim_{h\to 0} \frac{0.2·3.5^2+0.2·3.5·h+h^2-0.2·3.5^2}{h} = \lim_{h\to 0}\frac{0.2·2·3.5·h}{h} + \lim_{h\to 0}\frac{h^2}{h} = 1.4
\]

\item[Pizarra] ¿Y cuál es la derivada de esta función? $f(x) = 0.2·x^2 \to f'(x) = 0.4·x$ \[f'(3.5) = 0.4·3.5 = 1.4\]

\item Y si la derivada en un punto es la pendiente de la recta tangente... ¡Podemos calcular rectas tangentes a las gráficas!

\item[Pizarra]
Repaso de Geometría de 4ESO: Ecuación de la recta punto-pendiente: Dado un punto $P_0(x_0,y_0)$ y una pendiente $m$, la ecuación de la recta que pasa por $P_0$ y $m$ tiene la siguiente ecuación: $y-y_0=m·(x-x_0)$.

Repaso Geometría de 4ESO II: $y=mx+n_1$ y la recta $y=\frac{-1}{m}·x + n_2$ son perpendiculares.


\textbf{Ejemplo: Calcular las recta tangente y normal (perpendicular) a la gráfica $f(x) = \rfrac{1}{3}·x^3$ en el punto $x=3$.}


\textbf{Tangente:}
Punto: $\left(x,f(x)\right) = \left(3,\rfrac{1}{3}·3^3\right) = (3,9)$

$f'(x) = x^2$
Pendiente: $m=f'(3) = 9$

Ec. de la recta: $y-9 = 9·(x-3) \to y=9x-18$

\textbf{Normal:}

Punto $(3,9)$.

Pendiente (por el repaso de 4ESO): $m_n=\frac{-1}{9}$

Ecuación de la recta: $y-y_0 = m_n(x-x_0) \to y-9=\frac{-1}{9}(x-3)$

\textit{\\ Demostración en geogebra.}

Ejercicios para ellos + reto: $\displaystyle\lim_{h\to 0 } \frac{\sen{(x+h)}-\sen{(x)}}{h}$

\item Si da tiempo: ¿Recta tangente del valor absoluto?

\end{itemize}


\section*{Clase 2} Función derivada y ¿De dónde sale la tabla de las derivadas?

\begin{itemize}
\item  ¿De dónde sale la tabla de las derivadas? Ejemplo $f(x) = x^2+x$
\[
\lim_{h\to 0} \frac{f(x+h)-f(x)}{h} = \lim_{h\to 0} \frac{(x+h)^2+(x+h) - x^2-x}{h} = \lim_{h\to 0}\frac{x^2+2xh+h^2+x+h-x^2-x}{h} =\]
\[ \lim_{h\to 0}{2xh+h+h^2}{h} = \lim_{h\to 0}\frac{h·(2x+1+h)}{h} = 2x+1
\]

\item  ¿De dónde sale la tabla de las derivadas? Ejemplo $f(x) = \sen(x)$
\[
\lim_{h\to 0} \frac{f(x+h)-f(x)}{h} = \lim_{h\to 0} \frac{\sen(x+h) - \sen(x)}{h} = ??? = \cos(x)
\]

Demostración gráfica con Geogebra.

\end{itemize}

\section*{Clase 3} Dominio de derivabilidad, funciones no derivables.

\[
\left[f(x) = \frac{x^2-5x+6}{x-2}\right] == \left[g(x) = x-3\right]
\]

No son exactamente la misma función. Sí tienen el mismo valor de los límites, pero una función es continua en $\real$ y la otra en $\real-\{2\}$, con lo que no son iguales.
\\
Algo parecido pasa con las derivadas.



\printindex
\end{document}
\grid
