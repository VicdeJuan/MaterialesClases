\documentclass[palatino,nosec]{Docencia}


\usepackage{breqn}


\title{Cuaderno de clase}
\author{Víctor de Juan}
\date{17/18}

\begin{abstract}
Cuaderno de clase de Matemáticas I, con el desarrollo continuado (sin estar separado por sesiones).
\end{abstract}

% Paquetes adicionales

\usepackage[author={Víctor de Juan, 2017}]{pdfcomment}

\makeatletter
\newcommand{\annotate}[2][]{%
\pdfstringdef\x@title{#1}%
\edef\r{\string\r}%
\pdfstringdef\x@contents{#2}%
\pdfannot
width 2\baselineskip
height 2\baselineskip
depth 0pt
{
/Subtype /Text
/T (\x@title)
/Contents (\x@contents)
}%
}
\makeatother

% --------------------
\newcommand{\cimplies}{\text{\hl{$\implies$}}}
\renewcommand{\vx}{\overset{\rightarrow}{x}}
\renewcommand{\vy}{\overset{\rightarrow}{y}}
\renewcommand{\vz}{\overset{\rightarrow}{z}}
\newcommand{\vi}{\overset{\rightarrow}{i}}
\newcommand{\vj}{\overset{\rightarrow}{j}}
\renewcommand{\vec}[1]{\overset{\rightarrow}{#1}}

\usepackage{pgf,tikz}
\usetikzlibrary{arrows}


\begin{document}
\pagestyle{plain}

Resumen de fórmulas de probabilidad:

\[
P(A\cup B) = P(A) + P(B) - P(A\cap B)
\]

Básica: $P(A\cap B) = P(A) · P(B\tq A)$. 

De aquí se deducen muchas otras:

\begin{itemize}
 	\item 
 $P(A\cap B) = P(A) · P(B\tq A) \to P(B\tq A) = \displaystyle\frac{P(A\cap B)}{P(A)}$

 \item Probabilidad total (es bastante más elaborado que sólo esto. Puedes mirarlo en el libro):
 \[
 	P(A) = P(B)·P(A\tq B) + P(\overline{B})·P(A\tq \overline{B})
 \]
\item 
\small
\[
\left.\begin{array}{c}
	P(A\cap B) = P(A) · P(B\tq A)\\
	P(A\cap B) = P(B) · P(A\tq B)
\end{array} \right\} 
\rightarrow 
	P(A) · P(B\tq A) = P(B) · P(A\tq B) \dimplies 
	\underbrace{\left\{ \begin{array}{l} 
		P(A) = \displaystyle\frac{P(B) · P(A\tq B)}{P(A\tq B)}\\
		\\
		P(B) = \displaystyle\frac{P(A) · P(B\tq A)}{P(B\tq A)}
	\end{array}\right.}_{\text{Teorema de Bayes}}
\]

\textbf{Si son independientes:} $P(A\tq B) = P(A)$

Por lo tanto, $P(A\cap B) = P(A)·P(B)$, sólo si son independientes.
\end{itemize} 
 

\end{document}

