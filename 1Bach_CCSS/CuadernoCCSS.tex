\documentclass[palatino,nosec]{Docencia}


\title{Cuaderno de clase}
\author{Víctor de Juan}
\date{17/18}

\begin{abstract}
Cuaderno de clase de Matemáticas I, con el desarrollo continuado (sin estar separado por sesiones).
\end{abstract}

% Paquetes adicionales

\usepackage[author={Víctor de Juan, 2017}]{pdfcomment}

\makeatletter
\newcommand{\annotate}[2][]{%
\pdfstringdef\x@title{#1}%
\edef\r{\string\r}%
\pdfstringdef\x@contents{#2}%
\pdfannot
width 2\baselineskip
height 2\baselineskip
depth 0pt
{
/Subtype /Text
/T (\x@title)
/Contents (\x@contents)
}%
}
\makeatother

% --------------------
\newcommand{\cimplies}{\text{\hl{$\implies$}}}
\renewcommand{\vx}{\overset{\rightarrow}{x}}
\renewcommand{\vy}{\overset{\rightarrow}{y}}
\renewcommand{\vz}{\overset{\rightarrow}{z}}
\newcommand{\vi}{\overset{\rightarrow}{i}}
\newcommand{\vj}{\overset{\rightarrow}{j}}
\renewcommand{\vec}[1]{\overset{\rightarrow}{#1}}

\usepackage{pgf,tikz}
\usetikzlibrary{arrows}

\begin{document}
\pagestyle{plain}
\maketitle
\tableofcontents


%% Contenido.

\chapter{Introducción a la asignatura}


Hola, soy Víctor y voy a ser vuestro profesor de matemáticas este año. Algunos me conoceréis, tal vez de campamento o de verme en misa. Para otros tal vez soy totalmente nuevo. ¡Genial! \hl{Soy matemático, entre otras cosas}, y espero poder transmitiros la belleza que encierran las matemáticas.

Hay una cosa que espero de vosotros este curso.  Si tuvieras que elegir una cualidad, ¿qué diríais? Lluvia de ideas durante 4 minutos. Sal y escribe en la pizarra. Yo diría \hl{madurez} claramente. Me encantaría que fuera amor la verdad… porque para amar de verdad hay que tener una cierta madurez, pero lo que realmente espero es madurez.

Se madura reflexionando en general. Sobre lo que te pasa, la vida… los problemas de tu alrededor…  ¿Cómo te fueron las \textbf{Matemáticas el año pasado}? Buscamos conclusiones del tipo: 
\begin{itemize}
	\item lo dejaba todo para el último día y no era capaz. \ul{¡Trabaja!}
	\item Descubrí que era capaz, aunque lo descubriera tarde. \ul{¡No te rindas!}
\end{itemize}

Te va a tocar currar mucho. Desde el principio más absoluto. Por eso hace falta madurez. Por cierto, \hl{examen el 10 de octubre}

\subsection{Temario}

\begin{itemize}
	\item Álgebra. Ecuaciones. Como la primera evaluación de 4º con algún añadido (Gauss) [Terminado para la preevaluación]
	\item Estadística. Con excel y a mano, porque para desgracia del universo, se os va a examinar de estadística a mano el resto de vuestra vida… así que, así os vais acostumbrando.
	\item “Finanzas”: aplicación concreta. Haremos un trabajo real, que tendréis que ir a los bancos, etc. 
	\item Estadística y probabilidad.
	\item Funciones.
\end{itemize}


\section{Criterios de evaluación}

Esta asignatura tiene evaluación continua. Eso significa que no hay una recuperación exclusiva para quienes suspendan una evaluación, sino que la primera evaluación es contenido evaluable de la segunda, y la segunda de la tercera.
%
(La primera de la tercera no).

Hay 2 exámenes por evaluación, el primero vale \hl{$\rfrac{1}{3}$} y el segundo \hl{$\rfrac{2}{3}$}. 
%
En ese primer examen, la mitad corresponderá a la evaluación pasada (en caso de que haya evaluación pasada).
%
Si tienes la evaluación suspensa, tienes 5 puntos en el primer examen para recuperar.
%
Si tienes la evaluación aprobada, tienes 5 puntos en el primer examen para repasar.
%
En el segundo examen, en principio, no pondremos contenido de la evaluación anterior.

Y las notas de clase podrán influir en el redondeo a criterio del profesor. 

\paragraph{0 tonterías:} ¡Ah! se me olvidaba. 3 normas especiales de esta clase.

\begin{itemize}
	\item $(a+b)^2 = a^2 + b^2$ [no hay cosa que me moleste más... Una pena que no pueda poner partes de incidencias por semejante barbaridad...]
	\item "Matemáticas para tontos"
	\item "Es que soy/eres de letras"
\end{itemize}

% -*- root: ../Cuaderno.tex -*-

\chapter{Álgebra}

\section{Repaso de 4º}

\subsection{Logaritmos}

\paragraph{Introducción}

Vamos a aprender una nueva manera de multiplicar. En realidad ya sabéis, aunque no seáis conscientes.\footnote{Fuente: \href{https://www.youtube.com/watch?v=FB3\_BeukBBk\&t=99s}{Mark Foskey, youtube.com}}


\begin{itemize}
\item Caso 1: $1000000·10000000 = 10^6·10^7 = 10^{13}$. ¿Y podremos hacer esto con otros números que no sean el 10?

\item Caso 2: $64·128 = 2^6·2^7 = 2^{13} = 8192$
\item ¿Caso 3?: Me construyo la tabla del 3.
\begin{center}
\begin{tabular}{cccccccccccc}
1& 3& 9& 27& 81& 243& 729& 2187& 6561& 19683& 59049& 177147\\
\textcolor{red}{0} & \textcolor{red}{1} & \textcolor{red}{2} & \textcolor{red}{3} & \textcolor{red}{4} & \textcolor{red}{5} & \textcolor{red}{6} & \textcolor{red}{7} & \textcolor{red}{8} & \textcolor{red}{9} & \textcolor{red}{10} & \textcolor{red}{11}
\end{tabular}
\end{center}
\item Caso 4: ¿Y para números que no son potencias enteras? Por ejemplo, $64*40$. Pues si $32=2^5$ y $64=2^6$, $40=2^{5,...}$ ¿Tiene sentido?

\item Caso 5: Lo que hicieron, Yost y Napier, fue coger la tabla del 1,0001 en lugar de la tabla del 3 y dividir por mil los números rojos, dando lugar a la tabla de logaritmos:

\begin{center}
	\begin{tabular}{cl}
		0.0 & 1.0\\
		0.001 & 1.001\\
		0.002 & 1.002\\
		0.003 & 1.003\\
		0.004 & 1.00401\\
		0.005 & 1.00501\\
		0.006 & 1.00602\\
		0.007 & 1.00702\\
		0.008 & 1.00803\\
		0.009 & 1.00904\\
		$\vdots$ & \quad\quad$\vdots$\\
		0.991 & 2.69259\\
		0.992 & 2.69529\\
		0.993 & 2.69798\\
		0.994 & 2.70068\\
		0.995 & 2.70338\\
		0.996 & 2.70608\\
		0.997 & 2.70879\\
		0.998 & 2.7115\\
		0.999 & 2.71421\\
		1.0 & \hl{2.71692} (Una aproximación de $e$)\\
	\end{tabular}
\end{center}

\end{itemize}

$40=2^{5...}$. Ese $5...$ es lo que llamamos "logaritmo" en base 2 de 40. ¡Los logaritmos son exponentes! Es el \textit{exponente al que hay que elevar} ...


\begin{defn}[Logaritmo]
Sean $a\in ℝ>0,a≠1$ y $N\in\real$.

Se llama logaritmo en base $a$ de $N$ al exponente $x$ que cumple: $a^x = N$ y se escribe:
\[
	\log_aN=x\dimplies a^x=N
\]
\end{defn}

\textbf{Conectando con otras operaciones matemáticas:} 
\[
	\begin{array}{c}
		2^3=8\\
		\sqrt[3]{8}=2\\
		\text{??? }= 3
	\end{array}
\]

La relación $2^3=8$ se puede expresar de otras maneras, dando como resultado el 2 (raíz cúbica) y dando como resultado el 3 (logaritmo).

\nota{De la propia definición se entiende:}
\begin{enumerate}
	\item $y\in\real, y<0 \implies ∀a,\nexists\log_a y$
	\item $\log_a 1 = 0 \dimplies a^0 = 1$
	\item $\log_a a = 1 \dimplies a^1 = a$
	\item $\log_a a^q = q \dimplies a^q = a^q$ por definición.
	\item $a^{\log_a N} = N$
\end{enumerate}

\paragraph{Propiedades}

Vamos a razonar las propiedades de los logaritmos teniendo en cuenta que son exponentes. De hecho, las propiedades de los logaritmos no son otra cosa que las propiedades de las potencias escritas de otra manera.

\begin{itemize}
	\item $\log_a(AB) = \log_a(A) + \log_a(B)$
	\subitem Por ejemplo:
	\[9·8 = 2^x \dimplies x = \log_2 (9·8) = \log_2 8 + \log_2 9\]
	\subitem Como los logaritmos son exponentes, esta propiedad se podría leer: \textit{el exponente de un producto es la suma de los exponentes}.
	\item $\log_a\left(\rfrac{A}{B}\right) = \log_a(A) - \log_a(B)$
	\item Como los logaritmos son exponentes, ¿cómo se leería esta propiedad?
	\item $\log_a(A)^n = n·\log_a(A)$ 

	\item $\displaystyle\log_aA = \frac{\log_bA}{\log_ba}$ \textbf{(Cambio de base})
\end{itemize}

\paragraph{Tomar logaritmos:}

\[
A = B \overset{A>0; a>0; a\neq 1}{\dimplies} a^{\log_a A} = a^{\log_a B} \dimplies \log_aA=\log_aB
\]


\subsubsection{Ejemplos con logaritmos}

Calcula:
\begin{itemize}
	\item $\log \sqrt{0,0001}$
	\item \textit{[Ejercicio de examen de años anteriores]} Demuestra $\log_a b · \log_b a = 1$ 
\end{itemize}


\subsection{Factorización}

\begin{itemize}
	\item Alguien sale a la pizarra a factorizar. Utilizando 
https://www.wolframalpha.com/problem-generator/quiz/?category=Algebra\&topic=FactorPolynomial 
	\item Diferencia raíz y factor.
	\item Verdadero o falso.
	\subitem Un polinomio de grado 2 tiene 2 raíces reales. (Falso: $x^2+1$)
	\subitem Un polinomio con 2 raíces tiene 2 factores. (Discutible: $(x+1)^2(x-1)$)
	\subitem Un polinomio con 2 factores tiene 2 raíces.
	\subitem Un polinomio es irreducible si no tiene raíces reales. (Falso: $P(x) = x^4+2x^2+1$)
	\subitem 2 polinomios con las mismas raíces son iguales: (Falso: $(x^2+1)(x-2)$ y $(x-2)$)
	\subitem 2 polinomios con los mismos factores son iguales: (Falso: $2(x-1)$ y $(x-1)$)

	\item Halla “m” para que $2x^3-2x^2+m·x+4$ sea divisible por $(x-2)$

	\item Deberes: ejercicios 6-9
\end{itemize}


\subsection{Teoremas de factorización}

\paragraph{Ejemplo}
Factoriza: $P(x) = 3x^3-x^2+9x-3 = 3(x^2+3)\left(x-\rfrac{1}{3}\right)$

La factorización del polinomio. ¿Qué raíces puede tener? Ni $\pm1,\pm3$, ¿entonces? Teorema de las raíces racionales.


\hl{\textit{Entregar en hoja aparte: Teoremas de Polinomios}}

\begin{theorem}[Teorema\IS del factor]
Sea $P(x) = a_nx^n+a_{n-1}x^{n-1}+...+a_1x+a_0$ con $a_n≠0$ y $a_n,a_{n-1},...,a_1,a_0\in\real$. 
Sea $α\in\real$.

\[
	P(α) = 0 \dimplies \frac{P(x)}{(x-α)} = Q(x)
\]
\end{theorem}

De hecho este teorema es un caso particular del teorema del resto:
\begin{theorem}[Teorema\IS del resto]
Sea $P(x) = a_nx^n+a_{n-1}x^{n-1}+...+a_1x+a_0$ con $a_n≠0$ y $a_n,a_{n-1},...,a_1,a_0\in\real$.

Entonces, el resto de $\frac{P(x)}{x-α} = P(α)$
\end{theorem}


\begin{theorem}[Teorema\IS de la factorización]
Sea $P(x) = a_nx^n+a_{n-1}x^{n-1}+...+a_1x+a_0$ con $a_n≠0$ y $a_n,a_{n-1},...,a_1,a_0\in\real$ y $α_1,α_2,...,α_n\in\real$ las raíces o ceros de $P(x)$. 

Entonces,\[P(x) = a_n(x-α_1)(x-α_2)...(x-α_n)\]
\end{theorem}


\begin{theorem}[Teorema\IS de las raíces enteras]
Sea $P(x) = a_nx^n+a_{n-1}x^{n-1}+...+a_1x+a_0$, con $a_n≠0$, una raíz entera $r$ de $P(x)$ tiene que ser divisor del término independiente.
\end{theorem}



\begin{theorem}[Teorema\IS de las raíces racionales]
Sea $P(x) = a_nx^n+a_{n-1}x^{n-1}+...+a_1x+a_0$, con $a_n≠0$,$a_i\inℤ$ una raíz fraccionaria $\rfrac{n}{m}$ del polinomio $P(x)$ tiene que cumplir $n|a_0$ y $m|a_n$.
\end{theorem}


\subsubsection{Ejercicios:}

\begin{enumerate}
\item Alguien en la pizarra. Corrijo lo que se haya dejado, escribiendo los teoremas, etc.

Sea $P(x) = 3x^3-3x^2-3x+3$ .¡Factoriza! $P(x) = 3(x-1)(x+1)^2$

\begin{itemize}
	\item ¿Es divisible por $(x-1)$? Comprobamos $P(1) = 3-3-3+3 = 0 \overset{T.F}{\implies}$ Sí.
\end{itemize}

\textit{¡Mira que tontería dice el teorema del factor si miras el polinomio factorizado!}

\item (Ellos) Sea $P(x) = 6x^3-10x^2+4x = 6x(x-1)(x-\rfrac{2}{3})$ 
\begin{itemize}
	\item Factoriza.
	\subitem $P(0) = 0$. Por el teorema del factor sabemos que $x-0$ es un factor.
	\subitem Posibles raíces: $n=\pm1,\pm2,\pm4$ y $m=\pm1,\pm2,\pm3,\pm6$	
	\subitem Por el teorema de la factorización, $Q(x) = 3x^3-5x+2x$ tendrá las mismas raíces que $P(x) = 6x^3-10x^2+4x$. \hl{(Ojo, no podemos simplificar, pero las raíces son las mismas)}. Ahora las posibles raíces son $\rfrac{n}{m}$ donde $n\in\{\pm1,\pm2\}$ y $m\in\{\pm1,\pm3\}$
	\subitem $P(1) = 0$. Por el teorema del factor sabemos que $0$ es una raíz. ¿Es esto más fácil que Ruffini? ¿Y ahora?
\end{itemize}

\item Sea $P(x) = 2x^3-2x^2+kx+4$.
\begin{itemize}
	\item Halla el valor de $k$ para que $P(x)$ sea divisible por $x-2$.
	\subitem Por el teorema del factor, buscamos $P(2) = 0$. Entonces:
	\[
		P(2) = 0 \dimplies 2^4-2^3+2k+4 = 0 \dimplies 16-12+2k = 0 \dimplies k = -2
	\]
\end{itemize}




\item Sea $P(x) = 4x^2+kx+1$.
\begin{itemize}
	\item Halla el valor de $k$ para que sea divisible por $\left(x-\rfrac{1}{3}\right)$. $k=\frac{13}{3}$.
	\item Pero, $3$ no divide a $4$. ¿Cómo podría ser una raíz $\rfrac{1}{3}$?
\end{itemize}


\item Sea $P(x) = 6x^3+ax^2+bx-1$, con $a,b\inℤ$
\begin{itemize}
	\item Halla el valor de $a,b$ para que $P(x)$ sea divisible por $(x-\rfrac{1}{3})$ y por $(x-\rfrac{1}{5})$.
	\subitem Por el teorema de las raíces racionales, $5$ no divide al coeficiente principal, por lo que $P(x)$ no puede ser divisible por $(x-\rfrac{1}{5})$.
	\item Halla el valor de $a,b$ para que $P(x)$ sea divisible por $(x-\rfrac{1}{3})$ y por $(x-\rfrac{1}{2})$.
	\subitem Por el teorema del factor, buscamos:
	\[
	\left\{
		\begin{array}{c}
			P(\rfrac{1}{2}) = 0 \dimplies \frac{6}{8} + \frac{a}{4} + \frac{b}{2} - 1 = 0\\
			P(\rfrac{1}{3}) = 0 \dimplies \frac{6}{27} + \frac{a}{9} + \frac{b}{3} - 1 = 0
		\end{array}\right\}\dimplies ... \quad (a,b) = (-1,-4)
	\]
\end{itemize}

\item\textbf{Ampliación, puesto pero sin corregir} Sea $P(x) = 4x^2+bx+1$, con $b∈ℤ$. 
\begin{itemize}
	\item Sabemos que sus raíces $α_1,α_2$ son fraccionarias y negativas. ¿Cuáles son? ¿Cuánto vale $b$?
	\subitem Por el teorema de las raíces racionales, $α_1 = \rfrac{n_1}{m_1}$, sabemos que $n_1$ divide a $1$. Análogo para $α_2$.

	Por otro lado, sabemos que $m_2$ divide a 4. Las posibilidades son $2,4$, con lo que $α_1,α_2 \in \{\rfrac{1}{2},\rfrac{1}{4}\}$

	Por el teorema del factor, $P(\rfrac{1}{2}) = 1+b\rfrac{1}{2}+1 = 0 \implies b=-4$. 

	Por el teorema del factor, $P(\rfrac{1}{4}) = \rfrac{1}{4}+b\rfrac{1}{4}+1 = 0 \implies b=-2$.

	Si queremos que sea divisible por los 2 factores, b tiene que valer a la vez $4$ y $-2$. Entonces, necesariamente $P(x) = 4(x-\rfrac{1}{2})^2$ o $P(x) = 4(x-\rfrac{1}{4})^2$. 

	Desarrollando la segunda opción, obtenemos como término independiente $\rfrac{1}{4}≠1$, por lo que no es posible. 
	%
	Por otro lado, desarrollando la primera opción obtenemos algo con sentido.

	\[
		4\left(x+\rfrac{1}{2}\right)^2 = 4\left(x^2+x+\rfrac{1}{4}\right) = 4x^2+4x+1 \implies b=4
	\]

\end{itemize}

	\item Factorizar $P(x) = 9x^3-\frac{27}{2}x^2+\frac{13}{2}x-1 = 9·(x-1/2)(x-2/3)(x-1/3)$. Pista (para ahorraros pruebas innecesarias con Ruffini), todas las raíces son fraccionarias y positivas.

	\item Factorizar $P(x) = x^7+2x^4+x = x(x^3+1)^2$

	

\item Sea $P(x) = 21x^2+10x-2$. $P(x) + 3 = 21(x+1/3)(x+1/7)$.

\end{enumerate}




\section{Tema 2: Ecuaciones}

\subsection{Teoría sobre ecuaciones}


\begin{example}
\[
	-20 = -20 \dimplies 25-45 = 16-36 \dimplies 5^2-5·9 = 4^2-4·9 \dimplies 5^2-5·9+\left(\rfrac{9}{2}\right)^2 = 4^2-4·9+\left(\rfrac{9}{2}\right)^2 \dimplies
\]
\[
	\left(5-\rfrac{9}{2}\right)^2 = \left(4-\rfrac{9}{2}\right)^2 \text{\hl{\;\;;\;\;}} 5-\rfrac{9}{2} = 4-\rfrac{9}{2} \dimplies 5=4
\]
\end{example}


\obs Dividir por 0 no mantiene la equivalencia.
%
En general, tomar una raíz no mantiene equivalencia entre ecuaciones (tampoco elevar a una potencia).

\begin{defn}[Ecuaciones equivalentes]
Dos ecuaciones son equivalentes si tienen las mismas incógnitas y las mismas soluciones.
\end{defn}

\paragraph{Clasificación de ecuaciones}

Las ecuaciones según sus soluciones pueden ser:
\begin{itemize}
	\item Incompatible: no tiene ninguna solución. Ejemplo: $5x=5x+2$
	\item Compatible determinada: tiene un número finito de soluciones. Ejemplo: $3x=6$.
	\item Compatible indeterminada: tiene infinitas soluciones. Ejemplo $2x-\frac{3x-1}{3} = x+\frac{1}{3}$. Solución: $x=λ, ∀λ∈ℝ$.
\end{itemize}


\begin{example}
	¿Son equivalentes?
	\begin{itemize}
		\item $9x=3x^2 \dimplies 9=3x$ [CD]
		\item $9=3x \dimplies x=3$ [CD]
		\item $4=5 \dimplies 1=0$ [IN]
		\item $9x=3^2x \dimplies 0x=0$ [CI]
		\item[difícil] $9x=\frac{(3x)^2}{x} \dimplies 9x=9x$ [CI]

\paragraph{Conclusiones:} \textbf{¡Ojo con simplificar ecuaciones!} Cuando "desaparezcan" incógnitas mirar con cuidado, porque estaremos perdiendo soluciones en la inmensa mayoría de los casos.

¿Cuándo no?

	\item $\frac{\sqrt{9x^2}}{x^2}=\frac{3x}{x^2} \dimplies \frac{9}{x} = \frac{9}{x} \dimplies 0=0$
	\end{itemize}
\end{example}



\subsection{Racionales}

Ecuaciones racionales.

\paragraph{Ejemplo}
\[
	\frac{2x}{x-2} + \frac{3x}{x+2} = \frac{7x^2}{x^2-4} \dimplies \frac{2x(x+2)}{(x-2)(x+2)} + \frac{3x(x-2)}{(x+2)(x-2)} = \frac{7x^2}{x^2-4} \dimplies 
\]
\[
	\frac{2x(x+2)+3x(x-2)}{x^2-4} = \frac{7x^2}{x^2-4} \text{\hl{$\implies$}} 2x^2+4x+3x^2-6x=7x^2 \dimplies 5x^2-7x^2-2x = 0 \dimplies 
\]
\[
	x(-x-1) = 0 \dimplies x_1 = 0 \wedge x_2 = -1
\]

\hl{¿Son soluciones las 2?}

La equivalencia la hemos perdido si $x\neq \pm2$, por lo que las soluciones $x_1 = 0 \wedge x_2 = -1$ son válidas (\ul{y no es necesario hacer la comprobación}).



\hl{Ejercicios: 84 ad + propios con trampa de equivalencias}


\paragraph{Cuidado:} casuística nueva de equivalencias e implicaciones. Hasta ahora, los valores peculiares de las implicaciones que no son equivalencias sólo nos ahorraban alguna comprobación\footnote{que tampoco está demás hacer para asegurarnos que hemos operado bien}.
%
Pero puede darse el caso de que pasen otras cosas. Por ejemplo, ¿qué pasa en esta simplificación?

\[(x-1)·\frac{x}{x+1} = (x-1)·(x^2-9) \text{\hl{$\implies$}} \frac{x}{x+1} = x^2-9\]

A la izquierda el 1 es solución y a la derecha no, luego las ecuaciones no son equivalentes. Pero, ¿qué pasa con el $x=1$?

En esta implicación he \ul{perdido una solución}. Es una equivalencia si $x\neq 1$, y en el caso $x=1$, tengo una solución. 

\paragraph{Más ejemplos}
\begin{itemize}
	\item
	\[
		\frac{1}{1-\frac{1}{x+1}} = \frac{x+1}{x} \dimplies \frac{1}{\frac{x+1-1}{x+1}}=\frac{x+1}{x} \dimplies \frac{x+1}{x} = \frac{x+1}{x} \text{\hl{$\implies$}} x=λ, ∀λ∈ℝ\setminus\{0,-1\}
	\]

	\item
	\[
		\frac{1+\displaystyle\frac{x+1}{x-1}}{2-\displaystyle\frac{x-1}{x+1}}=2 \dimplies \frac{\displaystyle\frac{x-1+x+1}{x-1}}{\displaystyle\frac{2x+2-x+1}{x+1}} = 2 \dimplies
	\]
	\[	
		\frac{\displaystyle\frac{2x}{x-1}}{\displaystyle\frac{x+3}{x+1}}=2 \text{\hl{$\implies$}} 2x^2+2x=x^2+4x-6 \dimplies 2x=6 \dimplies x=3
	\]

	\item

	\[
		\frac{3}{x} - \frac{x}{x+2} = \frac{5x-1}{x^2+x-2}
	\]

	\item Ejercicios 83 y siguientes del libro.
\end{itemize}

\subsection{Ecuaciones irrracionales}

\paragraph{Ejemplo:}
\[
	\sqrt{x+1} - \sqrt{x^2-5}=0 \text{\hl{$\implies$}} x+1 = x^2-5 \dimplies (x_0,x_1) = (3,-2)
\]

\textbf{Comprobamos} porque hemos perdido la equivalencia: 

$\sqrt{-2+1} = \sqrt{(-2)^2-5} \dimplies \sqrt{-1} = \sqrt{-1}$; -2 no es una solución en los reales.

Por otro lado: $\sqrt{3+1} = \sqrt{3^2-5} \dimplies \sqrt{2}=\sqrt{2}$

\textit{La comprobación no sería necesaria si hubiéramos reparado en que la equivalencia se mantendría siempre que el interior de las raíces fuera positivo, es decir $x\geq -1 \wedge x\geq \sqrt{5}$}

\paragraph{Ejercicio:} 
\[
	\sqrt{x+4}+\sqrt{x-1} = 5 \text{\hl{$\implies$}} (x+4)+(x-1) + 2\sqrt{(x+4)(x-1)} = 25 \text{\hl{$\implies$}} (22-2x)^2 = 4(x^2+3x-4) \dimplies 
\]
\[
	4x^2-88x + 484 = 4x^2+12x-16 \dimplies -100x + 500 = 0 \dimplies x=5
\]

Comprobamos:
\[
	\sqrt{5+4}+\sqrt{5-1} = 3+2 = 5
\]

\textit{Aquí la comprobación si resulta mucho más útil porque nos ahorra resolver la inecuación $(x+4)(x-1) \geq 0$}

\begin{itemize}
	\item Dudas.
	\item Corregir irracional. 89b.
	\item 85e,90a no lo hagáis.
	\item Empezamos 
\end{itemize}



\subsection{Exponenciales y logarítmicas}

Pregunta: $a^x=a^y \overset{?}{\dimplies} x=y$

$x=y\implies a^x=a^y$ Sí.

$a^x=a^y\implies x=y$ No. Contraejemplo: $1^2=1^3$.  Basicamente, si los logaritmos no mantenían la equivalencia, tampoco lo iban a hacer estas.

Siempre que la base no sea $0,±1$ sí serán equivalentes. ¿Y si tenemos un polinomio como base? Pues como puede ser uno de esos valores, no mantenemos la equivalencia o calculamos para qué valores sí sería equivalente o no.

Trabajamos 108abc, 109ac,111c (el más difícil de todos)

\subsection{Logarítmicas}

Los logaritmos tampoco conservan las equivalencias:

Versión innecesariamente larga:
\[
	-5 = -5 \dimplies -30+25 = 1-6 \dimplies -30+25+9 = 9+1-6 \dimplies (3-5)^2 = (3-1)^2 \dimplies 
\]
\[
	\log(3-5)^2 = \log(3-1)^2 \dimplies 2\log(3-5) = 2\log(3-1) \dimplies \log(3-5)=\log(3-1) \dimplies \log2=\log-2
\]

Versión corta:
\[
	(-2)^2 = (2)^2 \text{\hl{$\implies$}} 2\log(-2) = 2\log(2) \dimplies \log(-2) = \log(2) \dimplies -2=2
\]

Ecuación de ejemplo:


\[
	\log x=t \implies 5t=3t+\log6^2 \dimplies 2t=2\log6 \dimplies t=\log6 \dimplies \log x=\log6 \text{\hl{$\implies$}} x=6
\]

Si estás más versado en la abstracción algebraica:

\[
	5\log x=3\log x+2\log 6 \dimplies 2\log x=2\log6 \text{\hl{$\implies$}} x=6
\]


Incluso, habría una tercera manera:
 
\[
	5\log x=3\log x+2\log 6 \dimplies \log x^5=\log36x^3 \dimplies x^5=36x^3 \dimplies
\]
\[
	x^5-36x^3 = 0 \dimplies x^3(x^2-36) = 0 \dimplies x^3(x+6)(x-6) = 0
\]




\paragraph{Ejercicio}
\[
	\log\frac{2x-2}{x} = 2\log(x-1)-\log x \dimplies \log \frac{2x-2}{x}=\log\frac{(x-1)^2}{x} \cimplies \frac{2(x-1)}{x} = \frac{(x-1)^2}{x} \overset{1}{\cimplies}
	\]
	\[ 
	x-1=2 \dimplies x=3
\]
En 1 hemos simplificado 2 factores. $x$ y $(x-1)$. En esta simplificación podríamos haber perdido soluciones, en concreto, si $0,1$ fueran soluciones no lo obtendríamos. 

En este caso no son solución porque $\log 0$ no existe.

\paragraph{Ejercicio}
\[
\frac{\log (4-x)}{\log(x+2)}=2 \cimplies \log(4-x) = \log(x+2)^2 \cimplies 4-x=(x+2)^2 \dimplies 4-x=x^2+4x+4 \dimplies
	\]
	\[ x^2+5x=0 \dimplies x_1=0 \wedge x_2=-5
\]
$x_2=-5$ no es solución porque $\nexists\log(-5+2)=\log(-3)$. Por otro lado, $\frac{\log4}{\log2} = \log_24=2$ cqc.



\paragraph{Ejercicio} mientras corrigen

\[
	\log_x 3 = \ln \sqrt{3} \dimplies \frac{\ln3}{\ln x}=\ln\sqrt{3} \dimplies \frac{\ln3}{\ln x}=\frac{\ln3}{2} \dimplies \frac{1}{\ln x}=\frac{1}{2} \implies \ln x = 2 \implies e^2=x
\]

\paragraph{Ejercicio}
\[
	\log_332+\log_{\rfrac{1}{3}}(6-x) = \log_{\sqrt{3}}x \dimplies \log_332+\frac{\log_3(6-x)}{\log_3\rfrac{1}{3}} = \frac{\log_3x}{\log_3{\sqrt{3}}} \dimplies 
\]
\[
	\log_332-\log(6-x)=2\log_3x\dimplies \log_3\left(\frac{32}{6-x}\right)=\log_3x^2 \cimplies 32=x^2(6-x) \dimplies -x^3+6x^2-32 = 0
\]
\[
	-(-2)^3 + 6(-2)^2-32 = 8+24-32 = 0\implies x_1=-2 \wedge x_2=x_3=4
\]
 
Comprobamos:

\[
	\log_332+\log_{\rfrac{1}{3}}(6-4) = \log_{\sqrt{3}}4 \dimplies \log_332-\log_32=\log_34^2 \dimplies \log_3\frac{32}{2}=\log_316 \;\;\text{   cqc.}
\]



\section{Sistemas de ecuaciones}

Minimísimo repaso a la reducción como método para resolver sistemas de ecuaciones.

\subsection{Sistemas lineales: Gauss}

Por grupos, resolver:
\[
\left\{\begin{array}{lccccc}
e_1: &2x&+y&-2z&=&7\\
e_2: &x&+y&+z&=&0\\
e_3: &3x&+2y&+2z&=&1
\end{array}\right\} \dimplies (x,y,z) = (1,1,-2)
\]



Clase 1: Explicación de los apuntes de MariNieves y realización de 1 sistema.

Clase 2: Realización de un ejemplo por mi parte. Tiempo de trabajo para ellos.

Clase 3 (11/10/2017): Corrección ejercicio del libro + dudas del examen. 

\paragraph{Sesión 17/10:} Examen.

\paragraph{Sesión 18/10:} Sistemas de Gauss C.Indeterminados e Incompatibles.

\paragraph{Sesión 19/10:} 
\begin{itemize}
	\item Corregimos (si quieren) 113 incompatible y CI.
	\item ¿En qué consiste discutir un sistema? Escribir completo los 3 casos
	\item Sistemas con parámetros
\end{itemize}

\paragraph{Discusión de un sistema}

Una vez llegado al \hl{sistema escalonado} pueden darse 3 situaciones:

\begin{itemize}
	\item La ecuación con una única incógnita es incompatible $\implies$ Sistema Incompatible.
	\item Número de incógnitas > número de ecuaciones $\implies$ Compatible indeterminado.
	\item Número de incógnitas = número de ecuaciones, siendo la última ecuación compatible determinada $\implies$ Compatible determinado.
\end{itemize}


\subsection{Sistemas no lineales}

Ejercicios: 114cf (f es interesante),115acdf

\[
\left\{
	\begin{array}{c}
		x^2-2xy+y^2 = 1\\
		x^2-y^2 = 12\\
	\end{array}
\right\}
\]


\section{Estadística unidimensional descriptiva}

\paragraph{¿Para qué sirve la estadística?} Sirve para hacer predicciones. Estudiar los datos permite obtener información para poder hacer predicciones. 

Por ejemplo, si la nota media en una clase es de $\overline{x}=5$ y sabemos que los datos varían muy poco, es probable que si he perdido un examen, su nota fuera a ser un 5. En cambio, si la nota media es $\overline{x} = 5$ pero los datos están muy dispersos y hay desde $0$ hasta $10$, no tengo mucha posibilidad de estimar su nota sin equivocarme.

\subsection{Vocabulario estadístico}

Rellenar de ejemplos el margen del libro a lápiz.

\begin{itemize}
	\item Población
	\item Muestra
	\item Individuo
	\item Variables:
		\subitem Cualitativas o cuantitativas.
		\subitem Discretas o continuas.
\end{itemize}

\subsection{Tabla de frecuencias}
Ejercicio 4.

\subsection{Medidas}

\subsubsection{Centralización}
Ejercicios 10 y 12.

\subsubsection{Posición}

\subsubsection{Dispersión}

\section{Probabilidad, binomial y normal}

\subsection{Probabilidad}

\subsection{Binomial}

\subsection{Normal}


\section{Estadística bidimensional}




\chapter{Análisis}


\section{Funciones}

El objetivo de esta función es ser capaz de representar gráficas de funciones dada su expresión algebraica. 
%
Para ello vamos a ir desarrollando diferentes herramientas que nos van a resultar tremendamente útiles para este propósito.

\subsection{Concepto}

Lo primero es tener claro qué es una función.

\begin{defn}[Función]
Una función asigna a cada elemento de un conjunto $X$, un único elemento de otro, $Y$. Escribimos $f: X \to Y; x\in X\to f(x)\in Y$
\end{defn}

Ejemplos hablados. Añade 2 de tu propia cosecha.
\begin{itemize}
	\item A cada uno le asigno su edad.
	\item A cada uno su estatura.
	\item A cada uno su brazo.
	\item A un número su doble.
	\item A un número su raíz cuadrada.
\end{itemize}

\begin{itemize}
	\item A un número complejo su conjugado.
	\subitem Sí, pero sólo vamos a hablar de \concept{funciones reales de variable real}. 
\end{itemize}

Escribimos:  $f: D(f)\subset\real \to I\subset\real; x\to f(x)$

\textit{Como curiosidad hay funciones de variable ``funcional'', es decir, que asocia funciones con otras cosas. Este tipo de cosas se estudian en matemáticas.}


\subsubsection{Dominio y recorrido} Condiciones generales de cada función:

$f:\real\to\real\quad\quad D(f) = {x / \exists f(x)}$

Ejemplos:
\begin{itemize}
	\item $f_1(x) = 3x^2+2$
	\item Racionales $f_2(x) = \frac{x+2}{x-1}$
	\item Radicales $f_3(x) = \sqrt{x^2-9}$
	\item Radicales $f_4(x) = \sqrt[3]{x^2-16}$
	\item Exponenciales $f_5(x) = e^x$
	\item Logaritmos $f_6(x) = \log(x+2)$
	\item Trigonométricas $f_7(x) = \cos(x)$
	\item Trigonométricas $f_8(x) = \sin(x)$
	\item Combo: $f_9(x) = \frac{\log(x-5)}{x+7}$
	\item Combo: $f_{10}(x) = \frac{\sqrt{5-x}}{x+7}$
\end{itemize}


\paragraph{Clasificacion de funciones} \textit{Damos la definición formal y la intuitiva.}

\begin{itemize}
	\item Inyectiva: $x\neq y \implies f(x) \neq f(y)  $
	\item Sobreyectiva: $\forall y\in Y \exists x\in X / f(x) = y$
	\item Biyectiva: sobreyectiva e inyectiva a la vez.
\end{itemize}

\concept{Gráfica de una función:} $G(f) = {\left(a,f(a)\right)}$

\subsection{Operaciones con funciones}
\begin{itemize}
	\item Suma, resta $(f\pm g)(x) = f(x) \pm g(x)$
	\subitem $D(f\pm g) = D(f)\cap D(g)$.
	\item Producto
	\subitem $D(f·g) = D(f)\cap D(g)$.
	\item Cociente 
	\subitem $D\left(\rfrac{f(x)}{g(x)}\right) = D(f)\cap D(g) \setminus \{x/g(x) = 0\}$
	\item Composición $(g\circ f)(x) = g\left(f(x)\right)$ y se lee ``f compuesta con g''.
	\subitem El dominio se recalcula.
\end{itemize}

\concept{Función inversa} $f^{-1}(x):Y\to X$ y cumple $(f\circ f^{-1})(x) = x$

\paragraph{Ejemplo:} Comprueba si son inversas
\begin{itemize}
	\item $f(x) = x^3-1$; $f^{-1}(x) = \sqrt[3]{x+1}$ (sí)
	\item $f(x) = x^2+1$; $f^{-1}(x) = \sqrt{x}$ (no)
\end{itemize}

\paragraph{Ejemplo:} Calcula la función inversa de $f(x) = \frac{2x+1}{3}$ y de $g(x) = \frac{-1}{3x}$ 

\hl{13/03} Corregir pregunta: ¿son funciones inversas?

Construir funciones inversas.

\subsubsection{Clasificacion de funciones}

\begin{defn}[Inyectividad]
$f(x) : D(f) \to \real$ inyectiva $\dimplies\forall a,b\in D(f) a\neq b \implies f(a) \neq f(b)$
\end{defn} 

\begin{defn}[Sobreyectividad]
	$f(x) : D(f) \to \real$ sobreyectiva $\dimplies \forall b\in\real \exists a\in D(f) \tlq f(a) = b$

	Es decir, una función es sobreyectiva si su imagen son todos los números reales.
\end{defn}


\begin{defn}[Biyectividad] 
	Inyectividad y sobreyectividad
\end{defn}

\paragraph{Ejemplos:} Razonando gráficamente.

\begin{itemize}
	\item $f(x) = x^3$ (biyectiva)
	\item $f(x) = x^2$ (nada)
	\item Dibujo función cúbica sí sobreyectica, no inyectiva.
	\item $f(x) = e^x$ (inyectiva)
	\item $f(x) = \log(x)$ (inyectiva)
\end{itemize}

\obs Una función no inyectiva no puede tener inversa.

\section{Continuidad}
\subsection{Límites}

\subsubsection{Repaso de límites de 4º}

\paragraph{Introducción}

Con el excel y geogebra. Definición intuitiva de límite. \textit{Reto: escribe tu propia definición de límite y vete hablándola conmigo.}

\begin{itemize}
	\item Excel: valor del límite: me voy acercando sin llegar a tocar.
	\item Geogebra: mismo gráficamente. Hacer zoom hasta la saciedad.
\end{itemize}

\begin{defn}[Límite en un punto finito]
Sea $f(x):\real\to\real$.

\[\lim_{x\to a} f(x) = b\]

El límite de la función $f(x)$ en un punto $x=a$ es el valor $b$ al que se aproxima la función cuando la variable se aproxima al punto, sin \hl{nunca} alcanzarlo.
\end{defn}

Ejemplos:
\begin{itemize}
	\item $\lim_{x\to 3} x^2 + 1 = 10$
	\item $\lim_{x\to 0} \frac{\sin x}{x} = 1$ (cálculo con calculadora) ¿Existe la función en el punto? Nooooo ¿Existe el límite? Siiiii.
	\item $\lim_{x\to 0} \frac{1}{x^2}$ (cálculo con calculadora). ¿Existe la función en el punto? Nooooo ¿Existe el límite? Puede existir. Decimos que es $\pm\infty$.

\obs Puede ser que la función no se acerque a ningún valor, sino que cada vez se haga más grande. En ese caso el resultado del límite será $+\infty$.

\obs Puede ser que la función no se acerque a ningún valor, sino que cada vez se haga más pequeño. En ese caso el resultado del límite será $-\infty$.
\end{itemize}


\paragraph{Infinito, ¿concepto o valor?}
\subparagraph{Aritmética del infinito}

\hl{Empezar aquí}

\begin{defn}[Límite en el infinito]
Sea $f(x):\real\to\real$.

\[\lim_{x\to \pm\infty} f(x) = b\]

El límite de una función $f(x):\real\to\real$ en el infinito ($\pm\infty$) es el valor $b$ al que se aproxima la función cuando la variable toma valores \hl{arbitrariamente} grandes (o pequeños).
\end{defn}

\obs Se cumplen las mismas observaciones de antes.

\begin{example}
	\[\lim_{x\to\infty}x=\infty\]
	\[\lim_{x\to\infty}\frac{1}{x}=0\]
\end{example}

\begin{defn}[Exsitencia del límite en un punto finito] 
	El límite existe si existen y son iguales los límites laterales. 
\end{defn}


\paragraph{Propiedades de los límites} (me lo he saltado).

Sea $a\in\real\cup\{\infty,-\infty\}$, $f,g:\real\to\real$ y que dado $a$, $\exists\lim_{x\to a}f(x) \;;\; \exists\lim_{x\to a}g(x)$
\begin{itemize}
	\item $\lim_{x\to a} \left(f(x) \pm g(x)\right) = \lim_{x\to a} f(x) \pm \lim_{x\to a} g(x)$
	\item $\lim_{x\to a} \left(f(x) · g(x)\right) = \lim_{x\to a} f(x) · \lim_{x\to a} g(x)$
	\item $\lim_{x\to a} \left(f(x) \pm g(x)\right) = \lim_{x\to a} f(x) \pm g(x)$
	\item $\lim_{x\to a} \left(\frac{f(x)}{g(x)}\right) = \lim_{x\to a} f(x) \pm g(x)$
	\item $\lim_{x\to a} \left(f(x)\right)^{g(x)} = \left(\lim_{x\to a} f(x)\right)^{\lim_{x\to a} g(x)}$	
	\item $\lim_{x\to a} (f\circ g)(x) = ?$	
\end{itemize}

\begin{example}
\[\lim_{x\to 0}\frac{1}{x}\]
 \[\lim_{x\to 4}\left(\frac{x+1}{x}\right)^{\frac{1}{x-4}}\]
\end{example}

\paragraph{Indeterminaciones}

\begin{itemize}
	\item $k/0$
	\subitem Vistas (la $h$). Son indeterminaciones a medias, ya que sólo tienen 2 posibilidades: $\pm∞$
	\item $\infty/\infty$ (potencia más alta del denominador)
	\subitem $\displaystyle\lim_{x\to∞}\frac{x^2+5x+6}{x+\sqrt{x}} = \lim_{x\to∞}\frac{\frac{x^2}{x}+5\frac{x}{x}+\frac{6}{x}}{\frac{x}{x}+\sqrt{\frac{x}{x^2}}} = \lim_{x\to∞}\frac{x+5+\rfrac{6}{x}}{1+\frac{1}{\sqrt{x}}} = ∞$
	\item $0/0$ (simplificar)
	\subitem $\displaystyle\lim_{x\to1}\frac{x^2-5x+4}{x-1}$
	\subitem $\displaystyle\lim_{x\to1}\frac{x-1}{\sqrt{x-1}}$
	\item $\infty-\infty$
	\subitem $\displaystyle\lim_{x\to-∞}x^2+x = \lim_{x\to-∞} x(x+1) = (-∞)·(-∞) = ∞$
	\subitem El problema viene con 

	$\displaystyle\lim_{x\to∞}\sqrt{x^2+1}-x = \lim_{x\to∞}\frac{(\sqrt{x^2+1}-x)(\sqrt{x^2+1}+x)}{\sqrt{x^2+1}+x} = \lim_{x\to∞} \frac{x^2+1-x^2}{\sqrt{x^2+1}+x} = 0$
	\item $1^{+\infty}$
		\subitem Estas indeterminaciones aparecen cuando $\displaystyle\lim_{x\to a}f(x)^{g(x)}\;\lim_{x\to a}f(x) = 1\;\lim_{x\to a}g(x) = ∞$ y se resuelven utilizando:
		\[
			\displaystyle\lim_{x\to a}f(x)^{g(x)} = e^λ, \text{ donde } λ = \lim_{x\to a} (f(x)-1)g(x)
		\]
		\subitem $\displaystyle\lim_{x\to \infty}\left(\frac{x-1}{x}\right)^x = e^λ$ donde $λ=\displaystyle\lim_{x\to∞}\left(\frac{x-1}{x}-1\right)·x = -1$
\end{itemize}

\subsection{Continuidad}

\begin{defn}[Continuidad\IS en un punto]

Una función $f(x):ℝ\toℝ$ es continua en un punto $a$ si y sólo si se cumplen:
\begin{itemize}
	\item $\displaystyle∃\lim_{x\to a}f(x)$
	\item $\displaystyle∃f(a)$
	\item $\displaystyle\lim_{x\to a}f(x) = f(a)$
\end{itemize}

\obs Normalmente diremos: si existen y son iguales el límite y el valor en el punto.
\end{defn}

\paragraph{Tipos de discontinuidades}

Explicar gráficamente los casos.s

\begin{example}
\begin{itemize}
	\item ¿Es continua la función $f(x)= \frac{x^2-1}{x-1}$ en $x=2$?
	\item ¿Es continua la función $f(x)$ en $x=1$? 
	\item ¿Es continua la función $f(x) = \sqrt{x}$ en $x=0$?
	\item Funciones definidas a trozos (hasta aquí).
\end{itemize}
\end{example}

\begin{defn}[Continuidad\IS en un intervalo abierto]
Una función $f(x):ℝ\toℝ$ es continua en $(a,b)\inℝ$ si es continua $∀c\in(a,b)$.
\end{defn}

\begin{itemize}
	\item ¿Es continua la función $f(x) = \frac{x^2-1}{x-1}$ en $(2,∞)$?
	\item ¿Es continua la función $f(x) = \sqrt{x}$ en $(0,∞)$?
\end{itemize}



\begin{example}
Estudia la continuidad de la función $f(x) = \frac{\sqrt{x}}{x-4}$

\obs Una función sólo puede ser continua en su dominio. En este caso: $D(f) = \{x\inℝ\tq x≥0\}-\{x\in\real\tq x-4=0\} = [0,∞)-\{4\}$

La función es continua en $(0,4) ∪ (4,∞)$

\end{example}

\section{Derivabilidad}

\begin{defn}[Derivada\IS en un punto]
Sea $\appl{f}{ℝ}{ℝ}$ una función continua. Se define la derivada de $f$ en el punto $a$ como:

\[
	f'(a) = \lim_{x\to a}\frac{f(x)-f(a)}{x-a}
\]
\end{defn}

\obs $f$ \concept[Función\IS derivable]{derivable} en $x=a \dimplies \exists f'(a)$
\obs A la función inicial la llamaremos primitiva.

\begin{example}
Cálculo de la derivada de $f(x) = x^2+2x-1$ en $x=3$.

\[
	f'(3) = \lim_{\x\to3}\frac{x^2+2x-1 - (3^2+2·3-1)}{x-3} = \lim_{\x\to3}\frac{x^2+2x-15}{x-3} = \lim_{\x\to3}\frac{(x-3)(x+5)}{x-3} = \lim_{\x\to3}(x+5) = 8
\]

\end{example}

\begin{example}
Cálculo de la derivada de $f(x) = |x|$ en $x=0$.

\[
	f'(0) = \lim_{x\to0}\frac{|x| - |0|}{x-0} = \lim_{x\to0}\frac{|x|}{x} = \left\{\begin{array}{l}\displaystyle\lim_{x\to0^+}\frac{x}{x} = 1 \\ \displaystyle\lim_{x\to0^-}\frac{-x}{x}=-1\end{array}\right\}\implies \nexists\lim_{x\to0}f(x)
\]

Conclusión: La función $f(x)$ no es derivable en $x=0$. ¿Es continua? Sí.
\end{example}

\begin{example}
Cálculo de la derivada de $f(x) = \frac{1}{x}$ en $x=0$.

\[
	f'(0) = \nexists\lim_{x\to0}\frac{\rfrac{1}{x}-\rfrac{1}{0}}{x-0}
\]

No es derivable. ¿Es continua? No.
\end{example}

\obs \textbf{Derivable} $\implies$ \textbf{Continua}. \textit{Para que una función sea derivable en un punto es necesario que sea continua en ese punto}\footnote{Está en su tabla de derivadas}.

¿En física habéis derivado polinomios? ¿Cuál sería la "derivada" del polinomio? $f'(x) = 2x+2$. ¿Cuánto vale $f'(3)$? $f'(3) = 2·3+2=8$. ¿Casualidad?

Pero... ¿porqué esto es cierto? ¿De dónde sale ese $3x^2+2$? Esto es a lo que llamamos la función derivada:

\begin{defn}[Función\IS derivada]
Sea $\appl{f}{ℝ}{ℝ}$ una función continua. Se define la función derivada de $f$ como la función que a cada punto le asigna el valor de su derivada.

\[
	f'(x) = \lim_{h\to 0}\frac{f(x+h)-f(x)}{h}
\]
\end{defn}

\begin{example}
Cálculo de la función derivada $f'(x)$ de un polinomio, por ejemplo $f(x)=x^2+2x-1$

\[
	f'(x) = \lim_{h\to0}\frac{f(x+h)-f(x)}{h} = \lim_{h\to0}\frac{(x+h)^2+2(x+h)-1 - x^2+2x-1}{h} =
\]
\[
	\lim_{h\to0}\frac{(x^2+2xh+h^2+2x+2h-1 - x^2-2x+1}{h} = \lim_{h\to0}\frac{2xh+h^2+2h}{h} =
\]
\[
	\lim_{h\to0}\frac{h(2x+h+2)}{h} = \lim_{h\to0} 2x+h+2) = 2x+2
\]

Conclusión: la función derivada de $f(x) = x^2+2x-1$ es $f'(x) = 2x+2$.
\end{example}

\paragraph{Propiedades de la derivada:}
\begin{prop}[Cálculo operativo]
	Sean $f, g$ derivables en $a$. Entonces
	\begin{itemize}
		\item $(k·f(x))' = k·f'(x) ∀k\in\real$
		\item $(f\pm g)'(a)=f'(a)\pm g'(a)$
		\item $(fg)'(a)=f'(a)g(a)+f(a)g'(a)$
		\item Si $g(a)\neq 0 $, $\left(\frac{1}{g}\right)'(a)=\frac{-g'(a)}{(g(a))^2}$
		\item Si $g(a)\neq 0$, $\left(\frac{f}{g}\right)'(a)=\frac{f'(a)g(a)-f(a)g'(a)}{(g(a))^2}$
		\item $(g\circ f)'(a)= g'(f(a))f'(a)$
	\end{itemize}
\end{prop}

\hl{Explicación de la tabla de derivadas}

\paragraph{Regla de la cadena}

\paragraph*{A practicar:}
\begin{itemize}
	\item $\displaystyle f(x) = x\sqrt[3]{x^2}$
	\item $\displaystyle f(x) = x·e^x$
	\item $\displaystyle f(x) = x·\sen(x)$
	\item $\displaystyle f(x) = x·\ln(x)$
	\item $\displaystyle f(x) = \sen(x)·\cos(x)$
	\item $\displaystyle f(x) = \frac{x^2+1}{x}$
	\item $\displaystyle f(x) = (x^2+2x)·\sen(x)$
	\item $\displaystyle f(x) = (e^x - x)·\ln{x}$
	\item $\displaystyle f(x) = \sqrt{x^2+x}$
	\item $\displaystyle f(x) = (\arcsen{x})^3$
	\item $\displaystyle f(x) = \ln(4x)$
	\item $\displaystyle f(x) = (\cos{x})^2 = \cos^2{x}$
	\item $\displaystyle f(x) = \sen{(3x^2)}$
	\item $\displaystyle f(x) = \cos{(x^2+1)} $
	\item $\displaystyle f(x) = \tg{(x^2-3x)}$
	\item $\displaystyle f(x) = \sen{\sqrt{x^2+3x}} $
	\item $\displaystyle f(x) =  \cos{\frac{x-1}{x}}$
	\item $\displaystyle f(x) = \tg{\sqrt{x-1}} $
	\item $\displaystyle f(x) = -\sen{\frac{x}{-x^4+x-1}} $
	\item $\displaystyle f(x) = \tg{\frac{2}{\sqrt{1-x}}} $
\end{itemize}

\subsection{Interpretación geométrica de la derivada}



\section{Integrales inmediatas}



\section{Estudio sistemático de una función}

\begin{itemize}
	\item Dominio, siempre.
	\item Puntos de corte con los ejes.
	\item Simetría.
	\item Asíntotas, repaso de 4º.
	\item Monotonía.
	\item Curvatura.
\end{itemize}

\subsection{Monotonía}

Diferenciar los intervalos en los que la función crece de aquellos en los que la función decrece. También, ¿dónde están los máximos y los mínimos?

\begin{itemize}
	\item Ejemplo: dada la función $f(x) = 3x^2+2x+5$, encuentra sus extremos relativos.
	\item Los extremos relativos se encuentran en los puntos donde la derivada se hace 0. ¿Por qué? Interpretación gráfica.
	\item ¿Cómo distinguir máximo de mínimo? En este caso, la parábola va hacia arriba, por lo que debe ser mínimo. 
	\subitem Por un lado negativa (función decrece), por otro positiva (función creciente).
	\subitem Segunda derivada positiva, lo que marca que es un mínimo.
	\item Concavidad: segunda derivada positiva -> mínimo, convexa.
\end{itemize}

\begin{itemize}
	\item $f'(x) > 0 \implies f(x)$ crece. 
	\item $f'(x) < 0 \implies f(x)$ decrece. 
	\item $f'(x) = 0 \implies $ extremo relativo. ¿Máximo, mínimo?
	\item $f''(x) > 0 \implies $ mínimo y convexa-contenta (desde el semieje negativo de $y$).
	\item $f''(x) < 0 \implies $ máximo y cóncava desde el semieje negativo de $y$.
	\item 
\end{itemize}

Clase del jueves:

\begin{itemize}
	\item Corregir: demuestra que el vértice de la parábola es $\rfrac{-b}{2a}$
	\item Halla las rectas tangentes a la función $f(x)$ en los máximos (no es necesario intervalos de crecimiento y decrecimiento).
	\[
		f(x) = 3x^4-4x^3-36x^2
	\]
	\item Estudia la monotonía (intervalos de crecimiento, decrecimiento y extremos relativos y absolutos) de $f(x) = x^4-2x^2$
	
	\item Estudia sistemáticamente la función: $f(x) = \frac{2(x-2)+1}{(x-2)^2} = \frac{2x-1}{x^2-4x+4}$
	\subitem Punto de corte eje x: $(1.5,0)$, eje y: $(0,-0.75)$, mínimo absoluto: $(1,-1)$
	
	\item Estudia las asíntotas de $f(x) =\frac{x^3-4x^2+x+6}{2x^3-14x^2+32x-24}$
	
	\item Deriva $f(x) = \tan(x^2-3x)$
\end{itemize}


\printindex
\end{document}
\grid
