\documentclass[palatino,nosec]{Docencia}


\usepackage{breqn}


\title{Cuaderno de clase}
\author{Víctor de Juan}
\date{19/20}

\begin{abstract}
Cuaderno de clase de Matemáticas I, con el desarrollo continuado (sin estar separado por sesiones).
\end{abstract}

% Paquetes adicionales

\usepackage[author={Víctor de Juan, 2018}]{pdfcomment}

\makeatletter
\newcommand{\annotate}[2][]{%
\pdfstringdef\x@title{#1}%
\edef\r{\string\r}%
\pdfstringdef\x@contents{#2}%
\pdfannot
width 2\baselineskip
height 2\baselineskip
depth 0pt
{
/Subtype /Text
/T (\x@title)
/Contents (\x@contents)
}%
}
\makeatother

% --------------------
\newcommand{\cimplies}{\text{\hl{$\implies$}}}
\renewcommand{\vx}{\overset{\rightarrow}{x}}
\renewcommand{\vy}{\overset{\rightarrow}{y}}
\renewcommand{\vz}{\overset{\rightarrow}{z}}
\newcommand{\vi}{\overset{\rightarrow}{i}}
\newcommand{\vj}{\overset{\rightarrow}{j}}
\renewcommand{\vec}[1]{\overset{\rightarrow}{#1}}

\usepackage{pgf,tikz}
\usetikzlibrary{arrows}

\begin{document}
\pagestyle{plain}
\maketitle
\tableofcontents


%% Contenido.

\chapter{Introducción a la asignatura}


\hl{Soy matemático, entre otras cosas}, y espero poder transmitiros la belleza que encierran las matemáticas.

Hay una cosa que espero de vosotros este curso.
%
Si tuvieras que elegir una cualidad, ¿qué diríais? Lluvia de ideas durante 4 minutos. Yo diría \hl{madurez} claramente. Me encantaría que fuera amor la verdad… porque para amar de verdad hay que tener una cierta madurez, pero lo que realmente espero es madurez.

Se madura reflexionando en general. Sobre lo que te pasa, la vida… los problemas de tu alrededor…  ¿Cómo te fueron las \textbf{Matemáticas el año pasado}? Buscamos conclusiones del tipo: 
\begin{itemize}
	\item lo dejaba todo para el último día y no era capaz. \ul{¡Trabaja!}
	\item Descubrí que era capaz, aunque lo descubriera tarde. \ul{¡No te rindas!}
\end{itemize}

Te va a tocar currar mucho. Desde el principio más absoluto. Por eso hace falta madurez. 

\subsection{Temario}

\begin{itemize}
	\item Estadística y Probabilidad. Pobrecitos los de ciencias... Estadística con calculadora, en general.
	\item Álgebra. Ecuaciones. Como la primera evaluación de 4º con algún añadido (Gauss) [Terminado para la preevaluación]
	\item “Finanzas”: aplicación concreta. A ver si podemos hacer un trabajo real, que tendréis que ir a los bancos, etc. 
	\item Funciones.
\end{itemize}


\subsection{Metodología de clase}

\subsection{Criterios de evaluación}


\begin{itemize}
	\item 10\%: llevarlo al día.
	    \subitem 10\%, 30\%, 60\%. Estudiar al día. Tutorías a demanda. Cosas que tenéis que saber, aunque, seminarios.

    	\subitem Pruebecita del lunes, tal vez con teoría.

	\item 30\%: primer parcial.
	\item 60\%: segundo parcial.
\end{itemize}

\subsection{0 tonterías:} ¡Ah! se me olvidaba. 3 normas especiales de esta clase.

\begin{itemize}
	\item $(a+b)^2 = a^2 + b^2$ [no hay cosa que me moleste más... Una pena que no pueda poner partes de incidencias por semejante barbaridad...]
	\item "Matemáticas para tontos"
	\item "Es que soy/eres de letras"
\end{itemize}



\section{Estadística unidimensional descriptiva}

\paragraph{¿Para qué sirve la estadística?} Sirve para hacer predicciones. Estudiar los datos permite obtener información para poder hacer predicciones. 

Por ejemplo, si la nota media en una clase es de $\overline{x}=5$ y sabemos que los datos varían muy poco, es probable que si he perdido un examen, su nota fuera a ser un 5. En cambio, si la nota media es $\overline{x} = 5$ pero los datos están muy dispersos y hay desde $0$ hasta $10$, no tengo mucha posibilidad de estimar su nota sin equivocarme.

\subsection{Vocabulario estadístico}

Rellenar de ejemplos el margen del libro a lápiz.

\begin{itemize}
	\item Población
	\item Muestra
	\item Individuo
	\item Variables:
		\subitem Cualitativas o cuantitativas.
		\subitem Discretas o continuas.
\end{itemize}

\subsection{Tabla de frecuencias}
Comentar la tabla resuelta. 

\begin{itemize}
	\item ¿Por qué $\sqrt{n}$? Por simetría. Para tener el mismo número de intervalos que de datos por intervalo.
	\item Las $h_i$ y $H_i$ son porcentajes. ¿Cuánta gente está por debajo de $x$ kilos? Miramos la $H_i$ que corresponda.
	\item Interpretación de $x_i$. ¿Para qué sirven entonces los intervalos? Porque en realidad la variable con la que trabajo es una variable continua.
\end{itemize}

\subsection{Medidas}

\subsubsection{Centralización y posición}


\paragraph{Medidas de posición}

Las medidas de posición son los percentiles, aunque cuando dividimos la muestra en 4 partes hablamos de cuartiles. Al dividir la muestra en 10, hablamos de deciles. Al dividir la muestra en 2, hablamos de la mediana.



\paragraph{Utilidad de la mediana:} comparemos 2 muestras:
\[A=\{0,0,0,0,0,9,9,9,9\} \to \overline{x}_A = \frac{9*4}{9}=4\;\;\; Me_A = 0\]
\[B=\{4,4,4,4,4,4,4,4,4\} \to \overline{x}_A = \frac{9*4}{9}=4\;\;\; Me_A = 4\]

Estas 2 muestras tienen la misma media. La mediana nos ayuda a distinguir. 

\subparagraph{Sensibilidad a datos atípicos}
Supongamos una variante de la muestra $A$ anterior:

\[A'=\{0,0,0,0,0,9,9,9,90\} \to \overline{x}_A = \frac{9*3+90}{9}=29,25\;\;\; Me_A = 0\]

Por un dato atípico, la media sale realmente disparada. La mediana, en cambio, no se ve afectada.

\paragraph{Ejercicio 54 del libro}

\paragraph{Ejercicios prácticos de calcular percentiles} sobre la tabla resuelta de la página 223. 
%
Calculamos los cuartiles, deciles y percentiles.

\begin{defn}[Percentil n]
Es el dato que tiene a su izquerda el $n\%$ de los datos de la muestra.
\end{defn} 

\[
\begin{array}{c|c|c}
x_i & f_i&F_i\\\hline
1&15&15\\
2&7&22\\
3&33&56\\
4&10&6
\end{array}
\]

Vamos a calcular $P_{47}$

\[
P_{47} \to 47\% · 66 = 22,82 \to P_{47} = d_{23} = 3
\]

El $P_{47}$ debería ser el dato que está en la posición $22,82$. Como esa posición no existe, podríamos pensar en hacer la media entre el $d_{22}$ y el $d_{23}$. Incluso, una media ponderada. Pero no vamos a meternos tanto. 
%
Para calcular percentiles nos quedaremos con la primera frecuencia acumulada que supera, es decir, $\displaystyle P_k = d_{i}$ donde $i$ es el mínimo que cumple $F_i\geq k\%·N$

Cuando trabajamos con muestras de datos pequeños, los percentiles no funcionan bien. Cuando trabajamos con muestras grandes sí. ¿Qué sentido tiene esto? 
%
Si queremos medir la dispersión de 4 datos, no tiene mucho sentido, porque ya, simplemente viendo los datos nos lo imaginamos bien. 
%
¿Cuándo tienen sentido los percentiles? 
%
Cuando un bebé nace, querríamos saber si su peso está dentro de unos márgenes normales. 
%
Si el peso del hijo es $P_{99}$, es de los bebés más gordos que han nacido, ya que el $99\%$ de los bebés nacidos han pesado menos que él. 

\textit{En el ejemplo de los bebés, al tener tantísimos datos, los percentiles se comportan de una manera razonable. ¿Cómo corregir los percentiles para que funcionen en muestras de datos pequeñas? Estudia Matemáticas.}

\obs Con la mediana sí nos podemos poner exquisitos y calcular su valor exacto.

\subsubsection{Dispersión}

\textit{Que todo el mundo copie esto. Por si lo necesitas para tu vida futura. Yo no me sé las fórmulas de memoria. Sé de dónde salen y las razono y eso voy a intentar con vosotros.}

\begin{defn}[Rango]
\end{defn}

Una medida de la dispersión bastante intuitiva sería: ¿Cuánto se separa cada dato de la media? Y si sumo todas esas desviaciones, debería obtener una medida general. ¿Y por qué no cuánto se separa de la mediana? Total, la mediana también es una medida de centralidad. \footnote{Por esto necesitamos que la mediana sea un valor concreto y no puede ser un intervalo.}
%
Por motivos más complejos (de los que hablaremos al llegar a regresión), la que se utiliza habitualmente es la media, aunque en ocasiones sí se utiliza la mediana. ¿Por qué? Porque es menos sensible a datos atípicos.

Supongamos una muestra $A=\{5,5,5,5,5,5,5,5\}$ ¿Cuál sería su desviación? Sea cual sea lo que entendamos por desviación, debería ser 0.

Supongamos otra muestra, algo más compleja: $B = \{ 2,2,2,3,3,3,4,4,4,5,5,5 \} \to \overline{x}_B=3,5$
\paragraph{Opción 1:} Suma de las desviaciones

\[ \sum_{i=1}^{12} (x_i-\overline{x}) = 0\]

¿Porqué ocurre esto? Claramente los datos no datos uniformes. Tomando $i=1\to x_1-\overline{x} = -1,5$. Por otro lado, $i=12 \to x_{12}-\overline{x} = 1,5$, los datos se cancelan al sumarlos.

\paragraph{Opción 2:} Para solucionarlo podríamos pensar en el valor absoluto, para que esto no ocurra.

\[ \sum_{i=1}^{12} \abs{x_i-\overline{x}} = 12\]

Esto ya parece algo más razonable, pero el valor absoluto es una operación matemática con la que resulta muy complicado trabajar, porque aparecen ramas... 

\subparagraph{Opción 2.2: } Media de las desviaciones ¿Y si tengo el doble de datos pero igual de distribuidos?  Esta manera de calcular la desviación aumenta. ¿Cómo corregirlo? Dividiendo por el número de datos.

\[ \frac{\sum_{i=1}^{12} \abs{x_i-\overline{x}}}{12} = 1\]

\paragraph{Opción 3:} Otra manera de poner positivo algo que es negativo sería elevar a una potencia de índice par. Por ejemplo, 2.

\[ \frac{\sum_{i=1}^{12} \left(x_i-\overline{x}\right)^2}{12} = \frac{5}{4} = 1.25\]

\subparagraph{Opción 3.2} ¿Y podría elevar a orden 4? Hombre, por poder sí, pero entonces las diferencias pequeñas se hacen demasiado pequeñas y las grandes demasiado grandes.

\[ \frac{\sum_{i=1}^{12} \left(x_i-\overline{x}\right)^4}{12} = \frac{41}{16} > 2·\frac{5}{4}\]


\paragraph{Opción 4:} A ver, si hemos elevado al cuadrado, hemos introducido artificialmente un mucho ruido en la medición. ¿Cómo podríamos contrarrestar eso? Haciendo la raíz cuadrada.

\[ \sqrt{\frac{\displaystyle\sum_{i=1}^{12} \left(x_i-\overline{x}\right)^2}{12}} = 1.12\]

\begin{table}[hbtp]
\centering
\label{SD:medidasDispersion}
\caption{Distintas opciones para medir la dispersión con respecto a la media.}
\begin{tabular}{c|ccc}
\textbf{Nombre} & \textbf{Medidas} & \textbf{Medida razonable} & \textbf{Contras}\\\hline\hline
 & $\displaystyle\sum_{i=1}^{N} (x_i-\overline{x})$ & No & Puede resultar 0 en datos sí dispersos\\\\
Desv. media  & $\displaystyle\frac{\displaystyle\sum_{i=1}^{N} \abs{x_i-\overline{x}}}{N}$ & Sí & Difícil de trabajar con el valor absoluto (derivadas)\\\\
Varianza $\sigma^2$ & $\displaystyle \frac{\displaystyle\sum_{i=1}^{N} \left(x_i-\overline{x}\right)^2}{N}$ & Sí. & Medida algo falseada por elevar al cuadrado\\\\
Desv. típica $\sigma$ & $\displaystyle \sqrt{\frac{\displaystyle\sum_{i=1}^{N} \left(x_i-\overline{x}\right)^2}{N}}$ &Sí. & Corrección artificial de elevar al cuadrado\\\\\hline
\end{tabular}
\\
\textit{\textbf{Obs:} Todas estas mismas medidas se pueden calcular respecto de la mediana. Cambiarían los nombres}
\end{table}


\textbf{Repasamos medidas de dispersión: } fórmula y ya está. Tizas de colores para las letras.

\begin{defn}[Varianza]
\[
	\sigma^2 = \frac{\displaystyle\sum_{i=1}^{N} \left(x_i-\overline{x}\right)^2}{N}
\]
Donde $N$ es el número total de datos, $x_i$ cada uno de los datos y $\overline{x}$ la media.

\obs En caso de haber datos $x_i$ repetidos, existirá un número de veces que cada dato se repite $f_i$. 
%
Podemos calcular la varianza de una manera más sencilla:
\[
	\sigma^2 = \frac{\displaystyle\sum_{i=1}^{n} f_i·\left(x_i-\overline{x}\right)^2}{N}
\]
Donde $n$ el número de clases (filas de la tabla de frecuencias) y lo demás se mantiene.
\end{defn}

\begin{prop}
	\[\sigma^2 = \frac{\displaystyle\sum_{i=1}^{N} \left(x_i-\overline{x}\right)^2}{N} = \frac{\displaystyle\sum_{i=1}^N x_i^2}{N} - \overline{x}^2\]
\obs Por los datos repetidos, se cumple:
	\[\sigma^2 = \frac{\displaystyle\sum_{i=1}^{n} f_i\left(x_i-\overline{x}\right)^2}{N} = \frac{\displaystyle\sum_{i=1}^N f_ix_i^2}{N} - \overline{x}^2\]
\end{prop}

Si alguien me demuestra esta proposición fuera de clase +0,25 en el examen (que te recuerdo que vale 2 tercios).

\begin{proof}
	\begin{dmath}
	\sigma^2 = \frac{\displaystyle\sum_{i=1}^{N} \left(x_i-\overline{x}\right)^2}{N} =
	  \frac{\displaystyle\sum_{i=1}^{N} \left(x_i^2+\overline{x}^2-2x_i\overline{x}\right)}{N} = 
	  \frac{\displaystyle\sum_{i=1}^{N} x_i^2}{N} + \underbrace{\frac{\displaystyle\sum_{i=1}^{N}\overline{x}^2}{N}}_{\small\displaystyle\sum_{i=1}^9 z = 9·z}-\frac{\displaystyle\sum_{i=1}^{N} 2x_i\overline{x}}{N} =
	  \frac{\displaystyle\sum_{i=1}^{N} x_i^2}{N} + \frac{N·\overline{x}^2}{N}-2\overline{x}\underbrace{\frac{\displaystyle\sum_{i=1}^N x_i}{N}}_{\text{ def. }\overline{x}} = 
	  \frac{\displaystyle\sum_{i=1}^{N} x_i^2}{N} + \overline{x}^2-2\overline{x}^2 = 
	  \frac{\displaystyle\sum_{i=1}^{N} x_i^2}{N} - \overline{x}^2
	\end{dmath}
	En caso de datos repetidos, donde $N$ no es el número total de datos, sino de filas de la tabla. Llamamos $n$ al número total de datos
	\begin{dmath}
		\sigma^2 = \frac{\displaystyle\sum_{i=1}^{n} \left(x_i-\overline{x}\right)^2}{N} =
	  \frac{\displaystyle\sum_{i=1}^{n} f_i\left(x_i^2+\overline{x}^2-2x_i\overline{x}\right)}{N} = 
	  \frac{\displaystyle\sum_{i=1}^{n} f_ix_i^2}{N} + \underbrace{\frac{\displaystyle\sum_{i=1}^{n}f_i\overline{x}^2}{N}}_{\small\displaystyle\sum_{i=1}^9 z = 9·z}-\frac{f_i\displaystyle\sum_{i=1}^{n} 2x_i\overline{x}}{N} =
	  \frac{\displaystyle\sum_{i=1}^{n} x_i^2}{N} + \frac{N·\overline{x}^2}{N}-2\overline{x}\underbrace{\frac{\displaystyle\sum_{i=1}^N x_i}{N}}_{\text{ def. }\overline{x}} = 
	  \frac{\displaystyle\sum_{i=1}^{n} x_i^2}{N} + \overline{x}^2-2\overline{x}^2 = 
	  \frac{\displaystyle\sum_{i=1}^{n} x_i^2}{N} - \overline{x}^2
	\end{dmath}
\end{proof}

En la varianza se eleva al cuadrado cada diferencia de una manera algo arbitraria. 
%
Para corregir ese falseamiento se podría utilizar la raíz cuadrada.

\begin{defn}[Desviación típica]
	\[ 
		\sigma = \sqrt{\sigma^2} = \sqrt{\displaystyle\frac{\displaystyle\sum_{i=1}^{N} \left(x_i-\overline{x}\right)^2}{N}}
	\]
\end{defn}

\obs Para la varianza se eleva al cuadrado, pero se podría elevar a cualquier número par. Lo único que buscamos es que la suma se vuelva positiva.

\begin{defn}[Mi varianza personal]
\[
	\sigma^{2a} = \frac{\displaystyle\sum_{i=1}^{N} \left(x_i-\overline{x}\right)^{2a}}{N}
\]
Donde $N$ es el número total de datos, $x_i$ cada uno de los datos, $\overline{x}$ la media y $a\in\mathbb{Z}$
\end{defn}


\begin{defn}[Mi desviación típica personal]
\[
	\sigma^{a} = \sqrt{\displaystyle\frac{\displaystyle\sum_{i=1}^{N} \left(x_i-\overline{x}\right)^{2a}}{N}}
\]
Donde $N$ es el número total de datos, $x_i$ cada uno de los datos, $\overline{x}$ la media y $a\in\mathbb{Z}$
\end{defn}

\newcommand{\coefdemierda}{Coeficiente de variación\xspace}
\begin{defn}[\coefdemierda]
Se define como $\frac{\sigma}{\overline{x}}$
\end{defn}

\obs ¿Qué ocurre con el \coefdemierda si $\overline{x} = 0$? 
Por eso no se utiliza este coeficiente.



\subsection{Problema práctico}

Problema 21 y problema 50.

Trabajar en 27, 28, 29, 30 (puede ser esquemático),32 ,35 ,39 ,43 ,51 ,53


\section{Probabilidad, binomial y normal}

\subsection{Combinatoria}

Principio multiplicativo:

Caminos para ir de $A$ a $C$, pasando por $B$. ¿Cuántos caminos posibes hay? 

% https://www.fing.edu.uy/tecnoinf/mvd/cursos/pye/materiales/practico/pye-pr02.pdf
\begin{enumerate}
	\item En un centro escolar hay 40 en 1º de ESO, 35 en 2º, 32 en 3º y 28 en 4º. Para hablar con la dirección se quiere formar una comisión que esté integrada por un alumno de cada curso. ¿Cuántas comisiones se pueden formar?
	{$40·35·32·28=54400$}
	\subitem Si el alumno de 1º de la ESO ya ha sido elegido. ¿Cuántas comisiones distintas se pueden formar?
	{$35·32·28=360$}
	\item ¿Cuántos números de tres cifras se pueden formar con las cifras 1, 2, 3 y 4 sin que se repita ninguna? 
	{$4·3·2·1 = 4! = 24$}
		\subitem b) ¿Cuántos terminan en 34? 
		{$2$, el $1234$ y el $2134$}
		\subitem c) ¿Cuántos habrá que sean mayores que 300?
		{24, porque el más pequeño, $1234$ es mayor que 300}
	\item ¿De cuántas maneras 4 personas en 10 sitios?
	{$10·9·8·7 = 5040$}
	\subitem ¿Y si pueden sentarse unos encima de otros? 
	{$10·10·10·10 = 10000$}
	\item ¿De cuántas maneras puedo ordenar 4 bolas?
	\item En una urna hay 6 bolas distintas, ¿De cuántas maneras distintas pueden sacarse? 
	\item En una urna hay 3 bolas rojas, 1 azul, 1 amarilla y 1 naranja ¿De cuántas maneras distintas pueden sacarse? 
	\item En una urna hay 8 bolas: 4 verdes y las otras 4 distintas. ¿De cuántas maneras distintas pueden sacarse?
	\item En una urna hay 10 bolas: 2 rojas, 4 verdes y las otras 4 distintas. ¿De cuántas maneras pueden sacarse?
	\item En una urna hay tres bolas rojas, tres verdes, cuatro negras y dos azules. ¿De cuántas maneras distintas pueden sacarse, bola a bola, de la urna?

%%%%%%%%%%%%%%%%%%%%%%%%%%%%%%%
	\item 10 alumnos. Se reparten 3 premios. Se dan varios casos: Premios iguales o distintos. Una persona puede recibir varios premios o no.
	\subitem ¿En qué caso deberían salir más posibilidades?
	\subitem Comprueba tu intuición calculando las posibilidades de cada caso.
	\item Número de diagonales de un cuadrado, de un hexágono, de un heptágono, de un polígono de $k$ lados.
	\item 5 hombres y 4 mujeres. Hombres en lugares impares y mujeres en lugares pares.
	\item Número de posibles matrículas.
	\item En un grupo de 10 amigos, ¿cuántas distribuciones de sus fechas de cumpleaños pueden darse al año?
	\item Lanzo 4 monedas al aire. ¿Cuántos resultados posibles puedo obtener?
	\item Un alumno tiene que elegir 7 de las 10 preguntas de un examen. ¿De cuántas maneras puede elegirlas? ¿Y si las 4 primeras son obligatorias?
	\item Una línea de tren tiene 25 estaciones. ¿Cuántos billetes distintos habrá que imprimir si cada billete lleva impresas las estaciones de origen y destino?
	\item ¿De cuántas maneras distintas pueden llegar 3 atletas a la meta? (Pueden llegar exactamente al mismo tiempo).
	\item ¿Cuántos números de DNI distintos existen? ¿Y números de teléfono?
	\item Una lista de Spotify tiene 30 canciones. ¿Cuántas posibles reproducciones de la lista completa, si las canciones se reproducen aleatoriamente, puede haber? ¿Y si la primera canción siempre debe ser la misma? ¿Y si no se repiten aleatoriamente, sino que siguen el orden alfabético?
	\subitem ¿Cuál es la probabilidad de que una lista de reproducción de 30 canciones reproducidas al azar siga el orden alfabético?
	\item Una persona tiene 6 chaquetas y 10 pantalones. ¿De cuántas formas distintas puede combinar estas prendas?
	\item Una familia, formada por los padres y tres hijos, van al cine. Se sientan en butacas consecutivas.
		\subitem a) ¿De cuántas maneras distintas pueden sentarse?
		\subitem b) ¿Y si los padres se sientan en los extremos?
	\item Con los números 3, 5, 6, 7 y 9 
		\subitem ¿cuántos productos distintos se pueden obtener multiplicando dos de estos números? 
		\subitem ¿Cuántos de ellos son múltiplos de 2? 
		\subitem ¿Cuántos cocientes distintos se pueden obtener dividiendo dos de estos números?
	\item ¿Cuántos números hay entre 2000 y 3000 que tengan sus cifras diferentes?
	\item ¿Cuántos resultados distintos pueden aparecer al lanzar un dado 4 veces?
	\item a) ¿Cuántos números de 6 cifras puedes escribir con los dígitos 1, 2 y 3?. b) ¿Cuántos de ellos contienen todos los dígitos 1, 2 y 3 al menos una vez? 
	\item ¿Cuántos números de 4 cifras puedes escribir con los dígitos 1,2,3 y 4? b) ¿Cuántos de ellos contienen los 4 dígitos? c) ¿Cuántos de ellos tienen alguna cifra repetida?
	\item Todas las personas que asisten a una reunión se estrechan la mano. Si hubo 105 apretones, ¿cuántas personas asistieron?
	\item ¿Cuántas columnas tenemos que cubrir para acertar seguro una quiniela?. Cada columna tiene 15 resultados a elegir entre 1, X, 2. 
	\item Ocho amigos van de viaje llevando para ello dos coches. Si deciden ir 4 en cada coche.
		\subitem a) ¿De cuántas formas pueden ir si todos tienen carnet de conducir?
		\subitem b) ¿De cuántas formas pueden ir si sólo tres tienen carnet de conducir?
	\item  ¿De cuántas maneras pueden ordenarse 6 libros en un estante si:
		\subitem a) es posible cualquier ordenación?
		\subitem b) 3 libros determinados deben estar juntos?
		\subitem c) dos libros determinados deben ocupar los extremos?
		\subitem d) tres libros son iguales entre sí?
	\item ¿Cuántas palabras distintas se pueden formar con las letras de la palabra SOBRE?
	\item ¿Cuántas palabras distintas se pueden formar con las letras de la palabra MATEMATICAS?
\end{enumerate}

\subsection{Probabilidad}

{¿Qu\'e es una probabilidad? La respuesta intuitiva será un número entre $0$ y $1$. Primera en la frente: \hl{es una función}. Siempre es la "probabilidad de un suceso", ¿no? Necesito saber de qué suceso. 

Una probabilidad es una función. ¿Cuántas cosas recibe una función?

Todas las funciones tienen \hl{un origen y un destino}. Una probabilidad llega a un número real. ¿De qué parte? (Ejemplo de raíces: parte de números positivos para dar un número real).

Dado un dado, ¿cuál es su espacio muestral? Razonado con el mítico dibujito de funciones 
$E=\{1,2,3,4,5,6\}$. Ahora puedo calcular $P(\{3\}) = ...$, ¿no? Perfecto. ¿La probabilidad, entendida como función, \hl{¿parte del espacio muestral?} No, porque ¿$P(\{1\}\cup\{2\})$? 

Vamos a ver una \hl{definición axiomática} de la probabilidad. ¿Axioma? ¿A alguien le dice algo? Os lo cuento, fundamentalmente, para que tengáis una \textbf{visi\'on real de lo que son las Matemáticas.}}

Un axioma es una propiedad que se acepta sin demostración. A partir de ahí construimos la teoría al completo.

{¿Cuántas propiedades crees que hacen falta aceptar sin demostrar para la teoría de la probabilidad?}


\begin{defn}[Axiomática de Kolmogorov]
Dado un conjunto de sucesos elementales $E$ (llamado espacio muestral), $\mathcal{P}(E)$ al conjunto de todos los subconjutos de $E$ y una función $P$ que asigna un valor real a cualquier suceso $A$.

Decimos que $P$ es una probabilidad del espacio $E$ si cumple las siguientes propiedades:

\begin{itemize}
	\item $\forall A\in \mathcal{P}(E), P(A)\geq 0$
	\item $P(E) = 1$
	\item $\forall A\in \mathcal{P}(E)$ con $\{A_1,A_2,..., A_k\}$ sucesos incompatibles dos a dos. Entonces \[ P\left(\bigcup_{j=1}^k A_j\right) = \sum_{j=1}^k P(A_j) \]
\end{itemize}

\obs Llamamos \concept{suceso} a los elementos de $\mathcal{P}(E)$.

\end{defn}

\textit{¿Y esta definición por qué? ¿Dónde ha quedado eso de casos favorables entre posibles? Lo seguiremos utilizando, pero es una consecuencia lógica de estas propiedades.}

\textit{Sólo con estas 3 propiedades vamos a poder construir todas las demás. ¿Te acuerdas cuando el año pasado decía que las Matem\'aticas de 4º son un poco "de andar por casa"?}

\begin{example} Un dado. 

$E = \{1,2,3,4,5,6\}$

$\mathcal{P}(E) = \{1,2,3,4,5,6,\{1,2\},\{1,3\},\{1,4\} ... , \{6,6\},\{1,2,3\} ..., \emptyset \}$

$\emptyset \equiv$ "no sale nada"

Por cierto, ¿cuántos elementos tiene? $6+6·5+6·5·4+6·5·4·3+... + 6!+1$

Definiendo $P(e) = \rfrac{1}{6} \forall e\in E$, con la tercera propiedad podemos calcular:

$P(\{1\} \cup \{2\} = P(\{1\}) + P(\{2\}) = \rfrac{2}{6} = \rfrac{1}{3}$

\textit{Hacer énfasis en que la probabilidad recibe conjuntos, no números}

Calcular la probabilidad de sacar un número par y mayor que $3$ (Razonamiento con dibujos aplicando el tercer axioma).

\end{example}

\begin{example}
Sea $E=\{e_1,e_2,e_3\}$. Estudia si las siguientes funciones son espacios de probabilidad:

\begin{itemize}
	\item $P_1(e_1) = 1/2 ; P_1(e_2) = 1/2 ; P_1(e_3) = 1/3$
	\item $P_2(e_1) = 1/3 ; P_2(e_2) = 0 ; P_2(e_3) = 2/3$
\end{itemize}

\obs Fíjate lo que nos hemos abstraído, que no necesitamos nada de nada. 
\end{example}

\begin{prop}[Propiedades de una función de Probabilidad]
$\forall A,B \in \mathcal{P}(E)$
\begin{itemize}
	\item $P(\overline{A}) = 1 - P(A)$
	\begin{proof}
		\[1=P(E) = P(A\cup\overline{A} = P(A) + P(\overline{A})\]
	\end{proof}
	\item $P(\emptyset) = 0$ porque $\overline{\emptyset} = E$
	\item $A\subset B \dimplies P(A) \leq P(B)$
	\item $0\leq P(A)\leq 1$
	\item $P(A\cup B) = P(A) + P(B) - P(A\cap B)$
\end{itemize}
\end{prop}

\begin{prop}[Regla de Laplace]
En un espacio muestral equiprobable $E$ con $n$ elementos, podemos calcular:

\[\forall A\in \mathcal{P}(E)\;\;\; P(A) = \frac{\text{fav.}}{\text{pos}}\]
\end{prop}

Problemas 11,12,13,14


\subsection{Empezando probabilidad de 0}

\subsubsection{Operaciones con conjuntos}

\begin{prop}[Propiedades de conjuntos]
Algunas propiedades de conjuntos, necesarias para probabilidad:

	\begin{itemize}
		\item $A\cup A=A$ y $A\cap A = A$
		\item $A\cup B=B\cup A$ y $A\cap B = B\cap A$
		\item $A\cup \emptyset=A$ y $A\cap \emptyset = \emptyset$
		\item $A\cup (B\cap C) = (A\cup B)\cap (A\cup C)$
		\item $A\cap (B\cup C) = (A\cap B)\cup (A\cap C)$
		\item $\overline{\overline{B}} = B$
		\item $\overline{A}\cap A = \emptyset$
		\item $A\setminus B = A\cap \overline{B}$
		\item[DeMorgan] $\overline{A\cup B} = \overline{A}\cap\overline{B}$
		\item[DeMorgan] $\overline{A\cap B} = \overline{A}\cup\overline{B}$
	\end{itemize}
\end{prop}
\begin{proof}
Dibujaremos gráficamente cada propiedad.
\end{proof}

\begin{problem}[1]
Simplifica las siguientes igualdades:

\ppart $(A\cup B)\cap (A\cup B)\cap A=A$

\ppart $(B\cup A)\cap \overline{\left(\overline{B}\cap \overline{A}\right)} = A\cup B$

\ppart $\left[A\setminus (A\cup B)\right]\cup B = B$

\ppart $(A\cap B) \cup (A\cap \overline{B}) = A$

\ppart $(A\setminus B) \cap B = \emptyset$


\solution
\end{problem}

Problemas 11 y 12.

\begin{problem}[2]
En una urna de lotería hay bolas 4 bolas amarillas, 3 rojas, 5 naranjas y 1 morada.

Calcula las siguientes probabilidades:

(Sacando de una en una)
\ppart No morada.

\textbf{Regla de LaPlace}

\ppart Ni roja ni amarilla.

(Sacando de varias en varias)

\ppart Las 3 bolas rojas.

\ppart 2 bolas rojas y una amarilla.

\solution
\end{problem}

\textbf{Tablas de contingencia:} explicar probabilidad condicionada desde tablas de contingencia.

En una tabla, traducir frases a probabilidades y sucesos con condicionadas.

\textit{Para resolver este problema en uno de los tercios de la pizarra hacemos una raya vertical discontinua a la derecha para ir escribiendo el resumen teórico con las fórmulas y demás. Sé que no les gusta y prefieren primero la teoría y luego ejemplos, pero considero importante que vean de dónde salen las fórmulas y que las razonen desde el ejemplo.}


\begin{problem}[G1]
En una clase hay 140 mujeres, de las cuales 60 llevan gafas. Hay también 60 hombres con gafas y 40 hombres sin ellas.

Si elegimos una persona al azar,
\ppart ¿Cuál es la probabilidad de que lleve gafas?
\ppart ¿Cuál es la probabilidad de que lleve gafas, sabiendo que es mujer?
\ppart ¿Cuál es la probabilidad de que lleve gafas, sabiendo que es hombre?

\solution



\spart Hay 120 (60+60) personas que llevan gafas. En total hay 240 personas (140+60+40), luego, utilizando la regla de Laplace:  $P(G) = \frac{90}{210} = \frac{3}{7}$

\obs Errores cometidos. ¿Qué es "G"?

\obs Enseñar a simplificar fracciones con la calculadora.

\obs Construimos la tabla de "contingencia", que es una buena manera de agrupar la información.

\hl{Construir tabla}

\spart Ahora sólo me importan mujeres, ¿no? Es como si no hubiera hombres. Para la regla de Laplace, mis casos posibles son 140 (las mujeres = $M$) y los favorables son 60 (las mujeres que llevan gafas = $M\cap G$)

Es decir: $P(\text{ gafas, sabiendo que son mujeres} ) = \frac{P(G\cap M)}{P(G)} = \frac{60}{140}$

\spart ...

\end{problem}


\begin{problem}[G2]
En una clase hay 140 mujeres, de las cuales 60 llevan gafas. Hay también 30 hombres con gafas y 40 hombres sin ellas.

Si elegimos una persona al azar,
\ppart ¿Cuál es la probabilidad de que lleve gafas?
\ppart ¿Cuál es la probabilidad de que lleve gafas, sabiendo que es mujer?
\ppart ¿Cuál es la probabilidad de que lleve gafas, sabiendo que es hombre?

\solution

\spart Construimos la tabla y contestamos a las preguntas.

\spart ...

\spart ...

\end{problem}

\subsubsection{Independencia}

\begin{defn}[Independencia]
2 sucesos son independientes si $P(A \tq B) = P(A)$
\end{defn}

\textbf{Ejemplos}
\begin{itemize}
	\item En el problema $G1$, ¿género y gafas son independientes?
	\item En el problema $G2$, ¿género y gafas son independientes?
\end{itemize}

Vamos a darle a la maquinaria de la lógica.


\hl{Corregir 17 y 18. 
Explicar independencia desde la intersección.
}
\begin{corol}
Si $A$ y $B$ son sucesos independientes ($P(A) = P(B\tq A)$)

$P(B\tq A) = \frac{P(B\cap A)}{P(A)} \to P(A\cap B) = P(A) · P(B\tq A) = P(A) · P(B)$
\end{corol}



\hl{Resolver el 17 como árbol}

\textbf{Problemas de Árbol}
Problema 76,75,77

\textbf{Probabilidad total}
Problema 82,83,84...

Resolviendo un problema de diagrama de árbol, razonar el teorema.
\begin{theorem}[Probabilidad Total]
Sea $\{A_i\}_0^n$ una partición del espacio muestral $E$.

Sea $B$ un proceso del espacio muestral.

$P(B) = \sum_{i=0}^n P(B\tq A_i)$
\end{theorem}


\textbf{Bayes}
Problema 87 (o alguno de los del libro de esa página)

\begin{theorem}[Bayes]
\begin{gather*}
\left\{
\begin{array}{c}
	P(A\tq B) = \frac{P(A\cap B)}{P(B)} \to P(A\cap B) = P(A\tq B)· P(B)\\
	P(B\tq A) = \frac{P(B\cap A)}{P(A)} \to P(B\cap A) = P(B\tq A)· P(A)
\end{array} \right\} 
\\
\implies P(A\tq B)· P(B) = P(B\tq A)· P(A) \dimplies 
\\
P(B\tq A) = \frac{P(A\tq B)· P(B)}{P(A)}
\end{gather*}
\end{theorem}

Problemas 88 y 89. \hl{¿Entran así en selectividad?}


\textbf{Cuestiones teóricas}


\section{Distribuciones de probabilidad}

\begin{problem}
Se tira un dado de 6 caras 100 veces. Calcula la probabilidad de:

\ppart A="sacar 100 veces un 6"

\ppart B="sacar exactamente una vez un 5 y todo lo demás 6"

\ppart C="sacar exactamente una vez un 5"

\ppart D="sacar exactamente 2 veces un 5"

\ppart E="sacar exactamente 3 veces un 5"

\ppart F="sacar exactamente $k$ veces un 5"

\solution

\end{problem}

\textit{Fijaos, hemos llegado a una fórmula general de cálculo de probabilidades para resolver este caso específico sin necesidad de hacer el árbol ni nada de nada. Ya sabéis que me gusta partir de ejemplos para deducir las fórmulas generales.
\\
Esto, tiene un nombre propio y es con lo que empezamos tema nuevo: Distribuciones de probabilidad.
\\
En el libro viene bastante bien, asíque lo vamos a seguir. Página 294.
}

4 conceptos a distinguir, leyendo las definiciones del libro.
\begin{itemize}
	\item Distribución de probabilidad: binomial.
	\subitem \textit{La manera que tiene la probabilidad de distribuirse entre los sucesos}.
	\item Variable aleatoria: en los apartados C-F, sería el número de cincos.
	\subitem $P(C) = P(X=1)$.
	\subitem $P(E) = P(X=3)$.
	\item Función de probabilidad: función que devuelve la probabilidad.
	\item Función de distribución [para dentro de unos días. Básicamente es la probabilidad acumulada.].
\end{itemize}

Ejemplo página 292.
\begin{itemize}
	\item Podría definir cualquier variable aleatoria.
\end{itemize}


\textit{Cuando estén claros:} 

¿Podemos calcular la media de una distribución? En el ejemplo de la 292 y después el ejemplo del número de cincos. ¿Se puede calcular? Y, calculando la media, ¿se puede calcular la desviación típica?

Ya hemos visto las páginas 292-294. Lo interesante de este tema son 2 distribuciones especiales y sólo haremos ejercicios relativos a ellas. Lo demás, solamente es útil para pasar a las distribuciones que verdaderamente se utilizan.

Página 295: distribución binomial. Ejemplo resuelto.

Reconocer distribuciones binomiales, ejercicio 43.

Deberes: 46


______

Ejercicio resuelto página 296. Ejercicios tipo para hacer con esto.

\subsection{Binomial}

Problemas. Media y varianza - desviación típica.

\subsection{Normal}

Desde la binomial a la normal.

% Código en Sage para dibujar.
%import scipy.stats
%n=200
%p=0.5
%binom_dist = scipy.stats.binom(n,p)
%bar_chart([binom_dist.pmf(x) for x in range(n*p*2)])

Vamos dibujando binomiales cada vez con más datos y probabilidades cambiantes (primero de los extremos de la probabilidad (0,0009) hacia los no extremos (0,4)). ¿Hay alguna aproximación? Tal vez haya una curva suave que se parezca...

Sí la hay.  Te presento a distribución normal. En estadística se denomina normal a todo lo que sigue esta distribución. Centrado, y más pegado o menos pegado.

Casi todo está cerca de la media. Esta distribución sólo necesita de la media y la desviación típica (que no varianza). Según la media y la desviación típica que le demos sean, la gráfica será de una u otra manera.

Podemos ver que todas tienen la misma forma y conociendo cómo funciona una de ellas, podríamos conocer todas las demás. Solamente hay que mover el centro y la reajustar la tripa. A este "mover el centro y reajustar la tripa" se le llama "tipificar", volverlo típico, de la distribución normal típica, la 0,1. 

[Utilizar el PPT de apoyo]


De la misma manera: ejemplo resuelto 3. Leo yo el enunciado, antes de decirles que es del libro. Problemas 15 y 16. Para comparar datos y poder tomar decisiones. 


\hl{Clase del viernes 1/02}
Corregimos el 16.

Una vez entendida la tipificación, vamos con el cálculo de probabilidades. ¿Por qué no se calcula la probabilidad de "medir exactamente 1,75"? Porque... en realidad, 1.75 es entre 1.745 y 1.754, luego ya tengo un intervalo. 
Además, lo habitual será preguntarme por "¿Cuánta gente es más alta de 1.90?". En general, me interesarán intervalos, por ello la normal va por intervalos.

Tipificar el volver típico. Vamos a dedicarnos hoy a familiarizarnos con la distribución normal típica y con la tabla de la normal

Supongamos que tengo una normal de media 0 y de desviación típica 1. La tabla lo que me da es lo que dice el dibujo, el área, la suma de las barras, la probabilidad de estar por debajo. Como la distribución es simétrica, la media está en el medio, coincide con la mediana (esto no siempre es así). La mitad de la gente está por debajo, la mitad por encima.

\begin{itemize}
	\item $P(Z<0)$
	\item $P(Z<1)$
	\item $P(Z<1,13)$
	\item $P(Z<1,1234)$
	\item $P(Z > 1) = 1 - P(Z\leq 1) = ...$ [contrario]
	\item $P(Z > 3) = $ [contrario]
	\item $P(Z  > -1) = P(Z < 1)$ [simetría]
	\item $P(Z  > -1,21) = P(Z < 1)$ [simetría]
	\item $P(Z  < -1) = P(Z > 1) = 1 - P(Z<1)$ [simetría y suceso contrario.]
	\item $P(Z  < -4) = P(Z > 1) = 1 - P(Z<1)$ [simetría y suceso contrario.]
\end{itemize}

Ejercicios 66,67,68,69[abc]; 72; 74 ; Tipificar + De deberes 66-69[de]

Supongamos que la altura media de una mujer es 1,70cm y la desviación típica 7cm. ¿Cuántas personas podríamos esperar que midieran más menos de 1.75?

Necesitaríamos la tabla de la normal de media 1,70cm y de desviación típica 7cm, ¿no? Otra opción, más sencilla porque no podemos hacer estadística con Excel (todavía) sería \textit{tipificar} los datos, volverlos típicos, llevárnoslos a nuestra escala \textit{típica}, que sería la normal de media 0 y desviación típica 1. Tipificamos este valor: $\frac{1,85-1,70}{7} = $

Aprendemos a utilizar la tabla de la normal. 

A calcular los que nos faltan. Simetría y susceso contrario.

Una vez hayamos aprendido, empezaremos a resolver $P(Z<k) = 0,2325$ y $P(X<k) = 0,2325$ y derivados.


\begin{itemize}
	\item Procedimiento general: conseguimos un número mayor que 0,5 a la derecha y después arreglamos el interior del paréntesis jugando con la simetría.
	\item $P(X>k) = 0,8413$ [-1,00]
	\[P(X>k) = P(X<-k) = 0,8413 \implies -k = 1 \implies k=-1\]
	\item $P(X>k) = 0,9920$ [-2,41]
	\[P(X>k) = P(X<-k) = 0,9920 \implies -k = 2,41 \implies k=-2,41\]
	\hrule{}
	\item $P(X<k) = 0,1587$ [-1,00]
	\[P(X<k) = 0,1587 \dimplies P(X>k) = 0,8413 \dimplies P(X<-k) = 0,8413 \implies -k = 1 \dimplies k=-1\]
	\item $P(X<k) = 0,2877$ [-0,56]
	\[P(X<k) = 0,2877 \dimplies P(X>k) = 0,7123 \dimplies P(X<-k) = 0,7123 \implies -k = 0,56\]
	\hrule{}
	\item $P(X>k) = 0,1492$ [1,04]
	\[ P(X>k) = 0,1492 \dimplies P(X<k) = 1-0,1492 = 0,8508 \implies k=1.04\]
	Comprobación \[P(X>1.04) = 1-P(X<1.04) = 1- 0.8508 = 0,1492\]
	\item $P(X>k) = 0,0078$ [2,42]
	\[ P(X>k) = 0,0078 \dimplies P(X<k) = 1-0,0078 = 0,9922 \implies k=2.42\]
	Comprobación \[P(X>2.42) = 1-P(X<2.42) = 1- 0.9922 = 0,0078\]
	\subitem $P(X>2k+1) = 0,1736$ [0,93]
	\[ P(X>2k+1) = 0,1736 \dimplies P(X<2k+1) = 1-0,1736 = 0,8264 \implies 2k+1=0.94 \implies k=-0,03\]
	\subitem $P\left(X<\frac{k-2}{3}\right) = 0,2877$ [0,56]
	\[ P\left(X<\frac{k-2}{3}\right) = 0,2877 \dimplies PP\left(X>\frac{k-2}{3}\right) = P\left(X<-\frac{k-2}{3}\right) = 0,7123\]
	\[ \implies -\frac{k-2}{3} = 0,56 \implies k=0,56·3+2 = k=3,68\]

	\item Ejercicios 87 y 85

	Deberes: ejercicio 75

\end{itemize}

\hl{Clase del lunes 11: } Corregimos el 75, el 85 y hacemos más de calcular media y desviación típica.

¿Qué pasa si el valor buscado no coincide en la tabla? Que cogemos el más cercano.

Después, problemas de enunciado natural.

Hallar media y desviación típica desde unas pocas probabilidades.

\hl{Clase martes 12/02}
Corregir 85 cd. Hacerlos bien [Del solucionario si hace falta].
Resultados del 75 ce, que cometimos el mismo error.


\subsection{Aproximación de la binomial a la normal}

Si $X\equiv B(n,p)$, se puede aproximar por $X\equiv N(\mu=n·p, \sigma=\sqrt{n·p·(1-p)}$. ¿Cuándo se puede hacer esto? Cuando $np>5$ y $n(1-p)>5$

Problema 19.

Problema 91. Antes de nada, ¿es una binomial? 
\begin{itemize}
	\item Independencia.
	\item Cuento número de éxitos.
	\item Tengo una probabilidad y un número de intentos.
\end{itemize}

Deberes: Problema 92. También se puede hacer.

Comentamos problema 94. Es una binomial, pero si quieres aproximar con la normal, es perfectamente válido, si lo argumentas.

\subsubsection{Cálculo de media y desviación típica}

\begin{itemize}
	\item $X\equiv N(\mu,2); P(X<5) = 0,9332$. $[\mu = 2]$
	\[
	P\left(X<5\right) = P\left(\frac{X-\mu}{\sigma}<\frac{5-\mu}{\sigma}\right) = P\left(Z<\frac{5-\mu}{2}\right) = 0,9332
	\]
	Buscamos en la tabla 0,9332 y tenemos que:
	$\frac{5-\mu}{2} = 1,5 \implies \mu=5-3=2$
	\item $X\equiv N(3,\sigma); P(X<10) = 0,6368$. $[\sigma = 20]$
	\[
		P\left(X<10\right) = P\left(\frac{X-3}{\sigma}>\frac{10-3}{\sigma}\right) =0,6368 \implies \frac{10-3}{\sigma}=0,35 \implies 
	\]
	\item $X\equiv N(\mu,\sigma); P(X<12) = 0,5; P(X>6) = 0,8413$. $[\mu = 12, \sigma = 6]$
	\[
		P(X<12) = 0,5 \implies \mu = 12
	\]
	\[
		P\left(X>6\right) = \left(\frac{X-12}{\sigma}>\frac{6-12}{\sigma}\right) = \left(Z<-\frac{-1}{2\sigma}\right) = 0,8413
	\]
	Consultando en la tabla: $-\frac{-1}{2\sigma} = 0,8413 \implies \sigma = \frac{1}{2·0,8413} = 1,79$
\end{itemize}


\section{Estadística bidimensional}

Preparar PPT con los ejemplos del libro:

Clase 1) Tabla de datos bidimensionales. Construir una, calcular distribuciones marginales y condicionadas para interiorizar bien las tablas, lo que significan, etc.

Algún ejercicio de marginales + diagramas de dispersión. Covarianza vs correlación

Covarianza: cuánto varía cada dato respecto de la media. $\sigma_{xy}$. No hace falta que te sepas las fórmula porque vamos a utilizar la calculadora a mansalva. 

Esperadme en clase mañana sentados por grupos de calculadoras iguales.

Rectas de regresión e intuición gráfica.

Cálculo de todo con calculadora.

\hl{Clase 27/02}

Recta de regresión de Y sobre X (PPT)

Tablita resumen del coeficiente de correlación (PPT)

La recta de regresión siempre pasa por el punto $(\gor{x},\gor{y})$

¿Qué recta calcular? ¿Y sobre X o X sobre Y? Si quiero predecir Y, hago Y sobre X. Si quiero predecir X, hacemos X sobre Y.

Recomendación: https://www.vitutor.com/estadistica/bi/ejercicios\_correlacion.html

\subsubsection{Causalidad vs Correlación}

\begin{itemize}
	\item https://www.telecinco.es/informativos/ciencia/beneficios-chupar-chupete-bebe_0_2661750040.html
	\item https://www.infosalus.com/nutricion/noticia-bebidas-endulzadas-artificialmente-reducen-riesgo-recurrencia-cancer-colon-muerte-20180723074434.html
	\item https://www.infobae.com/america/tendencias-america/2018/10/23/un-estudio-afirma-que-la-comida-organica-reduce-un-25-el-riesgo-de-cancer/
\end{itemize}

Ejercicios para clase: 64 b,d,e ; 65
Ejercicios para practicar: 62, 63




\chapter{Álgebra (Matrices y vectores)}

\section{Matrices}

\paragraph{Definición de matriz}

\paragraph{Operaciones con matrices}

\subparagraph{Traspuesta}

Ejercicio: \textbf{Demuestra que cualquier matriz puede escribirse como suma de una matriz simétrica y otra antisimétrica}

\subparagraph{Producto de matrices}

\subsection{Matriz inversa}
Se puede calcular de 3 formas. Definición, Gauss-Jordan y matriz adjunta. Vamos a ver ahora los 2 primeros métodos.

\paragraph{Definición y propiedades}

\subsection{Algunas ecuaciones matriciales sencillas}

\paragraph{Gauss-Jordan}
La base del método de Gauss es que toda transformación lineal de Gauss se puede expresar como una matriz. Simplemente buscamos la matriz que transforma la matriz dada en la identidad. Para ello, ponemos la identidad a la derecha. (Espero que leyendo esta explicación te hayas enterado)


\subsection{Utilidades: Grafos}

\section{Determinantes}

\begin{defn}[Determinante]
$\appl{|\;\;|}{\mathcal{M}_n}{\real}$
\end{defn}

\subsection{Cálculo de determinantes de orden 3}

\subsection{Propiedades}

\subsection{Cálculo de determinantes de orden 4 o más}

\paragraph{Menores, Gauss}

\subsection{Matriz inversa por determinantes}

\subsection{Ecuaciones matriciales a tope}

\subsection{Rango}
\paragraph{Gauss}

\paragraph{Determinantes}

Si $|A| \neq 0$, significa que no hay 2 filas (ni 2 columnas) linealmente independientes. Si las hubiera, $|A| = 0$.

Por lo tanto, si $|A|\neq 0 \dimplies rg(A) = \text{ máx}$

¿Qué ocurre si $A\not\in\mathcal{M}_n$?

\begin{example}
\[
    A=\begin{pmatrix}2&2&3&4\\4&4&2&1\end{pmatrix}
\]

En este ejemplo, cogiendo $\left|\begin{matrix}2&2\\4&4\end{matrix}\right| = 0$, pero $\left|\begin{matrix}3&4\\2&1\end{matrix}\right| \neq 0$, por lo que estas 2 filas tienen que ser linealmente independientes, por lo que la matriz tiene rango 2.

También valdría argumentarlo desde $\left|\begin{matrix}2&3\\4&2\end{matrix}\right| \neq 0$
\end{example}

\begin{prop}[Cálculo del rango por menores]
Sea $M_p$ un menor de orden $p$ de la matriz $A\in\mathcal{M}_{n\times m}$
\[\exists M^p \tlq M_p \neq 0 \dimplies rg(A) \geq p\]
\[\forall M^p \;\; M_p = 0 \dimplies rg(A) < p\]
\end{prop}

\subsubsection{Matriz de Vandermonde}
\[
V=\begin{bmatrix}
1 & \alpha_1 & \alpha_1^2 & \dots & \alpha_1^{n-1}\\
1 & \alpha_2 & \alpha_2^2 & \dots & \alpha_2^{n-1}\\
1 & \alpha_3 & \alpha_3^2 & \dots & \alpha_3^{n-1}\\
\vdots & \vdots & \vdots & \ddots &\vdots \\
1 & \alpha_n & \alpha_n^2 & \dots & \alpha_n^{n-1}\\
\end{bmatrix}\]

\paragraph{Determinante: } El determinante se calcula con la siguiente fórmula:

\[\begin{vmatrix} V \end{vmatrix}=\prod_{1 \le i<j\le n}(\alpha_j-\alpha_i)\]

\begin{example}
\[
\begin{vmatrix}
1&2&4&8\\
1&3&9&27\\
1&4&16&64\\
1&5&25&125
\end{vmatrix} = \underbrace{\overbrace{(3-2)}^{j=2}\overbrace{(4-2)}^{j=3}\overbrace{(5-2)}^{j=4}}_{i=1}\underbrace{\overbrace{(4-3)}^{j=3}\overbrace{(5-3)}^{j=4}}_{i=2}\underbrace{\overbrace{(5-4)}^{j=4}}_{i=3} = 1·2·3·1·2·1 = 6
\]
\end{example}

\begin{proof}[por Inducción]

\paragraph{Base: n=2} Es fácil notar que en el caso de una matriz de 2×2 el resultado es correcto.
\[\begin{vmatrix} V \end{vmatrix}=v_{1,1}v_{2,2} - v_{1,2}v_{2,1}=\alpha_2-\alpha_1=\prod_{1\le i<j\le 2} (\alpha_j-\alpha_i)\]

\paragraph{Paso}
Suponiendo cierta la fórmula para el caso $n-1$, procedemos a calcular el determinante de orden $n$. Para ello, basta con realizar la siguiente operación elemental sobre cada columna: $C_{j}\rightarrow C_{j}-(\alpha_1 \times C_{j-1})$. Esta operación no afecta al determinante, por lo que se obtiene lo siguiente:

\[
\begin{vmatrix} V \end{vmatrix}=\begin{vmatrix}
1 & \alpha_1 & \alpha_1^2 & \dots & \alpha_1^{n-1}\\
1 & \alpha_2 & \alpha_2^2 & \dots & \alpha_2^{n-1}\\
1 & \alpha_3 & \alpha_3^2 & \dots & \alpha_3^{n-1}\\
\vdots & \vdots & \vdots & \ddots &\vdots \\
1 & \alpha_n & \alpha_n^2 & \dots & \alpha_n^{n-1}\\
\end{vmatrix}=\begin{vmatrix}
1 & 0 & 0 & \dots & 0\\
1 & \alpha_2-\alpha_1 & \alpha_2(\alpha_2-\alpha_1) & \dots & \alpha_2^{n-2}(\alpha_2-\alpha_1)\\
1 & \alpha_3-\alpha_1 & \alpha_3(\alpha_3-\alpha_1) & \dots & \alpha_3^{n-2}(\alpha_3-\alpha_1)\\
\vdots & \vdots & \vdots & \ddots &\vdots \\
1 & \alpha_n-\alpha_1 & \alpha_n(\alpha_n-\alpha_1) & \dots & \alpha_n^{n-2}(\alpha_n-\alpha_1)\\
\end{vmatrix}
\]

Desarrollando por los adjuntos de la primera fila: 

\[\begin{vmatrix} V \end{vmatrix}=\begin{vmatrix}
\alpha_2-\alpha_1 & \alpha_2(\alpha_2-\alpha_1) & \dots & \alpha_2^{n-2}(\alpha_2-\alpha_1)\\
\alpha_3-\alpha_1 & \alpha_3(\alpha_3-\alpha_1) & \dots & \alpha_3^{n-2}(\alpha_3-\alpha_1)\\
\vdots & \vdots & &\vdots \\
\alpha_n-\alpha_1 & \alpha_n(\alpha_n-\alpha_1) & \dots & \alpha_n^{n-2}(\alpha_n-\alpha_1)\\
\end{vmatrix}\]
Extrayendo de cada fila un factor, obtenemos:
\[\begin{vmatrix} V \end{vmatrix}=
(\alpha_2-\alpha_1)(\alpha_3-\alpha_1)\dots(\alpha_n-\alpha_1)
\underbrace{\begin{vmatrix}
1 & \alpha_2 & \alpha_2^2 & \dots & \alpha_2^{n-2}\\
1 & \alpha_3 & \alpha_3^2 & \dots & \alpha_3^{n-2}\\
1 & \alpha_4 & \alpha_4^2 & \dots & \alpha_4^{n-2}\\
\vdots & \vdots & \vdots & &\vdots \\
1 & \alpha_n & \alpha_n^2 & \dots & \alpha_n^{n-2}\\
\end{vmatrix}}_{(1)}\]

(1): es una matriz de Vandermonde de orden $n-1$, por lo que podemos aplicar la fórmula por la hipótesis de inducción, quedando así demostrada la fómrula del determinante de Vandermonde para orden $n$
\end{proof}

\section{Sistemas de ecuaciones}

Sistemas, expresión matricial de sistemas. 
 
Rouché-Frobenius, corregimos. 

Resolución de sistemas escalonados y método de Gauss Jordan.
 
"Repaso" de Sistema Compatible Indeterminado. 2 sistemas resueltos por mi. El primero con ecuaciones. El segundo con matricial.


\begin{problem}

Discute y resuelve el siguiente sistema:

\[
\left\{\begin{array}{lcccl}
x&+2y&-2z&=&4\\
2x&+5y&-2z&=&10\\
4x&+9y&-6z&=&18
\end{array}\right\}
\]

\solution

\[
\left\{\begin{array}{lcccl}
x&+2y&-2z&=&4\\
2x&+5y&-2z&=&10\\
4x&+9y&-6z&=&18
\end{array}\right\}
\overset{(1)}{\dimplies}
\left\{\begin{array}{lcccl}
x&+2y&-2z&=&4\\
 &y&+2z&=&2 \\
4x&+9y&-6z&=&18
\end{array}\right\}
\overset{(2)}{\dimplies}\]
\[
\left\{\begin{array}{lcccl}
x&+2y&-2z&=&4\\
 &y&+2z&=&2 \\
 &y&+2z&=&2 \\
\end{array}\right\}
\dimplies
\underbrace{\left\{\begin{array}{lcccl}
x&+2y&-2z&=&4\\
 &y&+2z&=&2 
\end{array}\right\}}_{\text{Discusión: C.I (*)}}
\]

(*): Es un sistema compatible indeterminado porque es un sistema escalonado con más incógnitas que ecuaciones.

Al ser compatible indeterminado, el sistema tiene infinitas soluciones (que no se calculan en 1º de Bachillerato).


\paragraph{Resolución:} Aunque un sistema de ecuaciones Compatible Indeterminado tiene infinitas soluciones, no cualquier trío de números es solución. 
%
Por ejemplo, en este caso, la terna $(x,y,z) = (0,0,0)$ no es solución.
%
\textbf{Infinitas soluciones no significa que todo sea solución}.

La pregunta lógica sería, ¿cómo podemos escribir \textbf{todas} las soluciones del sistema? Utilizando un parámetro.
%
Al dar un valor a una incógnita, ya forzamos los otros 2 valores. 
%
Para cada valor inventado de $x$, solo hay un único valor posible de $y$ y de $z$ (normalmente).

En este caso, vamos a dar un valor concreto a $y$, pero en forma de parámetro.
%
Tomamos $y=λ$ y sustituimos en $E_2$.

\[y+2z=2 \dimplies λ+2z=2 \dimplies z=\frac{2-λ}{2}\]

Sustituimos $y=λ,z=\frac{2-λ}{2}$ en $E_1$:

\[x+2y-2z = 4 \dimplies x= 4+2z-2y = 4+2\left(\frac{2-λ}{2}\right)-2λ = 4+2-λ-2λ = 6-3λ = 3(2-λ)\]

\textbf{Solución:} $(x,y,z) = \left(3(2-λ),λ,\frac{2-λ}{2}\right)$

\paragraph{1)} $E_2=E_2-2E_1$

\[
\left\{\begin{array}{lcccl}
2x&+4y&-4z&=&8\\
2x&+5y&-2z&=&10\\
\hline
&-y&-2z&=&-2 
\end{array}\right\}
\]

\paragraph{2)} $E_3=E_2-4E_1$

\[
\left\{\begin{array}{lcccl}
4x&+9y&-6z&=&18\\
4x&+10y&-4z&=&20\\
\hline
&-y&-2z&=&-2 
\end{array}\right\}
\]


\paragraph*{Comprobación:} Sustituimos $(x,y,z) = \left(3(2-λ),λ,\frac{2-λ}{2}\right)$ en el sistema inicial:


\[
\left\{\begin{array}{lcccll}
x&+2y&-2z&=&4 &\to 6-3λ + 2λ - 2\displaystyle\left(\frac{2-λ}{2}\right) = 6-λ-2+λ = 4\\
2x&+5y&-2z&=&10 &\to 12-6λ +5λ - 2\displaystyle\left(\frac{2-λ}{2}\right) = 12-λ-2+λ = 10\\
4x&+9y&-6z&=&18 &\to 24-12λ + 9λ - 6\displaystyle\left(\frac{2-λ}{2}\right) = 24-3λ-6+3λ = 18
\end{array}\right\}\begin{array}{c}\\\\\\\\\text{cqc}\end{array}
\]

\end{problem}

\begin{problem}

Discute y resuelve el siguiente sistema:

\[
\left\{\begin{array}{rcccl}
3x&-y&+z&=&3\\
6x&-2y&+2z&=&6\\
-3x&+y&-z&=&-3
\end{array}\right\}
\]

\solution


\[
\left\{\begin{array}{rcccl}
3x&-y&+z&=&3\\
6x&-2y&+2z&=&6\\
-3x&+y&-z&=&-3
\end{array}\right\} \implies
\left(\begin{array}{ccc|c}
3&-1&1&3\\
6&-2&2&6\\
-3&1&-1&-3
\end{array}\right)
\dimplies\]
\[
\text{\hl{Ojo con el cambio de columnas}}
\left(\begin{array}{ccc|c}
1&-1&3&3\\
2&-2&6&6\\
-1&1&-3&-3
\end{array}\right)
\dimplies
\left(\begin{array}{ccc|c}
1&-1&3&3\\
0&0&0&0\\
0&0&0&0
\end{array}\right)
\]

Tiene grado de indeterminación 2, por lo que necesitaremos 2 parámetros.

Llamamos $x=\lambda$ e $y = \mu$ con $\mu,\lambda\in\real$ y sustituimos para hallar $z$.

$$z-y+3x=3 \implies z - \mu + 3\lambda = 3 \dimplies z = 3+\mu-3\lambda$$

Solución: $(x,y,z) = \left(\lambda, \mu, 3+\mu - 3\lambda\right), \forall\lambda,\mu\in\real$

\end{problem}

Ejercicio 59 de deberes.

\textbf{Resolución por inversa de stma}. ¿Funciona siempre? Sólo en sistemas de Cramer, es decir, matriz de coeficientes cuadrada con rango máximo. 

Deberes el 15b,16b.

\subsection{Regla de Cramer}

En todos los sistemas cuya matriz de coeficientes tenga inversa, puede generalizarse el método de la inversa.

Así, $Ax = B \dimplies x = A^{-1}·B$

\[
    A^{-1} = \frac{1}{|A|} · \left( \text{Adj}(A) \right)^T = \frac{1}{|A|} · \begin{pmatrix} 
    A_{11} & A_{21} & A_{31} & ... & A_{n1}\\
    A_{21} & A_{22} & A_{32} & ... & A_{n2}\\
    \vdots &        &       & \ddots & \vdots\\
    A_{1n} & A_{2n} & A_{3n} & ... & A_{nn}\end{pmatrix}
\]

Por lo tanto,
\[
    A^{-1}·B = \frac{1}{|A|} · 
    \begin{pmatrix} 
        A_{11} & A_{21} & A_{31} & ... & A_{n1}\\
        A_{12} & A_{22} & A_{32} & ... & A_{n2}\\
        \vdots &        &       & \ddots & \vdots\\
        A_{1n} & A_{2n} & A_{3n} & ... & A_{nn}\end{pmatrix}·
    \begin{pmatrix}
        b_1\\b_2\\\vdots\\ b_n
    \end{pmatrix}
    = 
    \begin{pmatrix}
    A_{11}b_1 + A_{21}b_2 + A_{31}b_3 \dots A_{n1}·b_n\\
    A_{12}b_1 + A_{22}b_2 + A_{32}b_3 \dots A_{n2}·b_n\\
    \vdots\\
    A_{1n}b_1 + A_{2n}b_2 + A_{3n}b_3 \dots A_{nn}·b_n\\
    \end{pmatrix}
\]

Deberes : 21b, 22b

Corregimos Cramer. 

Inconvenientes: ¿y si es incompatible?

Numéricos: 61a,b;62a,b

Parámetros:64a,c (ojo con eliminar una solución)



Deberes para el punete: 
57,60

\chapter{Geometría analítica}

% \paragraph{Espacio vectorial: $\real^3$}

% \begin{defn}[Espacio vectorial]
% Un espacio vectorial sobre $K$ es una estructura algebraica, $(V,+,·)$, donde $V$ es un conjunto cualquiera y $\appl{+}{V\times V}{V}$ y $\appl{·}{K\times V}{V}$


% Las operaciones $+$ y $·$ deben cumplir las siguientes propiedades:
% \begin{itemize}
%     \item $\vec{u} + (\vec{v} + \vec{w}) = (\vec{u} + \vec{v}) + \vec{w}, \qquad \forall \vec{u}, \vec{v}, \vec{w} \in V $  (asociativa)
%     \item $\vec{u} + \vec{v} = \vec{v} + \vec{u}, \qquad \forall \vec{u}, \vec{v} \in V$ (conmutativa)
% \item $ \exists{}\vec{e} \in{} V : $  $ \vec{u} + \vec{e} = \vec{u} , \forall{} \vec{u} \in{} V
% $ (elemento neutro)

% \item $
%    \forall{} \vec{u} \in{} V , \quad
%    \exists{} \vec{-u} \in{} V : $  $
%     \vec{u} + (\vec{-u}) = \vec{e}
% $ (elemento opuesto)

% \item $
%    \mathit{a} \cdot (\mathit{b} \cdot \vec{u})=(\mathit{a} \cdot \mathit{b}) \cdot \vec{u} ,$  $
%    \forall{} \mathit{a} ,\mathit{b} \in{}K , $  $
%    \forall{} \vec{u} \in{} V
% $

% \item$
%    \exists{e} \in{K}: $ 
%    e \cdot \vec{u}   = \vec{u} , 
%    \forall{} \vec{u} \in{} V
% $ (elemento neutro del producto)
% \item $
%    \mathit{a} \cdot (\vec{u}+ \vec{v}) =
%    \mathit{a} \cdot \vec{u}+ \mathit{a} \cdot \vec{v} , $  $
%    \forall{} \mathit{a}\in{}K , $  $
%    \forall{} \vec{u}, \vec{v} \in{} V
% $ (propiedad distributiva)
% \item $
%    (\mathit{a} + \mathit{b}) \cdot \vec{u} =
%    \mathit{a} \cdot \vec{u} + \mathit{b} \cdot \vec{u} , $  $
%    \forall{} \mathit{a}, \mathit{b} \in{} K , $  $
%    \forall{} \vec{u} \in{} V
% $ (propiedad distributiva)
% \end{itemize}
% \end{defn}

%\section{Introducción}

\begin{defn}[Espacio vectorial]
Un espacio vectorial es una estructura algebraica, $(\mathcal{V},+,·)$, donde $\mathcal{V}$ es un conjunto cualquiera y $\appl{+}{\mathcal{V}\times \mathcal{V}}{\mathcal{V}}$ y $\appl{·}{\real\times \mathcal{V}}{\mathcal{V}}$
\end{defn}

En nuestro caso, el espacio vectorial con el que trabajaremos será $\mathcal{V}^3 = (\real^3, + , ·)$ sobre $\real$, siendo la operación $+$ la suma de vectores habitual y $·$ el producto por un escalar real.

Los elementos de $\mathcal{V}^3$ se denominan vectores y son ternas de números (reales) con los que se pueden hacer operaciones. 
%
Escribiremos $\vec{u} = (u_1,u_2,u_3)$.

Las operaciones $+$ y $·$ son las habituales. 
%
Recordamos:

\begin{example}
Sean $u_1 = (1,2,3)$ y $u_2 = (0,1,-2)$, tenemos:
  \begin{itemize}
      \item $u_1 + u_2 = \hide{(1+0, 2+1, 3-2)}$
      \item $-u_2 = (0,-1,2)$
      \item $u_1-u_2 = u_1+ (-u_2) = (1,2,3) + (0,-1,2) = (1,1,1)$
      \item $2·u_1 + 3·u_2 = \hide{(2,4,6) + (0,3,-6) = (2,7,0)}$
      \obs \hide{A esta operación la llamamos \concept[Combinación lineal\IS de vectores]{combinación lineal de vectores}}.
  \end{itemize}
\end{example}

\obs Las matrices también forman un espacio vectorial. Ver libro página 179.

\obs Todavía no hemos definido nada de producto de vectores. \textit{Explicación de porqué el orden seguido es diferente}

\paragraph{Bases de espacios vectoriales y coordenadas}

\begin{defn}[Subespacio generado] 
Sean $ G = \{v_1,v_2,...,v_n\}$ un conjunto de vectores de un espacio vectorial.

Llamamos subespacio generado al conjunto de todas las combinaciones lineales de vectores de $G$.

\obs Llamaremos a $G$ \concept[Espacio vectorial\IS Sistema de generadores]{Sistema de generadores}.
\end{defn}

\begin{example}
\begin{itemize}
  \item El subespacio generado por un vector sería una recta.
  \item El subespacio generado por 2 vectores sería un plano.
  \item El subespacio generado por 3 vectores sería el espacio.
\end{itemize}
\end{example}

Un concepto fundamental a la hora de trabajar con vectores es la "base del espacio vectorial". \hide{(Libro página 257)}

\begin{defn}[Base de un espacio vectorial][Espacio vectorial\IS Base]
Un conjunto de vectores $\mathcal{B} = \{b_1,b_2,b_3\}$ es una base de $\mathcal{V}^3$ si:
  \begin{itemize}
      \item Son linealmente independientes.
      \item El subespacio que generan es $\mathcal{V}^3$. (Es decir, que cualquier vector de $\mathcal{V}^3$ puede escribirse como combinación lineal de los vectores de la base).
  \end{itemize}
\end{defn}


\begin{defn}[Dimensión de un espacio vectorial][Espacio vectorial\IS Dimensión]
Sea $\mathcal{V}$ un espacio vectorial y $\mathcal{B}$ una base del mismo.

Llamamos \textbf{Dimensión} del espacio vectorial al número de vectores de $\mathcal{B}$ (que son linealmente independientes por definición de \textit{base de un espacio vectorial})
\end{defn}

En este curso, para saber si un conjunto de vectores es base de un espacio vectorial, basta comprobar que son linealmente independientes y que hay tantos vectores linealmente independientes como dimensión tiene el espacio vectorial.


\begin{defn}[Coordenadas de un vector]
Sea $\mathcal{B} = \{\vec{b}_1,\vec{b}_2,\vec{b}_3\}$ es una base de $\mathcal{V}$.

Si $\vec{u} = c_1 · \vec{b_1} +  c_2·\vec{b_2} + c_3\vec{b_3}$, decimos que $(c_1,c_2,c_3)$ son las coordenadas del vector $\vec{u}$ en la base $\mathcal{B}$.
\end{defn}

\obs "Sea el vector $\vec{u} = (1,2,3)$" deja de tener sentido, ya que necesitamos estar refiriéndonos a una base. 
%
Por ello, a partir de ahora, intentaremos escribir los vectores $\vec{u} = \vec{i} + 2\vec{j} + 3\vec{k}$, dejando bien claro en qué base estamos trabajando.

\begin{problem}

  Determina si los siguientes conjuntos son bases de $\mathcal{V}^3$ (que tiene dimensión: \hide{3})
    
\ppart $\mathcal{B}_1 = \{(1,0,0), (0,0,1)\}$
\ppart $\mathcal{B}_2 = \{(1,0,0), (0,1,0),(0,0,1)\}$
\ppart $\mathcal{B}_3 = \{(1,0,0), (0,1,0),(0,0,1),(1,1,1)\}$
\ppart $\mathcal{B}_4 = \{(1,0,3),(1,2,-1),(0,1,2)\}$
\ppart $\mathcal{B}_5 = \{(2,-1,5),(1,-2,-4),(4,-5,-3)\}$
    \solution

        
        \spart $\mathcal{B}_1 = \{(1,0,0), (0,0,1)\}$
        \subitem \hide{No es base porque hay vectores de $\mathcal{V}^3$ que no se pueden expresar como combinación lineal de los vectores de $\mathcal{B}_1$, por ejemplo, $(0,1,0)$}
        
        \spart $\mathcal{B}_2 = \{(1,0,0), (0,1,0),(0,0,1)\}$
        \subitem Sí es una base porque $\mathcal{B}_2$ tiene 3 vectores linealmente independientes: $Rg\displaystyle\begin{pmatrix}1&0&0\\0&1&0\\0&0&1\end{pmatrix} = 3$.
        
        \spart $\mathcal{B}_3 = \{(1,0,0), (0,1,0),(0,0,1),(1,1,1)\}$
        \subitem No es una base porque el vector $(1,1,1)$ se puede expresar como combinación lineal de los 3 primeros. 
        
        Así, aunque $\displaystyle Rg\begin{pmatrix}1&0&0\\0&1&0\\0&0&1\\1&1&1\end{pmatrix} = 3$, no podríamos decir que $\mathcal{B}_3$ fuera una base. Solo podríamos decir que \hide{es un \textbf{Sistema de generadores de $\mathcal{V}^3$}}
        
        
        \spart $\mathcal{B}_4 = \{(1,0,3),(1,2,-1),(0,1,2)\}$
        \subitem Estudiamos $\displaystyle \begin{pmatrix}\vec{u_1}\\\vec{u_2}\\\vec{u_3}\end{pmatrix} = \begin{pmatrix}1&0&3\\1&2&-1\\0&1&2\end{pmatrix}$ 
        Buscamos si tiene rango máximo. Para ello, $\begin{vmatrix}1&0&3\\1&2&-1\\0&1&2\end{vmatrix} \neq 0 \implies $ por lo que podemos decir que son linealmente independientes\footnote{Si no lo fueran, el determinante sería 0}. Así, tenemos 3 vectores linealmente independientes, por lo que podemos decir que \textbf{sí son una base de $\mathcal{V}^3$.}
        
        
        \spart $\mathcal{B}_5 = \{(2,-1,5),(1,-2,-4),(4,-5,-3)\}$
        \subitem \hide{Estudiamos $\displaystyle \begin{pmatrix}\vec{u_1}\\\vec{u_2}\\\vec{u_3}\end{pmatrix} = \begin{pmatrix}2&-1&5\\1&-2&-4\\4&-5&-3\end{pmatrix}$ Su determinante es 0, por lo que no tiene rango 3, por lo que no son linealmente independientes.}
    
\end{problem}

\paragraph{Cambios de base}

\begin{problem}

Sea 
$\mathcal{B}_1 = \{ u_1=(1,1,1), u_2(0,1,0), u_3=(0,0,1)\}$
y
$\mathcal{B}_2 = \{ w_1=(1,2,3), w_2=(1,1,0), w_3=(3,-1,1)\}$

\ppart Si $\vec{z} = (2,-2,1)$ son las coordenadas en la base $\mathcal{B}_1$, halla las coordenadas de $\vec{z}$ en la base canónica.


\ppart Si $\vec{q} = (2,-2,1)$ son las coordenadas en la base $\mathcal{B}_2$, halla las coordenadas de $\vec{q}$ en la base canónica.


\ppart Halla las coordenadas de $\vec{z}$ en $\mathcal{B}_2$

\obs El vector $\vec{u_1}$ de la base $\mathcal{B}_1$ son las coordenadas respecto de una base concreta. 
%
Si no se dice nada, suponemos la canónica.

\solution

\spart
\[\vec{z}_1 = 2\vec{u_1} -2\vec{u_2} + \vec{u_3} = 2·(\vec{i} +\vec{j} + \vec{k}) - 2·(\vec{j} + \vec{k}) = (2\vec{i}+3\vec{k})\]

Este $\vec{z}_c = (2,0,3)$ son las coordenadas de $\vec{u}$ en la base canónica.


\spart 

\[
\vec{q}_1 = 2\vec{w_1} -2\vec{w_2} + \vec{w_3} = 
2·(\vec{i} +2\vec{j} + 3\vec{k}) - 2·(\vec{i}+\vec{j}) + 3(\vec{i} -\vec{j} + \vec{k}) = 
3\vec{i} -\vec{j} + 9\vec{k}
\]

Este $\vec{q} = (2,0,3)$ son las coordenadas de $\vec{u}$ en la base canónica.

\spart
Buscamos 
$\vec{z} = (2\vec{u_1}-\vec{u_2}+\vec{u_3}) = c_1(1,2,3) + c_2 (1,1,0) + c_3 ( 3,-1,1)$.

\[
2·(1,1,1) - (0,1,0) + (0,0,1) = c_1(1,2,3) + c_2 (1,1,0) + c_3 ( 3,-1,1)\dimplies \left\{
  \begin{array}{c}
    2 = c_1 + c_2 + 3c_3\\
    1 = 2c_1 + c_2 -c_3\\
    3 = 3c_1  \quad\quad +c_3 
  \end{array}
  \right\}
\]

Resolvemos el sistema de 3 ecuaciones con 3 incógnitas, obteniendo: $(x,y,z) = \left(\rfrac{6}{13},\rfrac{11}{13},\rfrac{-3}{13}\right)$

Estas son las coordenadas de $\vec{z}$ en la base $\mathcal{B}_2$

\end{problem}

Hecho 1, hechos todos. Problemas recomendados, especialmente 267.48,49 + 269.60,61: 
\begin{itemize}
  \item Página 267.48,49
  \item Página 269.56-65
  \item Página 271.102,103
\end{itemize}

\section{Geometría afín (salto al tema 11)}

Dejamos el mundo abstracto de las estructuras algebraicas para empezar con la geometría. 

El planeta tierra existe antes de que se inventen las coordenadas GPS. 


Lo que vamos a hacer es determinar las "coordenadas GPS" de todos los puntos del espacio tridimensional.
%
Para ello, necesitaremos tener un sistema de referencia consensuado.
%
Así, fijando un sistema de referencia, podemos empezar a trabajar con elementos que se encuentren en algún lugar de algún espacio. 



\begin{defn}[Sistema de referencia] (Página 277)
Sea $V$ un espacio vectorial con base $\mathcal{B}$ y sea $O$ un punto, que llamaremos \textbf{origen de coordenadas}

Llamamos Sistema de referencia afín al conjunto $S = \{\mathcal{B},O\}$
\end{defn}

\begin{example}
En el sistema de referencia $S=\{O = (1,2,3); \mathcal{B} = \{\vec{u_1} = (1,1,1), \vec{u_2} = (2,1,0), \vec{u_3} = (0,0,1)\}\}$

En este sistema de referencia, $\vec{a} = (2,7,3)$ significa:

$\vec{a} =  (1,2,3) + 2\cdot u_1 + 7\cdot u_2 + 3\cdot u_3$ si quisiéramos escribirlo en función del sistema de referencia habitual.

\label{example::origen_ref}

Si se va a trabajar y a hacer operaciones siempre con el mismo sistema de referencia, con origen $(1,2,3)$, trabajaremos con $\vec{a} = 2\cdot u_1 + 7\cdot u_2 + 3\cdot u_3$.
%
Al final, uno puede prácticamente ignorar el origen, trabajar como si fuera otro y despues sumar o restar para corregir el punto de origen.

\end{example}

\obs En el ejemplo anterior, se ha realizado la operación: $\vec{a} =  (1,2,3) + 2·u_1 + 7·u_2 + 3·u_3$. \ul{¿Cómo se suman puntos y vectores?} 
%
No es posible sumar puntos con vectores (como no es posible sumar peras con manzanas). La manera de hacer esta operaciónes considerar $(1,2,3)$ como un vector al que llamaremos vector de posición.


Un sistema de referencia permite definir \concept[Vector\IS de posición]{vectores de posición} para situar los puntos unívocamente en el espacio.
%
El vector de posición $\vec{OP}$ es el vector que une el punto $P$ con el punto $O$, origen de coordenadas. 
%
La \textbf{ventaja de los vectores de posición} es que permiten hacer operaciones, ya que \ul{\textbf{los puntos no se pueden sumar}}.



\begin{problem}

Dado el sistema de referencia $S_1=\{O = (0,0,0); \mathcal{B} = \{\vec{i},\vec{j},\vec{k}\}\}$

\ppart Si $\vec{a} = (13,1,-5)$ está expresado en función de $S_1$, halla sus coordenadas en $S_2=\{O = (0,1,-5); \mathcal{B} = \{\vec{i},\vec{j},\vec{k}\}\}$

\ppart Haz el mismo ejercicio, tomando $S_1=\{O = (1,2,3); \mathcal{B} = \{\vec{i},\vec{j},\vec{k}\}\}$

\solution
\hide{
\spart 
\[
  \vec{a} = (13,1,-5) = (0,1,-5) + \lambda_1\vec{i} + \lambda_2\vec{j} + \lambda_3\vec{k} \dimplies \left\{
    \begin{array}{c}
      13 = \lambda_1\\
      1 = 1 + \lambda_2\\
      -5 = -5 + \lambda_3
    \end{array}
  \right\} \]
\[\dimplies (\lambda_1,\lambda_2,\lambda_3) = (13,0,0)
\]

\spart 
En realidad, $\vec{a}$ sería el vector de posición del punto $(14,3,-2)$, teniendo en cuenta que $O(1,2,3)$, por lo que:
\[
  \vec{a} = (14,3,-2) = (0,1,-5) + \lambda_1\vec{i} + \lambda_2\vec{j} + \lambda_3\vec{k} \dimplies \left\{
    \begin{array}{c}
      14 = \lambda_1\\
      3 = 1 + \lambda_2\\
      -2 = -5 + \lambda_3
    \end{array}
  \right\} \]
\[\dimplies (\lambda_1,\lambda_2,\lambda_3) = (14,2,3)
\]
}


\end{problem}


\obs Ya tenemos 1 punto y un sistema de referencia para movernos entre puntos. 
%
Habitualmente trabajaremos con $S= \{O(0,0,0), \mathcal{B} = \{(1,0,0), (0,1,0), (0,0,1)\}\}$ como sistema de referencia de $\mathcal{V}^3$.



\paragraph{Vectores libres vs vectores fijos}

%$E_3$$ = (puntos, espacio vectorial, relación de equipolencia en la que los elementos del espacio vectorial son las clases de equivalencia de la relación de equipolencia de vectores fijos y libres)

Hasta ahora, hemos llamado vector a un elemento del espacio vectorial $V^3$. 
%
Estos elementos son \concept[Vector\IS libre]{vectores libres}.

Sin embargo, al fijar un sistema de referencia, podemos considerar los vectores con un origen y un final. 
%
¿Sabéis de dónde viene la palabra \textit{vector}? 
%
Etimológicamente significa, el que mueve algo de un sitio a otro. 
%
Así al menos son las interpretaciones físicas. 
%
Los vectores así interpretados tienen un origen y un destino.

\begin{figure}[hptb]
    \centering
    \includegraphics[width=0.9\textwidth]{img/Fijos-libres.png}
    \caption{Los diferentes vectores fijos de $V^2$ que tienen las mismas coordenadas forman parte del vector libre. 
    \newline
    De la misma manera que, la idea $\rfrac{1}{2}=\rfrac{2}{4}=\rfrac{3}{6}$ se escribiría, formalmente: $\left[\rfrac{1}{2}\right] = \left\{\rfrac{1}{2},\rfrac{2}{4},\rfrac{3}{6} ... \right\}$}
    \label{fig:plano}
\end{figure}


Denotaremos por $[\vec{AB}]$ al vector libre cuyas coordenadas son las del vector que va de $A$ a $B$.

Denotaremos por $\vec{AB}$ al vector fijo que va de $A$ a $B$.





\begin{defn}[Vector\IS definido por dos puntos]
(Página 278)

Dados dos puntos $A=(a_1,a_2,a_3)$ y $B = (b_1,b_2,b_3)$, tenemos 
    \begin{itemize}
        \item $\vec{AB} = \hide{(b_1-a_1, b_2-a_2, b_3-a_3)}$
        \item $\vec{BA} = \hide{(a_1-b_1, a_2-b_2, a_3-b_3)}$ 
    \end{itemize}
\end{defn}

\begin{proof}
Lo único que tenemos para trabajar son vectores de posición. 
%
En este caso son $\vec{a} = \vec{OA}$ y $\vec{b} = \vec{OB}$. 

Tenemos que $\vec{a} + \vec{AB} = \vec{b} \dimplies \vec{AB} = \vec{b}-\vec{a}$.
\end{proof}

\begin{example}
    Halla un vector definido por $A(1,0,3)$ y $B(2,1,2)$.
    
    \hide{\[\vec{AB} = (2-1, 1-0, 2-3) = (1,1,-1)\]}
\end{example}

\begin{problem}

Utilizando vectores de posición, demuestra que el punto medio entre los puntos $A$ y $B$, que llamamos $M_{AB}$, tiene por coordenadas:

\[M_{AB} = \left(\frac{a_1+b_1}{2},\frac{a_2+b_2}{2},\frac{a_3+b_3}{2}\right)\]

\ppart \textbf{Ampliación:} Dados los puntos $A(a_1,a_2,a_3)$ y $B(b_1,b_2,b_3)$, calcula las coordenadas del punto que divide el segmento $AB$ dejando $\rfrac{1}{3}$ de distancia con $A$ y $\rfrac{2}{3}$ de distancia con $B$. \textit{Pista: planteamiento parecido al punto medio.}

\solution

\[
\left\{
\begin{array}{c}
    \vec{AM} + \vec{MB} = \vec{AB}\\
    \vec{AM} = \vec{MB}\\
    \vec{OM} = ?
\end{array}\right\}
 \implies 
  2\vec{AM} = \vec{AB} \dimplies
  \]\[ 
  2\left(\vec{OM} - \vec{OA}\right) = \vec{AB} \dimplies
  2\vec{OM} = \vec{AB} + 2\vec{OA} = 
  \]\[
  2\vec{OM} = \left(
  b_1-a_1,
  b_2-a_2,
  b_3-a_3
  \right) + 
  \left(
  2a_1 - 0,
  2a_2 - 0,
  2a_3 - 0
  \right) = 
  \left(
  b_1+a_1,
  b_2+a_2,
  b_3+a_3
  \right)\]\[
  \vec{OM} = \left(
  \frac{b_1+a_1}{2},
  \frac{b_2+a_2}{2},
  \frac{b_3+a_3}{2}
  \right)
\]

\end{problem}

% \begin{problem}

% Dado el punto definido por el vector de posición $\vec{OP} = (1,1,1)$ en el sistema de referencia $\{(1,0,0), \mathcal{B} = \{u_1 = (1,0,1), u_2 = (0,2,1), u_3 = (0,0,4)\}\}$, expresa sus coordenadas en el sistema de referencia: 
% %
% $\{(1,1,0), \mathcal{B} = \{\vec{i},\vec{j},\vec{k}\}\}$

% \solution

% ¿Qué significa $\vec{OP} = (1,-2,3)$? Las coordenadas son los coeficientes de los vectores de la base, por lo tanto:

% \[
% P = (1,0,0) + 1·u_1 - 2·u_2+ 3·u_3  \dimplies (1,0,0) + 1·(1,0,1) - 2·(0,2,1)  + 3·(0,0,4) = (2,-4,11)
% \]

% Expresamos el vector $(2,-4,11)$ en el sistema de referencia pedido:

% $(2,-4,11) = (1,1,0) + \lambda_1\vec{i}+ \lambda_2\vec{j}+ \lambda_3\vec{k} \implies (\lambda_1,\lambda_2,\lambda_3) = (1,4,11)$

% \end{problem}

\begin{problem}

Dado el punto definido por el vector de posición $\vec{OP} = (1,1,1)$ en el sistema de referencia $S_1 = \{O_1(1,0,0), \mathcal{B} = \{u_1 = (1,0,1), u_2 = (0,2,1), u_3 = (0,0,4)\}\}$, expresa sus coordenadas en el sistema de referencia: 
%
$S_2=\{O_2(1,1,0), \mathcal{B} = \{\vec{i},\vec{j},\vec{k}\}\}$

\solution

En realidad $\vec{OP} = \vec{O_1P}$, por estar expresado en ese sistema de referencia. 

Entendemos que $O_1(1,0,0)$ no está expresado en función de la base de $S_1$.

¿Qué significa $\vec{O_1P} = (1,1,1)$? Las coordenadas son los coeficientes de los vectores de la base, por lo tanto, tomando $O(0,0,0)$ tenemos:

\[
\vec{OP} = \vec{OO_1} + \vec{O_1P} = (1,0,0) + 1·u_1 + 1·u_2+ 1·u_3  \dimplies\]
\[ (1,0,0) + 1·(1,0,1) +1 ·(0,2,1)  + 1·(0,0,4) = (2,2,6)
\]

Expresamos el vector $\vec{OP}=(2,2,6)$ en el sistema de referencia pedido:
\[\vec{OP} = (2,2,6) = (1,1,0) + \lambda_1\vec{i}+ \lambda_2\vec{j}+ \lambda_3\vec{k} \implies (\lambda_1,\lambda_2,\lambda_3) = (1,1,6) = \vec{O_2P}\]

\vspace{-0.3cm}
\paragraph{Plan b:} 
\[\vec{OP} = \vec{OO_2} + \vec{O_2P} \dimplies \vec{O_2P} = \vec{OP} - \vec{OO_2} = \underbrace{(2,2,6)}_{\ast} - \underbrace{(1,1,0)}_{\Delta} = (1,1,6)\]
\obs Para este planteamiento es \textbf{necesario} que los vectores que operamos (en este caso $\ast$ y $\Delta$) en coordenadas estén expresados en la \textbf{misma base}. 
%
En este caso, dicha base es la base canónica.

\vspace{-0.3cm}
\paragraph{Plan c: } No es necesario pasar por el origen $O(0,0,0)$, aunque pueda resultar más fácil de interpretar así. 

Podríamos haber empezado desde el principio haciendo un razonamiento parecido al plan b. Buscamos $\vec{O_2P}$ y sabemos $\vec{O_1P}$, por lo que podemos plantear:

\[
  \vec{O_1O_2} + \vec{O_2P} = \vec{O_1P} \dimplies \vec{O_2P} = \vec{O_1P} - \vec{O_1O_2}
\]

Como para poder hacer operaciones de vectores en coordenadas necesitamos la misma base, tenemos:

\[
  \vec{O_2P} = \vec{O_1P} - \vec{O_1O_2} = (\textcolor{red}{1},\textcolor{blue}{1},\textcolor{green}{1})_{S_1} - (0,1,0)_{SH} = (\textcolor{red}{1}\vec{u_1} + \textcolor{blue}{1}\vec{u_2} + \textcolor{green}{1}\vec{u_3}) - (0,1,0)
\]\[
  = 1·(1,0,1) +1 ·(0,2,1)  + 1·(0,0,4) - (0,1,0) = (1,1,6)
\]

\begin{itemize}
   \item $(0,1,0)_{SH}$ quiere decir "el vector $(0,1,0)$ expresado en el sistema de referencia habitual".
   \item $(1,1,1)_{S_1}$ quiere decir "el vector $(1,1,1)$ expresado en el sistema de referencia $S_1$".
 \end{itemize} 
\end{problem}


\subsection{La recta}

\obs Dado que la única manera que tenemos de determinar los puntos es a través de vectores de posición, utilizaremos $\vec{p} = (x,y,z)$ de forma equivalente a $[\vec{OP}]$ para referirnos a un punto cualquiera del plano, al que accedemos a través de su vector de posición.

Formas de determinar una recta:
\begin{enumerate}
  \item Un punto\footnote{O su vector de posición} y un vector director.
  \subitem 2 puntos (se reduce al caso anterior)
  \subitem Un punto y una condición de paralelismo (se reduce al primer caso)
  \item 2 planos secantes.
  \item Un punto y un plano perpendicular (se verá en geometría euclídea).
\end{enumerate}

Como todo con lo que vamos a trabajar son operaciones con vectores con coordenadas en un sistema de referencia, en realidad el punto de origen del sistema de referencia no es relevante. (Ver ejemplo \ref{example::origen_ref}).

\subsubsection{Ecuaciones de la recta}

La 
%
\concept[Ecuación de la recta\IS vectorial]{ecuación vectorial de la recta} $r$ determinada por el punto $A$, cuyo vector de posición es $\vec{a}$, con vector director $\vec{u_r} $  es $r : \vec{p} = \vec{a} + \lambda \vec{u_r}$, con $\lambda\in\real, \forall P\in\real$.
%
De esta manera quedan determinados los vectores de posición de todos los puntos de la recta $r$.

¿Es posible construir la recta sin ese parámetro $\lambda$? En realidad,
$$r : \vec{p} = \vec{a} + \lambda \vec{u_r}\dimplies \underbrace{r:\displaystyle \left\{
\begin{array}{c} 
  x = a_1 + \lambda u_1\\
  y = a_2 + \lambda u_2 \\ 
  z = a_3 + \lambda u_3
\end{array}\right\}}_{(1)} \implies \underbrace{r:\frac{x-a_1}{u_1} = \frac{y-a_2}{u_2} = \frac{z-a_3}{u_3}}_{(2)}$$

\begin{itemize}
    \item A $(1)$ lo denominamos \concept[Ecuación de la recta\IS paramétrica]{ecuación paramétrica de la recta}
    \item A $(2)$ lo denominamos \concept[Ecuación de la recta\IS continua]{ecuación continua de la recta}
\end{itemize}

\[
\frac{x-a_1}{u_1} = \frac{y-a_2}{u_2} = \frac{z-a_3}{u_3}\overset{(1)}{\implies} \left\{
\begin{array}{c}
     \displaystyle\frac{x-a_1}{u_1} = \frac{y-a_2}{u_2}\\
     \displaystyle\frac{y-a_2}{u_2} = \frac{z-a_3}{u_3}
\end{array}\right\} \implies
\underbrace{\left\{\begin{array}{cccc}
     Ax + &By    &     & = D\\
          &B'y   &+ C'z  & = D
\end{array}\right\}}_{(2)}
\]


\begin{figure}[hptb]
    \centering
    \includegraphics[width=0.4\textwidth]{img/EcVectorialRecta.png}
    \caption{Representación gráfica de la ecuación vectorial de la recta.}
    \label{fig:plano}
\end{figure}


Donde:
\begin{itemize}
    \item (1) \hide{Se han elegido estas 2 parejas, pero podrían haberse elegido otras, dando lugar a otras ecuaciones implícitas de la misma recta.}
    \item (2) \hide{\concept[Ecuación de la recta\IS implícita]{ecuación implícita de la recta}}
    \subitem \obs \hide{Si interpretáramos las ecuaciones implícitas de la recta como un sistema de ecuaciones, tendríamos un sistema compatible indeterminado con grado de libertad 1.}
    \subitem \obs La dimensión de la recta es 1.
\end{itemize}

\begin{problem}
    \ppart 
    Halla todas las ecuaciones de la recta que pasa por $A(0,1,2)$ y es paralela a la que pasa por $B(1,-2,-1)$ y $C(1,0,0)$
    \ppart 
    Halla un vector director de la recta $r:\displaystyle\frac{x-2}{1} = \frac{y-3}{5} = \frac{z-1}{4}$
    \ppart 
    Halla un vector director de la recta $r:\displaystyle\left\{\begin{array}{c} 2x+3y=4\\2x-y+3z=0\end{array}\right\}$
    \solution

\end{problem}

\textbf{Deberes:} 
\begin{itemize}
  \item Página 279.13,14.
  \item Página 281.18-21.
\end{itemize}

\subsection{El plano}

\subsubsection{Ecuaciones del plano}

Un plano queda determinado por \hide{un punto y dos vectores linealmente independientes}

La 
%
\concept[Ecuación del plano\IS vectorial]{ecuación vectorial del plano} 
%
$\pi$ determinada por el punto $A$ y los vectores linealmente independientes $\vec{V_{\pi}}$ y $\vec{W_{\pi}}$  es $r : \vec{p} = \vec{A} + \lambda \vec{V_{\pi}} + \mu\vec{W_{\pi}}$, con $\lambda\in\real$. De esta manera quedan determinados los vectores de posición de todos los puntos del plano.

Como en el caso de la recta, podemos escribir esta ecuación en forma de sistema con parámetros:

$$\pi : \vec{p} = \vec{A} + \lambda \vec{V_{\pi}} + \mu\vec{W_{\pi}}\dimplies \underbrace{\pi:\displaystyle \left\{\begin{array}{c} x = a_1 + \lambda v_1 + \mu w_1\\y = a_2 + \lambda v_2 + \mu w_2 \\ z = a_3 + \lambda v_3 + \mu w_3 \end{array}\right\}}_{(1)} $$

 A $(1)$ lo denominamos \concept[Ecuación del plano\IS paramétrica]{ecuación paramétrica del plano}

\[
\pi:\displaystyle \left\{
\begin{array}{c} 
x - a_1 = \lambda v_1 + \mu w_1\\
y - a_2 = \lambda v_2 + \mu w_2 \\ 
z - a_3 = \lambda v_3 + \mu w_3 
\end{array}\right\}
\]
Como los vectores $\vec{v},\vec{w}$ son linealmente independientes y el vector $AX$ es una combinación lineal de los otros 2 (ver \ref{fig:plano}, tenemos:

\[
\left|
\begin{array}{ccc} 
x - a_1 & v_1 & w_1\\
y - a_2 & v_2 & w_2 \\ 
z - a_3 & v_3 & w_3 
\end{array}\right| = 0
\]

Desarrollando esta ecuación, tendríamos una ecuación del tipo $Ax+By+Cz + D = 0$, que llamamos \concept[Ecuación del plano\IS implícita]{Ecuación implícita del plano}.


\begin{figure}[hptb]
    \centering
    \includegraphics[width=0.65\textwidth]{img/ecplanos.png}
    \caption{Plano generado por un punto y dos vectores}
    \label{fig:plano}
\end{figure}

\begin{problem}
    \textbf{Calcula las ecuaciones del plano que pasa por los puntos $A(1,1,1), B(2,2,2), C(1,2,3)$}

    \solution 

    \hide{
    El primer paso sería calcular 2 vectores linealmente independientes de estos 3 puntos, para comprobar que los 3 puntos forman un plano (y no una recta).

    $\vec{AB} = (1,1,1) \quad\quad \vec{AC} = (0,1,2)$, que son linealmente independientes al no ser proporcionales.

    Ecuación vectorial: $\pi: \vec{p} = (1,1,1) + \mu\vec{AB} + \lambda\vec{AC}, \lambda,\mu\in\real$

    Ecuación paramétrica: $
    \displaystyle\left\{ \begin{array}{c}
      x = 1 + \mu\\
      y = 1 + \mu + \lambda\\
      z = 1 + \mu + 2\lambda
    \end{array} \right\}\text{ con } \lambda,\mu\in\real$

    Ecuación implícita: 

    \[
      \displaystyle \begin{vmatrix}
      x - 1 & 1 & 0\\
      y - 1 & 1 & 1\\
      z - 1 & 1 & 2\\
    \end{vmatrix} = 0 \dimplies \cdots \dimplies x - 2y+z=0
    \]
    }
\end{problem}


\begin{problem}
\ppart Halla la ecuación de 2 rectas que pertenezcan al mismo plano.
\ppart Halla un vector director del plano: $\pi_1: x+y+z = 3$
\ppart Halla el plano paralelo a $\pi_2: x+y+z = 3$ que pase por el origen de coordenadas.
\ppart Halla el plano paralelo al $XY$ que pasa por $A(-1,2,-2)$.
\obs Llamamos plano $XY$ al plano "del suelo", es decir, al plano $z=0$.

\ppart Página 283, ejercicios 25-28.

\solution

\end{problem}

\subsection{Posiciones relativas}

\subsubsection{Entre 2 planos}  

\begin{framed}
\textbf{Ecuaciones vectoriales o paramétricas:}
  \begin{itemize}
    \item Si 2 planos comparten 3 puntos, entonces son el mismo plano.
    \item Si los 4 vectores directores de los 2 planos son linealmente independientes, entonces los planos son secantes en un punto.
    \item Si la matriz formada por los 4 vectores directores de los 2 planos tiene rango 3, los planos son secantes en una recta.
    \item Si la matriz formada por los 4 vectores directores de los 2 planos tiene rango 2, los planos son paralelos.
    \item Si 2 planos paralelos comparten un punto, entonces son coincidentes.
  \end{itemize}
\end{framed}



\subparagraph{Ecuaciones implícitas}

\[
\left\{\begin{array}{c}
\pi_1: Ax+By+Cz = D\\
\pi_2: A'x+B'y+C'z = D'
\end{array}\right\}
\]

Obtenemos las matrices: $M = \displaystyle\begin{pmatrix}A&B&C\\A'&B'&C'\end{pmatrix}$ y $M^* = \displaystyle\begin{pmatrix}A&B&C&D\\A'&B'&C'&D'\end{pmatrix}$

\begin{framed}
  \begin{itemize}
    \item $Rg(M) = Rg(M^*) = 1 $\hide{ coincidentes.}
    \item $Rg(M) < Rg(M^*) = 2 $\hide{ paralelos.}
    \item $Rg(M) = Rg(M^*) = 2 $\hide{ secantes.}
  \end{itemize}
\obs En realidad, sería como las ecuaciones implícitas de la recta.
\end{framed}

\subsubsection{Entre 3 planos}

\subparagraph{Ecuaciones vectorial o paramétricas}

Se pasa a implícitas.

\subparagraph{Ecuaciones implícitas}
\[
\left\{\begin{array}{c}
\pi_1: Ax+By+Cz = D\\
\pi_2: A'x+B'y+C'z = D'\\
\pi_3: A''x+B''y+C''z = D''\\
\end{array}\right\}
\]

Obtenemos las matrices: 
$M  = \displaystyle\begin{pmatrix}
A&B&C\\
A'&B'&C'\\
A''&B''&C''
\end{pmatrix}
$ y 
$M^* = \displaystyle\begin{pmatrix}
A&B&C&D\\
A'&B'&C'&D'\\
A''&B''&C''&D''
\end{pmatrix}
$

Las posibilidades son: (ver figura \ref{fig:PosicionesRelativasPlanos})
\begin{framed}
  \begin{itemize}
    \item $Rg(M) = Rg(M^*) = 1 $\hide{ SCI, secantes en un plano [grado de indeterminación 2, por lo que hay dos parámetros. \textbf{Coincidentes}.}
    \item $Rg(M) < Rg(M^*) = 2 $\hide{ paralelos.}
    \item $Rg(M) = Rg(M^*) = 2 $\hide{ SCI, secantes en una recta [grado de indeterminación 1, por lo que hay un parámetro.}
    \item $Rg(M) = 2 < Rg(M^*) = 3 $\hide{ no se cortan los 3. Sistema incompatible}
    \item $Rg(M) = Rg(M^*) = 3 $\hide{ SCD, secantes en un punto que es la solución del sistema.}
  \end{itemize}  
\end{framed}

\begin{figure}[hptb]
    \centering
    \includegraphics[width=0.65\textwidth]{img/Captura1.png}
    \includegraphics[width=0.95\textwidth]{img/Captura2.png}
    \includegraphics[width=1.1\textwidth]{img/Captura3.png}
    \includegraphics[width=1.1\textwidth]{img/Captura4.png}
    \caption{Representación gráfica de las posiciones relativas de 3 planos.}
    \label{fig:PosicionesRelativasPlanos}
\end{figure}




\subsection{Posiciones relativas entre recta y plano}

\paragraph{Ecuaciones implícitas: } si tanto la recta como el plano están dados en ecuaciones implícitas, estaríamos en la posición relativa de 3 planos, sabiendo que 2 de ellos son secantes en una recta.

\paragraph{Ecuaciones paramétrica: } $\pi: \{P_{\pi},\vec{v_{\pi}}, \vec{w_{\pi}}\}$ 

\begin{center}
\begin{tabular}{ccc}
$P_r \in \pi $ & $\vec{v_r}$ LD de $v_{\pi}, \vec{w}_{\pi}$ & Conclusión\\
No & No & Secantes en un punto\\
No & Sí & Recta paralela\\
Sí & No & Secantes en un punto\\
Sí & Sí & Recta contenida en el plano\\
\end{tabular}
\end{center}

\paragraph{Plano implícito, recta paramétrica: } comprobamos si existe algún valor de $\lambda_{r}$ para el que se cumpla la ecuación implícita del plano. 
\begin{itemize}
  \item $\exists!\lambda \implies $ secante.
  \item $\lambda \in \real \implies$ contenida.
  \item $\not\exists \lambda \implies $ paralela.
\end{itemize}

Deberes: Hoja resumen de teoría,105ab,107ab

\subsubsection{Posiciones relativas entre 2 rectas:}

Dadas las rectas 
$r:\vec{p} = \vec{a} + \lambda\vec{u},\quad \lambda\in\real$
y
$s:\vec{p} = \vec{b} + \lambda\vec{w},\quad \lambda\in\real$. 

Si los vectores directores son paralelos (proporcionales), las rectas pueden ser paralelas o coincidentes. 
%
Para poder distinguir , podríamos ver si un vector formado por un punto de cada recta es también proporcional (entonces serían coincidentes) o si no (entonces serían secantes).

De la misma manera, si los vectores son linealmente independientes las rectas pueden cruzarse o cortarse. 
%
Para distinguir estos 2 casos, podríamos ver si un vector formado por un punto de cada recta es linealmente dependiente a los otros 2 (entonces serían secantes porque formarían un plano que contiene al vector) o si no (entonces se cortarían en el espacio).

Así, buscamos estudiar la dependencia lineal de los 2 vectores directores ($\vec{u},\vec{w}$) y de los 2 vectores directores respecto de un vector formado, arbitrariamente, con 2 puntos de las rectas ($\vec{AB}$, con $A(a_1,a_2,a_3)\in r$ y $B(b_1,b_2,b_3)\in s$). 
%
Para ello, formamos las matrices:

$M  = \displaystyle\begin{pmatrix}
u_1&w_1\\
u_2&w_2\\
u_3&w_3
\end{pmatrix}
$ y 
$M^* = \displaystyle\begin{pmatrix}
u_1&w_1&b_1-a_1\\
u_2&w_2&b_2-a_2\\
u_3&w_3&b_3-a_3\\
\end{pmatrix}
$

\begin{framed}
  \begin{itemize}
    \item $Rg(M) = Rg(M^*) = 1 $\hide{ SCI, secantes en una recta [grado de indeterminación 1, por lo que hay dos parámetros. \textbf{Coincidentes}.]}
    \item $Rg(M) = 1 < Rg(M^*) = 2 $\hide{ paralelas.} 
    \item $Rg(M) = Rg(M^*) = 2 $\hide{ SCI, secantes en un plano [dimensión 2]. $\vec{AB}$ se puede escribir como combinación lineal de $\vec{u}$ y $\vec{w}$}
    \item $Rg(M) = 2 < Rg(M^*) = 3 $\hide{ se cruzan en el espacio.} 
  \end{itemize}  
\end{framed}
\obs También se podría trabajar con las matrices traspuestas si uno está más familiarizado con estudiar el rango como combinaciones lineales de filas en lugar de columnas.

\begin{problem}
Página 289, ejercicios 52, calculando puntos de cortes
\solution

\end{problem}


\subsubsection{Haz (no se pide en selectividad, así que se salta)}
\paragraph{Haz de rectas paralelas: } cambia el punto, manteniendo fijo el vector. 
\paragraph{Haz de rectas secantes: } cambia el vector (sin ser nunca nulo), mantiene fijo el punto.
\paragraph{Haz de planos paralelelos: } cambia el punto, mantiene los vectores
\paragraph{Haz de planos secantes en una recta: } mantiene un vector y un punto, cambia el otro vector.

\begin{problem}
Tema 11: 56,57,59,60.
\solution
\end{problem}

Deberes: 118,136,143

\subsubsection{Practicamos en general}

Tema 11: 
122,127,130,134,135,136,139,140,143,145,149,150

Tema 11:
\begin{itemize}
  \item Básicos: 83,91,92,93a,98,100,101a,103a,104a,105,106,107d,108
  \item Síntesis: 111-119
  \item Completos: 122-124,127,129-133
\end{itemize}




\section{Geometría euclídea (del 25/01 al final)}

Llamamos geometría euclídea al espacio en el que podemos medir, cosa que hasta ahora no era posible.

Todo surge desde el módulo de un vector. Llamamos \concept[Módulo de un vector]{módulo de un vector} a la longitud que tiene. Dado $\vec{v}$, se define el módulo como $|\vec{v}|$.

\subsection{Producto, escalar, vectorial y mixto}

\paragraph{Introducción sobre el origen del producto escalar y vectorial}

Fuentes consultadas:
\begin{itemize}
  \item \href{http://www.suitcaseofdreams.net/Geometric_multiplication.htm}{Relación forma polar y binómica del producto complejo}
  \vspace{-0.4cm}
  \item \href{https://www2.clarku.edu/faculty/djoyce/complex/mult.html}{Interpretación geométrica del producto complejo}
  \vspace{-0.4cm}
  \item \href{https://www.quora.com/Who-invented-the-dot-product-and-cross-product}{Historia y aplicación de los cuaterniones los productos}
  \vspace{-0.4cm}
  \item \href{https://es.wikipedia.org/wiki/Cuaterni%C3%B3n}{ Extensión de los complejos al grupo de los quaterniones}
\end{itemize}

Dados 2 números complejos $z_1 = a_1+b_1i$, $z_2 = a_2+b_2i$. Expresando estos números complejos en forma polar tenemos: $z_1=r_{\alpha_1}$ y $z_2 = s_{\alpha_2}$.  

$z_1·z_2 = (a_1a_2 - b_1b_2) + (a_1b_2+a_2b_1)i = r·s_{\alpha_1+\alpha_2}$.

Tomando $z_1·\bar{z_2} = (a_1a_2 + b_1b_2) + (a_1b_2-a_2b_1)i = r·s_{\alpha_1-\alpha_2}
$

\subparagraph{Estudio de la parte real (producto escalar)}

En $Re(z_1·\bar{z_2}) = a_1a_2 + b_1b_2 = Re(r·s_{\alpha_1-\alpha_2})$

Para calcular $Re(r·s_{\alpha_1-\alpha_2}) = Re(r·s·\cos(\alpha_1-\alpha_2) + i·r·s·\sen(\alpha_1-\alpha_2)$, por lo que podemos completar:

$ a_1a_2 + b_1b_2 = |z_1||z_2|·\cos(\alpha_1-\alpha_2)$, y, siendo conscientes que $\alpha_1-\alpha_2$ es el ángulo que forman los 2 vectores, obtenemos la expresión del producto escalar de 2 vectores.


\subparagraph{Estudio de la parte imaginaria (producto vectorial)}

Al extender este razonamiento al grupo de los cuaterniones, tendríamos:

\newcommand{\quat}{\vec}

$Re(z_1\bar{z_2}) = Re\left((b_1\quat{i}+c_1\quat{j}+d_1\quat{k})·(-b_2\quat{i}-c_2\quat{j}-d_2\quat{k})\right) = (b_1b_2+c_1c_2+d_1d_2)$

Lo espectacular viene al considerar la parte "imaginaria" (aunque no tengamos claro cómo se define ese concepto en los cuaterniones):

$Im(z_1\bar{z_2}) = f(\quat{i},\quat{j},\quat{k})$, cuya expresión analítica es la del producto vectorial, ya que en grupo de los cuaterniones $\quat{i}\quat{j}=\quat{k}$ y todo eso.

\subsubsection{Producto escalar}

\begin{itemize}
  \item Definición.
  \item Base ortogonal y ortonormal.
  \item Expresión analítica.
  \item Cálculo del módulo (porque Pitágoras tridimensional no funciona).
  \item Base canónica.
  \item Interpretación geométrica (proyección).
\end{itemize}

\subsubsection{Producto vectorial}
\begin{itemize}
  \item Definición.
  \subitem Regla de la mano derecha
  \item Expresión analítica.
  \item Interpretación geométrica. 
\end{itemize}


\subsubsection{Producto mixto}

\begin{itemize}
  \item ¿Qué ocurre si en el producto vectorial meto un vector en lugar de $\vec{i},\vec{j},\vec{k}$?
  \item Definición.
  \item Expresión analítica.
  \item Interpretación geométrica. 
\end{itemize}

\subsubsection{Practicamos en general ejercicios del tema 9}

\subsection{Aplicación de los productos}

\subsubsection{Vector normal del plano}
\subsection{Perpendicular común}

\subsection{Proyecciones y medidas}
\subsubsection{Simetría de un punto respecto de otro punto}
\subsubsection{Simetría de un punto respecto de una recta}
\subsubsection{Simetría de un punto respecto de un plano}

\newpage
\subsection{Ángulos}
\subsubsection{Ángulo formado por dos rectas: secantes, se cruzan y paralelas/coincidentes}
\subsubsection{Ángulo formado por dos planos}
\subsubsection{Ángulo formado por recta y plano}

\subsection{Distancias}
\subsubsection{Entre 2 puntos}
\subsubsection{Plano mediador}
\subsubsection{Entre punto y plano}
\subsubsection{Planos bisectores}
\subsubsection{Entre 2 planos}
\subsubsection{Entre recta y plano}
\subsubsection{Entre punto y recta}
\subsubsection{Entre 2 rectas paralelas}
\subsubsection{Entre 2 rectas no paralelas}

\subsubsection{Volumen del paralelepípedo}


\chapter{Análisis}

\begin{defn}[Función real de variable real]
Una función real de una variable real es una aplicación definida entre dos conjuntos de números reales tal que a cada elemento del primer conjunto le corresponde un único elemento del segundo conjunto.

$\appl{f}{D\subset\real}{\real}$

\begin{itemize}
	\item $f$: símbolo de la función.
	\item $D(f) = \{x\in\real \tq \exists f(x)\}$
	\item $Rec(f) = \{y\in\real \tq \exists x \in\real f(x)=y\}$
\end{itemize}
\end{defn}

\paragraph{Ejemplos de dominios de funciones:} hojita impresa de repaso, incluyendo funciones trigonométricas y de las trigonométricas inversas.

\begin{problem}
Halla el dominio de las siguientes funciones:
\ppart $f(x) = \sqrt{\displaystyle\frac{x^2+5x+4}{x+4}}$ \obs Ojo con simplificar.

\ppart  $f(x) = \log\left(x^2+1\right)$

\ppart $f(x) = \arcsen\left(x^2-9\right)$

\solution

\end{problem}

\section{Límites y continuidad}

\begin{theorem}[Teorema\IS de existencia del límite]
\[\exists \lim_{x\to a}f(x) \dimplies \lim_{x\to a^+} f(x) = \lim_{x\to a^-}f(x)\]
\end{theorem}

\begin{problem}
Página 13, ejercicios 11-14. 
\solution
\end{problem}

\subsection{Indeterminaciones}

\begin{itemize}
	\item Racionales: $\rfrac{0}{0},\rfrac{\infty}{\infty},\infty-\infty,0·\infty, \rfrac{k}{0}$ (Ver por el libro, se dan por supuestas)
	\item Exponenciales $1^{\infty}; 0^0; \infty^0$
\end{itemize}


\begin{problem}
\textit{Para repasar en sus casas:}
\ppart 19b,d
\ppart 23g,h
\ppart 25ac
\solution
\end{problem}


\begin{prop}[Indeterminación\IS $1^{\infty}$]
\index{Indeterminación\IS $1^{\infty}$}
\[
\lim_{x\to a}f(x)^{g(x)} \to 1^{\infty} \implies \lim_{x\to a}f(x)^{g(x)} = e^\lambda, \lambda = \lim_{x\to a} \left(g(x)·[f(x)-1]\right)
\]
\end{prop}

\begin{proof}
\begin{align*}
\lim_{x\to a}f(x)^{g(x)} &= \lim_{x\to a}(1+f(x)-1)^{g(x)}
\\
&= \lim_{x\to a}\left(1+\frac{1}{\left(\frac{1}{f(x)-1}\right)}\right)^{g(x)}
= \lim_{x\to a}\left(1+\frac{1}{\left(\frac{1}{f(x)-1}\right)}\right)^{g(x)\frac{f(x)-1}{f(x)-1}}
\\
&= \lim_{x\to a}\left[\left(1+\frac{1}{\left(\frac{1}{f(x)-1}\right)}\right)^{\frac{1}{f(x)-1}}\right]^{g(x)(f(x)-1)}
\\
&= \lim_{x\to a}\left[\left(1+\frac{1}{\left(\frac{1}{f(x)-1}\right)}\right)^{\frac{1}{f(x)-1}}\right]^{\displaystyle\lim_{x\to a}g(x)(f(x)-1)} 
\\
&= e^{\displaystyle\lim_{x\to a}g(x)(f(x)-1)}
\end{align*}
\end{proof}

\subsubsection{Infinitésimos equivalentes}

Libro página 19. Ver tabla y hacer ejercicio 30.

\begin{defn}[Infinitésimos equivalentes]
Dadas $f(x)$, $g(x)$, $a\in\real$ decimos que $f(x)$ y $g(x)$ son infinitésimos equivalentes \textbf{en $x=a$} si y sólo si $\displaystyle\lim_{x\to a}{f(x)}{g(x)} = 1$
\end{defn}

\begin{problem}
Ejercicio 30
\solution
\end{problem}

\subsection{Continuidad (¡Tiene resumen para entregar!)}

\begin{defn}[Continuidad\IS en un punto]
Sea $\appl{f}{D\subset\real}{\real}$.

Se dice que $f(x)$ es continua en $x=a$ sí y solo si se cumplen las 3 condiciones siguientes:
\begin{itemize}
	\item $\exists \displaystyle\lim_{x\to a} f(x)$
	\item $\exists f(a)$
	\item $\displaystyle\lim_{x\to a}f(x) = f(a)$
\end{itemize}

\textit{También se puede decir de manera abreviada: \[f(x) \text{ continua en } x=a\dimplies \text{ existen y son iguales } \lim_{x\to a}f(x) \text{ y } f(a)\]}
\end{defn}

\begin{problem}
Halla el valor de $m$ para que la función sea continua en $x=0$.
\[f(x) = 
	\begin{cases}
		2x-5 & \text{ si }x\leq 0\\ 
		\frac{2x+m}{x+1} & \text{ si } x>0
	\end{cases}\]
\solution

Aplicamos: "$f(x)$ continua en $x=0$ si existen y son iguales $\displaystyle\lim_{x\to 0}$ y $f(0)$".

\begin{itemize}
	\item $f(0) =  2·0-5 = -5$
	\item $\displaystyle\lim_{x\to0} f(x)$. Para calcularlo necesitamos los límites laterales:
	\subitem $\displaystyle\lim_{x\to0^+} f(x) = \lim_{x\to0^+} \frac{2x+m}{x+1} = \frac{2·0+m}{0+1} = m$
	\subitem $\displaystyle\lim_{x\to0^-} f(x) = \lim_{x\to0^-} 2·x-5 = -5$

	\[\lim_{x\to 0} f(x) = -5 \dimplies m=-5 \text{ porque } \begin{cases}\displaystyle\lim_{x\to0^+}f(x) = m\\ \displaystyle\lim_{x\to0^-}f(x) = -5\end{cases}\]
	\item Si $m=-5$, $f(x)$ es continua en $x=0$.
\end{itemize}

\end{problem}

\paragraph{Tipos de discontinuidad}

\[
\begin{cases} 
	\text{Continua}\\
	\text{ Discontinua }
		\begin{cases}
			\text{Evitable}\\
			\text{Esencial}
				\begin{cases}
					\text{1ª especie}
						\begin{cases}
							\text{Salto finito}\\
							\text{Salto infinito}
						\end{cases}\\
					\text{2ª especie}
				\end{cases}
		\end{cases}
\end{cases}
\]

\textbf{Descripción: } Libro página 22. 
\begin{itemize}
	\item Evitables: $\exists \displaystyle\lim_{x\to a}f(x)$ pero $\begin{cases}\displaystyle\lim_{x\to a}f(x) \neq f(a)\\\text{o}\\\nexists f(a)\end{cases}$
	\subitem Para evitarlas, se define una nueva función: $f(x) = \begin{cases}f(x) & x\neq a\\ b & x=a \end{cases}$
	\item Esenciales (o inevitables): 
	\begin{itemize}
		\item De primera especie:
		\subitem De salto finito: ambos límites laterales son finitos pero distinto.
		\subitem De salto infinito: al menos un límite lateral es infinito.
		\item De 2ª especie: al menos un límite lateral no existe. $f(x) = \sqrt{x}$ y $f(x) = \log(x)$ en $x=0$.
	\end{itemize}
\obs Ver \fref{fig::fun-tipos-discontinuidad}.
\end{itemize}


\begin{figure}
\centering
\includegraphics[scale=0.25]{img/Funs/funcion_xy_discontin_4hStg}
\caption{Ejemplo de discontinuidad evitable}
\label{fig::fun-tipos-discontinuidad}
\includegraphics[scale=0.25]{img/Funs/funcion_xy_discontin_4jSGg}
\caption{Ejemplo de discontinuidad de 2ª especie}
\includegraphics[scale=0.25]{img/Funs/funcion_xy_discontin_6ueR5}
\caption{Ejemplo de discontinuidad de salto infinito}
\includegraphics[scale=0.25]{img/Funs/funcion_xy_discontin_U72G2}
\caption{Ejemplo de discontinuidad de salto finito}
\end{figure}



\begin{defn}[Continuidad\IS en un intervalo abierto]
	$f(x)$ es continua en $(a,b) \dimplies \forall x\in(a,b), f(x)$ es continua.
\end{defn}


\begin{defn}[Continuidad\IS por la izquierda y/o por la derecha]
	\begin{itemize}
		\item Una función $(x)$ es continua por la derecha de $a$ si $\displaystyle\lim_{x\to a^+}f(x) = f(x)$
		\item Una función $(x)$ es continua por la izquierda de $a$ si $\displaystyle\lim_{x\to a^-}f(x) = f(x)$
	\end{itemize}
\end{defn}

\begin{defn}[Continuidad\IS en un intervalo cerrado]
$f(x)$ es continua en $[a,b] \dimplies \forall x\in[a,b],\begin{cases} f(x) \text{ es continua en } (a,b)\\
\text{Continua por la derecha de } a\\
\text{Continua por la izquierda de } b\\\end{cases}$
\end{defn}


\begin{example}
Dada $f(x) = +\sqrt{x}$.

$f(x)$ es continua en su dominio, por ser función radical. $f(x)$ continua en $(0,\infty)$.

$f(x)$ no es continua en $x=0$, pero sí es continua \textit{por la derecha} en $x=0$, ya que $\displaystyle\lim_{x\to 0^+} f(x) = 0$ en $[0,\infty)$.
\end{example}


\begin{table}[hbtp]
\begin{tabular}{|l|l|}
\hline
Funciones polinómicas & Continuas en $\real$\\\hline\hline
$f(x) = \frac{P(x)}{Q(x)}$ con $P(x),Q(x)$ polinomios& Continuas en $\real-\{x \tq Q(x)=0\}$\\\hline
$f(x) = \sqrt[n]{P(x)}$ & $\begin{cases}
n\text{ impar: continua en } \real\\\hline
n\text{ par: continua en } \{x\in\real\tq P(x) \geq 0\}\end{cases}$\\\hline
$f(x) = a^x$ con $a>0$ & Continua en $\real$\\\hline
$f(x) = \log_ax$ (con $a>0, a\neq 1$) & Continua en $\real$\\\hline
$f(x) = \cos(x)$ y $f(x) = \sen(x)$  & Continua en $\real$\\\hline
$f(x) = \tg(x)$   & Continua en $\real-\left\{x=\rfrac{\pi}{2}k, k\in\mathbb{Z}\right\}$\\\hline
$f(x) = \arcsen(x)$ & Continua en su dominio.\\
$f(x) = \arccos(x)$ & Continua en su dominio.\\
$f(x) = \arctg(x)$ & Continua en su dominio.\\\hline
\end{tabular}
\label{tbl::ContinuidadFunElementales}
\caption{Continuidad de las funciones elementales}
\end{table}

\begin{problem}
Ejercicios 23.41 y 23.42.
\solution
\end{problem}

\begin{problem}Estudia la continuidad de las siguientes funciones y clasifica sus discontinuidades

\begin{itemize}
	\item $f_1(x) = \displaystyle\frac{x^2+5x+6}{x^2+2x}$
	\item $f_2(x) = \begin{cases}\log{x+1} & x\leq -1\\x & x>1\end{cases}$
\end{itemize}
\solution
\end{problem}



\begin{theorem}[Teorema\IS de Bolzano]
Sea $\appl{f}{D}{\real}$.

\[
\left.\begin{array}{c}f(x) \text{ continua en } [a,b]\\\text{Signo}(f(a))\neq \text{Signo}(f(b))\end{array}\right\}\implies \exists c\in(a,b) \tq f(c) = 0
\]

\end{theorem}
\obs Es una condición suficiente, no necesaria. Es decir, es $\implies $. Por ejemplo, $f(x) = (x-1)^2$ corta en $x=1$, pero no cumple las condiciones.

\begin{problem} Demuestra que la ecuación $x^3-7x^2-1$ tiene al menos una solución real en el intervalo $[0,10]$.
\solution

Sea $f(x) = x^3-7x^2-1$. Se trata de demostrar que $\exists c\in[0,10]\tq f(c) = 0$. Comprobamos que cumple el teorema de Bolzano.

\[
\left\{
	\begin{array}{c}
		f(x) \text{ es continua en } [0,10] \text{ por ser polinómica}\\
		\left.\begin{array}{c}
		f(x) = -1 \\
		f(10) = 299\end{array}
		\right\}\implies Signo(f(0))\neq Signo(f(10))
	\end{array}
\right\}\implies \exists c\in(0,10)\tq f(c)=0 
\]
\end{problem}

\begin{theorem}[Teorema\IS del valor intermedio]
Sea $\appl{f}{D}{\real}$.
\[
\left.\begin{array}{c}f(x) \text{ continua en } [a,b]\\
\exists k\in\real \begin{cases}f(a)\leq k\leq f(b)\\f(b)\leq k \leq f(a)\end{cases}\end{array}\right\}\implies \exists c\in(a,b) \tq f(c) = k
\]
\end{theorem}

\obs Para $k=0$, el teorema del valor intermedio se convierte en teorema de Bolzano.


\begin{problem}
Demuestra que las funciones $f(x) = \cos(x)$ y $g(x) = x$ se cortan en algún punto.
\solution

Basta considerar la función $h(x) = f(x) - g(x) = \cos(x)-x$. Así, demostrar que las funciones se cortan será equivalente a demostrar que $\exists c\in\real\tq h(x)=0$. 

Esta función es continua en $\real$, por ser resta de funciones continuas en $\real$.

Buscamos $c\in\real\tq h(c)=0$. Consideramos $h(0) = 1$ y $h\left(\rfrac{\pi}{2}\right) = \rfrac{-\pi}{2}<0$.

Por el teorema de Bolzano, $\exists c\in\left(0,\rfrac{\pi}{2}\right)\tq h(c) = 0$, por lo que podemos concluir que las funciones se cortan.
\end{problem}

\begin{problem}
\begin{itemize}
	\item 35.108 (continuidad)
	\item 36.113 (continuidad)
	\item 36.117 (problema - continuidad)
	\item 36.119,120 (límites)
	\item Página 37.autoevaluación.
\end{itemize}
\solution
\end{problem}

\begin{theorem}[Teorema\IS de Weierstrass]
Si $f(x)$ es continua en $[a,b]$, entonces tiene un máximo y un mínimo absolutos en $[a,b]$.
\end{theorem}

\obs Este teorema se utilizará en el siguiente tema.


%%%%%%%%%%%%%%%%%%%%%%%%%%%%%%%%%%%%%%%%%%%%%%%%%%%%%%%%%%%%%%%%%%
%%%%%%%%%%%%%%%%%%%%%%%%%%%%%%%%%%%%%%%%%%%%%%%%%%%%%%%%%%%%%%%%%%
%%%%%%%%%%%%%%%%%%%%%%%%%%%%%%%%%%%%%%%%%%%%%%%%%%%%%%%%%%%%%%%%%%
%%%%%%%%%%%%%%%%%%%%%%%%%%%%%%%%%%%%%%%%%%%%%%%%%%%%%%%%%%%%%%%%%%
%%%%%%%%%%%%%%%%%%%%%%%%%%%%%%%%%%%%%%%%%%%%%%%%%%%%%%%%%%%%%%%%%%
%%%%%%%%%%%%%%%%%%%%%%%%%%%%%%%%%%%%%%%%%%%%%%%%%%%%%%%%%%%%%%%%%%
%%%%%%%%%%%%%%%%%%%%%%%%%%%%%%%%%%%%%%%%%%%%%%%%%%%%%%%%%%%%%%%%%%
%%%%%%%%%%%%%                                    %%%%%%%%%%%%%%%%%
%%%%%%%%%%%%%                                    %%%%%%%%%%%%%%%%%
%%%%%%%%%%%%%                                    %%%%%%%%%%%%%%%%%
%%%%%%%%%%%%%                                    %%%%%%%%%%%%%%%%%
%%%%%%%%%%%%%            DERIVABILIDAD           %%%%%%%%%%%%%%%%%
%%%%%%%%%%%%%                                    %%%%%%%%%%%%%%%%%
%%%%%%%%%%%%%                                    %%%%%%%%%%%%%%%%%
%%%%%%%%%%%%%                                    %%%%%%%%%%%%%%%%%
%%%%%%%%%%%%%                                    %%%%%%%%%%%%%%%%%
%%%%%%%%%%%%%%%%%%%%%%%%%%%%%%%%%%%%%%%%%%%%%%%%%%%%%%%%%%%%%%%%%%
%%%%%%%%%%%%%%%%%%%%%%%%%%%%%%%%%%%%%%%%%%%%%%%%%%%%%%%%%%%%%%%%%%
%%%%%%%%%%%%%%%%%%%%%%%%%%%%%%%%%%%%%%%%%%%%%%%%%%%%%%%%%%%%%%%%%%
%%%%%%%%%%%%%%%%%%%%%%%%%%%%%%%%%%%%%%%%%%%%%%%%%%%%%%%%%%%%%%%%%%
%%%%%%%%%%%%%%%%%%%%%%%%%%%%%%%%%%%%%%%%%%%%%%%%%%%%%%%%%%%%%%%%%%
%%%%%%%%%%%%%%%%%%%%%%%%%%%%%%%%%%%%%%%%%%%%%%%%%%%%%%%%%%%%%%%%%%
%%%%%%%%%%%%%%%%%%%%%%%%%%%%%%%%%%%%%%%%%%%%%%%%%%%%%%%%%%%%%%%%%%

\newpage\section{Derivadas y derivabilidad}

\subsection{Introducción y repaso}
\begin{defn}[Pendiente de una recta]
Sea la recta $y=mx+n$.

Se define \textbf{pendiente de la recta}, $m=\frac{\Delta y}{\Delta x}$
\end{defn}

\begin{defn}[Derivada\IS en un punto]
Se define $f'(a)$ como la derivada de $f(x)$ en el punto $x=a$.

\[f'(a) = \lim_{x\to a}\frac{f(x)-f(a)}{x-a} \overset{(1)}{=} \lim_{h\to 0}\frac{f(a+h)-f(a)}{h}\]

$(1): h=x-a \dimplies x=a+h$

\textit{Ver \fref{fig::funinterpretacionderivadapunto}}.
\end{defn}

\begin{example}
Dada $f(x) = |x|$, calcula $f'(0)$.

\[
f(x) = \begin{cases}x&\text{ si } x>0 \\ -x & \text{ si }x\leq 0\end{cases}
\]

Calculamos:

\[
\lim_{x\to 0}\frac{f(x)-f(0)}{x-0} = \begin{cases}
\displaystyle\lim_{x\to 0^+} \frac{f(x)}{x} = \displaystyle\lim_{x\to 0^+} \frac{x}{x} = 1\\\\
\displaystyle\lim_{x\to 0^-} \frac{f(x)}{x} = \displaystyle\lim_{x\to 0^-} \frac{-x}{x} = -1
\end{cases}
\]

\label{derivEjemplo}

Los límites laterales no coinciden, por lo tanto, $\nexists \displaystyle\lim_{x\to 0}\frac{f(x)-f(a)}{x-a} \dimplies \nexists f'(a)$
\end{example}




\subsubsection{Interpretación geométrica de la derivada}

Ver \fref{fig::funinterpretacionderivadapunto}.

\begin{figure}[hbp]
\centering
\includegraphics[scale=0.5]{img/DerivadaInterGeometrica}
\label{fig::funinterpretacionderivadapunto}
\caption{Interpretación geométrica de la derivada} Cuando $h\to0$, el punto $B$ se acercará cada vez más al punto $A$, dando lugar a la recta tangente. 
%
Para una mejor comprensión consultar la versión de Geogebra: https://www.geogebra.org/m/jwtw6mdt\#material/f52nQ7T5

\end{figure} 


\subsubsection{Derivabilidad}
\begin{defn}[Derivabilidad\IS en un punto]
\[f(x) \text{ derivable en } x=a\dimplies \exists f'(a)\]
\end{defn}

\begin{prop}
$f(x)$ derivable en $x=a \implies f(x)$ continua en $x=a$
\obs El recíproco no es cierto. Basta comprobar el ejemplo \ref{derivEjemplo} ($f(x) = |x|$ es continua en $x=0$, pero no derivable en $x=0$).
\end{prop}

\begin{defn}[Derivabilidad\IS en un intervalo abierto]
\[f(x) \text{ derivable en } x\in(a,b) \dimplies \forall c\in(a,b) \exists f'(c) \]
\end{defn}

\begin{defn}[Dominio de derivabilidad]
El dominio de derivabilidad de una función $f(x)$ es el mayor conjunto en el que la función es derivable.
\end{defn}

\begin{example}
$f(x) = |x|$ no es derivable en $x=0$.

El dominio de derivabilidad de $f(x)$ es $\real-\{0\}$
\end{example}


\paragraph{Derivabilidad lateral:} De la misma manera que existía la \textit{continuidad lateral}, también podemos hablar de \textit{derivabilidad lateral}. 

\begin{defn}[Derivada lateral]
La derivada lateral de $f$ en $x=a$ por la derecha, escrita $f'(a^+)$, si existe:

\[f'(a^+) = \lim_{h\to 0^+} \frac{f(x+h)-f(x)}{h}\]

La derivada lateral de $f$ en $x=a$ por la izquierda, escrita $f'(a^-)$, si existe:

\[f'(a^-) = \lim_{h\to 0^-} \frac{f(x+h)-f(x)}{h} =  \lim_{h\to 0^+} \frac{f(x-h)-f(x)}{-h}\]

\obs Esta última igualdad se debe a que $h\to 0^-\implies h<0$. Si resulta menos confuso, puede elegirse trabajar siempre con $h>0$ y así los signos quedan explicitados.
\end{defn}

\begin{problem} Estudia la derivabilidad en $x=0$ de la función 
\label{prb::derivab1}
\[f(x) = \begin{cases} x^2+3x & \text{ si } x\leq 0\\ 3·\left(\frac{x^2+x}{x+1}\right)&\text{ si } x>0\end{cases}\]
\solution

$f(x)$ será derivable en $x=0$ si $\exists f'(0)$. Dado que $f(x)$ está definida a trozos, calculamos la derivadas laterales.

\[f'(0^-) = \lim_{h\to 0^+} \frac{f(0-h)-f(0)}{-h} = \lim_{h\to 0^-} \frac{(0-h)^2+3·(0-h)-(0^2+3·0)}{-h} = \lim_{h\to 0^-} \frac{h(h-3)}{-h} = +3\]
\[f'(0^+) = \lim_{h\to 0^+} \frac{f(0+h)-f(0)}{h} = \lim_{h\to 0^+} \frac{3·\frac{(0+h)^2+(0+h)}{0+h+1} - (0^2-3·0)}{h} = \lim_{h\to 0^+} \frac{3·\frac{h^2+h}{h+1}}{h} = \]
\[=\lim_{h\to 0^+} \frac{3·h·(h+1)}{h(h+1)} = 3 \]

\textbf{Conclusión:} Dado que  $f'(0^-) \eq f'(0^+) \implies f'(0) = 3$, por lo que la función es derivable en $x=0$.

\begin{figure}[h!]
\centering
\includegraphics[scale=0.5]{img/DerivabilidadEjer1}
\label{fig::DerivabEjer1}
\caption{Representación gráfica del problema \ref{prb::derivab1}.}
Claramente la derivada en $x=0$ no puede existir, dado que la función no es continua.
\end{figure}

\obs También podríamos haber calculado la derivada lateral por la izquierda de la siguiente manera:

\[f'(0^-) = \lim_{h\to 0^-} \frac{f(0+h)-f(0)}{h} = \lim_{h\to 0^-} \frac{h^2+3·h-(0^2+3·0)}{h} = \lim_{h\to 0^-} \frac{h(h+3)}{h} = +3\]

\end{problem}

\begin{problem}[Página 43, 14.]
¿Es la siguiente función derivable en $x=-2,x=0,x=2$?

\[f(x) = \begin{cases}
x^2 & \text{ si } x<-2\\
-4(x+1) & \text{ si } -2<x\leq0\\
3x^2-4 & \text{ si } 0<x\leq2\\
12x+1 & \text{ si } x>2
\end{cases}\]
\solution
\end{problem}

\begin{problem}[Página 56, ejercicio 78.]
Dada la función $f(x)$, calcula $a,b,c\in\real$ para que la función sea derivable en $x=1$, sabiendo que $f(0) = f(4)$.
\solution

Solución: $a=-\rfrac{7}{4}; b=1; c=\rfrac{1}{4}$

\begin{figure}[h!]
\centering
\includegraphics[scale=1.1]{img/DerivabilidadEjer5678}
\label{ejercicioDerivabilidad}
\caption{Ejercicio sacado del libro de SM}
\end{figure}

\end{problem}


\subsubsection{Función derivada}

\begin{defn}[Función derivada]
Dada $\appl{f}{D(f)\subset\real}{\real}$. Sea $Dv(f)$ el dominio de derivabilidad de $f$.

La función derivada denotada por $\appl{f'(x)}{Dv(f)\subset\real}{\real}$ hace corresponder a cada $a\in Dv(F)$ el valor $f'(a)$.
\end{defn}

% \begin{table}[hbp]
% \centering
% \begin{tabular}{|c|c|}\hline
% Función & Derivada\\
% \hline
% a&b\\\hline
% \end{tabular}
% \caption{Tabla de derivadas}
% \label{tbl::Derivadas}
% \end{table}

\begin{problem}

\begin{itemize}
	\item 51.58 (2 trigonométricas inversas)
	\item 59.113 (8 variadas. Solo la h tiene una trigonométrica inversa.)
	\item 57.79-81 (Ejercicios resueltos)
\end{itemize}

\solution
\end{problem}


\subsection{Aplicaciones de la derivada}

\subsubsection{Recta tangente y recta normal}

\paragraph{Ecuación de la recta tangente:} Utilizando la ecuación de la recta punto-pendiente y la interpretación gráfica de la derivada (ver \ref{fig::funinterpretacionderivadapunto}), se obtiene fácilmente la siguiente ecuación:

\begin{mdframed}
	\begin{equation}
		\label{eq::rectatangente}
		\text{Recta tangente a }f(x)\text{ en }x_0 \to y-f(x_0) = f'(x_0)·(x-x_0)
	\end{equation}
\end{mdframed}

\begin{problem}
Demuestra que la recta $y=-x$ es tangente a la curva dada por la ecuación: $y=x^3+6x^2+8x$
\solution

Consideramos $f(x) = x^3-6x^2+8x \to f'(x) = 3x^2-12x+8$.

Buscamos $c\in\real\tq f'(c) = -1$.

\[
	3c^2+12c+8=-1 \dimplies \begin{cases}c_1 = 1\\c_2=3\end{cases}
\]

Los posibles puntos de tangencia son $P_1(c_1,f(c_1)) = (1,3)$ y $P_2(c_2,f(c_2)) = (3,-3)$.

Es necesario comprobar que dichos puntos son realmente de tangencia, es decir, que pertenecen a la recta y a la gráfica.

\[P_1: 3\neq -1 \implies \text{ no pertenece a la recta}\]
\[P_2: -3\eq -3 \implies \text{ sí pertenece a la recta}\]

\textbf{Conclusión: } El punto de tangencia de la recta $y=-x$ a la gráfica $f(x) = x^3-6x^2+8x$ es $P_2(3,-3)$
\end{problem}

\paragraph{Ecuación de la recta normal:} Dos rectas dadas, en 2 dimensiones, $r: y=m_rx+n_r\quad;\quad s:y=m_sx+n_s$ son perpendiculares si y sólo si $m_r·m_s = -1 \dimplies m_s = \rfrac{-1}{m_r}$.
%
Aplicando este resultado a la fórmula de la recta tangente anterior, tenemos:


\begin{mdframed}
	\begin{equation}
		\label{eq::rectatangente}
		\text{Recta normal a }f(x)\text{ en }x_0 \to y-f(x_0) = \rfrac{-1}{f'(x_0)}·(x-x_0)
	\end{equation}
\end{mdframed}


\subsubsection{Teoremas de derivabilidad}

\begin{theorem}[Teorema\IS de Rolle]
\[
\left.
	\begin{array}{c}
		f(x)\text{ continua en } [a,b]\\
		f(x)\text{ derivable en } (a,b)\\
		f(a)=f(b)
	\end{array}
\right\}\implies \exists c\in(a,b)\tq f'(c)=0
\]
\end{theorem}
\obs ¿Puede haber más de un punto? ¡Claro que sí! Basta pensar en una función horizontal.

\begin{problem}
% Página 66.1
Sea $f(x) = 2x^5+x+a$.  Demuestra que $\exists!c\in\real\tq f(c)=0$
\solution
Por el teorema de Bolzano, hay al menos una raíz real.

\ul{Veamos que es única.} Si hubiera otra raíz real, $d$, tendríamos $f(c) = f(d)$ en una función continua en $[c,d]\subset\real$ y derivable en $(c,d)\subset\real$ por ser una función polinómica.
%
Así, se puede aplicar el teorema de Rolle, argumentando que $\exists c\in\real\tq f'(c) = 0$, pero $f'(x) = 10x^4+1 \neq 0 \;\forall x\in\real$.

\textbf{Conclusión: } dado que $f(x)$ cumple todas las hipótesis del teorema de Rolle, debemos concluir necesariamente que no hay otra raíz real $d$\footnote{Sino, el teorema fallaría y eso no es posible.}. 
\end{problem}

\begin{problem}
67.8
\solution
\end{problem}

\begin{theorem}[Teorema\IS del valor medio]
\[
\left.
	\begin{array}{c}
		f(x)\text{ continua en } [a,b]\\
		f(x)\text{ derivable en } (a,b)\\
	\end{array}
\right\}\implies \exists c\in(a,b)\tq f'(c)=\frac{f(b)-f(a)}{x-a}
\]

\obs El teorema de Rolle es un caso particular de este teorema que se da cuando $f(a) = f(b)$
\end{theorem}
\obs ¿Puede haber más de un punto? ¡Claro que sí!

\begin{problem}
Considera la función $f(x) = x·e^{-x^2}$.

Calcula el valor del parámetro real $a$ para el que se puede aplicar el teorema de Rolle en el intervalo $[0,1]$ a la función $g(x) = f(x) + ax$
\solution

\[g(x) = x·e^{-x^2} + ax = x·\left(a+e^{-x^2}\right)\]

Las hipótesis del teorema de Rolle son:

\[
\left.
	\begin{array}{l}
		g(x)\text{ continua en } [0,1] \to \text{ En este caso se cumple por ser suma de función polinómica y exponencial}\\
		g(x)\text{ derivable en } (0,1) \to \text{ En este caso se cumple por ser suma de función polinómica y exponencial}\\
		g(a)=g(b) 
	\end{array}
\right\}
\]

Necesitamos que $g(0) = g(1)$. 

\[0·\left(a+e^{-0^2}\right) = 1·\left(a+e^{-1^2}\right) \dimplies 0 = a+e^{-1} \dimplies a=-e^{-1}\]

\obs Este problema salió en las noticias de 2019 porque se preguntó en la EVAU de valencia y los alumnos se enfadaron mucho.
\end{problem}

\begin{problem}

\ppart[17.2-MadA] Estudia la derivabilidad en $x=0$ de $f(x) = \begin{cases}\displaystyle x·e^{2x} &\mbox{ si } x<0\\ \displaystyle\frac{\ln(x+1)}{x+1}&\mbox{ si } x\geq 0\end{cases}$ en $x=0$. 
\ppart[16.1-MadB] Determina el polinomio $f(x)$, sabiendo que $\forall x\in\real f'''(x) = 12$ y además verifica $f(1) = 3; f'(1) = 1; f''(1) = 4$.
\ppart[16.1-MadB] Estudie la continuidad y la derivabilidad en $x=0$ y en $x=1$ de $f(x) = \begin{cases} 0& \text{ si } x\leq 0\\|x\ln(x)&\text{ si} x>0\end{cases}$
\solution
\end{problem}

\subsubsection{Regla de L'Hôpital}

\subsubsection{Monotonía}

\begin{theorem}[Teorema\IS de monotonía]
\end{theorem}
\begin{theorem}[Teorema\IS de extremos relativos]
\end{theorem}

\subsubsection{Optimización}


\begin{theorem}[Teorema\IS de Weierstrass]
Si $f(x)$ es continua en $[a,b]$, entonces tiene un máximo y un mínimo absolutos en $[a,b]$.
\end{theorem}


\section{Análisis sistemático de una función}

\section{Integrales}

\subsection{Cálculo de áreas de recintos cerrados}


\printindex
\end{document}
\grid
