\documentclass[palatino,nosec]{apuntes}

\title{Derivadas}
\author{}
\date{16/17 C2}


\newcommand{\sen}{\operatorname{\sen}}
\newcommand{\arcsen}{\operatorname{\arcsen}}
\newcommand{\tg}{\operatorname{\tg}}
\newcommand{\arctg}{\operatorname{\arctg}}


% Paquetes adicionales

% --------------------

\begin{document}
%\pagestyle{plain}
%\maketitle

%\tableofcontents
%% Contenido.
%\Large
\section*{Ejercicio 193.95}

\begin{itemize}

\item[\textbf{a)}] $\displaystyle f(x) = \sen{(3x^2)}$

Utilizamos: $f(u) = \sen{u} \to f'(u) = \cos{(u)}·u'$, $f(u) = k·u, k\in\real \to f'(u) = k·u'$ y $f(x) = x^n \to n·x^{n-1}$

\[
	f(x) = \sen{(3x^2)} \to f'(x) = \cos{(3x^2)}·(3x^2)' = \cos{(3x^2)}·3·(x^2)' = \cos{(3x^2)}·6x
\]

\textbf{Error muy grave:}

\[
	\cos{3x^2}·6x \textcolor{red}{\neq} \cos{18x^3}
\]
¿Cómo evitarlo? Utilizando paréntesis, para indicar claramente a qué afecta el coseno. (Lo mismo para $\sen(x)$,$\tg(x)$ y $\ln(x)$)

\[
	\cos{\left(3x^2\right)}·6x
\]


\item[\textbf{b)}] $\displaystyle f(x) = \cos{(x^2+1)} $

Utilizamos: $f(x) = x^n \to n·x^{n-1}$, $f(u) = \cos(u) \to f'(u) = -\sen{(u)}·u'$
. También utilizamos, $f(x) = u \pm v \to f'(x) = u' \pm v'$

\[
	f(x) = \cos{(x^2+1)} \to f'(x) = -\sen{(x^2+1)}·2x = -2x\sen{(x^2+1)}
\]

\item[\textbf{c)}] $\displaystyle f(x) = \tg{(x^2-3x)}$

Utilizamos $f(u) = \tg{u} \to f'(u) = (\tg^2(u)+1)·u'$ y $f(x) = x^n \to f'(x) = nx^{n-1}$
. También utilizamos, $f(x) = u \pm v \to f'(x) = u' \pm v'$

\[
	f(x) = \tg{(x^2-3x)} \to f'(x) = \left(\tg^2(x^2-3x)+1\right)·(2x-3)
\]


Otra posibilidad: utilizando $f(u) = \tg{u} \to f'(u) = \frac{1}{\cos^2{u}}·u'$

\[
	f(x) = \tg{(x^2-3x)} \to f'(x) = \left(\frac{1}{\cos^2(x^2-3x)}\right)·(2x-3) = \frac{2x-3}{\cos^2(x^2-3x)}
\]

\textbf{Curiosidad:} Estas 2 soluciones son exactamente la misma. Escribiendo $\tg(x) = \frac{\sen(x)}{\cos(x)}$ obtenemos:

\[
\left(\tg^2(x^2-3x)+1\right)·(2x-3) = \left(\frac{\sen^2(x^2-3x)}{\cos^2(x^2-3x)}+1\right)·(2x-3) =
\]
\[
= \left(\frac{\sen^2(x^2-3x)+\cos^2(x^2-3x)}{\cos^2(x^2-3x)}\right)·(2x-3) \overset{(1)}{=}  \left(\frac{1}{\cos^2(x^2-3x)}\right)·(2x-3) = \frac{2x-3}{\cos^2(x^2-3x)}
\]

(1): Utilizando la igualdad $\sen^2(x) + \cos^2(x) = 1$.
\item[\textbf{d)}] $\displaystyle f(x) = \sen{\sqrt{x^2+3x}} $

Utilizamos: $f(u) = \sen(u) \to f'(u) = \cos(u)·u'$, $f(u) = \sqrt{u} \to f'(u) = \frac{1}{2\sqrt{u}}·u'$ y $f(x) = x^n \to f'(x) = nx^{n-1}$
. También utilizamos, $f(x) = u \pm v \to f'(x) = u' \pm v'$

\[
	f(x) = \sen{\sqrt{x^2+3x}} \to f'(x) = \cos{\left(\sqrt{x^2+3x}\right)}·\left(\sqrt{x^2+3x}\right)' \]

\[= \cos{\left(\sqrt{x^2+3x}\right)}·\frac{1}{2\sqrt{x^2+3x}}·(x^2+3x)'  = \frac{\cos{\left(\sqrt{x^2+3x}\right)}·(2x+3)}{2\sqrt{x^2+3x}}
\]


\item[\textbf{e)}] $\displaystyle f(x) =  \cos{\frac{x-1}{x}}$

Utilizamos: $f(u) = \cos{u} \to f'(u) = -\sen{(u)}·u'$, $f(x) = \frac{u}{v} \to f'(x) = \frac{u'v-uv'}{v^2}$
. También utilizamos, $f(x) = u \pm v \to f'(x) = u' \pm v'$

\[
	f(x) =  \cos{\frac{x-1}{x}} \to f'(x) = -\sen{\left(\frac{x-1}{x}\right)}·\underbrace{
	\frac{x-(x-1)}{x^2}}_{u'} = \frac{-\sen{\frac{x-1}{x}}}{x^2}
\]




\item[\textbf{f)}] $\displaystyle f(x) = \tg{\sqrt{x-1}} $

Utilizamos $f(u) = \tg{u} \to f'(u) = (\tg^2(u)+1)·u'$ y $f(u) = \sqrt{u} \to f'(u) = \frac{1}{2\sqrt{u}}·u'$.
 También utilizamos, $f(x) = u \pm v \to f'(x) = u' \pm v'$

\[
	f(x) = \tg{\sqrt{x-1}} \to f'(x) = \left(\tg^2(\sqrt{x-1})+1\right)·\frac{1}{2\sqrt{x-1}}·1 = \frac{\tg^2(\sqrt{x-1})+1}{2\sqrt{x-1}}
\]

Otra posibilidad: utilizando $f(u) = \tg{u} \to f'(u) = \frac{1}{\cos^2{u}}·u'$

\[
	f(x) = \tg{\sqrt{x-1}} \to f'(x) = \left(\frac{1}{\cos^2(\sqrt{x-1})}\right)·\frac{1}{2\sqrt{x-1}}·1 = \frac{1}{2·\sqrt{x-1}·\cos^2(\sqrt{x-1})}
\]

\item[\textbf{g)}] $\displaystyle f(x) = -\sen{\frac{x}{-x^4+x-1}} $

Utilizamos: $f(u) = \sen(u) \to f'(u) = \cos(u)·u'$ y $f(x) = \frac{u}{v} \to f(x) = \frac{u'v-uv'}{v^2}$
. También utilizamos, $f(x) = u \pm v \to f'(x) = u' \pm v'$

\[
	f(x) = -\sen{\frac{x}{-x^4+x-1}} \to f'(x) = -\cos{\left( \frac{x}{-x^4+x-1}\right)}·\frac{(-x^4+x-1)-x·(-4x^3+1)}{(-x^4+x-1)^2} =
	\]

\[ 	
 	-\cos{\left( \frac{x}{-x^4+x-1}\right)}·\frac{-x^4+x-1+4x^4-x}{(-x^4+x-1)^2} =
    -\cos{\left( \frac{x}{-x^4+x-1}\right)}·\frac{3x^4-1}{(-x^4+x-1)^2}
\]

\newpage
\item[\textbf{h)}] $\displaystyle f(x) = \tg{\frac{2}{\sqrt{1-x}}} $

Utilizamos $f(u) = \tg{u} \to f'(u) = (\tg^2(u)+1)·u'$, $f(u) = u^n \to f'(u) = n·u^{n-1}·u'$, $f(x) = u \pm v \to f'(x) = u' \pm v'$  y $f(u) = k·u, k\in\real \to f'(u) = k·u'$.

Podemos darnos cuenta que de la tangente no tenemos un cociente sino: $2·(1-x)^{\rfrac{-1}{2}}$.


\[
	f(x) = \tg{\frac{2}{\sqrt{1-x}}} \to f'(x) = \left[\tg^2\left(\frac{2}{\sqrt{1-x}}\right)+1\right] · \left(2·(1-x)^{\rfrac{-1}{2}}\right)' =\]

\[  \left[\tg^2\left(\frac{2}{\sqrt{1-x}}\right)+1\right] · 2\frac{\textcolor{red}{(-1)}}{2}·(1-x)^{\rfrac{-1}{2}-1}·\textcolor{red}{(-1)} = +\left[\tg^2\left(\frac{2}{\sqrt{1-x}}\right)+1\right] ·(1-x)^{\rfrac{-3}{2}}
\]

\[
	f'(x) = \left[\tg^2\left(\frac{2}{\sqrt{1-x}}\right)+1\right] \frac{1}{\sqrt{(1-x)^3}}
\]

Otra posibilidad: utilizando $f(u) = \tg{u} \to f'(u) = \frac{1}{\cos^2{u}}·u'$



\[
	f(x) = \tg{\frac{2}{\sqrt{1-x}}} \to f'(x) = \left[\frac{1}{\cos^2\left(\frac{2}{\sqrt{1-x}}\right)}\right] · \left(2·(1-x)^{\rfrac{-1}{2}}\right)' =\]

\[  \left[\frac{1}{\cos^2\left(\frac{2}{\sqrt{1-x}}\right)}\right] · 2\frac{\textcolor{red}{(-1)}}{2}·(1-x)^{\rfrac{-1}{2}-1}·\textcolor{red}{(-1)} = +\left[\frac{1}{\cos^2\left(\frac{2}{\sqrt{1-x}}\right)}\right] ·(1-x)^{\rfrac{-3}{2}}
\]

\[
	f'(x) = \frac{1}{\cos^2\left(\displaystyle\frac{2}{\sqrt{1-x}}\right)·\sqrt{(1-x)^3}}
\]


\end{itemize}


%% Apendices (ejercicios, examenes)
%\appendix\chapter{---}% -*- root: ../Derivadas.tex -*-




%\section*{Ejercicio 281.16}

%\begin{itemize}
%	\item $3x^2·\log_2(x)$
%	\item $f(x) = e^x\sen(x)$
%	\item $f(x) = \sqrt[3]{x}·\cos(x)$
%	\item $f(x) = \cos(x)·\tg(x)$
%	\item $f(x) = 4x·\sen(x) + x^3·\cos(x)$
%	\item $f(x) = \ln(x)\frac{1}{x^4} + x^2·e^x$
%	\item $f(x)  = (\sen(x) - \cos(x)) ·\tg(x)$
%	\item $f(x) = 4x\sqrt{x} + \frac{\sen(x)}{x}$

%\end{itemize}

\printindex
\end{document}
\grid
