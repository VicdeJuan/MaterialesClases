\documentclass[palatino,nochap]{Docencia}


\usepackage{breqn}


\title{Cuaderno de clase}
\author{Víctor de Juan}
\date{19/20}

\begin{abstract}
Cuaderno de clase de Matemáticas I, con el desarrollo continuado (sin estar separado por sesiones).
\end{abstract}

% Paquetes adicionales

\usepackage[author={Víctor de Juan, 2018}]{pdfcomment}

\makeatletter
\newcommand{\annotate}[2][]{%
\pdfstringdef\x@title{#1}%
\edef\r{\string\r}%
\pdfstringdef\x@contents{#2}%
\pdfannot
width 2\baselineskip
height 2\baselineskip
depth 0pt
{
/Subtype /Text
/T (\x@title)
/Contents (\x@contents)
}%
}
\makeatother

% --------------------
\newcommand{\cimplies}{\text{\hl{$\implies$}}}
\renewcommand{\vx}{\overset{\rightarrow}{x}}
\renewcommand{\vy}{\overset{\rightarrow}{y}}
\renewcommand{\vz}{\overset{\rightarrow}{z}}
\newcommand{\vi}{\overset{\rightarrow}{i}}
\newcommand{\vj}{\overset{\rightarrow}{j}}
\renewcommand{\vec}[1]{\overset{\rightarrow}{#1}}

\usepackage{pgf,tikz}
\usetikzlibrary{arrows}

\begin{document}
\pagestyle{plain}


\section{Derivabilidad}

\begin{defn}[Derivada\IS en un punto]
Sea $\appl{f}{ℝ}{ℝ}$ una función continua. Se define la derivada de $f$ en el punto $a$ como:

\[
	f'(a) = \lim_{x\to a}\frac{f(x)-f(a)}{x-a}
\]
\end{defn}


\obs Una manera alternativa de escribir el límite de la derivada es:

\[f'(a) = \lim_{x\to a}\frac{f(x)-f(a)}{x-a} \dimplies f'(a) = \lim_{h\to 0} \frac{f(a+h) - f(a)}{h}\]


\obs $f$ \concept[Función\IS derivable]{derivable} en $x=a \dimplies \exists f'(a)$
\obs A la función inicial la llamaremos primitiva.

\begin{example}
Cálculo de la derivada de $f(x) = x^2+2x-1$ en $x=3$.

\[
	f'(3) = \lim_{x\to3}\frac{x^2+2x-1 - (3^2+2·3-1)}{x-3} = \lim_{x\to3}\frac{x^2+2x-15}{x-3} = \lim_{x\to3}\frac{(x-3)(x+5)}{x-3} = \lim_{x\to3}(x+5) = 8
\]

\end{example}

\begin{example}
Cálculo de la derivada de $f(x) = |x|$ en $x=0$.

\[
	f'(0) = \lim_{x\to0}\frac{|x| - |0|}{x-0} = \lim_{x\to0}\frac{|x|}{x} = \left\{\begin{array}{l}\displaystyle\lim_{x\to0^+}\frac{x}{x} = 1 \\ \displaystyle\lim_{x\to0^-}\frac{-x}{x}=-1\end{array}\right\}\implies \nexists\lim_{x\to0}f(x)
\]

Conclusión: La función $f(x)$ no es derivable en $x=0$. ¿Es continua? Sí.
\end{example}

\begin{example}
Cálculo de la derivada de $f(x) = \frac{1}{x}$ en $x=0$.

\[
	f'(0) = \nexists\lim_{x\to0}\frac{\rfrac{1}{x}-\rfrac{1}{0}}{x-0}
\]

No es derivable. ¿Es continua? No.
\end{example}

\obs \textbf{Derivable} $\implies$ \textbf{Continua}. \textit{Para que una función sea derivable en un punto es necesario que sea continua en ese punto}

¿En física/economía habéis derivado polinomios? ¿Cuál sería la "derivada" del polinomio anterior $f(x) = x^2+2x-1$? $f'(x) = 2x+2$. ¿Cuánto vale $f'(3)$? $f'(3) = 2·3+2=8$. ¿Casualidad?

Pero... ¿porqué esto es cierto? ¿De dónde sale ese $2x+2$? Esto es a lo que llamamos la función derivada:


\begin{defn}[Función\IS derivada]
Sea $\appl{f}{ℝ}{ℝ}$ una función continua. Se define la función derivada de $f$ como la función que a cada punto le asigna el valor de su derivada.

\[
	f'(x) = \lim_{h\to 0}\frac{f(x+h)-f(x)}{h}
\]
\end{defn}

\begin{example}
Cálculo de la función derivada $f'(x)$ de un polinomio, por ejemplo $f(x)=x^2+2x-1$

\[
	f'(x) = \lim_{h\to0}\frac{f(x+h)-f(x)}{h} = \lim_{h\to0}\frac{(x+h)^2+2(x+h)-1 - (x^2+2x-1)}{h} =
\]
\[
	\lim_{h\to0}\frac{x^2+2xh+h^2+2x+2h-1 - x^2-2x+1}{h} = \lim_{h\to0}\frac{2xh+h^2+2h}{h} =
\]
\[
	\lim_{h\to0}\frac{h(2x+h+2)}{h} = \lim_{h\to0} 2x+h+2 = 2x+2
\]

Conclusión: la función derivada de $f(x) = x^2+2x-1$ es $f'(x) = 2x+2$.
\end{example}

\hl{Explicación de la tabla de derivadas}

\paragraph{Propiedades de la derivada:}
\begin{prop}[Cálculo operativo]
	Sean $f, g$ derivables en $a$. Entonces
	\begin{itemize}
		\item $(k·f(x))' = k·f'(x) ∀k\in\real$
		\item $(f\pm g)'(x)=f'(x)\pm g'(x)$
		\item $(fg)'(x)=f'(x)g(x)+f(x)g'(x)$
		\item Si $g(x)\neq 0 $, $\left(\frac{1}{g}\right)'(x)=\frac{-g'(x)}{(g(x))^2}$
		\item Si $g(x)\neq 0$, $\left(\frac{f}{g}\right)'(x)=\frac{f'(x)g(x)-f(x)g'(x)}{(g(x))^2}$
		\item $(g\circ f)'(x)= \left(g(f(x))\right)' = g'(f(x))f'(x)$
	\end{itemize}
\end{prop}


\paragraph{Regla de la cadena}

\newpage

\paragraph*{A practicar:}
\begin{itemize}
	\item $\displaystyle f(x) = x\sqrt[3]{x^2}$
	\item $\displaystyle f(x) = x·e^x$
	\item $\displaystyle f(x) = x·\sen(x)$
	\item $\displaystyle f(x) = x·\ln(x)$
	\item $\displaystyle f(x) = \sen(x)·\cos(x)$
	\item $\displaystyle f(x) = \frac{x^2+1}{x}$
	\item $\displaystyle f(x) = (x^2+2x)·\sen(x)$
	\item $\displaystyle f(x) = (e^x - x)·\ln{x}$
	\item $\displaystyle f(x) = \sqrt{x^2+x}$
	\item $\displaystyle f(x) = (\arcsen{x})^3$
	\item $\displaystyle f(x) = \ln(4x)$
	\item $\displaystyle f(x) = (\cos{x})^2 = \cos^2{x}$
	\item $\displaystyle f(x) = \sen{(3x^2)}$
	\item $\displaystyle f(x) = \cos{(x^2+1)} $
	\item $\displaystyle f(x) = \tg{(x^2-3x)}$
	\item $\displaystyle f(x) = \sen{\sqrt{x^2+3x}} $
	\item $\displaystyle f(x) =  \cos{\frac{x-1}{x}}$
	\item $\displaystyle f(x) = \tg{\sqrt{x-1}} $
	\item $\displaystyle f(x) = -\sen{\frac{x}{-x^4+x-1}} $
	\item $\displaystyle f(x) = \tg{\frac{2}{\sqrt{1-x}}} $
\end{itemize}


\end{document}
\grid
