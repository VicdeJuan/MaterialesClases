\documentclass[nobuilddate]{Docencia}

\title{Matemáticas II - Apuntes de clase}
\author{Departamento de Matemáticas}
\date{2019-2020}


\usepackage{tikz}
\usepackage{tikz-3dplot}
\usepackage{fancysprefs}
\usepackage{svg}


\usetikzlibrary{calc,patterns,angles,quotes}

\renewcommand{\ve}[1]{\ensuremath{\mathbf{#1}}}
\newcommand{\ud}[0]{\mathrm{d}}
\newcommand{\lgdlp}{Lugar geométrico de los puntos del espacio que equidistan de}


\tikzset{
    vector/.style = {
        thick,
        > = stealth',
    },
    axis/.style = {
        very thin,
        > = stealth',
    },
}


\begin{document}

\pagestyle{plain}
\maketitle
\tableofcontents
\newpage

\newcommand{\hide}[1]{#1}

\renewcommand{\vec}[1]{\overrightarrow{#1}}

\paragraph{Introducción}

\begin{itemize}
    \item 2º de Bachillerato no es un curso para preparar la EVAU. Es un curso orientado a la universidad y es para eso para lo que te vamos a preparar. 
    %
    Ya tendremos tiempo a final de curso para preparar la EVAU.

    \item Temario muy extenso (¿Aquí también? Sí hijo sí...). Cuidado.
    
    \item 2º de Bachillerato no es solo estudiar. Sed responsables y organizados para poder disfrutar del curso. ¡Tratad bien a vuestras familias! Que los exámenes no os hagan ser unos amargados.
    
    \item A vuestra disposición por correo y por el classroom para lo que necesitéis. Colgaremos en el classroom los enunciados de los exámenes, las soluciones, etc.

    \item Metodología de clase.
    
    \item Criterios de evaluación.

    \subitem 10\%, 30\%, 60\%. Estudiar al día. Tutorías a demanda. Cosas que tenéis que saber, aunque, seminarios.

    \subitem Pruebecita del lunes, tal vez con teoría.

\end{itemize}

\chapter{Probabilidad y Estadística}

\section{Probabilidad}
% Página 67 del cuaderno en papel

Los sucesos en probabilidad se escriben entre comillas y se representan con una letra mayúscula.

\begin{itemize}
    \item[m] $I =$ impar
    \item[m] $I =$ probabilidad de sacar un número impar
    \item[m] $I =$ "Probabilidad de sacar un número impar"
    \item[b] $I =$ "Sacar un número impar"
\end{itemize}

\textbf{Operaciones con conjuntos (o sucesos)}

$A\cup B = \{x\in A \vee x \in B\}$ (Unión)

$A\cap B = \{x\in A \wedge x \in B\}$ (Intersección - Probabilidad compuesta)

$A^c = \overline{A} = \{x\not\in A$ (Complementario) \textit{No se llama contrario}.

$A - B = \{x\in A \wedge x \not\in B\}$ (Diferencia)

\obs $A - B = A \cap \overline{B}$

\begin{defn}[Leyes\IS de de Morgan]
\[\overline{A\cap B} = \overline{A}\cup \overline{B}\]
\[\overline{A\cup B} = \overline{A}\cap \overline{B}\]
\end{defn}



\begin{defn}[Compatibilidad]
Sean $A,B$ dos sucesos.

Son compatibles si $A\cap B \not= \emptyset$. 
Son incompatibles si $A\cap B =\emptyset$
\end{defn}

\begin{defn}[Sistema completo de sucesos]
Sean $A_1,...,A_n$ sucesos de un cierto experimento aleatorio.
%
Se dice que forman un sistema completo de sucesos del espacio muestral $E$ cuando:

\begin{itemize}
    \item $\displaystyle\bigcup_{i=1}^n A_i = A_1\cup A_2 \cup ... \cup A_n =  E$
    \item $A_i\cap A_j = \emptyset\;\;\; \forall i,j=1...n$
\end{itemize}
\end{defn}

\begin{example}
Sea $E = \{1,2,3,4,5,6\}$. ¿Son sistemas completos de sucesos las siguientes agrupaciones?

\begin{itemize}
    \item $A_1 = \{1,2,3\}; A_2 = \{4,5\} ; A_3 = \{6\}$
    \item $A_1 = \{\text{múltiplos de 3}\} ; A_2 = \{\text{números pares}\} ; A_3 = \{1,5\}$
\end{itemize}
\end{example}


\begin{prop}Sea $A_1,...,A_n$ un sistema completo de sucesos. Entonces \[\displaystyle\sum_{i=1}^n P(A_i) = 1\]
\end{prop}


\begin{defn}[Regla de Laplace]
Si los socesos elementales de un experimento aleatorio son equiprobables, entonces, $P(A) = \frac{\text{casos favorables}}{\text{casos posibles}}$
\end{defn}
\obs ¿Y si no son equiprobables? 

\begin{defn}[Probabilidad\IS Ley de los grandes números]
Sea $A$ un suceso y $h(A)$ su frecuencia de ocurrencia relativa\footnote{El porcentaje de veces que ese suceso ocurre.} en $n$ repeticiones del experimento. Entonces \[P(A) = \lim_{n\leftrightarrow \infty}h(A)\]
\end{defn}


\begin{defn}[Probabilidad\IS Axiomática de Kolmogorov]
    Sea $p$ una función que asocia a cada suceso $A$ del espacio de sucesos $S$ un número real designado por $p(A)$.
    
    Decimos que $p$ es una probabilidad si cumple las siguientes propiedades:
    \begin{itemize}
        \item $0\leq p(A) \leq 1 \;\;\;\forall A\in S$
        \item $p(E) = 1$
        \item $A\cap B = \emptyset \implies p(A\cup B) = p(A) + p(B)$
    \end{itemize}
\end{defn}

\paragraph{Propiedades de la probabilidad:} Sea $A$ un suceso cualquiera:
\begin{itemize}
    \item $P(\overline{A}) = 1 - P(A)$
    \item $A\subset B \implies P(A) \leq P(B)$
    \item $0\leq P(A) \leq 1$
    \item $P(A\cup B) = P(A) + P(B) - P(A\cap B)$
\end{itemize}

\begin{center}
\includegraphics[scale=0.3]{img/Venn-diagram-visualization-of-a-3-event-probability-space-O.png}
\includegraphics[scale=0.3]{img/Venn-diagram-visualization-of-a-3-event-probability-space-O.png}
\includegraphics[scale=0.3]{img/Venn-diagram-visualization-of-a-3-event-probability-space-O.png}
\includegraphics[scale=0.3]{img/Venn-diagram-visualization-of-a-3-event-probability-space-O.png}
\end{center}

\begin{defn}[Probabilidad\IS condicionada]
Sean $A,B$ sucesos de un suceso aleatorio. 

Se define la probabilidad condicionada $p(A/B)$ como la probablidad de que se produzca $A$ si sabemos que se ha producido $B$.

\[P(A/B) = \frac{P(A\cap B)}{P(B)}\]
\end{defn}

\textit{Este es un buen momento para hacer algún ejercicio. Página 349, ejer 36 por ejemplo. Recordamos que $P(A\cup B) = P(A) + P(B) - P(A\cap B)$}

\begin{defn}[Independencia de sucesos]
$A,B$ son sucesos independientes si y sólo si \[P(A/B) = P(A/\overline{B}) = P(A)\]
\end{defn}

\begin{theorem}
\[A,B \text{ independientes } \dimplies P(A\cap B) = P(A)·P(B)\]
\end{theorem}
\begin{proof}
\[\left.\begin{array}{c}
P(A/B) = \frac{P(A\cap B)}{P(B)} \dimplies P(A\cap B) = P(B)·P(A/B)\\A,B \text{ independientes } \dimplies P(A/B) = P(A)\end{array}\right\}*\]
\[(*) \implies P(A\cap B) = P(B)·\underset{P(A/B)}{P(A)} = P(B)·P(A)\]
\end{proof}

\begin{theorem}[Teorema\IS Probabilidad Total]
Sea $A_1,A_2,...,A_n$ un sistema completo de sucesos y sea $B$ otro suceso.
\[  
    P(B) = P(A_1\cap B) + P(A_2\cap B) + ... + P(A_n\cap B) = 
\]
\[
    P(B/A_1)·P(A_1) + P(B/A_2) · P(A_2) + ... + P(B/A_n)·P(A_n)
\]
\end{theorem}
\begin{theorem}[Teorema\IS de Bayes]
\[P(B/A) = \frac{P(A/B)·P(B)}{P(A)}\]
\end{theorem}
\begin{proof}
\[
\left.
\begin{array}{c}
    P(A\cap B) = P(A)·P(B/A)\\
    P(A\cap B) = P(B)·P(A/B)
\end{array}\right\} \implies P(A)·P(B/A) =  P(B)·P(A/B) \dimplies \]\[\dimplies P(B/A) = \frac{P(A/B)·P(B)}{P(A)} 
\]
\end{proof}


\paragraph{Bayesian trap (obtenido de youtube.com/Veritasium)}

\begin{example}
Se sabe que una enfermedad rara sólo afecta al 1\% de la población. 
%
El porcentaje de falsos positivos de las pruebas médicas es del 10\%, y el porcentaje de falsos negativos es del 1\%.

Intuitivamente, ¿cuál dirías que es la probabilida de tener la enfermedad sabiendo que has dado positivo en el test? ¿Podrías calcularlo numéricamente?

Los datos son: $P(E) = 0.01$, $P(-|E) = 0.01$ y $P(+|\overline{E})=0.1$

Aplicando el teorema de Bayes:

\[P(E|+) = \frac{P(+|E)·P(E)}{P(+)} = \frac{P(+|E)·P(E)}{P(+|E)·P(E) + P(+|\overline{E})·P(\overline{E})}= 0.01 \]
\end{example}

\section{Distribuciones de probabilidad}

Introducimos el tema con 4 preguntas que iremos contestando a lo largo del tema. Acabaremos el tema cuando sepamos contestar a las 4 preguntas.

\begin{itemize}
    \item ¿Probabilidad de sacar 2 caras en 4 lanzamientos? Lo resolvemos como un árbol.
    \item ¿Probabilidad de sacar 2 caras en 10 lanzamientos? Binomial. Contestamos esta pregunta y deducimos la fórmula de la binomial. Algunos ejemplos más para interiorizar la intuición de la binomial.
    \item ¿Probabilidad de sacar exactamente 120 caras en 2000 lanzamientos? Casi 0.
    \item ¿Probabilidad de sacar menos de 120 caras en 2000 lanzamientos? Normal.
\end{itemize}


\begin{example}

Tiramos un dado de 6 caras 10 veces.
\begin{itemize}
    \item Calcula la probabilidad de sacar 1 uno.
    \item Calcula la probabilidad de sacar 2 unos.
\end{itemize}

Se resuelve utilizando la lógica y la combinatoria. 

\end{example}

Desde el ejemplo, generalizamos para obtener la fórmula general.

2 ejercicios de cálculo de binomial para entender la fórmula. 

Dada la fómrula, desde el libro, vamos viendo las definiciones de probabilidad. Esperanza(media) y varianza: ¿cuántas caras podemos esperar en 10 lanzamientos? Si el experimento de tirar 10 monedas lo hago 10000 veces, podría hacer la media. 

¡Unión de la probabilidad y de la estadística! Hago cosas de estadística (varianza, esperanza-media) con probabilidades.


\section{Distribuciones de probabilidad}

\begin{defn}[Variables aleatorias]
Una v.a. asociada a un experimento aleatorio es una variable que a cada suceso elemental del espacio muestral le asigna un número
\end{defn}
Sucesos de tirar 10 veces una moneda: "CCCCCCC..", "ZZZZ....","CZCZCZCZ...". Necesitamos asignar a cada suceso un número. 
%
Podríamos llamar "X" al número de caras obtenidas. "X" sería una variable aleatoria.

\begin{itemize}
    \item \textbf{Discretas:} puede asignar un número de valores finito. La más importante es la binomial.
    \item \textbf{Continuas: } puede asignar un número de valores infinito. La más importante es la normal.
\end{itemize}

\begin{defn}[Distribución\IS binomial]
Una variable aleatoria "X" sigue una distribución binomial con parámetros n y p (se escribe $X\equiv B(n,p)$) cuando se basa en un experimento aleatorio que se repite n veces y que sólo tiene 2 posibles resultados:
\begin{itemize}
    \item \textbf{Éxito:} que incrementa la variable aleatoria.
    \item \textbf{Fracaso:} no incrementa la variable aleatoria.
\end{itemize}
\end{defn}

\paragraph{Esperanza y varianza}
\textit{Aquí confluyen la estadística y la probabilidad ya que podemos hablar de "medias" (esperanza) y varianza de variables aleatorias. Es la intersección de estadística y probabilidad.}

La esperanza matemática es el número de éxitos que podemos esperar que ocurran si repitiéramos el experimento origen de la v.a. X un nº infinito de veces. Sería la media de los resultados: $E[X] = np$

La varianza mide la dispersión de los datos. $V[x] = np(1-p)$

Ejercicios y ejemplos. Página 369 el 9 y el 10.

\begin{defn}[Distribución\IS normal]
Una v.a. X sigue una distribución normal de parámetros $\mu$ [media\footnote{Tal vez fuera más técnico llamarla esperanza}] y $\sigma$ [desviación típica] cuando la probabilidad de que tome valores entre "a" y "b" es igual al área comprendida por la gráfica de la campana de Gauss (en función de $\mu$ y $\sigma$) entre el eje $OX$ y las rectas $x=a$ y $x=b$.
\end{defn}

\obs Esta distribución es continua.
\obs La campana de Gauss tiene esta forma y viene definida por la fórmula: \[f(\mu,\sigma) = \displaystyle\frac{1}{\sigma\sqrt{2\pi}}e^{\displaystyle-\frac{1}{2}·\left(\frac{x-\mu}{\sigma}\right)^2}\]

\begin{defn}[Función de probabilidad]
Sea $X$ una variable aleatoria. Llamamos función de probabilidad a $P(x=k)$.
\end{defn}
\obs La "formulita" de la binomial es una función de probabilidad.
\obs $X\equiv N(\mu,\sigma) \implies P(x=k) = 0$ (por ser una distribución continua).

\begin{defn}[Función de distribución]
Sea $X$ una variable aleatoria. Llamamos función de distribución $F(x) = P(x\leq k)$.
\end{defn}

\obs ¿Por qué existe la función de distribución?
%
Dado que la probabilidad puntual de la normal es prácticamente 0 en todos los casos, al trabajar con la distribución normal, trabajaremos con su función de distribución.

Para ello, utilizaremos la tabla de la distribución normal. En ella se encuentran los valores de la función de distribución dados números reales con 2 decimales.

\textit{\textbf{Explicación de la tabla} con algún ejemplo}

\obs $X\equiv N(\mu,\sigma)$. Si $P(X=k) = 0 \implies P(x<k) = P(x\leq k)$

\obs ¿Cómo utilizo la tabla si $X\equiv N(\mu,\sigma)$? 

\begin{defn}[Tipificación]
Tipificar consiste en convertir en \textit{típica} ($\mu=0,\sigma=1)$ una distribución normal cualquiera.

Si $X\equiv N(\mu,\sigma) \implies Z = \displaystyle\frac{x-\mu}{\sigma} \equiv N(0,1)$
\end{defn}

\begin{example} $X\equiv N(3,8)$. Calcula
\begin{itemize}
    \item $P(X\leq 5)$
    \item $P(X\leq 3)$
    \item $P(X>6) = P\left(\frac{X-3}{8} > \frac{6-3}{8}\right) = P(Z>\rfrac{3}{8}) = 1-P(Z\leq\rfrac{3}{8}) = ... $
    \item $P(X>2) = P\left(\frac{X-3}{8} > \frac{2-3}{8}\right) = P(Z>\rfrac{-1}{8})$ ¿Suceso contrario? No gano nada. Vamos a por otro argumento: 
    
    \concept[Simetría de la normal]{Simetría} 
    $P(Z>\rfrac{-1}{8}) = P(Z<\rfrac{1}{8}) = ...$
    \item $P(X<2) =  P\left(\frac{X-3}{8} < \frac{2-3}{8}\right) = P(Z<\rfrac{-1}{8}) = P(Z>\rfrac{1}{6}) = 1-P(Z\leq \rfrac{1}{6}
    $
    \item $P(1<X<8) = P(X<8) - P(X<1) = ...$
\end{itemize}
373.19 + problemita de la normal.

\subsection{Aproximación de la binomial por la normal}
PPT del departamento.

\paragraph{Corrección por continuidad}

\end{example}

\chapter{Álgebra (Matrices y determinantes)}

\section{Matrices}

\paragraph{Definición de matriz}

\paragraph{Operaciones con matrices}

\subparagraph{Traspuesta}

Ejercicio: \textbf{Demuestra que cualquier matriz puede escribirse como suma de una matriz simétrica y otra antisimétrica}

\subparagraph{Producto de matrices}

\subsection{Matriz inversa}
Se puede calcular de 3 formas. Definición, Gauss-Jordan y matriz adjunta. Vamos a ver ahora los 2 primeros métodos.

\paragraph{Definición y propiedades}

\subsection{Algunas ecuaciones matriciales sencillas}

\paragraph{Gauss-Jordan}
La base del método de Gauss es que toda transformación lineal de Gauss se puede expresar como una matriz. Simplemente buscamos la matriz que transforma la matriz dada en la identidad. Para ello, ponemos la identidad a la derecha. (Espero que leyendo esta explicación te hayas enterado)\footnote{Puedes encontrar \href{https://math.stackexchange.com/questions/1240055/why-does-the-gaussian-jordan-elimination-works-when-finding-the-inverse-matrix}{aquí} una respuesta más elaborada.}

\subsection{Utilidades: Grafos}

\section{Determinantes}

\begin{defn}[Determinante]
$\appl{|\;\;|}{\mathcal{M}_n}{\real}$
\end{defn}

\subsection{Cálculo de determinantes de orden 3}

\subsection{Propiedades}

\subsection{Cálculo de determinantes de orden 4 o más}

\paragraph{Menores, Gauss}

\subsection{Matriz inversa por determinantes}

\subsection{Ecuaciones matriciales a tope}

\subsection{Rango}
\paragraph{Gauss}

\paragraph{Determinantes}

Si $|A| \neq 0$, significa que no hay 2 filas (ni 2 columnas) linealmente independientes. Si las hubiera, $|A| = 0$.

Por lo tanto, si $|A|\neq 0 \dimplies rg(A) = \text{ máx}$

¿Qué ocurre si $A\not\in\mathcal{M}_n$?

\begin{example}
\[
    A=\begin{pmatrix}2&2&3&4\\4&4&2&1\end{pmatrix}
\]

En este ejemplo, cogiendo $\left|\begin{matrix}2&2\\4&4\end{matrix}\right| = 0$, pero $\left|\begin{matrix}3&4\\2&1\end{matrix}\right| \neq 0$, por lo que estas 2 filas tienen que ser linealmente independientes, por lo que la matriz tiene rango 2.

También valdría argumentarlo desde $\left|\begin{matrix}2&3\\4&2\end{matrix}\right| \neq 0$
\end{example}

\begin{prop}[Cálculo del rango por menores]
Sea $M_p$ un menor de orden $p$ de la matriz $A\in\mathcal{M}_{n\times m}$
\[\exists M^p \tlq M_p \neq 0 \dimplies rg(A) \geq p\]
\[\forall M^p \;\; M_p = 0 \dimplies rg(A) < p\]
\end{prop}

\subsubsection{Matriz de Vandermonde}
\[
V=\begin{bmatrix}
1 & \alpha_1 & \alpha_1^2 & \dots & \alpha_1^{n-1}\\
1 & \alpha_2 & \alpha_2^2 & \dots & \alpha_2^{n-1}\\
1 & \alpha_3 & \alpha_3^2 & \dots & \alpha_3^{n-1}\\
\vdots & \vdots & \vdots & \ddots &\vdots \\
1 & \alpha_n & \alpha_n^2 & \dots & \alpha_n^{n-1}\\
\end{bmatrix}\]

\paragraph{Determinante: } El determinante se calcula con la siguiente fórmula:

\[\begin{vmatrix} V \end{vmatrix}=\prod_{1 \le i<j\le n}(\alpha_j-\alpha_i)\]

\begin{example}
\[
\begin{vmatrix}
1&2&4&8\\
1&3&9&27\\
1&4&16&64\\
1&5&25&125
\end{vmatrix} = \underbrace{\overbrace{(3-2)}^{j=2}\overbrace{(4-2)}^{j=3}\overbrace{(5-2)}^{j=4}}_{i=1}\underbrace{\overbrace{(4-3)}^{j=3}\overbrace{(5-3)}^{j=4}}_{i=2}\underbrace{\overbrace{(5-4)}^{j=4}}_{i=3} = 1·2·3·1·2·1 = 6
\]
\end{example}

\begin{proof}[por Inducción]

\paragraph{Base: n=2} Es fácil notar que en el caso de una matriz de 2×2 el resultado es correcto.
\[\begin{vmatrix} V \end{vmatrix}=v_{1,1}v_{2,2} - v_{1,2}v_{2,1}=\alpha_2-\alpha_1=\prod_{1\le i<j\le 2} (\alpha_j-\alpha_i)\]

\paragraph{Paso}
Suponiendo cierta la fórmula para el caso $n-1$, procedemos a calcular el determinante de orden $n$. Para ello, basta con realizar la siguiente operación elemental sobre cada columna: $C_{j}\rightarrow C_{j}-(\alpha_1 \times C_{j-1})$. Esta operación no afecta al determinante, por lo que se obtiene lo siguiente:

\[
\begin{vmatrix} V \end{vmatrix}=\begin{vmatrix}
1 & \alpha_1 & \alpha_1^2 & \dots & \alpha_1^{n-1}\\
1 & \alpha_2 & \alpha_2^2 & \dots & \alpha_2^{n-1}\\
1 & \alpha_3 & \alpha_3^2 & \dots & \alpha_3^{n-1}\\
\vdots & \vdots & \vdots & \ddots &\vdots \\
1 & \alpha_n & \alpha_n^2 & \dots & \alpha_n^{n-1}\\
\end{vmatrix}=\begin{vmatrix}
1 & 0 & 0 & \dots & 0\\
1 & \alpha_2-\alpha_1 & \alpha_2(\alpha_2-\alpha_1) & \dots & \alpha_2^{n-2}(\alpha_2-\alpha_1)\\
1 & \alpha_3-\alpha_1 & \alpha_3(\alpha_3-\alpha_1) & \dots & \alpha_3^{n-2}(\alpha_3-\alpha_1)\\
\vdots & \vdots & \vdots & \ddots &\vdots \\
1 & \alpha_n-\alpha_1 & \alpha_n(\alpha_n-\alpha_1) & \dots & \alpha_n^{n-2}(\alpha_n-\alpha_1)\\
\end{vmatrix}
\]

Desarrollando por los adjuntos de la primera fila: 

\[\begin{vmatrix} V \end{vmatrix}=\begin{vmatrix}
\alpha_2-\alpha_1 & \alpha_2(\alpha_2-\alpha_1) & \dots & \alpha_2^{n-2}(\alpha_2-\alpha_1)\\
\alpha_3-\alpha_1 & \alpha_3(\alpha_3-\alpha_1) & \dots & \alpha_3^{n-2}(\alpha_3-\alpha_1)\\
\vdots & \vdots & &\vdots \\
\alpha_n-\alpha_1 & \alpha_n(\alpha_n-\alpha_1) & \dots & \alpha_n^{n-2}(\alpha_n-\alpha_1)\\
\end{vmatrix}\]
Extrayendo de cada fila un factor, obtenemos:
\[\begin{vmatrix} V \end{vmatrix}=
(\alpha_2-\alpha_1)(\alpha_3-\alpha_1)\dots(\alpha_n-\alpha_1)
\underbrace{\begin{vmatrix}
1 & \alpha_2 & \alpha_2^2 & \dots & \alpha_2^{n-2}\\
1 & \alpha_3 & \alpha_3^2 & \dots & \alpha_3^{n-2}\\
1 & \alpha_4 & \alpha_4^2 & \dots & \alpha_4^{n-2}\\
\vdots & \vdots & \vdots & &\vdots \\
1 & \alpha_n & \alpha_n^2 & \dots & \alpha_n^{n-2}\\
\end{vmatrix}}_{(1)}\]

(1): es una matriz de Vandermonde de orden $n-1$, por lo que podemos aplicar la fórmula por la hipótesis de inducción, quedando así demostrada la fómrula del determinante de Vandermonde para orden $n$
\end{proof}

\section{Sistemas de ecuaciones}

Sistemas, expresión matricial de sistemas. 
 
Rouché-Frobenius, corregimos. 

Resolución de sistemas escalonados y método de Gauss Jordan.
 
"Repaso" de Sistema Compatible Indeterminado. 2 sistemas resueltos por mi. El primero con ecuaciones. El segundo con matricial.


\begin{problem}

Discute y resuelve el siguiente sistema:

\[
\left\{\begin{array}{lcccl}
x&+2y&-2z&=&4\\
2x&+5y&-2z&=&10\\
4x&+9y&-6z&=&18
\end{array}\right\}
\]

\solution

\[
\left\{\begin{array}{lcccl}
x&+2y&-2z&=&4\\
2x&+5y&-2z&=&10\\
4x&+9y&-6z&=&18
\end{array}\right\}
\overset{(1)}{\dimplies}
\left\{\begin{array}{lcccl}
x&+2y&-2z&=&4\\
 &y&+2z&=&2 \\
4x&+9y&-6z&=&18
\end{array}\right\}
\overset{(2)}{\dimplies}\]
\[
\left\{\begin{array}{lcccl}
x&+2y&-2z&=&4\\
 &y&+2z&=&2 \\
 &y&+2z&=&2 \\
\end{array}\right\}
\dimplies
\underbrace{\left\{\begin{array}{lcccl}
x&+2y&-2z&=&4\\
 &y&+2z&=&2 
\end{array}\right\}}_{\text{Discusión: C.I (*)}}
\]

(*): Es un sistema compatible indeterminado porque es un sistema escalonado con más incógnitas que ecuaciones.

Al ser compatible indeterminado, el sistema tiene infinitas soluciones (que no se calculan en 1º de Bachillerato).


\paragraph{Resolución:} Aunque un sistema de ecuaciones Compatible Indeterminado tiene infinitas soluciones, no cualquier trío de números es solución. 
%
Por ejemplo, en este caso, la terna $(x,y,z) = (0,0,0)$ no es solución.
%
\textbf{Infinitas soluciones no significa que todo sea solución}.

La pregunta lógica sería, ¿cómo podemos escribir \textbf{todas} las soluciones del sistema? Utilizando un parámetro.
%
Al dar un valor a una incógnita, ya forzamos los otros 2 valores. 
%
Para cada valor inventado de $x$, solo hay un único valor posible de $y$ y de $z$ (normalmente).

En este caso, vamos a dar un valor concreto a $y$, pero en forma de parámetro.
%
Tomamos $y=λ$ y sustituimos en $E_2$.

\[y+2z=2 \dimplies λ+2z=2 \dimplies z=\frac{2-λ}{2}\]

Sustituimos $y=λ,z=\frac{2-λ}{2}$ en $E_1$:

\[x+2y-2z = 4 \dimplies x= 4+2z-2y = 4+2\left(\frac{2-λ}{2}\right)-2λ = 4+2-λ-2λ = 6-3λ = 3(2-λ)\]

\textbf{Solución:} $(x,y,z) = \left(3(2-λ),λ,\frac{2-λ}{2}\right)$

\paragraph{1)} $E_2=E_2-2E_1$

\[
\left\{\begin{array}{lcccl}
2x&+4y&-4z&=&8\\
2x&+5y&-2z&=&10\\
\hline
&-y&-2z&=&-2 
\end{array}\right\}
\]

\paragraph{2)} $E_3=E_2-4E_1$

\[
\left\{\begin{array}{lcccl}
4x&+9y&-6z&=&18\\
4x&+10y&-4z&=&20\\
\hline
&-y&-2z&=&-2 
\end{array}\right\}
\]


\paragraph*{Comprobación:} Sustituimos $(x,y,z) = \left(3(2-λ),λ,\frac{2-λ}{2}\right)$ en el sistema inicial:


\[
\left\{\begin{array}{lcccll}
x&+2y&-2z&=&4 &\to 6-3λ + 2λ - 2\displaystyle\left(\frac{2-λ}{2}\right) = 6-λ-2+λ = 4\\
2x&+5y&-2z&=&10 &\to 12-6λ +5λ - 2\displaystyle\left(\frac{2-λ}{2}\right) = 12-λ-2+λ = 10\\
4x&+9y&-6z&=&18 &\to 24-12λ + 9λ - 6\displaystyle\left(\frac{2-λ}{2}\right) = 24-3λ-6+3λ = 18
\end{array}\right\}\begin{array}{c}\\\\\\\\\text{cqc}\end{array}
\]

\end{problem}

\begin{problem}

Discute y resuelve el siguiente sistema:

\[
\left\{\begin{array}{rcccl}
3x&-y&+z&=&3\\
6x&-2y&+2z&=&6\\
-3x&+y&-z&=&-3
\end{array}\right\}
\]

\solution


\[
\left\{\begin{array}{rcccl}
3x&-y&+z&=&3\\
6x&-2y&+2z&=&6\\
-3x&+y&-z&=&-3
\end{array}\right\} \implies
\left(\begin{array}{ccc|c}
3&-1&1&3\\
6&-2&2&6\\
-3&1&-1&-3
\end{array}\right)
\dimplies\]
\[
\text{\hl{Ojo con el cambio de columnas}}
\left(\begin{array}{ccc|c}
1&-1&3&3\\
2&-2&6&6\\
-1&1&-3&-3
\end{array}\right)
\dimplies
\left(\begin{array}{ccc|c}
1&-1&3&3\\
0&0&0&0\\
0&0&0&0
\end{array}\right)
\]

Tiene grado de indeterminación 2, por lo que necesitaremos 2 parámetros.

Llamamos $x=\lambda$ e $y = \mu$ con $\mu,\lambda\in\real$ y sustituimos para hallar $z$.

$$z-y+3x=3 \implies z - \mu + 3\lambda = 3 \dimplies z = 3+\mu-3\lambda$$

Solución: $(x,y,z) = \left(\lambda, \mu, 3+\mu - 3\lambda\right), \forall\lambda,\mu\in\real$

\end{problem}

Ejercicio 59 de deberes.

\textbf{Resolución por inversa de stma}. ¿Funciona siempre? Sólo en sistemas de Cramer, es decir, matriz de coeficientes cuadrada con rango máximo. 

Deberes el 15b,16b.

\subsection{Regla de Cramer}

En todos los sistemas cuya matriz de coeficientes tenga inversa, puede generalizarse el método de la inversa.

Así, $Ax = B \dimplies x = A^{-1}·B$

\[
    A^{-1} = \frac{1}{|A|} · \left( \text{Adj}(A) \right)^T = \frac{1}{|A|} · \begin{pmatrix} 
    A_{11} & A_{21} & A_{31} & ... & A_{n1}\\
    A_{21} & A_{22} & A_{32} & ... & A_{n2}\\
    \vdots &        &       & \ddots & \vdots\\
    A_{1n} & A_{2n} & A_{3n} & ... & A_{nn}\end{pmatrix}
\]

Por lo tanto,
\[
    A^{-1}·B = \frac{1}{|A|} · 
    \begin{pmatrix} 
        A_{11} & A_{21} & A_{31} & ... & A_{n1}\\
        A_{12} & A_{22} & A_{32} & ... & A_{n2}\\
        \vdots &        &       & \ddots & \vdots\\
        A_{1n} & A_{2n} & A_{3n} & ... & A_{nn}\end{pmatrix}·
    \begin{pmatrix}
        b_1\\b_2\\\vdots\\ b_n
    \end{pmatrix}
    = 
    \begin{pmatrix}
    A_{11}b_1 + A_{21}b_2 + A_{31}b_3 \dots A_{n1}·b_n\\
    A_{12}b_1 + A_{22}b_2 + A_{32}b_3 \dots A_{n2}·b_n\\
    \vdots\\
    A_{1n}b_1 + A_{2n}b_2 + A_{3n}b_3 \dots A_{nn}·b_n\\
    \end{pmatrix}
\]

Deberes : 21b, 22b

Corregimos Cramer. 

Inconvenientes: ¿y si es incompatible?

Numéricos: 61a,b;62a,b

Parámetros:64a,c (ojo con eliminar una solución)



Deberes para el punete: 
57,60


\section{Geometría afín}

\subsection{Vectores y sistemas de referencia}





\subsection{Ecuaciones de la recta y del plano}

\obs Dado que la única manera que tenemos de determinar los puntos es a través de vectores de posición, utilizaremos $\vec{p} = (x,y,z)$ de forma equivalente a $[\vec{OP}]$ para referirnos a un punto cualquiera del espacio, al que accedemos a través de su vector de posición.

Formas de determinar una recta:
\begin{enumerate}
  \item Un punto (o su vector de posición) y un vector director.
  \subitem Dos puntos (se reduce al caso anterior)
  \subitem Un punto y una condición de paralelismo (se reduce al primer caso)
  \item Dos planos secantes.
  \item Un punto y un plano perpendicular (se verá en geometría euclídea).
\end{enumerate}

\obs Como todo con lo que vamos a trabajar son operaciones con vectores con coordenadas en un sistema de referencia, en realidad el punto de origen del sistema de referencia no es relevante. 
%
%(Ver ejemplo \ref{example::origen_ref}).

\subsubsection{Ecuaciones de la recta}

La 
%
\concept[Ecuación de la recta\IS vectorial]{ecuación vectorial de la recta} 
%
$r$ determinada por el punto $A\in\mathcal{E}_3$, cuyo vector de posición es $\vec{a}\in\mathcal{V}^3$, con vector director $\vec{u_r}\in\mathcal{V}^3 $  es 

\[ r: \vec{p} = \vec{a} + \lambda \vec{u_r}\]

donde $\lambda\in\real, \forall P\in\mathcal{E}_3$ (ver \fref{fig::recta_ecuacion_vectorial})
%
De esta manera quedan determinados los vectores de posición de todos los puntos de la recta $r$.


\begin{figure}[hptb]
    \centering
    \includegraphics[width=0.4\textwidth]{img/EcVectorialRecta.png}
    \caption{Representación gráfica de la ecuación vectorial de la recta.}
    \label{fig::recta_ecuacion_vectorial}
\end{figure}


¿Es posible construir la recta sin ese parámetro $\lambda$? En realidad,
$$\underbrace{r : \vec{p} = \vec{a} + \lambda \vec{u_r}}_{(0)}\dimplies \underbrace{r:\displaystyle \left\{
\begin{array}{c} 
  x = a_1 + \lambda u_1\\
  y = a_2 + \lambda u_2 \\ 
  z = a_3 + \lambda u_3
\end{array}\right\}}_{(1)} \implies \underbrace{r:\frac{x-a_1}{u_1} = \frac{y-a_2}{u_2} = \frac{z-a_3}{u_3}}_{(2)}$$

\begin{itemize}
    \item A $(1)$ lo denominamos \concept[Ecuación de la recta\IS paramétrica]{ecuación paramétrica de la recta}
    \item A $(2)$ lo denominamos \concept[Ecuación de la recta\IS continua]{ecuación continua de la recta}
\end{itemize}

\[
\frac{x-a_1}{u_1} = \frac{y-a_2}{u_2} = \frac{z-a_3}{u_3}\overset{(1)}{\implies} \left\{
\begin{array}{c}
     \displaystyle\frac{x-a_1}{u_1} = \frac{y-a_2}{u_2}\\
     \displaystyle\frac{y-a_2}{u_2} = \frac{z-a_3}{u_3}
\end{array}\right\} \implies
\underbrace{\left\{\begin{array}{cccc}
     Ax + &By    &     & = D\\
          &B'y   &+ C'z  & = D
\end{array}\right\}}_{(2)}
\]



Donde:
\begin{itemize}
    \item (1) \hide{Se han elegido estas 2 parejas, pero podrían haberse elegido otras, dando lugar a otras ecuaciones implícitas de la misma recta.}
    \item (2) \hide{\concept[Ecuación de la recta\IS implícita]{ecuación implícita de la recta}}
    \subitem \obs \hide{Si interpretáramos las ecuaciones implícitas de la recta como un sistema de ecuaciones, tendríamos un sistema compatible indeterminado con grado de libertad 1.}
    \subitem \obs La dimensión de la recta es 1. \emph{¿Como el grado de indeterminación? Guau...}
\end{itemize}

\begin{problem}
    \ppart 
    Halla todas las ecuaciones de la recta que pasa por $A(0,1,2)$ y es paralela a la que pasa por $B(1,-2,-1)$ y $C(1,0,0)$
    \ppart 
    Halla un vector director de la recta $r:\displaystyle\frac{x-2}{1} = \frac{y-3}{5} = \frac{z-1}{4}$
    \ppart 
    Halla un vector director de la recta $r:\displaystyle\left\{\begin{array}{c} 2x+3y=4\\2x-y+3z=0\end{array}\right\}$
    \solution

\end{problem}

\textbf{Ejercicios:} 
\begin{itemize}
  \item Página 279.13,14.
  \item Página 281.18-21.
\end{itemize}

\subsubsection{Ecuaciones del plano}

Un plano queda determinado por \hide{un punto y dos vectores linealmente independientes}

La 
%
\concept[Ecuación del plano\IS vectorial]{ecuación vectorial del plano} 
%
$\pi$ determinada por el punto $A\in\mathcal{E}_3$ y los vectores linealmente independientes $\vec{V_{\pi}}\in\mathcal{V}^3$ y $\vec{W_{\pi}}\in\mathcal{V}^3$  es $r : \vec{p} = \vec{A} + \lambda \vec{V_{\pi}} + \mu\vec{W_{\pi}}$, con $\lambda\in\real$ (ver ). De esta manera quedan determinados los vectores de posición de todos los puntos del plano.

\begin{figure}[hptb]
    \centering
    \includegraphics[width=0.65\textwidth]{img/ecplanos.png}
    \caption{Plano generado por un punto y dos vectores}
    \label{fig:plano}
\end{figure}


\paragraph{¿Sería posible evitar el parámetro $\lambda$?} Como en el caso de la recta, podemos escribir esta ecuación en forma de sistema:

$$\pi : \vec{p} = \vec{A} + \lambda \vec{V_{\pi}} + \mu\vec{W_{\pi}}\dimplies \underbrace{\pi:\displaystyle \left\{\begin{array}{c} x = a_1 + \lambda v_1 + \mu w_1\\y = a_2 + \lambda v_2 + \mu w_2 \\ z = a_3 + \lambda v_3 + \mu w_3 \end{array}\right\}}_{(1)} $$

 A $(1)$ lo denominamos \concept[Ecuación del plano\IS paramétrica]{ecuación paramétrica del plano}

\[
\pi:\displaystyle \left\{
\begin{array}{c} 
x - a_1 = \lambda v_1 + \mu w_1\\
y - a_2 = \lambda v_2 + \mu w_2 \\ 
z - a_3 = \lambda v_3 + \mu w_3 
\end{array}\right\}
\]
Como los vectores $\vec{v},\vec{w}$ son linealmente independientes y el vector $AX$ es una combinación lineal de los otros 2 (ver \ref{fig::ecuacion_implicita_plano}), tenemos:


\begin{figure}[hptb]
    \centering
    \includegraphics[width=0.3\textwidth]{img/Implicita1.PNG}
    \includegraphics[width=0.4\textwidth]{img/Implicita2.PNG}
    \caption{En la figura de la izquierda puede comprobarse que el vector rosa pertenece al plano, formado por el punto en común y los otros 2 vectores (naranjas). La versión tridimensional se encuentra en \href{https://www.geogebra.org/m/pyh6nnsp}{https://www.geogebra.org/m/pyh6nnsp}.}
    \label{fig::ecuacion_implicita_plano}
\end{figure}

\[
\left|
\begin{array}{ccc} 
x - a_1 & v_1 & w_1\\
y - a_2 & v_2 & w_2 \\ 
z - a_3 & v_3 & w_3 
\end{array}\right| = 0
\]

Desarrollando esta ecuación, tendríamos una ecuación del tipo $Ax+By+Cz + D = 0$, que llamamos \concept[Ecuación del plano\IS implícita]{Ecuación implícita del plano}.



\begin{problem}
    \textbf{Calcula las ecuaciones del plano que pasa por los puntos $A(1,1,1), B(2,2,2), C(1,2,3)$}

    \solution 

    \hide{
    El primer paso sería calcular 2 vectores linealmente independientes de estos 3 puntos, para comprobar que los 3 puntos forman un plano (y no una recta).

    $\vec{AB} = (1,1,1) \quad\quad \vec{AC} = (0,1,2)$, que son linealmente independientes al no ser proporcionales.

    Ecuación vectorial: $\pi: \vec{p} = (1,1,1) + \mu\cdot\vec{AB} + \lambda\cdot\vec{AC}$ con $\lambda,\mu\in\real$

    Ecuación paramétrica: $
    \displaystyle\left\{ \begin{array}{c}
      x = 1 + \mu\\
      y = 1 + \mu + \lambda\\
      z = 1 + \mu + 2\lambda
    \end{array} \right\}\text{ con } \lambda,\mu\in\real$

    Ecuación implícita: 

    \[
      \displaystyle \begin{vmatrix}
      x - 1 & 1 & 0\\
      y - 1 & 1 & 1\\
      z - 1 & 1 & 2\\
    \end{vmatrix} = 0 \dimplies \cdots \dimplies x - 2y+z=0
    \]
    }
\end{problem}


\begin{problem}
\ppart Halla la ecuación de 2 rectas que pertenezcan al mismo plano.
\ppart Halla un vector director del plano: $\pi_1: x+y+z = 3$
\ppart Halla el plano paralelo a $\pi_2: x+y+z = 3$ que pase por el origen de coordenadas.
\ppart Halla el plano paralelo al $XY$ que pasa por $A(-1,2,-2)$.
\obs Llamamos plano $XY$ al plano "del suelo", es decir, al plano $z=0$.

\ppart Página 283, ejercicios 25-28.

\solution

\end{problem}

\subsection{Posiciones relativas}

\subsubsection{Entre 2 planos}  

\begin{framed}
\textbf{Ecuaciones vectoriales o paramétricas:} A continuación, se ofrecen algunos criterios para realizar el estudio de la posición relativa de dos planos.
  \begin{itemize}
    \item Si 2 planos comparten 3 puntos, entonces son el mismo plano.
    \item Si la matriz formada por los 4 vectores directores de los 2 planos tiene rango 2, los planos son paralelos.
    \item Si 2 planos paralelos comparten un punto, entonces son coincidentes.
    \item Si la matriz formada por los 4 vectores directores de los 2 planos tiene rango 3, los planos son secantes en una recta.
    \item \textit{Criterio de perpendicularidad}
  \end{itemize}
\end{framed}



\subparagraph{Ecuaciones implícitas:} En el caso concreto de que los 2 planos estén expresados en ecuaciones implícitas, podremos estudiar su posición relativa discutiendo el sistema formado por ambas ecuaciones:

\[
\left\{\begin{array}{c}
\pi_1: Ax+By+Cz = D\\
\pi_2: A'x+B'y+C'z = D'
\end{array}\right\}
\]

Obtenemos las matrices: $M = \displaystyle\begin{pmatrix}A&B&C\\A'&B'&C'\end{pmatrix}$ y $M^* = \displaystyle\begin{pmatrix}A&B&C&D\\A'&B'&C'&D'\end{pmatrix}$

\begin{framed}
  \begin{itemize}
    \item $Rg(M) = Rg(M^*) = 1 $\hide{ coincidentes.}
    \item $Rg(M) < Rg(M^*) = 2 $\hide{ paralelos.}
    \item $Rg(M) = Rg(M^*) = 2 $\hide{ secantes.}
  \end{itemize}
\obs Las ecuaciones implícitas de la recta, en realidad, es la expresión de 2 planos secantes (que, como no puede ser de otra manera, forman una recta)
\end{framed}

\begin{problem}
Estudia la posición relativa de los siguientes planos:

\[
\pi: \left\{\begin{matrix}x=\lambda+\mu\\y=1-\lambda\\z=2-2\lambda+\mu\end{matrix}\right\}
\]
\[
\pi': x-2y+z=1
\]

\solution

%\vspace{10cm}

\end{problem}

\subsubsection{Entre 3 planos}

\subparagraph{Ecuaciones vectorial o paramétricas}

El estudio de posición relativa en este caso se hace demasiado complejo. Es preferible pasar los planos a ecuaciones implícitas para el estudio.

\subparagraph{Ecuaciones implícitas} Como antes, discutiremos el sistema formado por las ecuaciones implícitas del plano.

\[
\left\{\begin{array}{c}
\pi_1: Ax+By+Cz = D\\
\pi_2: A'x+B'y+C'z = D'\\
\pi_3: A''x+B''y+C''z = D''\\
\end{array}\right\}
\]

Obtenemos las matrices: 
$M  = \displaystyle\begin{pmatrix}
A&B&C\\
A'&B'&C'\\
A''&B''&C''
\end{pmatrix}
$ y 
$M^* = \displaystyle\begin{pmatrix}
A&B&C&D\\
A'&B'&C'&D'\\
A''&B''&C''&D''
\end{pmatrix}
$

Las posibilidades son: (ver figura \ref{fig:PosicionesRelativasPlanos})
\begin{framed}
  \begin{itemize}
    \item $Rg(M) = Rg(M^*) = 1 $\hide{ SCI, secantes en un plano [grado de indeterminación 2, por lo que hay dos parámetros. \textbf{Coincidentes}.}
    \item $Rg(M) < Rg(M^*) = 2 $\hide{ paralelos.}
    \item $Rg(M) = Rg(M^*) = 2 $\hide{ SCI, secantes en una recta [grado de indeterminación 1, por lo que hay un parámetro.}
    \item $Rg(M) = 2 < Rg(M^*) = 3 $\hide{ no se cortan los 3. Sistema incompatible}
    \item $Rg(M) = Rg(M^*) = 3 $\hide{ SCD, secantes en un punto que es la solución del sistema.}
  \end{itemize}  
\end{framed}

\newpage
\begin{figure}[hptb]
    \centering
    \includegraphics[width=0.65\textwidth]{img/Captura1.png}
    \includegraphics[width=0.95\textwidth]{img/Captura2.png}
    \includegraphics[width=1.1\textwidth]{img/Captura3.png}
    \includegraphics[width=1.1\textwidth]{img/Captura4.png}
    \caption{Representación gráfica de las posiciones relativas de 3 planos.}
    \label{fig:PosicionesRelativasPlanos}
\end{figure}


\begin{problem}
Estudia la posición relativa de los planos: 
\begin{align*}
\pi_1: 2x-3y+2z=7\\
\pi_2: -3+y-z=-5\\
\pi_3:2x-2x+3z=7
\end{align*}
\solution

\begin{align*}
\left\{
\begin{array}{c}
    \pi_1: 2x-3y+2z=7\\
    \pi_2: -3+y-z=-5\\
    \pi_3:2x-2x+3z=7
\end{array}
\right\} \to
\left\|
\begin{array}{c}
M = \begin{pmatrix}             2&-3&2\\-3&1&-1\\2&-2&3
\end{pmatrix} \\\\
M' = \begin{pmatrix}             2&-3&2&-7\\-3&1&-1&5\\2&-2&3&-7
\end{pmatrix} \end{array}\right.
\end{align*}

%\vspace{10cm}

\end{problem}


\subsubsection{Posiciones relativas entre recta y plano}

\paragraph{Ecuaciones implícitas: } Si la recta y como el plano están dados en ecuaciones implícitas, estaríamos en la posición relativa de 3 planos, sabiendo que 2 de ellos son secantes en una recta.

\paragraph{Ecuaciones paramétrica: } $\pi: \{P_{\pi},\vec{v_{\pi}}, \vec{w_{\pi}}\}$ 

\begin{center}
\begin{tabular}{ccc}
$P_r \in \pi $ & $\vec{v_r}$ LD de $v_{\pi}, \vec{w}_{\pi}$ & \textbf{Conclusión}\\\hline
No & No & Secantes en un punto\\
No & Sí & Recta paralela\\
Sí & No & Secantes en un punto\\
Sí & Sí & Recta contenida en el plano\\
\end{tabular}
\end{center}

\begin{problem}
Dada la recta $r: \left\{\begin{array}{c}
    x=\lambda\\
    y=1-2\lambda\\
    z=2
\end{array}\right\}$ y el plano $\pi: (0,1,1) + \mu\cdot (-1,0,0) + \lambda\cdot(2,0,1)$ 
\solution
%\vspace{10cm}

\end{problem}

\paragraph{Plano implícito, recta paramétrica: } comprobamos si existe algún valor de $\lambda_{r}$ para el que se cumpla la ecuación implícita del plano. 
\begin{itemize}
  \item $\exists!\lambda \implies $ secante.
  \item $\lambda \in \real \implies$ contenida.
  \item $\not\exists \lambda \implies $ paralela.
\end{itemize}
\begin{problem}
Estudia la posición relativa del plano $\pi: 2x-2y+z=0$ y de ña recta $r:\left\{\begin{array}{c}
     x=1-2\lambda\\
     y=1+2\lambda\\
     z=3+5\lambda 
\end{array}\right\}$
\solution

%\vspace{10cm}

\end{problem}

\begin{problem}
Estudia la posición relativa del plano $\pi: 2x-2y+z=0$ y de ña recta $r:\left\{\begin{array}{c}
     x=1+\lambda\\
     y=-2+2\lambda\\
     z=3-\lambda 
\end{array}\right\}$
\solution

Sustituimos los valores de $x,y,z$ de las ecuaciones de $r$ en el plano $\pi$:

\[3(1+\lambda) - (-2+2\lambda) + 3 - \lambda = 0 \dimplies 3+3\lambda+2-2\lambda-2\lambda+3-\lambda \dimplies 8=0\]

\textbf{Conclusión} La recta y el plano no tienen nada en común, por lo que tienen que ser paralelos.

\end{problem}


Ejercicios recomendados: 105ab,107ab

\subsubsection{Posiciones relativas entre dos rectas}

Las distintas posibilidades de posición relativa son:

  \begin{tabular}{c|c|c}
\textbf{Posición relativa} & \textbf{Vectores} &\textbf{ Puntos}\\\hline
Paralelas coincidentes & Paralelos & Todos en común (*)\\
Paralelas no coincidentes. & Paralelos & Ninguno en común\\
Secantes en un punto & No paralelos & Solamente uno en común.\\
Se cruzan en el espacio & No paralelos & Ninguno en común\\
\end{tabular}

\begin{figure}[hptb]
    \centering
    \includegraphics[width=0.4\textwidth]{img/rectasquesecruzan.png}
    \caption{Representación gráfica de dos rectas que se cruzan en el espacio.}
    \label{fig::rectas_que_se_cruzan_en_el_espacio}
\end{figure}


\begin{framed}
  Dadas las rectas 
$r:\vec{p} = \vec{a} + \lambda\vec{u},\quad \lambda\in\real$
y
$s:\vec{p} = \vec{b} + \lambda\vec{w},\quad \lambda\in\real$. 


\begin{itemize}
    \item $\vec{u}|| \vec{w} \to $ Paralelas o coincidentes.
    \subitem $A_r\in s$ ó $B_s\in r \implies$ coincidentes.
    \subitem En caso contrario, paralelas no coincidentes.
    \item $\vec{u}\not || \vec{w} \to $ secantes o se cruzan.
    \subitem Ver \textit{procedimiento general}.
\end{itemize}
\end{framed}

\paragraph{Procedimiento general:}
\label{txt::procedimiento_general}

Si los vectores directores son paralelos (proporcionales), las rectas pueden ser paralelas o coincidentes. 
%
Para poder distinguir, podríamos ver si un vector formado por un punto de cada recta es también proporcional (entonces serían coincidentes) o si no (entonces serían secantes).

De la misma manera, si los vectores son linealmente independientes las rectas pueden cruzarse o cortarse. 
%
Para distinguir estos 2 casos, podríamos ver si un vector formado por un punto de cada recta es linealmente dependiente a los otros 2 (entonces serían secantes porque formarían un plano que contiene al vector) o si no (entonces se cortarían en el espacio).

Así, buscamos estudiar la dependencia lineal de los 2 vectores directores ($\vec{u},\vec{w}$) y de los 2 vectores directores respecto de un vector formado, arbitrariamente, con 2 puntos de las rectas ($\vec{AB}$, con $A(a_1,a_2,a_3)\in r$ y $B(b_1,b_2,b_3)\in s$). 
%
Para ello, formamos las matrices:

$M  = \displaystyle\begin{pmatrix}
u_1&w_1\\
u_2&w_2\\
u_3&w_3
\end{pmatrix}
$ y 
$M^* = \displaystyle\begin{pmatrix}
u_1&w_1&b_1-a_1\\
u_2&w_2&b_2-a_2\\
u_3&w_3&b_3-a_3\\
\end{pmatrix}
$

\begin{framed}
  \begin{itemize}
    \item $Rg(M) = Rg(M^*) = 1 $\hide{ SCI, secantes en una recta [grado de indeterminación 1, por lo que hay dos parámetros.] \textbf{Coincidentes}.]}
    \item $Rg(M) = 1 < Rg(M^*) = 2 $\hide{ paralelas.} 
    \item $Rg(M) = Rg(M^*) = 2 $\hide{ SCI, secantes en un plano [dimensión 2]. $\vec{AB}$ se puede escribir como combinación lineal de $\vec{u}$ y $\vec{w}$}
    \item $Rg(M) = 2 < Rg(M^*) = 3 $\hide{ se cruzan en el espacio.} 
  \end{itemize}  
\end{framed}

\begin{problem}
Estudia la posición relativa de las siguientes rectas:

\[
r:\left\{
\begin{array}{c}
x=\lambda \\
y=-1+2\lambda\\
z=1-\lambda
\end{array}
\right\}
\;\;\;\;
s:\left\{
\begin{array}{c}
x=2\lambda \\
y=2+\lambda\\
z=2-3\lambda
\end{array}
\right\}
\]
\solution

Se toma un punto y un vector de cada recta:
\[
r: A(0,-1,1), \vec{u_r} =(1,2,-1)\quad\quad s: B(0,2,2), \vec{w_s} = (2,1,-3)
\]

Se consideran las matrices: $M=\left(\vec{u_r}, \vec{w_s}\right) \quad M^\ast = \left(\vec{u_r},\vec{w_s}, [\vec{AB}]\right)$ y se estudia el rango.

%\vspace{5cm}

Para hallar las coordenadas el punto de corte: 

%\vspace{5cm}

\end{problem}

\begin{problem}
Página 289, ejercicios 52, calculando puntos de cortes
\solution
\end{problem}


\paragraph{Ambas rectas en general:} 
%
Formamos la matriz de 4x4 y estudiamos la posición relativa de 4 planos.
%
Consultando \fref{fig:PosicionesRelativasPlanos} tenemos todos los casos cubiertos en los que uno de los planos es combinación lineal de los demás. 
%
En el caso en el que no haya ningún plano que sea combinación lineal de los demás, tendremos que la matriz ampliada tendrá rango 4 por lo que el sistema será incompatible. 
%
Geométricamente, tendremos rectas que se cruzan.

\begin{problem}
Estudia la posición relativa de las siguientes rectas:
\[
r: 
\left\{
\begin{array}{l}
x+y+z=1\\
-3x+y+z=1
\end{array}
\right\}\;\;\;\;
s: 
\left\{
\begin{array}{l}
2x-y+z=1\\
-x+z=0
\end{array}
\right\}
\]
\solution

\[
\left\{
\begin{array}{l}
x+y+z=1\\
-3x+y+z=1\\
2x-y+z=1\\
-x+z=0
\end{array}
\right\} \to M^\ast=\begin{pmatrix}
0&1&1&1\\-3&1&1&1\\2&-1&1&1\\-1&0&1&0
\end{pmatrix}
\]

Calculamos $|M^\ast|$
\[
|M^\ast|=\begin{pmatrix}
0&1&1&1\\-3&1&1&1\\2&-1&1&1\\-1&0&1&0
\end{pmatrix} = ... = -6
\]

$Rg(M) \leq 3 < 4 = Rg(M^\ast)$, por lo que el sistema es incompatible.
%
Por ello, las rectas se cruzan en el espacio.

\end{problem}

\paragraph{Haz de rectas}
%\paragraph{Haz de rectas paralelas: } cambia el punto, manteniendo fijo el vector. 
%\paragraph{Haz de rectas secantes: } cambia el vector (sin ser nunca nulo), mantiene fijo el punto.
%\paragraph{Haz de planos paralelelos: } cambia el punto, mantiene los vectores
%\paragraph{Haz de planos secantes en una recta: } mantiene un vector y un punto, cambia el otro vector.

%\begin{problem}
%Tema 11: 56,57,59,60.
%\solution
%\end{problem}

%Deberes: 118,136,143


\paragraph{Practicamos en general}
Tema 11: 
122,127,130,134,135,136,139,140,143,145,149,150

Tema 11:
\begin{itemize}
  \item Básicos: 83,91,92,93a,98,100,101a,103a,104a,105,106,107d,108
  \item Síntesis: 111-119
  \item Completos: 122-124,127,129-133
\end{itemize}





\section{Geometría euclídea (del 25/01 al final)}

Llamamos geometría euclídea al espacio en el que podemos medir, cosa que hasta ahora no era posible.

Todo surge desde el módulo de un vector. Llamamos \concept[Módulo de un vector]{módulo de un vector} a la longitud que tiene. Dado $\vec{v}$, se define el módulo como $|\vec{v}|$.

\subsection{Producto, escalar, vectorial y mixto}

\paragraph{Introducción sobre el origen del producto escalar y vectorial}

Fuentes consultadas:
\begin{itemize}
  \item \href{http://www.suitcaseofdreams.net/Geometric_multiplication.htm}{Relación forma polar y binómica del producto complejo}
  \vspace{-0.4cm}
  \item \href{https://www2.clarku.edu/faculty/djoyce/complex/mult.html}{Interpretación geométrica del producto complejo}
  \vspace{-0.4cm}
  \item \href{https://www.quora.com/Who-invented-the-dot-product-and-cross-product}{Historia y aplicación de los cuaterniones los productos}
  \vspace{-0.4cm}
  \item \href{https://es.wikipedia.org/wiki/Cuaterni%C3%B3n}{ Extensión de los complejos al grupo de los quaterniones}
\end{itemize}

Dados 2 números complejos $z_1 = a_1+b_1i$, $z_2 = a_2+b_2i$. Expresando estos números complejos en forma polar tenemos: $z_1=r_{\alpha_1}$ y $z_2 = s_{\alpha_2}$.  

$z_1·z_2 = (a_1a_2 - b_1b_2) + (a_1b_2+a_2b_1)i = r·s_{\alpha_1+\alpha_2}$.

Tomando $z_1·\bar{z_2} = (a_1a_2 + b_1b_2) + (a_1b_2-a_2b_1)i = r·s_{\alpha_1-\alpha_2}
$

\subparagraph{Estudio de la parte real (producto escalar)}

En $Re(z_1·\bar{z_2}) = a_1a_2 + b_1b_2 = Re(r·s_{\alpha_1-\alpha_2})$

Para calcular $Re(r·s_{\alpha_1-\alpha_2}) = Re(r·s·\cos(\alpha_1-\alpha_2) + i·r·s·\sen(\alpha_1-\alpha_2)$, por lo que podemos completar:

$ a_1a_2 + b_1b_2 = |z_1||z_2|·\cos(\alpha_1-\alpha_2)$, y, siendo conscientes que $\alpha_1-\alpha_2$ es el ángulo que forman los 2 vectores, obtenemos la expresión del producto escalar de 2 vectores.


\subparagraph{Estudio de la parte imaginaria (producto vectorial)}

Al extender este razonamiento al grupo de los cuaterniones, tendríamos:

\newcommand{\quat}{\vec}

$Re(z_1\bar{z_2}) = Re\left((b_1\quat{i}+c_1\quat{j}+d_1\quat{k})·(-b_2\quat{i}-c_2\quat{j}-d_2\quat{k})\right) = (b_1b_2+c_1c_2+d_1d_2)$

Lo espectacular viene al considerar la parte "imaginaria" (aunque no tengamos claro cómo se define ese concepto en los cuaterniones):

$Im(z_1\bar{z_2}) = f(\quat{i},\quat{j},\quat{k})$, cuya expresión analítica es la del producto vectorial, ya que en grupo de los cuaterniones $\quat{i}\quat{j}=\quat{k}$ y todo eso.

\subsubsection{Producto escalar}

\begin{itemize}
  \item Definición.
  \item Base ortogonal y ortonormal.
  \item Expresión analítica.
  \item Cálculo del módulo (porque Pitágoras tridimensional no funciona).
  \item Base canónica.
  \item Interpretación geométrica (proyección).
\end{itemize}

\subsubsection{Producto vectorial}
\begin{itemize}
  \item Definición.
  \subitem Regla de la mano derecha
  \item Expresión analítica.
  \item Interpretación geométrica. 
\end{itemize}


\subsubsection{Producto mixto}

\begin{itemize}
  \item ¿Qué ocurre si en el producto vectorial meto un vector en lugar de $\vec{i},\vec{j},\vec{k}$?
  \item Definición.
  \item Expresión analítica.
  \item Interpretación geométrica. 
\end{itemize}

\subsubsection{Practicamos en general ejercicios del tema 9}

\subsection{Aplicación de los productos}

\subsubsection{Vector normal del plano}

\subsubsection{Vector director de la recta en implícitas}

\begin{figure}[hbtp]
\centering
\includegraphics{img/directorrectaimplicitas.jpg}
\end{figure}

Dada una recta $r$ que pertenece a ambos planos, $\pi_1$ y $\pi_2$, $\vec{V_r}\perp n_{\pi_1} \wedge \vec{V_r}\perp n_{\pi_2}$. Por eso, para buscar un vector perpendicular a la vez a 2 vectores dados, el camino más corto es el producto vectorial.

$\vec{V_r} = n_{\pi_1}\times n_{\pi_2}$

\begin{problem}

Halla la recta perpendicular a $r:\frac{x-1}{2} = \frac{y+2}{3} = \frac{2z-2}{3}$ que pasa por el punto $P(1,-2,0)$.

\solution

La recta pedida es $t:\left\{\begin{array}{c}P\in t\\t\perp r\\\end{array}\right.$ 

Todas las rectas perpendiculares a una recta forman un plano. Llamamos $\pi$ a ese plano. Así, $ t \in \pi, \text{ con } \pi:\left\{\begin{array}{c}P\in\pi\\\pi\perp r\end{array}\right.$. Así, $t$ será la recta determinada por $A = t\cap\pi, P(1,-2,0)$

$\pi\perp r\implies n_{\pi} || \vec{V_r}$. Tomamos $n_{\pi}  = \left(2,3,\rfrac{3}{2}\right)$

Como $P\in\pi\implies 2P_1 + 3P_2 +\rfrac{3}{2}P_3 + D = 0 \dimplies 2-6+D = 0 \dimplies D=4$

Así, $\pi: 2x+3y+\rfrac{3}{2}z + 4 = 0$

2) Buscamos $A=\pi\cap r$. Tendremos $t:\{A,P\}$


\[
  \left\{\begin{array}{c}
    2x+3y+\rfrac{3}{2}z + 4 = 0\\
    x = 1 + 2\lambda\\
    y = -2 + 3\lambda\\
    z = 1 + \rfrac{3}{2}\lambda\end{array}\right\} \implies 2+4\lambda - 6 + 9\lambda + \rfrac{3}{2} + \rfrac{9}{4}\lambda + 4 = 0 \dimplies 
\]
\[
  \frac{3}{2}+\left(13+\rfrac{9}{4}\right)\lambda = 0 \dimplies \lambda = \frac{\rfrac{61}{4}}{\rfrac{3}{2}} = \frac{61}{6}
\]

Sustituimos $\lambda$ en la ecuación de la recta para obtener $A$:
$\left\{\begin{array}{c}
    x = 1 + 2\rfrac{61}{6} = \frac{64}{3}\\
    y = -2 + 3\rfrac{61}{6} = \frac{57}{2}\\
    z = 1 + \rfrac{3}{2}\rfrac{61}{6} = \frac{57}{4}
\end{array}\right\}$


\[A = \pi\cap r = \left(\frac{64}{3},\frac{57}{2},\frac{57}{4}\right)\]

3) Buscamos la recta $r$ que pasa por los puntos $A$ y $P$. 

$\vec{AP} = \left(\frac{64}{3}-1,\frac{57}{2}+2,\frac{57}{4}\right) =  \left(\frac{61}{3},\frac{61}{2},\frac{57}{4}\right)$

\[r : \{A,\vec{V_r}\} = \left\{\begin{array}{c}
    x = \rfrac{64}{3} + 2\lambda\\
    y = \rfrac{57}{2} + 3\lambda\\
    z = \rfrac{57}{4} + \rfrac{3}{2}\lambda
\end{array}\right\}\]

\end{problem}

127,128.

\begin{problem}[Junio 2019]

Dados los puntos $A(1,1,1), B(1,3,-3)$ y $C(-3,-1,1)$, se pide:
\ppart Determinar la ecuación del plano que contiene a los 3 puntos.
\ppart Obtener un punto $D$ (distinto a los anteriores) tal que los vectores $\vec{AB}$, $\vec{AC}$,$\vec{AD}$ sean linealmente dependientes.
\ppart Encontrar un punto $P$ del eje $OX$, de modo que el volumen del tetraedro de vértices $A$,$B$,$C$,$P$ sea igual a 1.

\solution

\spart 

Tomamos $\vec{AB} = (0,2,-4)$, $\vec{AC} = (-4,-2,0)$ como vectores directores del plano y calculamos su ecuación implícita:

\[
\pi: \left|\begin{array}{ccc}
x - 1 & 0 & -4\\
y - 1 & 2 & -2\\
z - 1 & -4& 0
\end{array}\right| = 0 \dimplies 16y-16 + 8z-8-8x+8 = 0\]
\[\dimplies -8x + 16y + 8z -16 = 0 \dimplies -x+2y+z-2=0
\]

\spart Basta con tomar $D\in\pi$. Por ejemplo: $D(0,1,0)\in\pi$.

Comprobamos que $\vec{AB},\vec{AC},\vec{AD}$ son linealmente dependientes calculando el determinante de la matriz que forman.

$\vec{AD} = (1,0,1)$

\[
|\vec{AB}\quad\vec{AC}\quad\vec{AD}| = 
\left|\begin{array}{ccc}
1 & 0 & -4\\
0 & 2 & -2\\
1 & -4& 0
\end{array}\right| = 0
\]

\spart 

(Ojo que aquí no se puede simplificar, aunque antes sí hubiéramos podido)

El volumen del tetraedro será $\rfrac{1}{6}$ del voluen del paralelepípedo:

\[
||\vec{AB}\quad\vec{AC}\quad\vec{AD}|| = 
\left|\left|\begin{array}{ccc}
1 & 0 & -4\\
1 & 2 & -2\\
a-1 & -4& 0
\end{array}\right|\right| = 2·2·\left|\left|\begin{array}{ccc}
1 & 0 & 2\\
1 & 1 & 1\\
z-1 & -2& 0
\end{array}\right|\right| = {6} 
\]
\[
 \left|\left|\begin{array}{ccc}
1 & 0 & 2\\
1 & 1 & 1\\
a-1 & -2& 0
\end{array}\right|\right| = \frac{3}{2}  \dimplies
|-4 -2a-2+2| = \frac{3}{2}\dimplies |-4-2a| = \rfrac{3}{2} \implies\]
\[
\left\{\begin{array}{c}
-4-2a_1 = \rfrac{3}{2} \dimplies a_1 = \frac{-11}{4} \to P_1\left(\rfrac{-11}{4},0,0\right)\\
4+2a_2 = \rfrac{3}{2} \dimplies a_2 = \frac{-5}{4} \to P_2\left(\rfrac{-5}{4},0,0\right)
\end{array}\right.
\]

\end{problem}


\subsection{Perpendicular común}


\subsection{Proyecciones y simetrías}

\subsubsection{Proyección}

\subsubsection{Simetría de un punto respecto de otro punto}

Tirao.

\subsubsection{Simetría de un punto respecto de una recta}

Plano perpendicular a la recta que contenga al punto, intersección con la recta.

\subsubsection{Simetría de un punto respecto de un plano}

Recta perpendicular al plano, que pasa por el punto.

\newpage
\subsection{Ángulos}
\subsubsection{Ángulo formado por dos rectas: secantes, se cruzan y paralelas/coincidentes}
\subsubsection{Ángulo formado por dos planos}
\subsubsection{Ángulo formado por recta y plano}

\subsection{Distancias}
\subsubsection{Entre 2 puntos}
\subsubsection{Plano mediador}
\subsubsection{Entre punto y plano}
\subsubsection{Planos bisectores}
\subsubsection{Entre 2 planos}
\subsubsection{Entre recta y plano}
\subsubsection{Entre punto y recta}
\subsubsection{Entre 2 rectas paralelas}
\subsubsection{Entre 2 rectas no paralelas}

\subsubsection{Volumen del paralelepípedo}


\chapter{Análisis}

\begin{defn}[Función real de variable real]
Una función real de una variable real es una aplicación definida entre dos conjuntos de números reales tal que a cada elemento del primer conjunto le corresponde un único elemento del segundo conjunto.

$\appl{f}{D\subset\real}{\real}$

\begin{itemize}
	\item $f$: símbolo de la función.
	\item $D(f) = \{x\in\real \tq \exists f(x)\}$
	\item $Rec(f) = \{y\in\real \tq \exists x \in\real f(x)=y\}$
\end{itemize}
\end{defn}

\paragraph{Ejemplos de dominios de funciones:} hojita impresa de repaso, incluyendo funciones trigonométricas y de las trigonométricas inversas.

\begin{problem}
Halla el dominio de las siguientes funciones:
\ppart $f(x) = \sqrt{\displaystyle\frac{x^2+5x+4}{x+4}}$ \obs Ojo con simplificar.

\ppart  $f(x) = \log\left(x^2+1\right)$

\ppart $f(x) = \arcsen\left(x^2-9\right)$

\solution

\end{problem}

\section{Límites y continuidad}

\begin{theorem}[Teorema\IS de existencia del límite]
\[\exists \lim_{x\to a}f(x) \dimplies \lim_{x\to a^+} f(x) = \lim_{x\to a^-}f(x)\]
\end{theorem}

\begin{problem}
Página 13, ejercicios 11-14. 
\solution
\end{problem}

\subsection{Indeterminaciones}

\begin{itemize}
	\item Racionales: $\rfrac{0}{0},\rfrac{\infty}{\infty},\infty-\infty,0·\infty, \rfrac{k}{0}$ (Ver por el libro, se dan por supuestas)
	\item Exponenciales $1^{\infty}; 0^0; \infty^0$
\end{itemize}


\begin{problem}
\textit{Para repasar en sus casas:}
\ppart 19b,d
\ppart 23g,h
\ppart 25ac
\solution
\end{problem}


\begin{prop}[Indeterminación\IS $1^{\infty}$]
\index{Indeterminación\IS $1^{\infty}$}
\[
\lim_{x\to a}f(x)^{g(x)} \to 1^{\infty} \implies \lim_{x\to a}f(x)^{g(x)} = e^\lambda, \lambda = \lim_{x\to a} \left(g(x)·[f(x)-1]\right)
\]
\end{prop}

\begin{proof}
\begin{align*}
\lim_{x\to a}f(x)^{g(x)} &= \lim_{x\to a}(1+f(x)-1)^{g(x)}
\\
&= \lim_{x\to a}\left(1+\frac{1}{\left(\frac{1}{f(x)-1}\right)}\right)^{g(x)}
= \lim_{x\to a}\left(1+\frac{1}{\left(\frac{1}{f(x)-1}\right)}\right)^{g(x)\frac{f(x)-1}{f(x)-1}}
\\
&= \lim_{x\to a}\left[\left(1+\frac{1}{\left(\frac{1}{f(x)-1}\right)}\right)^{\frac{1}{f(x)-1}}\right]^{g(x)(f(x)-1)}
\\
&= \lim_{x\to a}\left[\left(1+\frac{1}{\left(\frac{1}{f(x)-1}\right)}\right)^{\frac{1}{f(x)-1}}\right]^{\displaystyle\lim_{x\to a}g(x)(f(x)-1)} 
\\
&= e^{\displaystyle\lim_{x\to a}g(x)(f(x)-1)}
\end{align*}
\end{proof}

\subsubsection{Infinitésimos equivalentes}

Libro página 19. Ver tabla y hacer ejercicio 30.

\begin{defn}[Infinitésimos equivalentes]
Dadas $f(x)$, $g(x)$, $a\in\real$ decimos que $f(x)$ y $g(x)$ son infinitésimos equivalentes \textbf{en $x=a$} si y sólo si $\displaystyle\lim_{x\to a}{f(x)}{g(x)} = 1$
\end{defn}

\begin{problem}
Ejercicio 30
\solution
\end{problem}

\subsection{Continuidad (¡Tiene resumen para entregar!)}

\begin{defn}[Continuidad\IS en un punto]
Sea $\appl{f}{D\subset\real}{\real}$.

Se dice que $f(x)$ es continua en $x=a$ sí y solo si se cumplen las 3 condiciones siguientes:
\begin{itemize}
	\item $\exists \displaystyle\lim_{x\to a} f(x)$
	\item $\exists f(a)$
	\item $\displaystyle\lim_{x\to a}f(x) = f(a)$
\end{itemize}

\textit{También se puede decir de manera abreviada: \[f(x) \text{ continua en } x=a\dimplies \text{ existen y son iguales } \lim_{x\to a}f(x) \text{ y } f(a)\]}
\end{defn}

\begin{problem}
Halla el valor de $m$ para que la función sea continua en $x=0$.
\[f(x) = 
	\begin{cases}
		2x-5 & \text{ si }x\leq 0\\ 
		\frac{2x+m}{x+1} & \text{ si } x>0
	\end{cases}\]
\solution

Aplicamos: "$f(x)$ continua en $x=0$ si existen y son iguales $\displaystyle\lim_{x\to 0}$ y $f(0)$".

\begin{itemize}
	\item $f(0) =  2·0-5 = -5$
	\item $\displaystyle\lim_{x\to0} f(x)$. Para calcularlo necesitamos los límites laterales:
	\subitem $\displaystyle\lim_{x\to0^+} f(x) = \lim_{x\to0^+} \frac{2x+m}{x+1} = \frac{2·0+m}{0+1} = m$
	\subitem $\displaystyle\lim_{x\to0^-} f(x) = \lim_{x\to0^-} 2·x-5 = -5$

	\[\lim_{x\to 0} f(x) = -5 \dimplies m=-5 \text{ porque } \begin{cases}\displaystyle\lim_{x\to0^+}f(x) = m\\ \displaystyle\lim_{x\to0^-}f(x) = -5\end{cases}\]
	\item Si $m=-5$, $f(x)$ es continua en $x=0$.
\end{itemize}

\end{problem}

\paragraph{Tipos de discontinuidad}

\[
\begin{cases} 
	\text{Continua}\\
	\text{ Discontinua }
		\begin{cases}
			\text{Evitable}\\
			\text{Esencial}
				\begin{cases}
					\text{1ª especie}
						\begin{cases}
							\text{Salto finito}\\
							\text{Salto infinito}
						\end{cases}\\
					\text{2ª especie}
				\end{cases}
		\end{cases}
\end{cases}
\]

\textbf{Descripción: } Libro página 22. 
\begin{itemize}
	\item Evitables: $\exists \displaystyle\lim_{x\to a}f(x)$ pero $\begin{cases}\displaystyle\lim_{x\to a}f(x) \neq f(a)\\\text{o}\\\nexists f(a)\end{cases}$
	\subitem Para evitarlas, se define una nueva función: $f(x) = \begin{cases}f(x) & x\neq a\\ b & x=a \end{cases}$
	\item Esenciales (o inevitables): 
	\begin{itemize}
		\item De primera especie:
		\subitem De salto finito: ambos límites laterales son finitos pero distinto.
		\subitem De salto infinito: al menos un límite lateral es infinito.
		\item De 2ª especie: al menos un límite lateral no existe. $f(x) = \sqrt{x}$ y $f(x) = \log(x)$ en $x=0$.
	\end{itemize}
\obs Ver \fref{fig::fun-tipos-discontinuidad}.
\end{itemize}


\begin{figure}
\centering
\includegraphics[scale=0.25]{img/Funs/funcion_xy_discontin_4hStg}
\caption{Ejemplo de discontinuidad evitable}
\label{fig::fun-tipos-discontinuidad}
\includegraphics[scale=0.25]{img/Funs/funcion_xy_discontin_4jSGg}
\caption{Ejemplo de discontinuidad de 2ª especie}
\includegraphics[scale=0.25]{img/Funs/funcion_xy_discontin_6ueR5}
\caption{Ejemplo de discontinuidad de salto infinito}
\includegraphics[scale=0.25]{img/Funs/funcion_xy_discontin_U72G2}
\caption{Ejemplo de discontinuidad de salto finito}
\end{figure}



\begin{defn}[Continuidad\IS en un intervalo abierto]
	$f(x)$ es continua en $(a,b) \dimplies \forall x\in(a,b), f(x)$ es continua.
\end{defn}


\begin{defn}[Continuidad\IS por la izquierda y/o por la derecha]
	\begin{itemize}
		\item Una función $(x)$ es continua por la derecha de $a$ si $\displaystyle\lim_{x\to a^+}f(x) = f(x)$
		\item Una función $(x)$ es continua por la izquierda de $a$ si $\displaystyle\lim_{x\to a^-}f(x) = f(x)$
	\end{itemize}
\end{defn}

\begin{defn}[Continuidad\IS en un intervalo cerrado]
$f(x)$ es continua en $[a,b] \dimplies \forall x\in[a,b],\begin{cases} f(x) \text{ es continua en } (a,b)\\
\text{Continua por la derecha de } a\\
\text{Continua por la izquierda de } b\\\end{cases}$
\end{defn}


\begin{example}
Dada $f(x) = +\sqrt{x}$.

$f(x)$ es continua en su dominio, por ser función radical. $f(x)$ continua en $(0,\infty)$.

$f(x)$ no es continua en $x=0$, pero sí es continua \textit{por la derecha} en $x=0$, ya que $\displaystyle\lim_{x\to 0^+} f(x) = 0$ en $[0,\infty)$.
\end{example}


\begin{table}[hbtp]
\begin{tabular}{|l|l|}
\hline
Funciones polinómicas & Continuas en $\real$\\\hline\hline
$f(x) = \frac{P(x)}{Q(x)}$ con $P(x),Q(x)$ polinomios& Continuas en $\real-\{x \tq Q(x)=0\}$\\\hline
$f(x) = \sqrt[n]{P(x)}$ & $\begin{cases}
n\text{ impar: continua en } \real\\\hline
n\text{ par: continua en } \{x\in\real\tq P(x) \geq 0\}\end{cases}$\\\hline
$f(x) = a^x$ con $a>0$ & Continua en $\real$\\\hline
$f(x) = \log_ax$ (con $a>0, a\neq 1$) & Continua en $\real$\\\hline
$f(x) = \cos(x)$ y $f(x) = \sen(x)$  & Continua en $\real$\\\hline
$f(x) = \tg(x)$   & Continua en $\real-\left\{x=\rfrac{\pi}{2}k, k\in\mathbb{Z}\right\}$\\\hline
$f(x) = \arcsen(x)$ & Continua en su dominio.\\
$f(x) = \arccos(x)$ & Continua en su dominio.\\
$f(x) = \arctg(x)$ & Continua en su dominio.\\\hline
\end{tabular}
\label{tbl::ContinuidadFunElementales}
\caption{Continuidad de las funciones elementales}
\end{table}

\begin{problem}
Ejercicios 23.41 y 23.42.
\solution
\end{problem}

\begin{problem}Estudia la continuidad de las siguientes funciones y clasifica sus discontinuidades

\begin{itemize}
	\item $f_1(x) = \displaystyle\frac{x^2+5x+6}{x^2+2x}$
	\item $f_2(x) = \begin{cases}\log{x+1} & x\leq -1\\x & x>1\end{cases}$
\end{itemize}
\solution
\end{problem}



\begin{theorem}[Teorema\IS de Bolzano]
Sea $\appl{f}{D}{\real}$.

\[
\left.\begin{array}{c}f(x) \text{ continua en } [a,b]\\\text{Signo}(f(a))\neq \text{Signo}(f(b))\end{array}\right\}\implies \exists c\in(a,b) \tq f(c) = 0
\]

\end{theorem}
\obs Es una condición suficiente, no necesaria. Es decir, es $\implies $. Por ejemplo, $f(x) = (x-1)^2$ corta en $x=1$, pero no cumple las condiciones.

\begin{problem} Demuestra que la ecuación $x^3-7x^2-1$ tiene al menos una solución real en el intervalo $[0,10]$.
\solution

Sea $f(x) = x^3-7x^2-1$. Se trata de demostrar que $\exists c\in[0,10]\tq f(c) = 0$. Comprobamos que cumple el teorema de Bolzano.

\[
\left\{
	\begin{array}{c}
		f(x) \text{ es continua en } [0,10] \text{ por ser polinómica}\\
		\left.\begin{array}{c}
		f(x) = -1 \\
		f(10) = 299\end{array}
		\right\}\implies Signo(f(0))\neq Signo(f(10))
	\end{array}
\right\}\implies \exists c\in(0,10)\tq f(c)=0 
\]
\end{problem}

\begin{theorem}[Teorema\IS del valor intermedio]
Sea $\appl{f}{D}{\real}$.
\[
\left.\begin{array}{c}f(x) \text{ continua en } [a,b]\\
\exists k\in\real \begin{cases}f(a)\leq k\leq f(b)\\f(b)\leq k \leq f(a)\end{cases}\end{array}\right\}\implies \exists c\in(a,b) \tq f(c) = k
\]
\end{theorem}

\obs Para $k=0$, el teorema del valor intermedio se convierte en teorema de Bolzano.


\begin{problem}
Demuestra que las funciones $f(x) = \cos(x)$ y $g(x) = x$ se cortan en algún punto.
\solution

Basta considerar la función $h(x) = f(x) - g(x) = \cos(x)-x$. Así, demostrar que las funciones se cortan será equivalente a demostrar que $\exists c\in\real\tq h(x)=0$. 

Esta función es continua en $\real$, por ser resta de funciones continuas en $\real$.

Buscamos $c\in\real\tq h(c)=0$. Consideramos $h(0) = 1$ y $h\left(\rfrac{\pi}{2}\right) = \rfrac{-\pi}{2}<0$.

Por el teorema de Bolzano, $\exists c\in\left(0,\rfrac{\pi}{2}\right)\tq h(c) = 0$, por lo que podemos concluir que las funciones se cortan.
\end{problem}

\begin{problem}
\begin{itemize}
	\item 35.108 (continuidad)
	\item 36.113 (continuidad)
	\item 36.117 (problema - continuidad)
	\item 36.119,120 (límites)
	\item Página 37.autoevaluación.
\end{itemize}
\solution
\end{problem}

\begin{theorem}[Teorema\IS de Weierstrass]
Si $f(x)$ es continua en $[a,b]$, entonces tiene un máximo y un mínimo absolutos en $[a,b]$.
\end{theorem}

\obs Este teorema se utilizará en el siguiente tema.


%%%%%%%%%%%%%%%%%%%%%%%%%%%%%%%%%%%%%%%%%%%%%%%%%%%%%%%%%%%%%%%%%%
%%%%%%%%%%%%%%%%%%%%%%%%%%%%%%%%%%%%%%%%%%%%%%%%%%%%%%%%%%%%%%%%%%
%%%%%%%%%%%%%%%%%%%%%%%%%%%%%%%%%%%%%%%%%%%%%%%%%%%%%%%%%%%%%%%%%%
%%%%%%%%%%%%%%%%%%%%%%%%%%%%%%%%%%%%%%%%%%%%%%%%%%%%%%%%%%%%%%%%%%
%%%%%%%%%%%%%%%%%%%%%%%%%%%%%%%%%%%%%%%%%%%%%%%%%%%%%%%%%%%%%%%%%%
%%%%%%%%%%%%%%%%%%%%%%%%%%%%%%%%%%%%%%%%%%%%%%%%%%%%%%%%%%%%%%%%%%
%%%%%%%%%%%%%%%%%%%%%%%%%%%%%%%%%%%%%%%%%%%%%%%%%%%%%%%%%%%%%%%%%%
%%%%%%%%%%%%%                                    %%%%%%%%%%%%%%%%%
%%%%%%%%%%%%%                                    %%%%%%%%%%%%%%%%%
%%%%%%%%%%%%%                                    %%%%%%%%%%%%%%%%%
%%%%%%%%%%%%%                                    %%%%%%%%%%%%%%%%%
%%%%%%%%%%%%%            DERIVABILIDAD           %%%%%%%%%%%%%%%%%
%%%%%%%%%%%%%                                    %%%%%%%%%%%%%%%%%
%%%%%%%%%%%%%                                    %%%%%%%%%%%%%%%%%
%%%%%%%%%%%%%                                    %%%%%%%%%%%%%%%%%
%%%%%%%%%%%%%                                    %%%%%%%%%%%%%%%%%
%%%%%%%%%%%%%%%%%%%%%%%%%%%%%%%%%%%%%%%%%%%%%%%%%%%%%%%%%%%%%%%%%%
%%%%%%%%%%%%%%%%%%%%%%%%%%%%%%%%%%%%%%%%%%%%%%%%%%%%%%%%%%%%%%%%%%
%%%%%%%%%%%%%%%%%%%%%%%%%%%%%%%%%%%%%%%%%%%%%%%%%%%%%%%%%%%%%%%%%%
%%%%%%%%%%%%%%%%%%%%%%%%%%%%%%%%%%%%%%%%%%%%%%%%%%%%%%%%%%%%%%%%%%
%%%%%%%%%%%%%%%%%%%%%%%%%%%%%%%%%%%%%%%%%%%%%%%%%%%%%%%%%%%%%%%%%%
%%%%%%%%%%%%%%%%%%%%%%%%%%%%%%%%%%%%%%%%%%%%%%%%%%%%%%%%%%%%%%%%%%
%%%%%%%%%%%%%%%%%%%%%%%%%%%%%%%%%%%%%%%%%%%%%%%%%%%%%%%%%%%%%%%%%%

\newpage\section{Derivadas y derivabilidad}

\subsection{Introducción y repaso}
\begin{defn}[Pendiente de una recta]
Sea la recta $y=mx+n$.

Se define \textbf{pendiente de la recta}, $m=\frac{\Delta y}{\Delta x}$
\end{defn}

\begin{defn}[Derivada\IS en un punto]
Se define $f'(a)$ como la derivada de $f(x)$ en el punto $x=a$.

\[f'(a) = \lim_{x\to a}\frac{f(x)-f(a)}{x-a} \overset{(1)}{=} \lim_{h\to 0}\frac{f(a+h)-f(a)}{h}\]

$(1): h=x-a \dimplies x=a+h$

\textit{Ver \fref{fig::funinterpretacionderivadapunto}}.
\end{defn}

\begin{example}
Dada $f(x) = |x|$, calcula $f'(0)$.

\[
f(x) = \begin{cases}x&\text{ si } x>0 \\ -x & \text{ si }x\leq 0\end{cases}
\]

Calculamos:

\[
\lim_{x\to 0}\frac{f(x)-f(0)}{x-0} = \begin{cases}
\displaystyle\lim_{x\to 0^+} \frac{f(x)}{x} = \displaystyle\lim_{x\to 0^+} \frac{x}{x} = 1\\\\
\displaystyle\lim_{x\to 0^-} \frac{f(x)}{x} = \displaystyle\lim_{x\to 0^-} \frac{-x}{x} = -1
\end{cases}
\]

\label{derivEjemplo}

Los límites laterales no coinciden, por lo tanto, $\nexists \displaystyle\lim_{x\to 0}\frac{f(x)-f(a)}{x-a} \dimplies \nexists f'(a)$
\end{example}




\subsubsection{Interpretación geométrica de la derivada}

Ver \fref{fig::funinterpretacionderivadapunto}.

\begin{figure}[hbp]
\centering
\includegraphics[scale=0.5]{img/DerivadaInterGeometrica}
\label{fig::funinterpretacionderivadapunto}
\caption{Interpretación geométrica de la derivada} Cuando $h\to0$, el punto $B$ se acercará cada vez más al punto $A$, dando lugar a la recta tangente. 
%
Para una mejor comprensión consultar la versión de Geogebra: https://www.geogebra.org/m/jwtw6mdt\#material/f52nQ7T5

\end{figure} 


\subsubsection{Derivabilidad}
\begin{defn}[Derivabilidad\IS en un punto]
\[f(x) \text{ derivable en } x=a\dimplies \exists f'(a)\]
\end{defn}

\begin{prop}
$f(x)$ derivable en $x=a \implies f(x)$ continua en $x=a$
\obs El recíproco no es cierto. Basta comprobar el ejemplo \ref{derivEjemplo} ($f(x) = |x|$ es continua en $x=0$, pero no derivable en $x=0$).
\end{prop}

\begin{defn}[Derivabilidad\IS en un intervalo abierto]
\[f(x) \text{ derivable en } x\in(a,b) \dimplies \forall c\in(a,b) \exists f'(c) \]
\end{defn}

\begin{defn}[Dominio de derivabilidad]
El dominio de derivabilidad de una función $f(x)$ es el mayor conjunto en el que la función es derivable.
\end{defn}

\begin{example}
$f(x) = |x|$ no es derivable en $x=0$.

El dominio de derivabilidad de $f(x)$ es $\real-\{0\}$
\end{example}


\paragraph{Derivabilidad lateral:} De la misma manera que existía la \textit{continuidad lateral}, también podemos hablar de \textit{derivabilidad lateral}. 

\begin{defn}[Derivada lateral]
La derivada lateral de $f$ en $x=a$ por la derecha, escrita $f'(a^+)$, si existe:

\[f'(a^+) = \lim_{h\to 0^+} \frac{f(x+h)-f(x)}{h}\]

La derivada lateral de $f$ en $x=a$ por la izquierda, escrita $f'(a^-)$, si existe:

\[f'(a^-) = \lim_{h\to 0^-} \frac{f(x+h)-f(x)}{h} =  \lim_{h\to 0^+} \frac{f(x-h)-f(x)}{-h}\]

\obs Esta última igualdad se debe a que $h\to 0^-\implies h<0$. Si resulta menos confuso, puede elegirse trabajar siempre con $h>0$ y así los signos quedan explicitados.
\end{defn}

\begin{problem} Estudia la derivabilidad en $x=0$ de la función 
\label{prb::derivab1}
\[f(x) = \begin{cases} x^2+3x & \text{ si } x\leq 0\\ 3·\left(\frac{x^2+x}{x+1}\right)&\text{ si } x>0\end{cases}\]
\solution

$f(x)$ será derivable en $x=0$ si $\exists f'(0)$. Dado que $f(x)$ está definida a trozos, calculamos la derivadas laterales.

\[f'(0^-) = \lim_{h\to 0^+} \frac{f(0-h)-f(0)}{-h} = \lim_{h\to 0^-} \frac{(0-h)^2+3·(0-h)-(0^2+3·0)}{-h} = \lim_{h\to 0^-} \frac{h(h-3)}{-h} = +3\]
\[f'(0^+) = \lim_{h\to 0^+} \frac{f(0+h)-f(0)}{h} = \lim_{h\to 0^+} \frac{3·\frac{(0+h)^2+(0+h)}{0+h+1} - (0^2-3·0)}{h} = \lim_{h\to 0^+} \frac{3·\frac{h^2+h}{h+1}}{h} = \]
\[=\lim_{h\to 0^+} \frac{3·h·(h+1)}{h(h+1)} = 3 \]

\textbf{Conclusión:} Dado que  $f'(0^-) \eq f'(0^+) \implies f'(0) = 3$, por lo que la función es derivable en $x=0$.

\begin{figure}[h!]
\centering
\includegraphics[scale=0.5]{img/DerivabilidadEjer1}
\label{fig::DerivabEjer1}
\caption{Representación gráfica del problema \ref{prb::derivab1}.}
Claramente la derivada en $x=0$ no puede existir, dado que la función no es continua.
\end{figure}

\obs También podríamos haber calculado la derivada lateral por la izquierda de la siguiente manera:

\[f'(0^-) = \lim_{h\to 0^-} \frac{f(0+h)-f(0)}{h} = \lim_{h\to 0^-} \frac{h^2+3·h-(0^2+3·0)}{h} = \lim_{h\to 0^-} \frac{h(h+3)}{h} = +3\]

\end{problem}

\begin{problem}[Página 43, 14.]
¿Es la siguiente función derivable en $x=-2,x=0,x=2$?

\[f(x) = \begin{cases}
x^2 & \text{ si } x<-2\\
-4(x+1) & \text{ si } -2<x\leq0\\
3x^2-4 & \text{ si } 0<x\leq2\\
12x+1 & \text{ si } x>2
\end{cases}\]
\solution
\end{problem}

\begin{problem}[Página 56, ejercicio 78.]
Dada la función $f(x)$, calcula $a,b,c\in\real$ para que la función sea derivable en $x=1$, sabiendo que $f(0) = f(4)$.
\solution

Solución: $a=-\rfrac{7}{4}; b=1; c=\rfrac{1}{4}$

\begin{figure}[h!]
\centering
\includegraphics[scale=1.1]{img/DerivabilidadEjer5678}
\label{ejercicioDerivabilidad}
\caption{Ejercicio sacado del libro de SM}
\end{figure}

\end{problem}


\subsubsection{Función derivada}

\begin{defn}[Función derivada]
Dada $\appl{f}{D(f)\subset\real}{\real}$. Sea $Dv(f)$ el dominio de derivabilidad de $f$.

La función derivada denotada por $\appl{f'(x)}{Dv(f)\subset\real}{\real}$ hace corresponder a cada $a\in Dv(F)$ el valor $f'(a)$.
\end{defn}

% \begin{table}[hbp]
% \centering
% \begin{tabular}{|c|c|}\hline
% Función & Derivada\\
% \hline
% a&b\\\hline
% \end{tabular}
% \caption{Tabla de derivadas}
% \label{tbl::Derivadas}
% \end{table}

\begin{problem}

\begin{itemize}
	\item 51.58 (2 trigonométricas inversas)
	\item 59.113 (8 variadas. Solo la h tiene una trigonométrica inversa.)
	\item 57.79-81 (Ejercicios resueltos)
\end{itemize}

\solution
\end{problem}


\subsection{Aplicaciones de la derivada}

\subsubsection{Recta tangente y recta normal}

\paragraph{Ecuación de la recta tangente:} Utilizando la ecuación de la recta punto-pendiente y la interpretación gráfica de la derivada (ver \ref{fig::funinterpretacionderivadapunto}), se obtiene fácilmente la siguiente ecuación:

\begin{mdframed}
	\begin{equation}
		\label{eq::rectatangente}
		\text{Recta tangente a }f(x)\text{ en }x_0 \to y-f(x_0) = f'(x_0)·(x-x_0)
	\end{equation}
\end{mdframed}

\begin{problem}
Demuestra que la recta $y=-x$ es tangente a la curva dada por la ecuación: $y=x^3+6x^2+8x$
\solution

Consideramos $f(x) = x^3-6x^2+8x \to f'(x) = 3x^2-12x+8$.

Buscamos $c\in\real\tq f'(c) = -1$.

\[
	3c^2+12c+8=-1 \dimplies \begin{cases}c_1 = 1\\c_2=3\end{cases}
\]

Los posibles puntos de tangencia son $P_1(c_1,f(c_1)) = (1,3)$ y $P_2(c_2,f(c_2)) = (3,-3)$.

Es necesario comprobar que dichos puntos son realmente de tangencia, es decir, que pertenecen a la recta y a la gráfica.

\[P_1: 3\neq -1 \implies \text{ no pertenece a la recta}\]
\[P_2: -3\eq -3 \implies \text{ sí pertenece a la recta}\]

\textbf{Conclusión: } El punto de tangencia de la recta $y=-x$ a la gráfica $f(x) = x^3-6x^2+8x$ es $P_2(3,-3)$
\end{problem}

\paragraph{Ecuación de la recta normal:} Dos rectas dadas, en 2 dimensiones, $r: y=m_rx+n_r\quad;\quad s:y=m_sx+n_s$ son perpendiculares si y sólo si $m_r·m_s = -1 \dimplies m_s = \rfrac{-1}{m_r}$.
%
Aplicando este resultado a la fórmula de la recta tangente anterior, tenemos:


\begin{mdframed}
	\begin{equation}
		\label{eq::rectatangente}
		\text{Recta normal a }f(x)\text{ en }x_0 \to y-f(x_0) = \rfrac{-1}{f'(x_0)}·(x-x_0)
	\end{equation}
\end{mdframed}


\subsubsection{Teoremas de derivabilidad}

\begin{theorem}[Teorema\IS de Rolle]
\[
\left.
	\begin{array}{c}
		f(x)\text{ continua en } [a,b]\\
		f(x)\text{ derivable en } (a,b)\\
		f(a)=f(b)
	\end{array}
\right\}\implies \exists c\in(a,b)\tq f'(c)=0
\]
\end{theorem}
\obs ¿Puede haber más de un punto? ¡Claro que sí! Basta pensar en una función horizontal.

\begin{problem}
% Página 66.1
Sea $f(x) = 2x^5+x+a$.  Demuestra que $\exists!c\in\real\tq f(c)=0$
\solution
Por el teorema de Bolzano, hay al menos una raíz real.

\ul{Veamos que es única.} Si hubiera otra raíz real, $d$, tendríamos $f(c) = f(d)$ en una función continua en $[c,d]\subset\real$ y derivable en $(c,d)\subset\real$ por ser una función polinómica.
%
Así, se puede aplicar el teorema de Rolle, argumentando que $\exists c\in\real\tq f'(c) = 0$, pero $f'(x) = 10x^4+1 \neq 0 \;\forall x\in\real$.

\textbf{Conclusión: } dado que $f(x)$ cumple todas las hipótesis del teorema de Rolle, debemos concluir necesariamente que no hay otra raíz real $d$\footnote{Sino, el teorema fallaría y eso no es posible.}. 
\end{problem}

\begin{problem}
67.8
\solution
\end{problem}

\begin{theorem}[Teorema\IS del valor medio]
\[
\left.
	\begin{array}{c}
		f(x)\text{ continua en } [a,b]\\
		f(x)\text{ derivable en } (a,b)\\
	\end{array}
\right\}\implies \exists c\in(a,b)\tq f'(c)=\frac{f(b)-f(a)}{x-a}
\]

\obs El teorema de Rolle es un caso particular de este teorema que se da cuando $f(a) = f(b)$
\end{theorem}
\obs ¿Puede haber más de un punto? ¡Claro que sí!

\begin{problem}
Considera la función $f(x) = x·e^{-x^2}$.

Calcula el valor del parámetro real $a$ para el que se puede aplicar el teorema de Rolle en el intervalo $[0,1]$ a la función $g(x) = f(x) + ax$
\solution

\[g(x) = x·e^{-x^2} + ax = x·\left(a+e^{-x^2}\right)\]

Las hipótesis del teorema de Rolle son:

\[
\left.
	\begin{array}{l}
		g(x)\text{ continua en } [0,1] \to \text{ En este caso se cumple por ser suma de función polinómica y exponencial}\\
		g(x)\text{ derivable en } (0,1) \to \text{ En este caso se cumple por ser suma de función polinómica y exponencial}\\
		g(a)=g(b) 
	\end{array}
\right\}
\]

Necesitamos que $g(0) = g(1)$. 

\[0·\left(a+e^{-0^2}\right) = 1·\left(a+e^{-1^2}\right) \dimplies 0 = a+e^{-1} \dimplies a=-e^{-1}\]

\obs Este problema salió en las noticias de 2019 porque se preguntó en la EVAU de valencia y los alumnos se enfadaron mucho.
\end{problem}

\begin{problem}

\ppart[17.2-MadA] Estudia la derivabilidad en $x=0$ de $f(x) = \begin{cases}\displaystyle x·e^{2x} &\mbox{ si } x<0\\ \displaystyle\frac{\ln(x+1)}{x+1}&\mbox{ si } x\geq 0\end{cases}$ en $x=0$. 
\ppart[16.1-MadB] Determina el polinomio $f(x)$, sabiendo que $\forall x\in\real f'''(x) = 12$ y además verifica $f(1) = 3; f'(1) = 1; f''(1) = 4$.
\ppart[16.1-MadB] Estudie la continuidad y la derivabilidad en $x=0$ y en $x=1$ de $f(x) = \begin{cases} 0& \text{ si } x\leq 0\\|x\ln(x)&\text{ si} x>0\end{cases}$
\solution
\end{problem}

\subsubsection{Regla de L'Hôpital}

\subsubsection{Monotonía}

\begin{theorem}[Teorema\IS de monotonía]
\end{theorem}
\begin{theorem}[Teorema\IS de extremos relativos]
\end{theorem}

\subsubsection{Optimización}


\begin{theorem}[Teorema\IS de Weierstrass]
Si $f(x)$ es continua en $[a,b]$, entonces tiene un máximo y un mínimo absolutos en $[a,b]$.
\end{theorem}


\section{Análisis sistemático de una función}

\section{Integrales}

\subsection{Cálculo de áreas de recintos cerrados}

\newpage
\printindex
\listoffigures
\listoftables

\end{document}
