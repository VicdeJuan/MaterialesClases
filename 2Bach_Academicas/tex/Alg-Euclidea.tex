
\section{Geometría euclídea (del 25/01 al final)}

Llamamos geometría euclídea al espacio en el que podemos medir, cosa que hasta ahora no era posible.

Todo surge desde el módulo de un vector. Llamamos \concept[Módulo de un vector]{módulo de un vector} a la longitud que tiene. Dado $\vec{v}$, se define el módulo como $|\vec{v}|$.

\subsection{Producto, escalar, vectorial y mixto}

\paragraph{Introducción sobre el origen del producto escalar y vectorial}

Fuentes consultadas:
\begin{itemize}
  \item \href{http://www.suitcaseofdreams.net/Geometric_multiplication.htm}{Relación forma polar y binómica del producto complejo}
  \vspace{-0.4cm}
  \item \href{https://www2.clarku.edu/faculty/djoyce/complex/mult.html}{Interpretación geométrica del producto complejo}
  \vspace{-0.4cm}
  \item \href{https://www.quora.com/Who-invented-the-dot-product-and-cross-product}{Historia y aplicación de los cuaterniones los productos}
  \vspace{-0.4cm}
  \item \href{https://es.wikipedia.org/wiki/Cuaterni%C3%B3n}{ Extensión de los complejos al grupo de los quaterniones}
\end{itemize}

Dados 2 números complejos $z_1 = a_1+b_1i$, $z_2 = a_2+b_2i$. Expresando estos números complejos en forma polar tenemos: $z_1=r_{\alpha_1}$ y $z_2 = s_{\alpha_2}$.  

$z_1·z_2 = (a_1a_2 - b_1b_2) + (a_1b_2+a_2b_1)i = r·s_{\alpha_1+\alpha_2}$.

Tomando $z_1·\bar{z_2} = (a_1a_2 + b_1b_2) + (a_1b_2-a_2b_1)i = r·s_{\alpha_1-\alpha_2}
$

\subparagraph{Estudio de la parte real (producto escalar)}

En $Re(z_1·\bar{z_2}) = a_1a_2 + b_1b_2 = Re(r·s_{\alpha_1-\alpha_2})$

Para calcular $Re(r·s_{\alpha_1-\alpha_2}) = Re(r·s·\cos(\alpha_1-\alpha_2) + i·r·s·\sen(\alpha_1-\alpha_2)$, por lo que podemos completar:

$ a_1a_2 + b_1b_2 = |z_1||z_2|·\cos(\alpha_1-\alpha_2)$, y, siendo conscientes que $\alpha_1-\alpha_2$ es el ángulo que forman los 2 vectores, obtenemos la expresión del producto escalar de 2 vectores.


\subparagraph{Estudio de la parte imaginaria (producto vectorial)}

Al extender este razonamiento al grupo de los cuaterniones, tendríamos:

\newcommand{\quat}{\vec}

$Re(z_1\bar{z_2}) = Re\left((b_1\quat{i}+c_1\quat{j}+d_1\quat{k})·(-b_2\quat{i}-c_2\quat{j}-d_2\quat{k})\right) = (b_1b_2+c_1c_2+d_1d_2)$

Lo espectacular viene al considerar la parte "imaginaria" (aunque no tengamos claro cómo se define ese concepto en los cuaterniones):

$Im(z_1\bar{z_2}) = f(\quat{i},\quat{j},\quat{k})$, cuya expresión analítica es la del producto vectorial, ya que en grupo de los cuaterniones $\quat{i}\quat{j}=\quat{k}$ y todo eso.

\subsubsection{Producto escalar}

\begin{itemize}
  \item Definición.
  \item Base ortogonal y ortonormal.
  \item Expresión analítica.
  \item Cálculo del módulo (porque Pitágoras tridimensional no funciona).
  \item Base canónica.
  \item Interpretación geométrica (proyección).
\end{itemize}

\subsubsection{Producto vectorial}
\begin{itemize}
  \item Definición.
  \subitem Regla de la mano derecha
  \item Expresión analítica.
  \item Interpretación geométrica. 
\end{itemize}


\subsubsection{Producto mixto}

\begin{itemize}
  \item ¿Qué ocurre si en el producto vectorial meto un vector en lugar de $\vec{i},\vec{j},\vec{k}$?
  \item Definición.
  \item Expresión analítica.
  \item Interpretación geométrica. 
\end{itemize}

\subsubsection{Practicamos en general ejercicios del tema 9}

\subsection{Aplicación de los productos}

\subsubsection{Vector normal del plano}
\subsection{Perpendicular común}

\subsection{Proyecciones y medidas}
\subsubsection{Simetría de un punto respecto de otro punto}
\subsubsection{Simetría de un punto respecto de una recta}
\subsubsection{Simetría de un punto respecto de un plano}

\newpage
\subsection{Ángulos}
\subsubsection{Ángulo formado por dos rectas: secantes, se cruzan y paralelas/coincidentes}
\subsubsection{Ángulo formado por dos planos}
\subsubsection{Ángulo formado por recta y plano}

\subsection{Distancias}
\subsubsection{Entre 2 puntos}
\subsubsection{Plano mediador}
\subsubsection{Entre punto y plano}
\subsubsection{Planos bisectores}
\subsubsection{Entre 2 planos}
\subsubsection{Entre recta y plano}
\subsubsection{Entre punto y recta}
\subsubsection{Entre 2 rectas paralelas}
\subsubsection{Entre 2 rectas no paralelas}

\subsubsection{Volumen del paralelepípedo}
