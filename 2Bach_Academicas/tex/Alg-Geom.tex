
\chapter{Geometría analítica}

% \paragraph{Espacio vectorial: $\real^3$}

% \begin{defn}[Espacio vectorial]
% Un espacio vectorial sobre $K$ es una estructura algebraica, $(V,+,·)$, donde $V$ es un conjunto cualquiera y $\appl{+}{V\times V}{V}$ y $\appl{·}{K\times V}{V}$



% \end{defn}

\section{Espacio vectorial}

\begin{defn}[Espacio vectorial]
Un espacio vectorial es una estructura algebraica, $(\mathcal{V},+,·)$, donde $\mathcal{V}$ es un conjunto cualquiera y $\appl{+}{\mathcal{V}\times \mathcal{V}}{\mathcal{V}}$ y $\appl{·}{\real\times \mathcal{V}}{\mathcal{V}}$

Las operaciones $+$ y $·$ deben cumplir las siguientes propiedades:
\begin{itemize}
    \item $\vec{u} + (\vec{v} + \vec{w}) = (\vec{u} + \vec{v}) + \vec{w}, \qquad \forall \vec{u}, \vec{v}, \vec{w} \in V $  (asociativa)
    \item $\vec{u} + \vec{v} = \vec{v} + \vec{u}, \qquad \forall \vec{u}, \vec{v} \in V$ (conmutativa)
\item $ \exists{}\vec{0} \in{} V : $  $ \vec{u} + \vec{0} = \vec{u} , \forall{} \vec{u} \in{} V
$ (elemento neutro)

\item $
   \forall{} \vec{u} \in{} V , \quad
   \exists{} \vec{-u} \in{} V : $  $
    \vec{u} + (\vec{-u}) = \vec{0}
$ (elemento opuesto)

\item $
   \mathit{a} \cdot (\mathit{b} \cdot \vec{u})=(\mathit{a} \cdot \mathit{b}) \cdot \vec{u} ,$  $
   \forall{} \mathit{a} ,\mathit{b} \in{}K , $  $
   \forall{} \vec{u} \in{} V
$

\item$
   \exists{e} \in{K}: $  $
   e \cdot \vec{u}   = \vec{u} , 
   \forall{} \vec{u} \in{} V
$ (elemento neutro del producto)
\item $
   \mathit{a} \cdot (\vec{u}+ \vec{v}) =
   \mathit{a} \cdot \vec{u}+ \mathit{a} \cdot \vec{v} , $  $
   \forall{} \mathit{a}\in{}K , $  $
   \forall{} \vec{u}, \vec{v} \in{} V
$ (propiedad distributiva)
\item $
   (\mathit{a} + \mathit{b}) \cdot \vec{u} =
   \mathit{a} \cdot \vec{u} + \mathit{b} \cdot \vec{u} , $  $
   \forall{} \mathit{a}, \mathit{b} \in{} K , $  $
   \forall{} \vec{u} \in{} V
$ (propiedad distributiva)
\end{itemize}

\end{defn}

En nuestro caso, el espacio vectorial con el que trabajaremos será $\mathcal{V}^3 = (\real^3, + , ·)$ sobre $\real$, siendo la operación $+$ la suma de vectores habitual y $·$ el producto por un \textbf{escalar real}.

Los elementos de $\mathcal{V}^3$ se denominan vectores y son ternas de números (reales) con los que se pueden hacer operaciones. 
%
Escribiremos $\vec{u} = (u_1,u_2,u_3)$.


%
\begin{example}
Recordamos:
Sean $u_1 = (1,2,3)$ y $u_2 = (0,1,-2)$, tenemos:
  \begin{itemize}
      \item $u_1 + u_2 = \hide{(1+0, 2+1, 3-2) = (1,3,1)}$
      \item $-u_2 = (0,-1,2)$
      \item $u_1-u_2 = u_1+ (-u_2) = (1,2,3) + (0,-1,2) = (1,1,1)$
      \item $2·u_1 + 3·u_2 = \hide{(2,4,6) + (0,3,-6) = (2,7,0)}$
      \obs \hide{A esta operación la llamamos \concept[Combinación lineal\IS de vectores]{combinación lineal de vectores}}.
  \end{itemize}
\end{example}

\obs Las matrices también forman un espacio vectorial. Ver libro página 179.

\obs Todavía no hemos definido nada de producto de vectores. \textit{Explicación de porqué el orden seguido es diferente}

\obs Todavía no hemos hablado (ni lo vamos a hacer de momento) de vectores libres y vectores fijos.

\subsection{Bases de espacios vectoriales y coordenadas}

\begin{defn}[Subespacio generado] 
Sean $ G = \{v_1,v_2,...,v_n\}$ un conjunto de vectores de un espacio vectorial.

Llamamos subespacio generado al conjunto de todas las combinaciones lineales de vectores de $G$.

\obs Llamaremos a $G$ \concept[Espacio vectorial\IS Sistema de generadores]{Sistema de generadores}.
\end{defn}

\begin{example}
Dado $G=\{(1,2,1) , (0,0,1)\}$. ¿Podemos decir que $\vec{a} = (2,2,3) \in G$? ¿Y que $\vec{b} = (1,2,0)\in G$?


\subparagraph{a)} $\vec{a} = (2,2,3)$\\

\subparagraph{b)} $\vec{b} = (1,2,0)$\\

\end{example}

\begin{example}

Siempre que tengamos un origen de coordenadas y un sistema de referencia...

\begin{itemize}
  \item El subespacio generado por un vector sería una recta.
  \item El subespacio generado por 2 vectores sería un plano.
  \item El subespacio generado por 3 vectores sería el espacio.
\end{itemize}
\end{example}

Un concepto fundamental a la hora de trabajar con vectores es la "base del espacio vectorial". \hide{(Libro página 257)}

\begin{defn}[Base de un espacio vectorial][Espacio vectorial\IS Base]
Un conjunto de vectores $\mathcal{B} = \{b_1,b_2,b_3\}$ es una base de $\mathcal{V}^3$ si:
  \begin{itemize}
      \item Son linealmente independientes.
      \item El subespacio que generan es $\mathcal{V}^3$. (Es decir, que cualquier vector de $\mathcal{V}^3$ puede escribirse como combinación lineal de los vectores de la base).
  \end{itemize}
\end{defn}


\begin{defn}[Dimensión de un espacio vectorial][Espacio vectorial\IS Dimensión]
Sea $\mathcal{V}$ un espacio vectorial y $\mathcal{B}$ una base del mismo.

Llamamos \textbf{Dimensión} del espacio vectorial al número de vectores de $\mathcal{B}$.
\end{defn}

En este curso, para saber si un conjunto de vectores es base de un espacio vectorial, basta comprobar que son linealmente independientes y que hay tantos vectores linealmente independientes como dimensión tiene el espacio vectorial.

\begin{problem}

  Determina si los siguientes conjuntos son bases de $\mathcal{V}^3$ (que tiene dimensión: \hide{3})
    
\ppart $\mathcal{B}_1 = \{(1,0,0), (0,0,1)\}$
\ppart $\mathcal{B}_2 = \{(1,0,0), (0,1,0),(0,0,1)\}$
\ppart $\mathcal{B}_3 = \{(1,0,0), (0,1,0),(0,0,1),(1,1,1)\}$
\ppart $\mathcal{B}_4 = \{(1,0,3),(1,2,-1),(0,1,2)\}$
\ppart $\mathcal{B}_5 = \{(2,-1,5),(1,-2,-4),(4,-5,-3)\}$
    \solution

        
        \spart $\mathcal{B}_1 = \{(1,0,0), (0,0,1)\}$
        \subitem \hide{No es base porque hay vectores de $\mathcal{V}^3$ que no se pueden expresar como combinación lineal de los vectores de $\mathcal{B}_1$, por ejemplo, $(0,1,0)$}
        
        \spart $\mathcal{B}_2 = \{(1,0,0), (0,1,0),(0,0,1)\}$
        \subitem Sí es una base porque $\mathcal{B}_2$ tiene 3 vectores linealmente independientes: $Rg\displaystyle\begin{pmatrix}1&0&0\\0&1&0\\0&0&1\end{pmatrix} = 3$.
        
        \spart $\mathcal{B}_3 = \{(1,0,0), (0,1,0),(0,0,1),(1,1,1)\}$
        \subitem No es una base porque el vector $(1,1,1)$ se puede expresar como combinación lineal de los 3 primeros. 
        
        Así, aunque $\displaystyle Rg\begin{pmatrix}1&0&0\\0&1&0\\0&0&1\\1&1&1\end{pmatrix} = 3$, no podríamos decir que $\mathcal{B}_3$ fuera una base. Solo podríamos decir que \hide{es un \textbf{Sistema de generadores de $\mathcal{V}^3$}}
        
        
        \spart $\mathcal{B}_4 = \{(1,0,3),(1,2,-1),(0,1,2)\}$
        \subitem Estudiamos $\displaystyle \begin{pmatrix}\vec{u_1}\\\vec{u_2}\\\vec{u_3}\end{pmatrix} = \begin{pmatrix}1&0&3\\1&2&-1\\0&1&2\end{pmatrix}$ 
        Buscamos si tiene rango máximo. Para ello, $\begin{vmatrix}1&0&3\\1&2&-1\\0&1&2\end{vmatrix} \neq 0 \implies $ por lo que podemos decir que son linealmente independientes\footnote{Si no lo fueran, el determinante sería 0}. Así, tenemos 3 vectores linealmente independientes, por lo que podemos decir que \textbf{sí son una base de $\mathcal{V}^3$.}
        
        
        \spart $\mathcal{B}_5 = \{(2,-1,5),(1,-2,-4),(4,-5,-3)\}$
        \subitem \hide{Estudiamos $\displaystyle \begin{pmatrix}\vec{u_1}\\\vec{u_2}\\\vec{u_3}\end{pmatrix} = \begin{pmatrix}2&-1&5\\1&-2&-4\\4&-5&-3\end{pmatrix}$ 

        Su determinante es 0, por lo que no tiene rango 3, por lo que no son linealmente independientes.

        \obs También podríamos haber estudiado $\begin{pmatrix}\vec{u_1} \vec{u_2} \vec{u_3}\end{pmatrix}$ }
    
\end{problem}


\begin{defn}[Coordenadas de un vector]
Sea $\mathcal{B} = \{\vec{b}_1,\vec{b}_2,\vec{b}_3\}$ es una base de $\mathcal{V}$.

Si $\vec{u} = c_1 · \vec{b_1} +  c_2·\vec{b_2} + c_3\vec{b_3}$, decimos que $(c_1,c_2,c_3)$ son las coordenadas del vector $\vec{u}$ en la base $\mathcal{B}$.
\end{defn}

\obs "Sea el vector $\vec{u} = (1,2,3)$" deja de tener sentido, ya que necesitamos estar refiriéndonos a una base. 
%
Por ello, a partir de ahora, intentaremos escribir los vectores $\vec{u} = \vec{i} + 2\vec{j} + 3\vec{k}$, dejando bien claro en qué base estamos trabajando. 

\textbf{Notación:} $\vec{i},\vec{j},\vec{k}$ forman la base canónica, con $\vec{i} = (1,0,0); \vec{j} = (0,1,0); \vec{k} = (0,0,1)$


\paragraph{Cambios de base}

\begin{problem}

Sea 
$\mathcal{B}_1 = \{ u_1=(1,1,1), u_2(0,1,0), u_3=(0,0,1)\}$
y
$\mathcal{B}_2 = \{ w_1=(1,2,3), w_2=(1,1,0), w_3=(3,-1,1)\}$

\ppart Si $\vec{z} = (2,-2,1)$ son las coordenadas en la base $\mathcal{B}_1$, halla las coordenadas de $\vec{z}$ en la base canónica.


\ppart Si $\vec{q} = (2,-2,1)$ son las coordenadas en la base $\mathcal{B}_2$, halla las coordenadas de $\vec{q}$ en la base canónica.


\ppart Halla las coordenadas de $\vec{z}$ en $\mathcal{B}_2$

\obs El vector $\vec{u_1}$ de la base $\mathcal{B}_1$ son las coordenadas respecto de una base concreta. 
%
Si no se dice nada, suponemos la canónica.

\solution

\hide{
\spart
\[\vec{z}_1 = 2\vec{u_1} -2\vec{u_2} + \vec{u_3} = 2·(\vec{i} +\vec{j} + \vec{k}) - 2·(\vec{j} + \vec{k}) = (2\vec{i}+3\vec{k})\]

Este $\vec{z}_c = (2,0,3)$ son las coordenadas de $\vec{u}$ en la base canónica.


\spart 

\[
\vec{q}_1 = 2\vec{w_1} -2\vec{w_2} + \vec{w_3} = 
2·(\vec{i} +2\vec{j} + 3\vec{k}) - 2·(\vec{i}+\vec{j}) + 3(\vec{i} -\vec{j} + \vec{k}) = 
3\vec{i} -\vec{j} + 9\vec{k}
\]

Este $\vec{q} = (2,0,3)$ son las coordenadas de $\vec{u}$ en la base canónica.

\spart
Buscamos 
$\vec{z} = (2\vec{u_1}-\vec{u_2}+\vec{u_3}) = c_1(1,2,3) + c_2 (1,1,0) + c_3 ( 3,-1,1)$.

\[
2·(1,1,1) - (0,1,0) + (0,0,1) = c_1(1,2,3) + c_2 (1,1,0) + c_3 ( 3,-1,1)\dimplies \left\{
  \begin{array}{c}
    2 = c_1 + c_2 + 3c_3\\
    1 = 2c_1 + c_2 -c_3\\
    3 = 3c_1  \quad\quad +c_3 
  \end{array}
  \right\}
\]

Resolvemos el sistema de 3 ecuaciones con 3 incógnitas, obteniendo: $(x,y,z) = \left(\rfrac{6}{13},\rfrac{11}{13},\rfrac{-3}{13}\right)$

Estas son las coordenadas de $\vec{z}$ en la base $\mathcal{B}_2$
}
\end{problem}


Problemas recomendados, especialmente 267.48,49 + 269.60,61: 
\begin{itemize}
  \item Página 267.48,49
  \item Página 269.56-65
  \item Página 271.102,103
\end{itemize}

