
\chapter{Análisis}

\begin{defn}[Función real de variable real]
Una función real de una variable real es una aplicación definida entre dos conjuntos de números reales tal que a cada elemento del primer conjunto le corresponde un único elemento del segundo conjunto.

$\appl{f}{D}{\real}$

\begin{itemize}
	\item $f$: símbolo de la función.
	\item $D(f) = \{x\in\real \tq \exists f(x)\}$
	\item $Rec(f) = \{y\in\real \tq \exists x \in\real f(x)=y\}$
\end{itemize}
\end{defn}

\paragraph{Ejemplos de dominios de funciones:} hojita impresa de repaso, incluyendo funciones trigonométricas y de las trigonométricas inversas.

\section{Límites y continuidad}

\begin{theorem}[Teorema\IS de existencia del límite]
\[\exists \lim_{x\to a}f(x) \dimplies \lim_{x\to a^+} f(x) = \lim_{x\to a^-}f(x)\]
\end{theorem}

\subsection{Indeterminaciones}

\begin{itemize}
	\item Racionales: $\rfrac{0}{0},\rfrac{\infty}{\infty},\infty-\infty,0·\infty, \rfrac{k}{0}$
	\item Exponenciales $1^{\infty}; 0^0; \infty^0$
\end{itemize}


\begin{prop}
\index{Indeterminación\IS $1^{\infty}$}
\[
\lim_{x\to a}f(x)^{g(x)} \to 1^{\infty} \implies \lim_{x\to a}f(x)^{g(x)} = e^\lambda, \lambda = \lim_{x\to a} \left(g(x)·[f(x)-1]\right)
\]
\end{prop}

\begin{example}

\end{example}

\paragraph{Infinitésimos equivalentes}

\subsection{Continuidad}

\begin{defn}[Continuidad\IS en un punto]
Sea $\appl{f}{D\subset\real}{\real}$.

Se dice que $f(x)$ es continua en $x=a$ sí y solo si se cumplen las 3 condiciones siguientes:
\begin{itemize}
	\item $\exists \lim_{x\to a} f(x)$
	\item $\exists f(a)$
	\item $\lim_{x\to a}f(x) = f(a)$
\end{itemize}

\textit{También se puede decir de manera abreviada: \[f(x) \text{ continua en } x=a\dimplies \text{ existen y son iguales } \lim_{x\to a}f(x) \text{ y } f(a)\]}
\end{defn}

\begin{example}
\[f(x) = 
	\begin{cases}
		2x-5 & \text{ si }x\leq 0\\ 
		\frac{x+2}{x+1} & \text{ si } x>0
	\end{cases}\]
\end{example}

\paragraph{Tipos de discontinuidad}

\begin{itemize}
	\item Evitables: $\exists \lim_{x\to a}f(x)$ pero $\begin{cases}\lim_{x\to a}f(x) \neq f(a)\\\nexists f(a)\end{cases}$
	\item Esenciales (o inevitables): 
	\begin{itemize}
		\item De primera especie:
		\subitem De salto finito: ambos límites laterales son finitos pero distinto
		\subitem De salto infinito: al menos un límite lateral es infinito.
	\end{itemize}
	\item De 2ª especie: al menos un límite lateral no existe.
\end{itemize}

\begin{figure}
\centering
\label{fig::fun-tipos-discontinuidad}
\caption{Tipos de distontinuidades}
\end{figure}

\begin{defn}[Continuidad\IS en un intervalo abierto]
$f(x)$ es continua en $(a,b) \dimplies \forall x\in(a,b), f(x)$ es continua.
\end{defn}

\begin{defn}[Continuidad\IS en un punto]
$f(x)$ es continua en $(a,b) \dimplies \begin{cases} f(x) \text{ continua en } (a,b)\\
\lim_{x\to a^+}f(x) = f(a)\\
\lim_{x\to a^-}f(x) = f(a)\\
 \end{cases}$
\end{defn}

\begin{example}

Dada $f(x) = +\sqrt{x}$.

$f(x)$ no es continua en $x=0$, pero sí es continua en $[0,\infty)$.

\end{example}

\begin{theorem}[Teorema\IS de Bolzano]
Sea $\appl{f}{D}{\real}$.

\[
\left.\begin{array}{c}f(x) \text{ continua en } [a,b]\\\text{Signo}(f(a))\neq \text{Signo}(f(b))\end{array}\right\}\implies \exists c\in(a,b) \tq f(c) = 0
\]

\obs Es una condición suficiente, no necesaria. Es decir, es $\implies $
\end{theorem}

\begin{problem} Demuestra que la ecuación $x^3-7x^2-1$ tiene al menos una solución real en el intervalo $[0,10]$.
\solution

Sea $f(x) = x^2-7x^2-1$. Se trata de demostrar que $\exists c\in[0,10]\tq f(c) = 0$. Comprobamos que cumple el teorema de Bolzano.

\[
\left\{
	\begin{array}{c}
		f(x) \text{ es continua en } [0,10] \text{ por ser polinómica}\\
		\left.\begin{array}{c}
		f(x) = -1 \\
		f(10) = 299\end{array}
		\right\}\implies Signo(f(0))\neq Signo(f(10))
	\end{array}
\right\}\implies \exists c\in(0,10)\tq f(c)=0 
\]
\end{problem}

\begin{theorem}[Teorema\IS del valor intermedio]
Sea $\appl{f}{D}{\real}$.
\[
\left.\begin{array}{c}f(x) \text{ continua en } [a,b]\\
\exists k\in\real \begin{cases}f(a)\leq k\leq f(b)\\f(b)\leq K \leq f(a)\end{cases}\end{array}\right\}\implies \exists c\in(a,b) \tq f(c) = k
\]
\end{theorem}

\obs Para $k=0$, el teorema del valor intermedio se convierte en teorema de Bolzano.

\begin{theorem}[Teorema\IS de Weierstrass]

\end{theorem}

\begin{problem}
Varios
\solution
\end{problem}


\section{Derivabilidad}

\begin{defn}[Derivabilidad\IS en un punto]

\end{defn}


\begin{defn}[Derivabilidad\IS en un intervalo abierto]

\end{defn}

\begin{defn}[Dominio de derivabilidad]

\end{defn}

\subsection{Interpretación geométrica de la derivada}

\paragraph{Ecuación de la recta tangente}

\paragraph{Ecuación de la recta normal}

\begin{defn}[Función derivada]

\end{defn}

\begin{theorem}[Teorema\IS de Rolle]

\end{theorem}


\begin{theorem}[Teorema\IS del valor medio]
\end{theorem}

\subsection{Regla de L'Hôpital}

\subsection{Monotonía}

\begin{theorem}[Teorema\IS de monotonía]
\end{theorem}
\begin{theorem}[Teorema\IS de extremos relativos]
\end{theorem}

\subsection{Optimización}

\section{Análisis sistemático de una función}

\section{Integrales}

\subsection{Cálculo de áreas de recintos cerrados}