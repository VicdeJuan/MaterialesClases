\chapter{Álgebra (Matrices y determinantes)}

\section{Matrices}

\paragraph{Definición de matriz}

\paragraph{Operaciones con matrices}

\subparagraph{Traspuesta}

Ejercicio: \textbf{Demuestra que cualquier matriz puede escribirse como suma de una matriz simétrica y otra antisimétrica}

\subparagraph{Producto de matrices}

\subsection{Matriz inversa}
Se puede calcular de 3 formas. Definición, Gauss-Jordan y matriz adjunta. Vamos a ver ahora los 2 primeros métodos.

\paragraph{Definición y propiedades}

\subsection{Algunas ecuaciones matriciales sencillas}

\paragraph{Gauss-Jordan}
La base del método de Gauss es que toda transformación lineal de Gauss se puede expresar como una matriz. Simplemente buscamos la matriz que transforma la matriz dada en la identidad. Para ello, ponemos la identidad a la derecha. (Espero que leyendo esta explicación te hayas enterado)\footnote{Puedes encontrar \href{https://math.stackexchange.com/questions/1240055/why-does-the-gaussian-jordan-elimination-works-when-finding-the-inverse-matrix}{aquí} una respuesta más elaborada.}

\subsection{Utilidades: Grafos}

\section{Determinantes}

\begin{defn}[Determinante]
$\appl{|\;\;|}{\mathcal{M}_n}{\real}$
\end{defn}

\subsection{Cálculo de determinantes de orden 3}

\subsection{Propiedades}

\subsection{Cálculo de determinantes de orden 4 o más}

\paragraph{Menores, Gauss}

\subsection{Matriz inversa por determinantes}

\subsection{Ecuaciones matriciales a tope}

\subsection{Rango}
\paragraph{Gauss}

\paragraph{Determinantes}

Si $|A| \neq 0$, significa que no hay 2 filas (ni 2 columnas) linealmente independientes. Si las hubiera, $|A| = 0$.

Por lo tanto, si $|A|\neq 0 \dimplies rg(A) = \text{ máx}$

¿Qué ocurre si $A\not\in\mathcal{M}_n$?

\begin{example}
\[
    A=\begin{pmatrix}2&2&3&4\\4&4&2&1\end{pmatrix}
\]

En este ejemplo, cogiendo $\left|\begin{matrix}2&2\\4&4\end{matrix}\right| = 0$, pero $\left|\begin{matrix}3&4\\2&1\end{matrix}\right| \neq 0$, por lo que estas 2 filas tienen que ser linealmente independientes, por lo que la matriz tiene rango 2.

También valdría argumentarlo desde $\left|\begin{matrix}2&3\\4&2\end{matrix}\right| \neq 0$
\end{example}

\begin{prop}[Cálculo del rango por menores]
Sea $M_p$ un menor de orden $p$ de la matriz $A\in\mathcal{M}_{n\times m}$
\[\exists M^p \tlq M_p \neq 0 \dimplies rg(A) \geq p\]
\[\forall M^p \;\; M_p = 0 \dimplies rg(A) < p\]
\end{prop}

\subsubsection{Matriz de Vandermonde}
\[
V=\begin{bmatrix}
1 & \alpha_1 & \alpha_1^2 & \dots & \alpha_1^{n-1}\\
1 & \alpha_2 & \alpha_2^2 & \dots & \alpha_2^{n-1}\\
1 & \alpha_3 & \alpha_3^2 & \dots & \alpha_3^{n-1}\\
\vdots & \vdots & \vdots & \ddots &\vdots \\
1 & \alpha_n & \alpha_n^2 & \dots & \alpha_n^{n-1}\\
\end{bmatrix}\]

\paragraph{Determinante: } El determinante se calcula con la siguiente fórmula:

\[\begin{vmatrix} V \end{vmatrix}=\prod_{1 \le i<j\le n}(\alpha_j-\alpha_i)\]

\begin{example}
\[
\begin{vmatrix}
1&2&4&8\\
1&3&9&27\\
1&4&16&64\\
1&5&25&125
\end{vmatrix} = \underbrace{\overbrace{(3-2)}^{j=2}\overbrace{(4-2)}^{j=3}\overbrace{(5-2)}^{j=4}}_{i=1}\underbrace{\overbrace{(4-3)}^{j=3}\overbrace{(5-3)}^{j=4}}_{i=2}\underbrace{\overbrace{(5-4)}^{j=4}}_{i=3} = 1·2·3·1·2·1 = 6
\]
\end{example}

\begin{proof}[por Inducción]

\paragraph{Base: n=2} Es fácil notar que en el caso de una matriz de 2×2 el resultado es correcto.
\[\begin{vmatrix} V \end{vmatrix}=v_{1,1}v_{2,2} - v_{1,2}v_{2,1}=\alpha_2-\alpha_1=\prod_{1\le i<j\le 2} (\alpha_j-\alpha_i)\]

\paragraph{Paso}
Suponiendo cierta la fórmula para el caso $n-1$, procedemos a calcular el determinante de orden $n$. Para ello, basta con realizar la siguiente operación elemental sobre cada columna: $C_{j}\rightarrow C_{j}-(\alpha_1 \times C_{j-1})$. Esta operación no afecta al determinante, por lo que se obtiene lo siguiente:

\[
\begin{vmatrix} V \end{vmatrix}=\begin{vmatrix}
1 & \alpha_1 & \alpha_1^2 & \dots & \alpha_1^{n-1}\\
1 & \alpha_2 & \alpha_2^2 & \dots & \alpha_2^{n-1}\\
1 & \alpha_3 & \alpha_3^2 & \dots & \alpha_3^{n-1}\\
\vdots & \vdots & \vdots & \ddots &\vdots \\
1 & \alpha_n & \alpha_n^2 & \dots & \alpha_n^{n-1}\\
\end{vmatrix}=\begin{vmatrix}
1 & 0 & 0 & \dots & 0\\
1 & \alpha_2-\alpha_1 & \alpha_2(\alpha_2-\alpha_1) & \dots & \alpha_2^{n-2}(\alpha_2-\alpha_1)\\
1 & \alpha_3-\alpha_1 & \alpha_3(\alpha_3-\alpha_1) & \dots & \alpha_3^{n-2}(\alpha_3-\alpha_1)\\
\vdots & \vdots & \vdots & \ddots &\vdots \\
1 & \alpha_n-\alpha_1 & \alpha_n(\alpha_n-\alpha_1) & \dots & \alpha_n^{n-2}(\alpha_n-\alpha_1)\\
\end{vmatrix}
\]

Desarrollando por los adjuntos de la primera fila: 

\[\begin{vmatrix} V \end{vmatrix}=\begin{vmatrix}
\alpha_2-\alpha_1 & \alpha_2(\alpha_2-\alpha_1) & \dots & \alpha_2^{n-2}(\alpha_2-\alpha_1)\\
\alpha_3-\alpha_1 & \alpha_3(\alpha_3-\alpha_1) & \dots & \alpha_3^{n-2}(\alpha_3-\alpha_1)\\
\vdots & \vdots & &\vdots \\
\alpha_n-\alpha_1 & \alpha_n(\alpha_n-\alpha_1) & \dots & \alpha_n^{n-2}(\alpha_n-\alpha_1)\\
\end{vmatrix}\]
Extrayendo de cada fila un factor, obtenemos:
\[\begin{vmatrix} V \end{vmatrix}=
(\alpha_2-\alpha_1)(\alpha_3-\alpha_1)\dots(\alpha_n-\alpha_1)
\underbrace{\begin{vmatrix}
1 & \alpha_2 & \alpha_2^2 & \dots & \alpha_2^{n-2}\\
1 & \alpha_3 & \alpha_3^2 & \dots & \alpha_3^{n-2}\\
1 & \alpha_4 & \alpha_4^2 & \dots & \alpha_4^{n-2}\\
\vdots & \vdots & \vdots & &\vdots \\
1 & \alpha_n & \alpha_n^2 & \dots & \alpha_n^{n-2}\\
\end{vmatrix}}_{(1)}\]

(1): es una matriz de Vandermonde de orden $n-1$, por lo que podemos aplicar la fórmula por la hipótesis de inducción, quedando así demostrada la fómrula del determinante de Vandermonde para orden $n$
\end{proof}

\section{Sistemas de ecuaciones}

Sistemas, expresión matricial de sistemas. 
 
Rouché-Frobenius, corregimos. 

Resolución de sistemas escalonados y método de Gauss Jordan.
 
"Repaso" de Sistema Compatible Indeterminado. 2 sistemas resueltos por mi. El primero con ecuaciones. El segundo con matricial.


\begin{problem}

Discute y resuelve el siguiente sistema:

\[
\left\{\begin{array}{lcccl}
x&+2y&-2z&=&4\\
2x&+5y&-2z&=&10\\
4x&+9y&-6z&=&18
\end{array}\right\}
\]

\solution

\[
\left\{\begin{array}{lcccl}
x&+2y&-2z&=&4\\
2x&+5y&-2z&=&10\\
4x&+9y&-6z&=&18
\end{array}\right\}
\overset{(1)}{\dimplies}
\left\{\begin{array}{lcccl}
x&+2y&-2z&=&4\\
 &y&+2z&=&2 \\
4x&+9y&-6z&=&18
\end{array}\right\}
\overset{(2)}{\dimplies}\]
\[
\left\{\begin{array}{lcccl}
x&+2y&-2z&=&4\\
 &y&+2z&=&2 \\
 &y&+2z&=&2 \\
\end{array}\right\}
\dimplies
\underbrace{\left\{\begin{array}{lcccl}
x&+2y&-2z&=&4\\
 &y&+2z&=&2 
\end{array}\right\}}_{\text{Discusión: C.I (*)}}
\]

(*): Es un sistema compatible indeterminado porque es un sistema escalonado con más incógnitas que ecuaciones.

Al ser compatible indeterminado, el sistema tiene infinitas soluciones (que no se calculan en 1º de Bachillerato).


\paragraph{Resolución:} Aunque un sistema de ecuaciones Compatible Indeterminado tiene infinitas soluciones, no cualquier trío de números es solución. 
%
Por ejemplo, en este caso, la terna $(x,y,z) = (0,0,0)$ no es solución.
%
\textbf{Infinitas soluciones no significa que todo sea solución}.

La pregunta lógica sería, ¿cómo podemos escribir \textbf{todas} las soluciones del sistema? Utilizando un parámetro.
%
Al dar un valor a una incógnita, ya forzamos los otros 2 valores. 
%
Para cada valor inventado de $x$, solo hay un único valor posible de $y$ y de $z$ (normalmente).

En este caso, vamos a dar un valor concreto a $y$, pero en forma de parámetro.
%
Tomamos $y=λ$ y sustituimos en $E_2$.

\[y+2z=2 \dimplies λ+2z=2 \dimplies z=\frac{2-λ}{2}\]

Sustituimos $y=λ,z=\frac{2-λ}{2}$ en $E_1$:

\[x+2y-2z = 4 \dimplies x= 4+2z-2y = 4+2\left(\frac{2-λ}{2}\right)-2λ = 4+2-λ-2λ = 6-3λ = 3(2-λ)\]

\textbf{Solución:} $(x,y,z) = \left(3(2-λ),λ,\frac{2-λ}{2}\right)$

\paragraph{1)} $E_2=E_2-2E_1$

\[
\left\{\begin{array}{lcccl}
2x&+4y&-4z&=&8\\
2x&+5y&-2z&=&10\\
\hline
&-y&-2z&=&-2 
\end{array}\right\}
\]

\paragraph{2)} $E_3=E_2-4E_1$

\[
\left\{\begin{array}{lcccl}
4x&+9y&-6z&=&18\\
4x&+10y&-4z&=&20\\
\hline
&-y&-2z&=&-2 
\end{array}\right\}
\]


\paragraph*{Comprobación:} Sustituimos $(x,y,z) = \left(3(2-λ),λ,\frac{2-λ}{2}\right)$ en el sistema inicial:


\[
\left\{\begin{array}{lcccll}
x&+2y&-2z&=&4 &\to 6-3λ + 2λ - 2\displaystyle\left(\frac{2-λ}{2}\right) = 6-λ-2+λ = 4\\
2x&+5y&-2z&=&10 &\to 12-6λ +5λ - 2\displaystyle\left(\frac{2-λ}{2}\right) = 12-λ-2+λ = 10\\
4x&+9y&-6z&=&18 &\to 24-12λ + 9λ - 6\displaystyle\left(\frac{2-λ}{2}\right) = 24-3λ-6+3λ = 18
\end{array}\right\}\begin{array}{c}\\\\\\\\\text{cqc}\end{array}
\]

\end{problem}

\begin{problem}

Discute y resuelve el siguiente sistema:

\[
\left\{\begin{array}{rcccl}
3x&-y&+z&=&3\\
6x&-2y&+2z&=&6\\
-3x&+y&-z&=&-3
\end{array}\right\}
\]

\solution


\[
\left\{\begin{array}{rcccl}
3x&-y&+z&=&3\\
6x&-2y&+2z&=&6\\
-3x&+y&-z&=&-3
\end{array}\right\} \implies
\left(\begin{array}{ccc|c}
3&-1&1&3\\
6&-2&2&6\\
-3&1&-1&-3
\end{array}\right)
\dimplies\]
\[
\text{\hl{Ojo con el cambio de columnas}}
\left(\begin{array}{ccc|c}
1&-1&3&3\\
2&-2&6&6\\
-1&1&-3&-3
\end{array}\right)
\dimplies
\left(\begin{array}{ccc|c}
1&-1&3&3\\
0&0&0&0\\
0&0&0&0
\end{array}\right)
\]

Tiene grado de indeterminación 2, por lo que necesitaremos 2 parámetros.

Llamamos $x=\lambda$ e $y = \mu$ con $\mu,\lambda\in\real$ y sustituimos para hallar $z$.

$$z-y+3x=3 \implies z - \mu + 3\lambda = 3 \dimplies z = 3+\mu-3\lambda$$

Solución: $(x,y,z) = \left(\lambda, \mu, 3+\mu - 3\lambda\right), \forall\lambda,\mu\in\real$

\end{problem}

Ejercicio 59 de deberes.

\textbf{Resolución por inversa de stma}. ¿Funciona siempre? Sólo en sistemas de Cramer, es decir, matriz de coeficientes cuadrada con rango máximo. 

Deberes el 15b,16b.

\subsection{Regla de Cramer}

En todos los sistemas cuya matriz de coeficientes tenga inversa, puede generalizarse el método de la inversa.

Así, $Ax = B \dimplies x = A^{-1}·B$

\[
    A^{-1} = \frac{1}{|A|} · \left( \text{Adj}(A) \right)^T = \frac{1}{|A|} · \begin{pmatrix} 
    A_{11} & A_{21} & A_{31} & ... & A_{n1}\\
    A_{21} & A_{22} & A_{32} & ... & A_{n2}\\
    \vdots &        &       & \ddots & \vdots\\
    A_{1n} & A_{2n} & A_{3n} & ... & A_{nn}\end{pmatrix}
\]

Por lo tanto,
\[
    A^{-1}·B = \frac{1}{|A|} · 
    \begin{pmatrix} 
        A_{11} & A_{21} & A_{31} & ... & A_{n1}\\
        A_{12} & A_{22} & A_{32} & ... & A_{n2}\\
        \vdots &        &       & \ddots & \vdots\\
        A_{1n} & A_{2n} & A_{3n} & ... & A_{nn}\end{pmatrix}·
    \begin{pmatrix}
        b_1\\b_2\\\vdots\\ b_n
    \end{pmatrix}
    = 
    \begin{pmatrix}
    A_{11}b_1 + A_{21}b_2 + A_{31}b_3 \dots A_{n1}·b_n\\
    A_{12}b_1 + A_{22}b_2 + A_{32}b_3 \dots A_{n2}·b_n\\
    \vdots\\
    A_{1n}b_1 + A_{2n}b_2 + A_{3n}b_3 \dots A_{nn}·b_n\\
    \end{pmatrix}
\]

Deberes : 21b, 22b

Corregimos Cramer. 

Inconvenientes: ¿y si es incompatible?

Numéricos: 61a,b;62a,b

Parámetros:64a,c (ojo con eliminar una solución)



Deberes para el punete: 
57,60
