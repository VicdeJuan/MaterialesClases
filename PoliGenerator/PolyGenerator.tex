\documentclass[palatino,nochap]{Docencia}


\usepackage{sagetex}
\usepackage{breqn}
\geometry{reset,margin=0.2in,rmargin=0.5in}


\title{Polinomios para factorizar.}
%\newcommand{\subtitle}[1]{\def\thesubtitle{#1}\@subsettrue}
%\subtitle{4º ESO Matemáticas Académicas}
\author{Departamento de Matemáticas}
\date{18/19}



\begin{abstract}

Ejercicios de polinomios generados automáticamente.

$x_2 = 3 [2]$, significa que la segunda raíz obtenida es $3$ y tiene multiplicidad $2$, es decir, el polinomio tiene 2 factores $(x-2)$.
%
Si entre corchetes no aparece ningún número, significa que su multiplicidad es 1.


\textit{Nota: Este documento se ha generado automáticamente utilizando \href{www.sagemath.org/es/}{Sage}, \LaTeX\xspace y \href{https://github.com/sagemath/sagetex}{sagetex} para la integración de las 2 herramientas mencionadas.}

\end{abstract}

% Paquetes adicionales

\usepackage[author={Víctor de Juan, 2018}]{pdfcomment}

\makeatletter
\newcommand{\annotate}[2][]{%
	\pdfstringdef\x@title{#1}%
	\edef\r{\string\r}%
	\pdfstringdef\x@contents{#2}%
	\pdfannot
	width 2\baselineskip
	height 2\baselineskip
	depth 0pt
	{
		/Subtype /Text
		/T (\x@title)
		/Contents (\x@contents)
	}%
}
\makeatother



% --------------------
\newcommand{\cimplies}{\text{\hl{$\implies$}}}
\renewcommand{\vx}{\overset{\rightarrow}{x}}
\renewcommand{\vy}{\overset{\rightarrow}{y}}
\renewcommand{\vz}{\overset{\rightarrow}{z}}
\newcommand{\vi}{\overset{\rightarrow}{i}}
\newcommand{\vj}{\overset{\rightarrow}{j}}
\renewcommand{\vec}[1]{\overset{\rightarrow}{#1}}




\begin{document}
\pagestyle{plain}
%\maketitle
%\tableofcontents
%\newpage

\begin{multicols}{3}
\section{Sin polinomios irreducibles de grado 2}
\subsection{Hasta 0 raices fraccionarias}
\textbf{Polinomios de grado 2} 
\subitem $P_{1}(x) = -x^2 - 4x - 4$

\subitem $P_{2}(x) = -2x^2 + 2$

\subitem $P_{3}(x) = 2x^2 - 8x + 6$

\subitem $P_{4}(x) = 2x^2 - 2x - 4$

\subitem $P_{5}(x) = x^2 - 5x + 6$

\subitem $P_{6}(x) = x^2 - 5x + 6$

\subitem $P_{7}(x) = -x^2 + 2x - 1$

\subitem $P_{8}(x) = 2x^2 - 10x + 12$

\subitem $P_{9}(x) = -x^2 + 4x - 3$

\subitem $P_{10}(x) = -x^2 - 3x - 2$

\subitem $P_{11}(x) = x^2 - 4$

\subitem $P_{12}(x) = x^2 - 6x + 9$

\subitem $P_{13}(x) = x^2 - 2x - 3$

\subitem $P_{14}(x) = -2x^2 + 8x - 8$

\subitem $P_{15}(x) = 2x^2 + 2x - 4$

\subitem $P_{16}(x) = x^2 - x - 2$

\subitem $P_{17}(x) = 2x^2 - 2$

\subitem $P_{18}(x) = 2x^2 - 2$

\subitem $P_{19}(x) = x^2 - x - 6$

\subitem $P_{20}(x) = x^2 - 4x + 3$

\subitem $P_{21}(x) = x^2 + 3x + 2$

\subitem $P_{22}(x) = 2x^2 + 8x + 8$

\subitem $P_{23}(x) = -2x^2 + 4x - 2$

\subitem $P_{24}(x) = -2x^2 + 2$

\subitem $P_{25}(x) = -2x^2 + 2x + 12$

\textbf{Polinomios de grado 3} 
\subitem $P_{26}(x) = x^3 - 3x^2 - 4x + 12$

\subitem $P_{27}(x) = 2x^3 - 4x^2 - 8x + 16$

\subitem $P_{28}(x) = -2x^3 - 2x^2 + 16x + 24$

\subitem $P_{29}(x) = 2x^3 - 14x - 12$

\subitem $P_{30}(x) = x^3 - 6x^2 + 11x - 6$

\subitem $P_{31}(x) = x^3 - x^2 - 5x - 3$

\subitem $P_{32}(x) = x^3 - 3x + 2$

\subitem $P_{33}(x) = x^3 + x^2 - 8x - 12$

\subitem $P_{34}(x) = -2x^3 - 4x^2 + 8x + 16$

\subitem $P_{35}(x) = x^3 - 2x^2 - 4x + 8$

\subitem $P_{36}(x) = 2x^3 - 8x^2 - 6x + 36$

\subitem $P_{37}(x) = x^3 - 7x^2 + 15x - 9$

\subitem $P_{38}(x) = 2x^3 + 8x^2 + 10x + 4$

\subitem $P_{39}(x) = x^3 + 5x^2 + 8x + 4$

\subitem $P_{40}(x) = -2x^3 + 4x^2 + 2x - 4$

\subitem $P_{41}(x) = x^3 + 2x^2 - x - 2$

\subitem $P_{42}(x) = 2x^3 + 4x^2 - 8x - 16$

\subitem $P_{43}(x) = x^3 - x^2 - 4x + 4$

\subitem $P_{44}(x) = x^3 + x^2 - 4x - 4$

\subitem $P_{45}(x) = -x^3 + 6x^2 - 12x + 8$

\subitem $P_{46}(x) = x^3 - x^2 - 4x + 4$

\subitem $P_{47}(x) = -x^3 + 7x + 6$

\subitem $P_{48}(x) = x^3 + x^2 - x - 1$

\subitem $P_{49}(x) = -x^3 - 2x^2 + x + 2$

\subitem $P_{50}(x) = -x^3 + x^2 + 5x + 3$

\textbf{Polinomios de grado 4} 
\subitem $P_{51}(x) = -2x^4 - 4x^3 + 6x^2 + 16x + 8$

\subitem $P_{52}(x) = -2x^4 + 6x^3 + 4x^2 - 24x + 16$

\subitem $P_{53}(x) = 2x^4 - 12x^3 + 10x^2 + 48x - 72$

\subitem $P_{54}(x) = -2x^4 - 12x^3 - 26x^2 - 24x - 8$

\subitem $P_{55}(x) = x^4 - 5x^3 + 5x^2 + 5x - 6$

\subitem $P_{56}(x) = x^4 - 3x^3 - 2x^2 + 12x - 8$

\subitem $P_{57}(x) = x^4 - 8x^2 + 16$

\subitem $P_{58}(x) = x^4 + 5x^3 + 9x^2 + 7x + 2$

\subitem $P_{59}(x) = x^4 - 9x^2 - 4x + 12$

\subitem $P_{60}(x) = x^4 + 2x^3 - 3x^2 - 4x + 4$

\subitem $P_{61}(x) = x^4 + x^3 - 7x^2 - 13x - 6$

\subitem $P_{62}(x) = x^4 - 4x^3 - 2x^2 + 12x + 9$

\subitem $P_{63}(x) = -2x^4 + 6x^3 - 2x^2 - 6x + 4$

\subitem $P_{64}(x) = -2x^4 + 12x^3 - 18x^2 - 8x + 24$

\subitem $P_{65}(x) = -x^4 + 4x^3 + x^2 - 16x + 12$

\subitem $P_{66}(x) = -x^4 + 6x^3 - 5x^2 - 24x + 36$

\subitem $P_{67}(x) = x^4 + x^3 - 6x^2 - 4x + 8$

\subitem $P_{68}(x) = x^4 - 3x^3 + x^2 + 3x - 2$

\subitem $P_{69}(x) = -2x^4 + 16x^3 - 46x^2 + 56x - 24$

\subitem $P_{70}(x) = x^4 - 2x^3 - 4x^2 + 2x + 3$

\subitem $P_{71}(x) = 2x^4 + 2x^3 - 12x^2 - 8x + 16$

\subitem $P_{72}(x) = x^4 - 4x^3 + 16x - 16$

\subitem $P_{73}(x) = -2x^4 + 4x^3 + 14x^2 - 16x - 24$

\subitem $P_{74}(x) = -2x^4 + 20x^3 - 74x^2 + 120x - 72$

\subitem $P_{75}(x) = -x^4 - 3x^3 + 6x^2 + 28x + 24$

\textbf{Polinomios de grado 5} 
\subitem $P_{76}(x) = 2x^5 - 4x^4 - 18x^3 + 28x^2 + 40x - 48$

\subitem $P_{77}(x) = -2x^5 + 6x^4 + 4x^3 - 12x^2 - 2x + 6$

\subitem $P_{78}(x) = x^5 - 3x^4 - 6x^3 + 10x^2 + 21x + 9$

\subitem $P_{79}(x) = x^5 + x^4 - 5x^3 - 5x^2 + 4x + 4$

\subitem $P_{80}(x) = 2x^5 - 8x^4 - 8x^3 + 44x^2 + 6x - 36$

\subitem $P_{81}(x) = 2x^5 - 10x^4 - 10x^3 + 90x^2 - 216$

\subitem $P_{82}(x) = x^5 - x^4 - 9x^3 + x^2 + 20x + 12$

\subitem $P_{83}(x) = x^5 + x^4 - 5x^3 - 5x^2 + 4x + 4$

\subitem $P_{84}(x) = x^5 - 4x^3 + 2x^2 + 3x - 2$

\subitem $P_{85}(x) = x^5 - 7x^3 - 2x^2 + 12x + 8$

\subitem $P_{86}(x) = -2x^5 + 12x^4 - 12x^3 - 32x^2 + 30x + 36$

\subitem $P_{87}(x) = 2x^5 - 18x^4 + 46x^3 + 18x^2 - 216x + 216$

\subitem $P_{88}(x) = x^5 - 6x^4 + 10x^3 - 11x + 6$

\subitem $P_{89}(x) = x^5 - x^4 - 8x^3 + 8x^2 + 16x - 16$

\subitem $P_{90}(x) = -x^5 + 5x^4 - 3x^3 - 13x^2 + 8x + 12$

\subitem $P_{91}(x) = x^5 - x^4 - 5x^3 + 5x^2 + 4x - 4$

\subitem $P_{92}(x) = -x^5 - 4x^4 - x^3 + 14x^2 + 20x + 8$

\subitem $P_{93}(x) = -2x^5 - 6x^4 + 10x^3 + 54x^2 + 64x + 24$

\subitem $P_{94}(x) = x^5 - 9x^4 + 26x^3 - 18x^2 - 27x + 27$

\subitem $P_{95}(x) = 2x^5 - 12x^4 + 20x^3 - 22x + 12$

\subitem $P_{96}(x) = x^5 - 2x^4 - 6x^3 + 8x^2 + 5x - 6$

\subitem $P_{97}(x) = -2x^5 + 14x^4 - 28x^3 + 4x^2 + 30x - 18$

\subitem $P_{98}(x) = x^5 - 11x^3 - 6x^2 + 28x + 24$

\subitem $P_{99}(x) = 2x^5 - 6x^4 + 4x^3 + 4x^2 - 6x + 2$

\subitem $P_{100}(x) = 2x^5 - 16x^3 - 12x^2 + 14x + 12$

\textbf{Polinomios de grado 6} 
\subitem $P_{101}(x) = x^6 - x^5 - 7x^4 + 9x^3 + 10x^2 - 20x + 8$

\subitem $P_{102}(x) = x^6 - 3x^5 - 11x^4 + 27x^3 + 46x^2 - 60x - 72$

\subitem $P_{103}(x) = -2x^6 + 20x^5 - 64x^4 + 28x^3 + 234x^2 - 432x + 216$

\subitem $P_{104}(x) = x^6 - 2x^5 - 11x^4 + 16x^3 + 40x^2 - 32x - 48$

\subitem $P_{105}(x) = x^6 - 6x^5 + 4x^4 + 30x^3 - 41x^2 - 24x + 36$

\subitem $P_{106}(x) = x^6 - 9x^5 + 26x^4 - 8x^3 - 96x^2 + 176x - 96$

\subitem $P_{107}(x) = -x^6 + 12x^5 - 58x^4 + 144x^3 - 193x^2 + 132x - 36$

\subitem $P_{108}(x) = x^6 - 2x^5 - 7x^4 + 12x^3 + 16x^2 - 16x - 16$

\subitem $P_{109}(x) = x^6 + 3x^5 - 3x^4 - 15x^3 - 6x^2 + 12x + 8$

\subitem $P_{110}(x) = x^6 + 3x^5 - 10x^3 - 15x^2 - 9x - 2$

\subitem $P_{111}(x) = -2x^6 + 12x^4 - 18x^2 + 8$

\subitem $P_{112}(x) = 2x^6 - 12x^5 + 2x^4 + 96x^3 - 112x^2 - 192x + 288$

\subitem $P_{113}(x) = 2x^6 - 12x^5 + 8x^4 + 60x^3 - 82x^2 - 48x + 72$

\subitem $P_{114}(x) = -x^6 + 3x^5 + 4x^4 - 10x^3 - 9x^2 + 7x + 6$

\subitem $P_{115}(x) = x^6 - 10x^4 - 4x^3 + 21x^2 + 4x - 12$

\subitem $P_{116}(x) = x^6 - 9x^5 + 28x^4 - 30x^3 - 11x^2 + 39x - 18$

\subitem $P_{117}(x) = x^6 - 6x^5 + 5x^4 + 28x^3 - 48x^2 - 16x + 48$

\subitem $P_{118}(x) = x^6 - 5x^5 + 26x^3 - 19x^2 - 21x + 18$

\subitem $P_{119}(x) = x^6 - 2x^5 - 12x^4 + 14x^3 + 47x^2 - 12x - 36$

\subitem $P_{120}(x) = -2x^6 + 16x^5 - 42x^4 + 32x^3 + 26x^2 - 48x + 18$

\subitem $P_{121}(x) = x^6 - 2x^5 - 8x^4 + 10x^3 + 19x^2 - 8x - 12$

\subitem $P_{122}(x) = x^6 - 2x^5 - 8x^4 + 14x^3 + 11x^2 - 28x + 12$

\subitem $P_{123}(x) = x^6 - 9x^4 + 24x^2 - 16$

\subitem $P_{124}(x) = x^6 - 8x^5 + 18x^4 + 4x^3 - 47x^2 + 12x + 36$

\subitem $P_{125}(x) = 2x^6 + 6x^5 - 6x^4 - 30x^3 - 12x^2 + 24x + 16$

\subsection{Hasta 1 raices fraccionarias}
\textbf{Polinomios de grado 2} 
\subitem $P_{126}(x) = 2x^2 - 3x - 2$

\subitem $P_{127}(x) = 2x^2 - 5x - 3$

\subitem $P_{128}(x) = 2x^2 + x - 1$

\subitem $P_{129}(x) = 2x^2 - 2x - 4$

\subitem $P_{130}(x) = 3x^2 - 8x + 4$

\subitem $P_{131}(x) = 2x^2 - 3x + 1$

\subitem $P_{132}(x) = 3x^2 + x - 2$

\subitem $P_{133}(x) = 2x^2 + 3x - 2$

\subitem $P_{134}(x) = x^2 - 4x + 4$

\subitem $P_{135}(x) = 8x^2 - 11x + 3$

\subitem $P_{136}(x) = 2x^2 + x - 6$

\subitem $P_{137}(x) = 2x^2 - 5x + 2$

\subitem $P_{138}(x) = 3x^2 - 5x - 2$

\subitem $P_{139}(x) = 2x^2 + 3x - 2$

\subitem $P_{140}(x) = 2x^2 + 6x + 4$

\subitem $P_{141}(x) = 3x^2 - 8x - 3$

\subitem $P_{142}(x) = 3x^2 - 2x - 1$

\subitem $P_{143}(x) = 7x^2 + 4x - 3$

\subitem $P_{144}(x) = x^2 - 5x + 6$

\subitem $P_{145}(x) = 2x^2 + 3x - 2$

\subitem $P_{146}(x) = 3x^2 - 4x - 4$

\subitem $P_{147}(x) = 2x^2 - 5x + 2$

\subitem $P_{148}(x) = 3x^2 + 7x + 2$

\subitem $P_{149}(x) = 2x^2 - 3x + 1$

\subitem $P_{150}(x) = 2x^2 - 7x + 3$

\textbf{Polinomios de grado 3} 
\subitem $P_{151}(x) = 3x^3 - 11x^2 + 12x - 4$

\subitem $P_{152}(x) = 2x^3 - x^2 - 8x + 4$

\subitem $P_{153}(x) = 2x^3 - 3x^2 - 11x + 6$

\subitem $P_{154}(x) = 2x^3 + 5x^2 + x - 2$

\subitem $P_{155}(x) = -2x^3 - 4x^2 + 8x + 16$

\subitem $P_{156}(x) = -x^3 - x^2 + 8x + 12$

\subitem $P_{157}(x) = 3x^3 + 5x^2 - 4x - 4$

\subitem $P_{158}(x) = 3x^3 + 8x^2 + 3x - 2$

\subitem $P_{159}(x) = x^3 - 3x^2 - x + 3$

\subitem $P_{160}(x) = 3x^3 - 4x^2 - 5x + 2$

\subitem $P_{161}(x) = 3x^3 - 14x^2 + 17x - 6$

\subitem $P_{162}(x) = 3x^3 - 4x^2 - 17x + 6$

\subitem $P_{163}(x) = 3x^3 - 7x^2 + 4$

\subitem $P_{164}(x) = 3x^3 - 2x^2 - 7x - 2$

\subitem $P_{165}(x) = 3x^3 - 7x^2 + 5x - 1$

\subitem $P_{166}(x) = 3x^3 + 8x^2 + 3x - 2$

\subitem $P_{167}(x) = 3x^3 - 8x^2 - 5x + 6$

\subitem $P_{168}(x) = 3x^3 + 2x^2 - 7x + 2$

\subitem $P_{169}(x) = 3x^3 - x^2 - 8x - 4$

\subitem $P_{170}(x) = 3x^3 - x^2 - 12x + 4$

\subitem $P_{171}(x) = 3x^3 + 10x^2 + 9x + 2$

\subitem $P_{172}(x) = 4x^3 - 3x^2 - 16x + 12$

\subitem $P_{173}(x) = 3x^3 - 16x^2 + 15x + 18$

\subitem $P_{174}(x) = 3x^3 + 11x^2 + 12x + 4$

\subitem $P_{175}(x) = 3x^3 - 2x^2 - 19x - 6$

\textbf{Polinomios de grado 4} 
\subitem $P_{176}(x) = 3x^4 - 8x^3 - 15x^2 + 32x + 12$

\subitem $P_{177}(x) = 2x^4 + x^3 - 14x^2 - 19x - 6$

\subitem $P_{178}(x) = 3x^4 - 7x^3 - 18x^2 + 28x + 24$

\subitem $P_{179}(x) = 8x^4 + 37x^3 + 49x^2 + 8x - 12$

\subitem $P_{180}(x) = 3x^4 + 8x^3 + x^2 - 8x - 4$

\subitem $P_{181}(x) = 3x^4 + 5x^3 - 5x^2 - 5x + 2$

\subitem $P_{182}(x) = 2x^4 - 2x^3 - 20x^2 + 8x + 48$

\subitem $P_{183}(x) = 3x^4 - 13x^3 + 7x^2 + 17x - 6$

\subitem $P_{184}(x) = 3x^4 + 4x^3 - 7x^2 - 4x + 4$

\subitem $P_{185}(x) = 5x^4 + 17x^3 + 13x^2 - 5x - 6$

\subitem $P_{186}(x) = 3x^4 - 7x^3 - x^2 + 7x - 2$

\subitem $P_{187}(x) = 2x^4 - 11x^3 + 19x^2 - 13x + 3$

\subitem $P_{188}(x) = 3x^4 - 10x^3 - 9x^2 + 40x - 12$

\subitem $P_{189}(x) = 3x^4 + 2x^3 - 4x^2 - 2x + 1$

\subitem $P_{190}(x) = -x^4 + 5x^2 - 4$

\subitem $P_{191}(x) = 2x^4 + 3x^3 - 7x^2 - 12x - 4$

\subitem $P_{192}(x) = 3x^4 - 8x^3 - 6x^2 + 8x + 3$

\subitem $P_{193}(x) = 3x^4 + 2x^3 - 25x^2 - 28x + 12$

\subitem $P_{194}(x) = 2x^4 - 8x^3 - 2x^2 + 32x - 24$

\subitem $P_{195}(x) = 3x^4 - 2x^3 - 9x^2 + 4$

\subitem $P_{196}(x) = x^4 - 8x^3 + 23x^2 - 28x + 12$

\subitem $P_{197}(x) = 3x^4 - 16x^3 + 29x^2 - 20x + 4$

\subitem $P_{198}(x) = 2x^4 - 3x^3 - 4x^2 + 3x + 2$

\subitem $P_{199}(x) = 3x^4 - 4x^3 - 19x^2 + 8x + 12$

\subitem $P_{200}(x) = x^4 - 9x^3 + 29x^2 - 39x + 18$

\textbf{Polinomios de grado 5} 
\subitem $P_{201}(x) = 2x^5 - 13x^4 + 16x^3 + 43x^2 - 96x + 36$

\subitem $P_{202}(x) = 3x^5 - 11x^4 + 9x^3 + 7x^2 - 12x + 4$

\subitem $P_{203}(x) = 3x^5 - x^4 - 15x^3 + 5x^2 + 12x - 4$

\subitem $P_{204}(x) = 2x^5 - x^4 - 4x^3 + 2x^2 + 2x - 1$

\subitem $P_{205}(x) = 2x^5 - 3x^4 - 5x^3 + 5x^2 + 3x - 2$

\subitem $P_{206}(x) = 2x^5 - 16x^4 + 40x^3 - 20x^2 - 42x + 36$

\subitem $P_{207}(x) = 3x^5 - 13x^4 + 13x^3 + 9x^2 - 16x + 4$

\subitem $P_{208}(x) = 3x^5 - 29x^4 + 98x^3 - 126x^2 + 27x + 27$

\subitem $P_{209}(x) = 2x^5 - 7x^4 - 3x^3 + 25x^2 - 23x + 6$

\subitem $P_{210}(x) = 7x^5 - 31x^4 + 26x^3 + 22x^2 - 33x + 9$

\subitem $P_{211}(x) = 3x^5 + 7x^4 - 7x^3 - 27x^2 - 20x - 4$

\subitem $P_{212}(x) = 2x^5 - 9x^4 - x^3 + 42x^2 - 28x - 24$

\subitem $P_{213}(x) = -x^5 + 5x^4 - 4x^3 - 16x^2 + 32x - 16$

\subitem $P_{214}(x) = 2x^5 - 9x^4 + 8x^3 + 6x^2 - 10x + 3$

\subitem $P_{215}(x) = 3x^5 - 5x^4 - 11x^3 + 21x^2 - 4x - 4$

\subitem $P_{216}(x) = 7x^5 - 24x^4 - 12x^3 + 86x^2 - 75x + 18$

\subitem $P_{217}(x) = 3x^5 - 13x^4 - 2x^3 + 38x^2 + 15x - 9$

\subitem $P_{218}(x) = 4x^5 - 7x^4 - 37x^3 + 46x^2 + 84x - 72$

\subitem $P_{219}(x) = 3x^5 - 2x^4 - 18x^3 - 12x^2 + 7x + 6$

\subitem $P_{220}(x) = 3x^5 - 7x^4 - 3x^3 + 11x^2 - 4$

\subitem $P_{221}(x) = 2x^5 + 5x^4 - 4x^3 - 19x^2 - 16x - 4$

\subitem $P_{222}(x) = 2x^5 - 11x^4 + 18x^3 - x^2 - 20x + 12$

\subitem $P_{223}(x) = -2x^5 - 4x^4 + 16x^3 + 32x^2 - 32x - 64$

\subitem $P_{224}(x) = -2x^5 + 6x^4 + 10x^3 - 30x^2 - 8x + 24$

\subitem $P_{225}(x) = 7x^5 - 17x^4 - 22x^3 + 26x^2 + 15x - 9$

\textbf{Polinomios de grado 6} 
\subitem $P_{226}(x) = -2x^6 + 8x^5 - 4x^4 - 24x^3 + 46x^2 - 32x + 8$

\subitem $P_{227}(x) = x^6 + 5x^5 + 5x^4 - 13x^3 - 34x^2 - 28x - 8$

\subitem $P_{228}(x) = 5x^6 + 12x^5 - 14x^4 - 32x^3 + 21x^2 + 20x - 12$

\subitem $P_{229}(x) = 2x^6 + 5x^5 - 8x^4 - 25x^3 - 2x^2 + 20x + 8$

\subitem $P_{230}(x) = 3x^6 + 5x^5 - 20x^4 - 54x^3 - 37x^2 + x + 6$

\subitem $P_{231}(x) = 3x^6 - 16x^5 + 9x^4 + 68x^3 - 96x^2 - 16x + 48$

\subitem $P_{232}(x) = 2x^6 - 5x^5 - 13x^4 + 25x^3 + 23x^2 - 20x - 12$

\subitem $P_{233}(x) = 4x^6 - 19x^5 + 97x^3 - 82x^2 - 84x + 72$

\subitem $P_{234}(x) = 3x^6 - 17x^5 + 19x^4 + 45x^3 - 118x^2 + 92x - 24$

\subitem $P_{235}(x) = x^6 - 4x^5 - 6x^4 + 28x^3 + 17x^2 - 48x - 36$

\subitem $P_{236}(x) = 3x^6 - 32x^5 + 130x^4 - 244x^3 + 191x^2 - 12x - 36$

\subitem $P_{237}(x) = 3x^6 - 20x^5 + 38x^4 - 53x^2 + 20x + 12$

\subitem $P_{238}(x) = 2x^6 - 3x^5 - 25x^4 + 15x^3 + 95x^2 + 24x - 36$

\subitem $P_{239}(x) = 3x^6 - x^5 - 17x^4 + 5x^3 + 22x^2 - 4x - 8$

\subitem $P_{240}(x) = -x^6 + 4x^5 + 2x^4 - 20x^3 + 11x^2 + 16x - 12$

\subitem $P_{241}(x) = 2x^6 + 9x^5 + 4x^4 - 29x^3 - 30x^2 + 20x + 24$

\subitem $P_{242}(x) = 3x^6 - 4x^5 - 10x^4 + 8x^3 + 11x^2 - 4x - 4$

\subitem $P_{243}(x) = 3x^6 - 8x^5 - 18x^4 + 40x^3 + 27x^2 - 32x - 12$

\subitem $P_{244}(x) = x^6 - 2x^5 - 7x^4 + 12x^3 + 16x^2 - 16x - 16$

\subitem $P_{245}(x) = x^6 - 2x^5 - 4x^4 + 10x^3 - x^2 - 8x + 4$

\subitem $P_{246}(x) = -x^6 + 7x^5 - 9x^4 - 31x^3 + 70x^2 + 12x - 72$

\subitem $P_{247}(x) = -x^6 + 4x^5 + 5x^4 - 32x^3 + 8x^2 + 64x - 48$

\subitem $P_{248}(x) = 3x^6 - 13x^5 - 8x^4 + 70x^3 - 13x^2 - 57x + 18$

\subitem $P_{249}(x) = 2x^6 + x^5 - 11x^4 - 5x^3 + 13x^2 + 4x - 4$

\subitem $P_{250}(x) = 3x^6 - 5x^5 - 25x^4 + 33x^3 + 38x^2 - 68x + 24$

\newpage\section{Soluciones}
\subitem \begin{dmath*}P_{1}(x) = -{\left(x + 2\right)}^{2} \end{dmath*}\vspace{-1.2cm}
\subitem \begin{dmath*}P_{2}(x) = -2 \, {\left(x + 1\right)} {\left(x - 1\right)} \end{dmath*}\vspace{-1.2cm}
\subitem \begin{dmath*}P_{3}(x) = 2 \, {\left(x - 1\right)} {\left(x - 3\right)} \end{dmath*}\vspace{-1.2cm}
\subitem \begin{dmath*}P_{4}(x) = 2 \, {\left(x + 1\right)} {\left(x - 2\right)} \end{dmath*}\vspace{-1.2cm}
\subitem \begin{dmath*}P_{5}(x) = {\left(x - 2\right)} {\left(x - 3\right)} \end{dmath*}\vspace{-1.2cm}
\subitem \begin{dmath*}P_{6}(x) = {\left(x - 2\right)} {\left(x - 3\right)} \end{dmath*}\vspace{-1.2cm}
\subitem \begin{dmath*}P_{7}(x) = -{\left(x - 1\right)}^{2} \end{dmath*}\vspace{-1.2cm}
\subitem \begin{dmath*}P_{8}(x) = 2 \, {\left(x - 2\right)} {\left(x - 3\right)} \end{dmath*}\vspace{-1.2cm}
\subitem \begin{dmath*}P_{9}(x) = -{\left(x - 1\right)} {\left(x - 3\right)} \end{dmath*}\vspace{-1.2cm}
\subitem \begin{dmath*}P_{10}(x) = -{\left(x + 2\right)} {\left(x + 1\right)} \end{dmath*}\vspace{-1.2cm}
\subitem \begin{dmath*}P_{11}(x) = {\left(x + 2\right)} {\left(x - 2\right)} \end{dmath*}\vspace{-1.2cm}
\subitem \begin{dmath*}P_{12}(x) = {\left(x - 3\right)}^{2} \end{dmath*}\vspace{-1.2cm}
\subitem \begin{dmath*}P_{13}(x) = {\left(x + 1\right)} {\left(x - 3\right)} \end{dmath*}\vspace{-1.2cm}
\subitem \begin{dmath*}P_{14}(x) = -2 \, {\left(x - 2\right)}^{2} \end{dmath*}\vspace{-1.2cm}
\subitem \begin{dmath*}P_{15}(x) = 2 \, {\left(x + 2\right)} {\left(x - 1\right)} \end{dmath*}\vspace{-1.2cm}
\subitem \begin{dmath*}P_{16}(x) = {\left(x + 1\right)} {\left(x - 2\right)} \end{dmath*}\vspace{-1.2cm}
\subitem \begin{dmath*}P_{17}(x) = 2 \, {\left(x + 1\right)} {\left(x - 1\right)} \end{dmath*}\vspace{-1.2cm}
\subitem \begin{dmath*}P_{18}(x) = 2 \, {\left(x + 1\right)} {\left(x - 1\right)} \end{dmath*}\vspace{-1.2cm}
\subitem \begin{dmath*}P_{19}(x) = {\left(x + 2\right)} {\left(x - 3\right)} \end{dmath*}\vspace{-1.2cm}
\subitem \begin{dmath*}P_{20}(x) = {\left(x - 1\right)} {\left(x - 3\right)} \end{dmath*}\vspace{-1.2cm}
\subitem \begin{dmath*}P_{21}(x) = {\left(x + 2\right)} {\left(x + 1\right)} \end{dmath*}\vspace{-1.2cm}
\subitem \begin{dmath*}P_{22}(x) = 2 \, {\left(x + 2\right)}^{2} \end{dmath*}\vspace{-1.2cm}
\subitem \begin{dmath*}P_{23}(x) = -2 \, {\left(x - 1\right)}^{2} \end{dmath*}\vspace{-1.2cm}
\subitem \begin{dmath*}P_{24}(x) = -2 \, {\left(x + 1\right)} {\left(x - 1\right)} \end{dmath*}\vspace{-1.2cm}
\subitem \begin{dmath*}P_{25}(x) = -2 \, {\left(x + 2\right)} {\left(x - 3\right)} \end{dmath*}\vspace{-1.2cm}
\subitem \begin{dmath*}P_{26}(x) = {\left(x + 2\right)} {\left(x - 2\right)} {\left(x - 3\right)} \end{dmath*}\vspace{-1.2cm}
\subitem \begin{dmath*}P_{27}(x) = 2 \, {\left(x + 2\right)} {\left(x - 2\right)}^{2} \end{dmath*}\vspace{-1.2cm}
\subitem \begin{dmath*}P_{28}(x) = -2 \, {\left(x + 2\right)}^{2} {\left(x - 3\right)} \end{dmath*}\vspace{-1.2cm}
\subitem \begin{dmath*}P_{29}(x) = 2 \, {\left(x + 2\right)} {\left(x + 1\right)} {\left(x - 3\right)} \end{dmath*}\vspace{-1.2cm}
\subitem \begin{dmath*}P_{30}(x) = {\left(x - 1\right)} {\left(x - 2\right)} {\left(x - 3\right)} \end{dmath*}\vspace{-1.2cm}
\subitem \begin{dmath*}P_{31}(x) = {\left(x + 1\right)}^{2} {\left(x - 3\right)} \end{dmath*}\vspace{-1.2cm}
\subitem \begin{dmath*}P_{32}(x) = {\left(x + 2\right)} {\left(x - 1\right)}^{2} \end{dmath*}\vspace{-1.2cm}
\subitem \begin{dmath*}P_{33}(x) = {\left(x + 2\right)}^{2} {\left(x - 3\right)} \end{dmath*}\vspace{-1.2cm}
\subitem \begin{dmath*}P_{34}(x) = -2 \, {\left(x + 2\right)}^{2} {\left(x - 2\right)} \end{dmath*}\vspace{-1.2cm}
\subitem \begin{dmath*}P_{35}(x) = {\left(x + 2\right)} {\left(x - 2\right)}^{2} \end{dmath*}\vspace{-1.2cm}
\subitem \begin{dmath*}P_{36}(x) = 2 \, {\left(x + 2\right)} {\left(x - 3\right)}^{2} \end{dmath*}\vspace{-1.2cm}
\subitem \begin{dmath*}P_{37}(x) = {\left(x - 1\right)} {\left(x - 3\right)}^{2} \end{dmath*}\vspace{-1.2cm}
\subitem \begin{dmath*}P_{38}(x) = 2 \, {\left(x + 2\right)} {\left(x + 1\right)}^{2} \end{dmath*}\vspace{-1.2cm}
\subitem \begin{dmath*}P_{39}(x) = {\left(x + 2\right)}^{2} {\left(x + 1\right)} \end{dmath*}\vspace{-1.2cm}
\subitem \begin{dmath*}P_{40}(x) = -2 \, {\left(x + 1\right)} {\left(x - 1\right)} {\left(x - 2\right)} \end{dmath*}\vspace{-1.2cm}
\subitem \begin{dmath*}P_{41}(x) = {\left(x + 2\right)} {\left(x + 1\right)} {\left(x - 1\right)} \end{dmath*}\vspace{-1.2cm}
\subitem \begin{dmath*}P_{42}(x) = 2 \, {\left(x + 2\right)}^{2} {\left(x - 2\right)} \end{dmath*}\vspace{-1.2cm}
\subitem \begin{dmath*}P_{43}(x) = {\left(x + 2\right)} {\left(x - 1\right)} {\left(x - 2\right)} \end{dmath*}\vspace{-1.2cm}
\subitem \begin{dmath*}P_{44}(x) = {\left(x + 2\right)} {\left(x + 1\right)} {\left(x - 2\right)} \end{dmath*}\vspace{-1.2cm}
\subitem \begin{dmath*}P_{45}(x) = -{\left(x - 2\right)}^{3} \end{dmath*}\vspace{-1.2cm}
\subitem \begin{dmath*}P_{46}(x) = {\left(x + 2\right)} {\left(x - 1\right)} {\left(x - 2\right)} \end{dmath*}\vspace{-1.2cm}
\subitem \begin{dmath*}P_{47}(x) = -{\left(x + 2\right)} {\left(x + 1\right)} {\left(x - 3\right)} \end{dmath*}\vspace{-1.2cm}
\subitem \begin{dmath*}P_{48}(x) = {\left(x + 1\right)}^{2} {\left(x - 1\right)} \end{dmath*}\vspace{-1.2cm}
\subitem \begin{dmath*}P_{49}(x) = -{\left(x + 2\right)} {\left(x + 1\right)} {\left(x - 1\right)} \end{dmath*}\vspace{-1.2cm}
\subitem \begin{dmath*}P_{50}(x) = -{\left(x + 1\right)}^{2} {\left(x - 3\right)} \end{dmath*}\vspace{-1.2cm}
\subitem \begin{dmath*}P_{51}(x) = -2 \, {\left(x + 2\right)} {\left(x + 1\right)}^{2} {\left(x - 2\right)} \end{dmath*}\vspace{-1.2cm}
\subitem \begin{dmath*}P_{52}(x) = -2 \, {\left(x + 2\right)} {\left(x - 1\right)} {\left(x - 2\right)}^{2} \end{dmath*}\vspace{-1.2cm}
\subitem \begin{dmath*}P_{53}(x) = 2 \, {\left(x + 2\right)} {\left(x - 2\right)} {\left(x - 3\right)}^{2} \end{dmath*}\vspace{-1.2cm}
\subitem \begin{dmath*}P_{54}(x) = -2 \, {\left(x + 2\right)}^{2} {\left(x + 1\right)}^{2} \end{dmath*}\vspace{-1.2cm}
\subitem \begin{dmath*}P_{55}(x) = {\left(x + 1\right)} {\left(x - 1\right)} {\left(x - 2\right)} {\left(x - 3\right)} \end{dmath*}\vspace{-1.2cm}
\subitem \begin{dmath*}P_{56}(x) = {\left(x + 2\right)} {\left(x - 1\right)} {\left(x - 2\right)}^{2} \end{dmath*}\vspace{-1.2cm}
\subitem \begin{dmath*}P_{57}(x) = {\left(x + 2\right)}^{2} {\left(x - 2\right)}^{2} \end{dmath*}\vspace{-1.2cm}
\subitem \begin{dmath*}P_{58}(x) = {\left(x + 2\right)} {\left(x + 1\right)}^{3} \end{dmath*}\vspace{-1.2cm}
\subitem \begin{dmath*}P_{59}(x) = {\left(x + 2\right)}^{2} {\left(x - 1\right)} {\left(x - 3\right)} \end{dmath*}\vspace{-1.2cm}
\subitem \begin{dmath*}P_{60}(x) = {\left(x + 2\right)}^{2} {\left(x - 1\right)}^{2} \end{dmath*}\vspace{-1.2cm}
\subitem \begin{dmath*}P_{61}(x) = {\left(x + 2\right)} {\left(x + 1\right)}^{2} {\left(x - 3\right)} \end{dmath*}\vspace{-1.2cm}
\subitem \begin{dmath*}P_{62}(x) = {\left(x + 1\right)}^{2} {\left(x - 3\right)}^{2} \end{dmath*}\vspace{-1.2cm}
\subitem \begin{dmath*}P_{63}(x) = -2 \, {\left(x + 1\right)} {\left(x - 1\right)}^{2} {\left(x - 2\right)} \end{dmath*}\vspace{-1.2cm}
\subitem \begin{dmath*}P_{64}(x) = -2 \, {\left(x + 1\right)} {\left(x - 2\right)}^{2} {\left(x - 3\right)} \end{dmath*}\vspace{-1.2cm}
\subitem \begin{dmath*}P_{65}(x) = -{\left(x + 2\right)} {\left(x - 1\right)} {\left(x - 2\right)} {\left(x - 3\right)} \end{dmath*}\vspace{-1.2cm}
\subitem \begin{dmath*}P_{66}(x) = -{\left(x + 2\right)} {\left(x - 2\right)} {\left(x - 3\right)}^{2} \end{dmath*}\vspace{-1.2cm}
\subitem \begin{dmath*}P_{67}(x) = {\left(x + 2\right)}^{2} {\left(x - 1\right)} {\left(x - 2\right)} \end{dmath*}\vspace{-1.2cm}
\subitem \begin{dmath*}P_{68}(x) = {\left(x + 1\right)} {\left(x - 1\right)}^{2} {\left(x - 2\right)} \end{dmath*}\vspace{-1.2cm}
\subitem \begin{dmath*}P_{69}(x) = -2 \, {\left(x - 1\right)} {\left(x - 2\right)}^{2} {\left(x - 3\right)} \end{dmath*}\vspace{-1.2cm}
\subitem \begin{dmath*}P_{70}(x) = {\left(x + 1\right)}^{2} {\left(x - 1\right)} {\left(x - 3\right)} \end{dmath*}\vspace{-1.2cm}
\subitem \begin{dmath*}P_{71}(x) = 2 \, {\left(x + 2\right)}^{2} {\left(x - 1\right)} {\left(x - 2\right)} \end{dmath*}\vspace{-1.2cm}
\subitem \begin{dmath*}P_{72}(x) = {\left(x + 2\right)} {\left(x - 2\right)}^{3} \end{dmath*}\vspace{-1.2cm}
\subitem \begin{dmath*}P_{73}(x) = -2 \, {\left(x + 2\right)} {\left(x + 1\right)} {\left(x - 2\right)} {\left(x - 3\right)} \end{dmath*}\vspace{-1.2cm}
\subitem \begin{dmath*}P_{74}(x) = -2 \, {\left(x - 2\right)}^{2} {\left(x - 3\right)}^{2} \end{dmath*}\vspace{-1.2cm}
\subitem \begin{dmath*}P_{75}(x) = -{\left(x + 2\right)}^{3} {\left(x - 3\right)} \end{dmath*}\vspace{-1.2cm}
\subitem \begin{dmath*}P_{76}(x) = 2 \, {\left(x + 2\right)}^{2} {\left(x - 1\right)} {\left(x - 2\right)} {\left(x - 3\right)} \end{dmath*}\vspace{-1.2cm}
\subitem \begin{dmath*}P_{77}(x) = -2 \, {\left(x + 1\right)}^{2} {\left(x - 1\right)}^{2} {\left(x - 3\right)} \end{dmath*}\vspace{-1.2cm}
\subitem \begin{dmath*}P_{78}(x) = {\left(x + 1\right)}^{3} {\left(x - 3\right)}^{2} \end{dmath*}\vspace{-1.2cm}
\subitem \begin{dmath*}P_{79}(x) = {\left(x + 2\right)} {\left(x + 1\right)}^{2} {\left(x - 1\right)} {\left(x - 2\right)} \end{dmath*}\vspace{-1.2cm}
\subitem \begin{dmath*}P_{80}(x) = 2 \, {\left(x + 2\right)} {\left(x + 1\right)} {\left(x - 1\right)} {\left(x - 3\right)}^{2} \end{dmath*}\vspace{-1.2cm}
\subitem \begin{dmath*}P_{81}(x) = 2 \, {\left(x + 2\right)}^{2} {\left(x - 3\right)}^{3} \end{dmath*}\vspace{-1.2cm}
\subitem \begin{dmath*}P_{82}(x) = {\left(x + 2\right)} {\left(x + 1\right)}^{2} {\left(x - 2\right)} {\left(x - 3\right)} \end{dmath*}\vspace{-1.2cm}
\subitem \begin{dmath*}P_{83}(x) = {\left(x + 2\right)} {\left(x + 1\right)}^{2} {\left(x - 1\right)} {\left(x - 2\right)} \end{dmath*}\vspace{-1.2cm}
\subitem \begin{dmath*}P_{84}(x) = {\left(x + 2\right)} {\left(x + 1\right)} {\left(x - 1\right)}^{3} \end{dmath*}\vspace{-1.2cm}
\subitem \begin{dmath*}P_{85}(x) = {\left(x + 2\right)} {\left(x + 1\right)}^{2} {\left(x - 2\right)}^{2} \end{dmath*}\vspace{-1.2cm}
\subitem \begin{dmath*}P_{86}(x) = -2 \, {\left(x + 1\right)}^{2} {\left(x - 2\right)} {\left(x - 3\right)}^{2} \end{dmath*}\vspace{-1.2cm}
\subitem \begin{dmath*}P_{87}(x) = 2 \, {\left(x + 2\right)} {\left(x - 2\right)} {\left(x - 3\right)}^{3} \end{dmath*}\vspace{-1.2cm}
\subitem \begin{dmath*}P_{88}(x) = {\left(x + 1\right)} {\left(x - 1\right)}^{2} {\left(x - 2\right)} {\left(x - 3\right)} \end{dmath*}\vspace{-1.2cm}
\subitem \begin{dmath*}P_{89}(x) = {\left(x + 2\right)}^{2} {\left(x - 1\right)} {\left(x - 2\right)}^{2} \end{dmath*}\vspace{-1.2cm}
\subitem \begin{dmath*}P_{90}(x) = -{\left(x + 1\right)}^{2} {\left(x - 2\right)}^{2} {\left(x - 3\right)} \end{dmath*}\vspace{-1.2cm}
\subitem \begin{dmath*}P_{91}(x) = {\left(x + 2\right)} {\left(x + 1\right)} {\left(x - 1\right)}^{2} {\left(x - 2\right)} \end{dmath*}\vspace{-1.2cm}
\subitem \begin{dmath*}P_{92}(x) = -{\left(x + 2\right)}^{2} {\left(x + 1\right)}^{2} {\left(x - 2\right)} \end{dmath*}\vspace{-1.2cm}
\subitem \begin{dmath*}P_{93}(x) = -2 \, {\left(x + 2\right)}^{2} {\left(x + 1\right)}^{2} {\left(x - 3\right)} \end{dmath*}\vspace{-1.2cm}
\subitem \begin{dmath*}P_{94}(x) = {\left(x + 1\right)} {\left(x - 1\right)} {\left(x - 3\right)}^{3} \end{dmath*}\vspace{-1.2cm}
\subitem \begin{dmath*}P_{95}(x) = 2 \, {\left(x + 1\right)} {\left(x - 1\right)}^{2} {\left(x - 2\right)} {\left(x - 3\right)} \end{dmath*}\vspace{-1.2cm}
\subitem \begin{dmath*}P_{96}(x) = {\left(x + 2\right)} {\left(x + 1\right)} {\left(x - 1\right)}^{2} {\left(x - 3\right)} \end{dmath*}\vspace{-1.2cm}
\subitem \begin{dmath*}P_{97}(x) = -2 \, {\left(x + 1\right)} {\left(x - 1\right)}^{2} {\left(x - 3\right)}^{2} \end{dmath*}\vspace{-1.2cm}
\subitem \begin{dmath*}P_{98}(x) = {\left(x + 2\right)}^{2} {\left(x + 1\right)} {\left(x - 2\right)} {\left(x - 3\right)} \end{dmath*}\vspace{-1.2cm}
\subitem \begin{dmath*}P_{99}(x) = 2 \, {\left(x + 1\right)} {\left(x - 1\right)}^{4} \end{dmath*}\vspace{-1.2cm}
\subitem \begin{dmath*}P_{100}(x) = 2 \, {\left(x + 2\right)} {\left(x + 1\right)}^{2} {\left(x - 1\right)} {\left(x - 3\right)} \end{dmath*}\vspace{-1.2cm}
\subitem \begin{dmath*}P_{101}(x) = {\left(x + 2\right)}^{2} {\left(x - 1\right)}^{3} {\left(x - 2\right)} \end{dmath*}\vspace{-1.2cm}
\subitem \begin{dmath*}P_{102}(x) = {\left(x + 2\right)}^{2} {\left(x + 1\right)} {\left(x - 2\right)} {\left(x - 3\right)}^{2} \end{dmath*}\vspace{-1.2cm}
\subitem \begin{dmath*}P_{103}(x) = -2 \, {\left(x + 2\right)} {\left(x - 1\right)} {\left(x - 2\right)} {\left(x - 3\right)}^{3} \end{dmath*}\vspace{-1.2cm}
\subitem \begin{dmath*}P_{104}(x) = {\left(x + 2\right)}^{2} {\left(x + 1\right)} {\left(x - 2\right)}^{2} {\left(x - 3\right)} \end{dmath*}\vspace{-1.2cm}
\subitem \begin{dmath*}P_{105}(x) = {\left(x + 2\right)} {\left(x + 1\right)} {\left(x - 1\right)} {\left(x - 2\right)} {\left(x - 3\right)}^{2} \end{dmath*}\vspace{-1.2cm}
\subitem \begin{dmath*}P_{106}(x) = {\left(x + 2\right)} {\left(x - 2\right)}^{4} {\left(x - 3\right)} \end{dmath*}\vspace{-1.2cm}
\subitem \begin{dmath*}P_{107}(x) = -{\left(x - 1\right)}^{2} {\left(x - 2\right)}^{2} {\left(x - 3\right)}^{2} \end{dmath*}\vspace{-1.2cm}
\subitem \begin{dmath*}P_{108}(x) = {\left(x + 2\right)} {\left(x + 1\right)}^{2} {\left(x - 2\right)}^{3} \end{dmath*}\vspace{-1.2cm}
\subitem \begin{dmath*}P_{109}(x) = {\left(x + 2\right)}^{2} {\left(x + 1\right)}^{2} {\left(x - 1\right)} {\left(x - 2\right)} \end{dmath*}\vspace{-1.2cm}
\subitem \begin{dmath*}P_{110}(x) = {\left(x + 1\right)}^{5} {\left(x - 2\right)} \end{dmath*}\vspace{-1.2cm}
\subitem \begin{dmath*}P_{111}(x) = -2 \, {\left(x + 2\right)} {\left(x + 1\right)}^{2} {\left(x - 1\right)}^{2} {\left(x - 2\right)} \end{dmath*}\vspace{-1.2cm}
\subitem \begin{dmath*}P_{112}(x) = 2 \, {\left(x + 2\right)}^{2} {\left(x - 2\right)}^{2} {\left(x - 3\right)}^{2} \end{dmath*}\vspace{-1.2cm}
\subitem \begin{dmath*}P_{113}(x) = 2 \, {\left(x + 2\right)} {\left(x + 1\right)} {\left(x - 1\right)} {\left(x - 2\right)} {\left(x - 3\right)}^{2} \end{dmath*}\vspace{-1.2cm}
\subitem \begin{dmath*}P_{114}(x) = -{\left(x + 1\right)}^{3} {\left(x - 1\right)} {\left(x - 2\right)} {\left(x - 3\right)} \end{dmath*}\vspace{-1.2cm}
\subitem \begin{dmath*}P_{115}(x) = {\left(x + 2\right)}^{2} {\left(x + 1\right)} {\left(x - 1\right)}^{2} {\left(x - 3\right)} \end{dmath*}\vspace{-1.2cm}
\subitem \begin{dmath*}P_{116}(x) = {\left(x + 1\right)} {\left(x - 1\right)}^{2} {\left(x - 2\right)} {\left(x - 3\right)}^{2} \end{dmath*}\vspace{-1.2cm}
\subitem \begin{dmath*}P_{117}(x) = {\left(x + 2\right)} {\left(x + 1\right)} {\left(x - 2\right)}^{3} {\left(x - 3\right)} \end{dmath*}\vspace{-1.2cm}
\subitem \begin{dmath*}P_{118}(x) = {\left(x + 2\right)} {\left(x + 1\right)} {\left(x - 1\right)}^{2} {\left(x - 3\right)}^{2} \end{dmath*}\vspace{-1.2cm}
\subitem \begin{dmath*}P_{119}(x) = {\left(x + 2\right)}^{2} {\left(x + 1\right)} {\left(x - 1\right)} {\left(x - 3\right)}^{2} \end{dmath*}\vspace{-1.2cm}
\subitem \begin{dmath*}P_{120}(x) = -2 \, {\left(x + 1\right)} {\left(x - 1\right)}^{3} {\left(x - 3\right)}^{2} \end{dmath*}\vspace{-1.2cm}
\subitem \begin{dmath*}P_{121}(x) = {\left(x + 2\right)} {\left(x + 1\right)}^{2} {\left(x - 1\right)} {\left(x - 2\right)} {\left(x - 3\right)} \end{dmath*}\vspace{-1.2cm}
\subitem \begin{dmath*}P_{122}(x) = {\left(x + 2\right)}^{2} {\left(x - 1\right)}^{3} {\left(x - 3\right)} \end{dmath*}\vspace{-1.2cm}
\subitem \begin{dmath*}P_{123}(x) = {\left(x + 2\right)}^{2} {\left(x + 1\right)} {\left(x - 1\right)} {\left(x - 2\right)}^{2} \end{dmath*}\vspace{-1.2cm}
\subitem \begin{dmath*}P_{124}(x) = {\left(x + 1\right)}^{2} {\left(x - 2\right)}^{2} {\left(x - 3\right)}^{2} \end{dmath*}\vspace{-1.2cm}
\subitem \begin{dmath*}P_{125}(x) = 2 \, {\left(x + 2\right)}^{2} {\left(x + 1\right)}^{2} {\left(x - 1\right)} {\left(x - 2\right)} \end{dmath*}\vspace{-1.2cm}
\subitem \begin{dmath*}P_{126}(x) = {\left(2 \, x + 1\right)} {\left(x - 2\right)} \end{dmath*}\vspace{-1.2cm}
\subitem \begin{dmath*}P_{127}(x) = {\left(2 \, x + 1\right)} {\left(x - 3\right)} \end{dmath*}\vspace{-1.2cm}
\subitem \begin{dmath*}P_{128}(x) = {\left(2 \, x - 1\right)} {\left(x + 1\right)} \end{dmath*}\vspace{-1.2cm}
\subitem \begin{dmath*}P_{129}(x) = 2 \, {\left(x + 1\right)} {\left(x - 2\right)} \end{dmath*}\vspace{-1.2cm}
\subitem \begin{dmath*}P_{130}(x) = {\left(3 \, x - 2\right)} {\left(x - 2\right)} \end{dmath*}\vspace{-1.2cm}
\subitem \begin{dmath*}P_{131}(x) = {\left(2 \, x - 1\right)} {\left(x - 1\right)} \end{dmath*}\vspace{-1.2cm}
\subitem \begin{dmath*}P_{132}(x) = {\left(3 \, x - 2\right)} {\left(x + 1\right)} \end{dmath*}\vspace{-1.2cm}
\subitem \begin{dmath*}P_{133}(x) = {\left(2 \, x - 1\right)} {\left(x + 2\right)} \end{dmath*}\vspace{-1.2cm}
\subitem \begin{dmath*}P_{134}(x) = {\left(x - 2\right)}^{2} \end{dmath*}\vspace{-1.2cm}
\subitem \begin{dmath*}P_{135}(x) = {\left(8 \, x - 3\right)} {\left(x - 1\right)} \end{dmath*}\vspace{-1.2cm}
\subitem \begin{dmath*}P_{136}(x) = {\left(2 \, x - 3\right)} {\left(x + 2\right)} \end{dmath*}\vspace{-1.2cm}
\subitem \begin{dmath*}P_{137}(x) = {\left(2 \, x - 1\right)} {\left(x - 2\right)} \end{dmath*}\vspace{-1.2cm}
\subitem \begin{dmath*}P_{138}(x) = {\left(3 \, x + 1\right)} {\left(x - 2\right)} \end{dmath*}\vspace{-1.2cm}
\subitem \begin{dmath*}P_{139}(x) = {\left(2 \, x - 1\right)} {\left(x + 2\right)} \end{dmath*}\vspace{-1.2cm}
\subitem \begin{dmath*}P_{140}(x) = 2 \, {\left(x + 2\right)} {\left(x + 1\right)} \end{dmath*}\vspace{-1.2cm}
\subitem \begin{dmath*}P_{141}(x) = {\left(3 \, x + 1\right)} {\left(x - 3\right)} \end{dmath*}\vspace{-1.2cm}
\subitem \begin{dmath*}P_{142}(x) = {\left(3 \, x + 1\right)} {\left(x - 1\right)} \end{dmath*}\vspace{-1.2cm}
\subitem \begin{dmath*}P_{143}(x) = {\left(7 \, x - 3\right)} {\left(x + 1\right)} \end{dmath*}\vspace{-1.2cm}
\subitem \begin{dmath*}P_{144}(x) = {\left(x - 2\right)} {\left(x - 3\right)} \end{dmath*}\vspace{-1.2cm}
\subitem \begin{dmath*}P_{145}(x) = {\left(2 \, x - 1\right)} {\left(x + 2\right)} \end{dmath*}\vspace{-1.2cm}
\subitem \begin{dmath*}P_{146}(x) = {\left(3 \, x + 2\right)} {\left(x - 2\right)} \end{dmath*}\vspace{-1.2cm}
\subitem \begin{dmath*}P_{147}(x) = {\left(2 \, x - 1\right)} {\left(x - 2\right)} \end{dmath*}\vspace{-1.2cm}
\subitem \begin{dmath*}P_{148}(x) = {\left(3 \, x + 1\right)} {\left(x + 2\right)} \end{dmath*}\vspace{-1.2cm}
\subitem \begin{dmath*}P_{149}(x) = {\left(2 \, x - 1\right)} {\left(x - 1\right)} \end{dmath*}\vspace{-1.2cm}
\subitem \begin{dmath*}P_{150}(x) = {\left(2 \, x - 1\right)} {\left(x - 3\right)} \end{dmath*}\vspace{-1.2cm}
\subitem \begin{dmath*}P_{151}(x) = {\left(3 \, x - 2\right)} {\left(x - 1\right)} {\left(x - 2\right)} \end{dmath*}\vspace{-1.2cm}
\subitem \begin{dmath*}P_{152}(x) = {\left(2 \, x - 1\right)} {\left(x + 2\right)} {\left(x - 2\right)} \end{dmath*}\vspace{-1.2cm}
\subitem \begin{dmath*}P_{153}(x) = {\left(2 \, x - 1\right)} {\left(x + 2\right)} {\left(x - 3\right)} \end{dmath*}\vspace{-1.2cm}
\subitem \begin{dmath*}P_{154}(x) = {\left(2 \, x - 1\right)} {\left(x + 2\right)} {\left(x + 1\right)} \end{dmath*}\vspace{-1.2cm}
\subitem \begin{dmath*}P_{155}(x) = -2 \, {\left(x + 2\right)}^{2} {\left(x - 2\right)} \end{dmath*}\vspace{-1.2cm}
\subitem \begin{dmath*}P_{156}(x) = -{\left(x + 2\right)}^{2} {\left(x - 3\right)} \end{dmath*}\vspace{-1.2cm}
\subitem \begin{dmath*}P_{157}(x) = {\left(3 \, x + 2\right)} {\left(x + 2\right)} {\left(x - 1\right)} \end{dmath*}\vspace{-1.2cm}
\subitem \begin{dmath*}P_{158}(x) = {\left(3 \, x - 1\right)} {\left(x + 2\right)} {\left(x + 1\right)} \end{dmath*}\vspace{-1.2cm}
\subitem \begin{dmath*}P_{159}(x) = {\left(x + 1\right)} {\left(x - 1\right)} {\left(x - 3\right)} \end{dmath*}\vspace{-1.2cm}
\subitem \begin{dmath*}P_{160}(x) = {\left(3 \, x - 1\right)} {\left(x + 1\right)} {\left(x - 2\right)} \end{dmath*}\vspace{-1.2cm}
\subitem \begin{dmath*}P_{161}(x) = {\left(3 \, x - 2\right)} {\left(x - 1\right)} {\left(x - 3\right)} \end{dmath*}\vspace{-1.2cm}
\subitem \begin{dmath*}P_{162}(x) = {\left(3 \, x - 1\right)} {\left(x + 2\right)} {\left(x - 3\right)} \end{dmath*}\vspace{-1.2cm}
\subitem \begin{dmath*}P_{163}(x) = {\left(3 \, x + 2\right)} {\left(x - 1\right)} {\left(x - 2\right)} \end{dmath*}\vspace{-1.2cm}
\subitem \begin{dmath*}P_{164}(x) = {\left(3 \, x + 1\right)} {\left(x + 1\right)} {\left(x - 2\right)} \end{dmath*}\vspace{-1.2cm}
\subitem \begin{dmath*}P_{165}(x) = {\left(3 \, x - 1\right)} {\left(x - 1\right)}^{2} \end{dmath*}\vspace{-1.2cm}
\subitem \begin{dmath*}P_{166}(x) = {\left(3 \, x - 1\right)} {\left(x + 2\right)} {\left(x + 1\right)} \end{dmath*}\vspace{-1.2cm}
\subitem \begin{dmath*}P_{167}(x) = {\left(3 \, x - 2\right)} {\left(x + 1\right)} {\left(x - 3\right)} \end{dmath*}\vspace{-1.2cm}
\subitem \begin{dmath*}P_{168}(x) = {\left(3 \, x - 1\right)} {\left(x + 2\right)} {\left(x - 1\right)} \end{dmath*}\vspace{-1.2cm}
\subitem \begin{dmath*}P_{169}(x) = {\left(3 \, x + 2\right)} {\left(x + 1\right)} {\left(x - 2\right)} \end{dmath*}\vspace{-1.2cm}
\subitem \begin{dmath*}P_{170}(x) = {\left(3 \, x - 1\right)} {\left(x + 2\right)} {\left(x - 2\right)} \end{dmath*}\vspace{-1.2cm}
\subitem \begin{dmath*}P_{171}(x) = {\left(3 \, x + 1\right)} {\left(x + 2\right)} {\left(x + 1\right)} \end{dmath*}\vspace{-1.2cm}
\subitem \begin{dmath*}P_{172}(x) = {\left(4 \, x - 3\right)} {\left(x + 2\right)} {\left(x - 2\right)} \end{dmath*}\vspace{-1.2cm}
\subitem \begin{dmath*}P_{173}(x) = {\left(3 \, x + 2\right)} {\left(x - 3\right)}^{2} \end{dmath*}\vspace{-1.2cm}
\subitem \begin{dmath*}P_{174}(x) = {\left(3 \, x + 2\right)} {\left(x + 2\right)} {\left(x + 1\right)} \end{dmath*}\vspace{-1.2cm}
\subitem \begin{dmath*}P_{175}(x) = {\left(3 \, x + 1\right)} {\left(x + 2\right)} {\left(x - 3\right)} \end{dmath*}\vspace{-1.2cm}
\subitem \begin{dmath*}P_{176}(x) = {\left(3 \, x + 1\right)} {\left(x + 2\right)} {\left(x - 2\right)} {\left(x - 3\right)} \end{dmath*}\vspace{-1.2cm}
\subitem \begin{dmath*}P_{177}(x) = {\left(2 \, x + 1\right)} {\left(x + 2\right)} {\left(x + 1\right)} {\left(x - 3\right)} \end{dmath*}\vspace{-1.2cm}
\subitem \begin{dmath*}P_{178}(x) = {\left(3 \, x + 2\right)} {\left(x + 2\right)} {\left(x - 2\right)} {\left(x - 3\right)} \end{dmath*}\vspace{-1.2cm}
\subitem \begin{dmath*}P_{179}(x) = {\left(8 \, x - 3\right)} {\left(x + 2\right)}^{2} {\left(x + 1\right)} \end{dmath*}\vspace{-1.2cm}
\subitem \begin{dmath*}P_{180}(x) = {\left(3 \, x + 2\right)} {\left(x + 2\right)} {\left(x + 1\right)} {\left(x - 1\right)} \end{dmath*}\vspace{-1.2cm}
\subitem \begin{dmath*}P_{181}(x) = {\left(3 \, x - 1\right)} {\left(x + 2\right)} {\left(x + 1\right)} {\left(x - 1\right)} \end{dmath*}\vspace{-1.2cm}
\subitem \begin{dmath*}P_{182}(x) = 2 \, {\left(x + 2\right)}^{2} {\left(x - 2\right)} {\left(x - 3\right)} \end{dmath*}\vspace{-1.2cm}
\subitem \begin{dmath*}P_{183}(x) = {\left(3 \, x - 1\right)} {\left(x + 1\right)} {\left(x - 2\right)} {\left(x - 3\right)} \end{dmath*}\vspace{-1.2cm}
\subitem \begin{dmath*}P_{184}(x) = {\left(3 \, x - 2\right)} {\left(x + 2\right)} {\left(x + 1\right)} {\left(x - 1\right)} \end{dmath*}\vspace{-1.2cm}
\subitem \begin{dmath*}P_{185}(x) = {\left(5 \, x - 3\right)} {\left(x + 2\right)} {\left(x + 1\right)}^{2} \end{dmath*}\vspace{-1.2cm}
\subitem \begin{dmath*}P_{186}(x) = {\left(3 \, x - 1\right)} {\left(x + 1\right)} {\left(x - 1\right)} {\left(x - 2\right)} \end{dmath*}\vspace{-1.2cm}
\subitem \begin{dmath*}P_{187}(x) = {\left(2 \, x - 1\right)} {\left(x - 1\right)}^{2} {\left(x - 3\right)} \end{dmath*}\vspace{-1.2cm}
\subitem \begin{dmath*}P_{188}(x) = {\left(3 \, x - 1\right)} {\left(x + 2\right)} {\left(x - 2\right)} {\left(x - 3\right)} \end{dmath*}\vspace{-1.2cm}
\subitem \begin{dmath*}P_{189}(x) = {\left(3 \, x - 1\right)} {\left(x + 1\right)}^{2} {\left(x - 1\right)} \end{dmath*}\vspace{-1.2cm}
\subitem \begin{dmath*}P_{190}(x) = -{\left(x + 2\right)} {\left(x + 1\right)} {\left(x - 1\right)} {\left(x - 2\right)} \end{dmath*}\vspace{-1.2cm}
\subitem \begin{dmath*}P_{191}(x) = {\left(2 \, x + 1\right)} {\left(x + 2\right)} {\left(x + 1\right)} {\left(x - 2\right)} \end{dmath*}\vspace{-1.2cm}
\subitem \begin{dmath*}P_{192}(x) = {\left(3 \, x + 1\right)} {\left(x + 1\right)} {\left(x - 1\right)} {\left(x - 3\right)} \end{dmath*}\vspace{-1.2cm}
\subitem \begin{dmath*}P_{193}(x) = {\left(3 \, x - 1\right)} {\left(x + 2\right)}^{2} {\left(x - 3\right)} \end{dmath*}\vspace{-1.2cm}
\subitem \begin{dmath*}P_{194}(x) = 2 \, {\left(x + 2\right)} {\left(x - 1\right)} {\left(x - 2\right)} {\left(x - 3\right)} \end{dmath*}\vspace{-1.2cm}
\subitem \begin{dmath*}P_{195}(x) = {\left(3 \, x - 2\right)} {\left(x + 1\right)}^{2} {\left(x - 2\right)} \end{dmath*}\vspace{-1.2cm}
\subitem \begin{dmath*}P_{196}(x) = {\left(x - 1\right)} {\left(x - 2\right)}^{2} {\left(x - 3\right)} \end{dmath*}\vspace{-1.2cm}
\subitem \begin{dmath*}P_{197}(x) = {\left(3 \, x - 1\right)} {\left(x - 1\right)} {\left(x - 2\right)}^{2} \end{dmath*}\vspace{-1.2cm}
\subitem \begin{dmath*}P_{198}(x) = {\left(2 \, x + 1\right)} {\left(x + 1\right)} {\left(x - 1\right)} {\left(x - 2\right)} \end{dmath*}\vspace{-1.2cm}
\subitem \begin{dmath*}P_{199}(x) = {\left(3 \, x + 2\right)} {\left(x + 2\right)} {\left(x - 1\right)} {\left(x - 3\right)} \end{dmath*}\vspace{-1.2cm}
\subitem \begin{dmath*}P_{200}(x) = {\left(x - 1\right)} {\left(x - 2\right)} {\left(x - 3\right)}^{2} \end{dmath*}\vspace{-1.2cm}
\subitem \begin{dmath*}P_{201}(x) = {\left(2 \, x - 1\right)} {\left(x + 2\right)} {\left(x - 2\right)} {\left(x - 3\right)}^{2} \end{dmath*}\vspace{-1.2cm}
\subitem \begin{dmath*}P_{202}(x) = {\left(3 \, x - 2\right)} {\left(x + 1\right)} {\left(x - 1\right)}^{2} {\left(x - 2\right)} \end{dmath*}\vspace{-1.2cm}
\subitem \begin{dmath*}P_{203}(x) = {\left(3 \, x - 1\right)} {\left(x + 2\right)} {\left(x + 1\right)} {\left(x - 1\right)} {\left(x - 2\right)} \end{dmath*}\vspace{-1.2cm}
\subitem \begin{dmath*}P_{204}(x) = {\left(2 \, x - 1\right)} {\left(x + 1\right)}^{2} {\left(x - 1\right)}^{2} \end{dmath*}\vspace{-1.2cm}
\subitem \begin{dmath*}P_{205}(x) = {\left(2 \, x - 1\right)} {\left(x + 1\right)}^{2} {\left(x - 1\right)} {\left(x - 2\right)} \end{dmath*}\vspace{-1.2cm}
\subitem \begin{dmath*}P_{206}(x) = 2 \, {\left(x + 1\right)} {\left(x - 1\right)} {\left(x - 2\right)} {\left(x - 3\right)}^{2} \end{dmath*}\vspace{-1.2cm}
\subitem \begin{dmath*}P_{207}(x) = {\left(3 \, x - 1\right)} {\left(x + 1\right)} {\left(x - 1\right)} {\left(x - 2\right)}^{2} \end{dmath*}\vspace{-1.2cm}
\subitem \begin{dmath*}P_{208}(x) = {\left(3 \, x + 1\right)} {\left(x - 1\right)} {\left(x - 3\right)}^{3} \end{dmath*}\vspace{-1.2cm}
\subitem \begin{dmath*}P_{209}(x) = {\left(2 \, x - 1\right)} {\left(x + 2\right)} {\left(x - 1\right)}^{2} {\left(x - 3\right)} \end{dmath*}\vspace{-1.2cm}
\subitem \begin{dmath*}P_{210}(x) = {\left(7 \, x - 3\right)} {\left(x + 1\right)} {\left(x - 1\right)}^{2} {\left(x - 3\right)} \end{dmath*}\vspace{-1.2cm}
\subitem \begin{dmath*}P_{211}(x) = {\left(3 \, x + 1\right)} {\left(x + 2\right)} {\left(x + 1\right)}^{2} {\left(x - 2\right)} \end{dmath*}\vspace{-1.2cm}
\subitem \begin{dmath*}P_{212}(x) = {\left(2 \, x + 1\right)} {\left(x + 2\right)} {\left(x - 2\right)}^{2} {\left(x - 3\right)} \end{dmath*}\vspace{-1.2cm}
\subitem \begin{dmath*}P_{213}(x) = -{\left(x + 2\right)} {\left(x - 1\right)} {\left(x - 2\right)}^{3} \end{dmath*}\vspace{-1.2cm}
\subitem \begin{dmath*}P_{214}(x) = {\left(2 \, x - 1\right)} {\left(x + 1\right)} {\left(x - 1\right)}^{2} {\left(x - 3\right)} \end{dmath*}\vspace{-1.2cm}
\subitem \begin{dmath*}P_{215}(x) = {\left(3 \, x + 1\right)} {\left(x + 2\right)} {\left(x - 1\right)}^{2} {\left(x - 2\right)} \end{dmath*}\vspace{-1.2cm}
\subitem \begin{dmath*}P_{216}(x) = {\left(7 \, x - 3\right)} {\left(x + 2\right)} {\left(x - 1\right)}^{2} {\left(x - 3\right)} \end{dmath*}\vspace{-1.2cm}
\subitem \begin{dmath*}P_{217}(x) = {\left(3 \, x - 1\right)} {\left(x + 1\right)}^{2} {\left(x - 3\right)}^{2} \end{dmath*}\vspace{-1.2cm}
\subitem \begin{dmath*}P_{218}(x) = {\left(4 \, x - 3\right)} {\left(x + 2\right)}^{2} {\left(x - 2\right)} {\left(x - 3\right)} \end{dmath*}\vspace{-1.2cm}
\subitem \begin{dmath*}P_{219}(x) = {\left(3 \, x - 2\right)} {\left(x + 1\right)}^{3} {\left(x - 3\right)} \end{dmath*}\vspace{-1.2cm}
\subitem \begin{dmath*}P_{220}(x) = {\left(3 \, x + 2\right)} {\left(x + 1\right)} {\left(x - 1\right)}^{2} {\left(x - 2\right)} \end{dmath*}\vspace{-1.2cm}
\subitem \begin{dmath*}P_{221}(x) = {\left(2 \, x + 1\right)} {\left(x + 2\right)} {\left(x + 1\right)}^{2} {\left(x - 2\right)} \end{dmath*}\vspace{-1.2cm}
\subitem \begin{dmath*}P_{222}(x) = {\left(2 \, x - 3\right)} {\left(x + 1\right)} {\left(x - 1\right)} {\left(x - 2\right)}^{2} \end{dmath*}\vspace{-1.2cm}
\subitem \begin{dmath*}P_{223}(x) = -2 \, {\left(x + 2\right)}^{3} {\left(x - 2\right)}^{2} \end{dmath*}\vspace{-1.2cm}
\subitem \begin{dmath*}P_{224}(x) = -2 \, {\left(x + 2\right)} {\left(x + 1\right)} {\left(x - 1\right)} {\left(x - 2\right)} {\left(x - 3\right)} \end{dmath*}\vspace{-1.2cm}
\subitem \begin{dmath*}P_{225}(x) = {\left(7 \, x - 3\right)} {\left(x + 1\right)}^{2} {\left(x - 1\right)} {\left(x - 3\right)} \end{dmath*}\vspace{-1.2cm}
\subitem \begin{dmath*}P_{226}(x) = -2 \, {\left(x + 2\right)} {\left(x - 1\right)}^{4} {\left(x - 2\right)} \end{dmath*}\vspace{-1.2cm}
\subitem \begin{dmath*}P_{227}(x) = {\left(x + 2\right)}^{2} {\left(x + 1\right)}^{3} {\left(x - 2\right)} \end{dmath*}\vspace{-1.2cm}
\subitem \begin{dmath*}P_{228}(x) = {\left(5 \, x - 3\right)} {\left(x + 2\right)}^{2} {\left(x + 1\right)} {\left(x - 1\right)}^{2} \end{dmath*}\vspace{-1.2cm}
\subitem \begin{dmath*}P_{229}(x) = {\left(2 \, x + 1\right)} {\left(x + 2\right)}^{2} {\left(x + 1\right)} {\left(x - 1\right)} {\left(x - 2\right)} \end{dmath*}\vspace{-1.2cm}
\subitem \begin{dmath*}P_{230}(x) = {\left(3 \, x - 1\right)} {\left(x + 2\right)} {\left(x + 1\right)}^{3} {\left(x - 3\right)} \end{dmath*}\vspace{-1.2cm}
\subitem \begin{dmath*}P_{231}(x) = {\left(3 \, x + 2\right)} {\left(x + 2\right)} {\left(x - 1\right)} {\left(x - 2\right)}^{2} {\left(x - 3\right)} \end{dmath*}\vspace{-1.2cm}
\subitem \begin{dmath*}P_{232}(x) = {\left(2 \, x + 1\right)} {\left(x + 2\right)} {\left(x + 1\right)} {\left(x - 1\right)} {\left(x - 2\right)} {\left(x - 3\right)} \end{dmath*}\vspace{-1.2cm}
\subitem \begin{dmath*}P_{233}(x) = {\left(4 \, x - 3\right)} {\left(x + 2\right)} {\left(x + 1\right)} {\left(x - 2\right)}^{2} {\left(x - 3\right)} \end{dmath*}\vspace{-1.2cm}
\subitem \begin{dmath*}P_{234}(x) = {\left(3 \, x - 2\right)} {\left(x + 2\right)} {\left(x - 1\right)}^{2} {\left(x - 2\right)} {\left(x - 3\right)} \end{dmath*}\vspace{-1.2cm}
\subitem \begin{dmath*}P_{235}(x) = {\left(x + 2\right)} {\left(x + 1\right)}^{2} {\left(x - 2\right)} {\left(x - 3\right)}^{2} \end{dmath*}\vspace{-1.2cm}
\subitem \begin{dmath*}P_{236}(x) = {\left(3 \, x + 1\right)} {\left(x - 1\right)} {\left(x - 2\right)}^{2} {\left(x - 3\right)}^{2} \end{dmath*}\vspace{-1.2cm}
\subitem \begin{dmath*}P_{237}(x) = {\left(3 \, x + 1\right)} {\left(x + 1\right)} {\left(x - 1\right)} {\left(x - 2\right)}^{2} {\left(x - 3\right)} \end{dmath*}\vspace{-1.2cm}
\subitem \begin{dmath*}P_{238}(x) = {\left(2 \, x - 1\right)} {\left(x + 2\right)}^{2} {\left(x + 1\right)} {\left(x - 3\right)}^{2} \end{dmath*}\vspace{-1.2cm}
\subitem \begin{dmath*}P_{239}(x) = {\left(3 \, x + 2\right)} {\left(x + 2\right)} {\left(x + 1\right)} {\left(x - 1\right)}^{2} {\left(x - 2\right)} \end{dmath*}\vspace{-1.2cm}
\subitem \begin{dmath*}P_{240}(x) = -{\left(x + 2\right)} {\left(x + 1\right)} {\left(x - 1\right)}^{2} {\left(x - 2\right)} {\left(x - 3\right)} \end{dmath*}\vspace{-1.2cm}
\subitem \begin{dmath*}P_{241}(x) = {\left(2 \, x - 3\right)} {\left(x + 2\right)}^{3} {\left(x + 1\right)} {\left(x - 1\right)} \end{dmath*}\vspace{-1.2cm}
\subitem \begin{dmath*}P_{242}(x) = {\left(3 \, x + 2\right)} {\left(x + 1\right)}^{2} {\left(x - 1\right)}^{2} {\left(x - 2\right)} \end{dmath*}\vspace{-1.2cm}
\subitem \begin{dmath*}P_{243}(x) = {\left(3 \, x + 1\right)} {\left(x + 2\right)} {\left(x + 1\right)} {\left(x - 1\right)} {\left(x - 2\right)} {\left(x - 3\right)} \end{dmath*}\vspace{-1.2cm}
\subitem \begin{dmath*}P_{244}(x) = {\left(x + 2\right)} {\left(x + 1\right)}^{2} {\left(x - 2\right)}^{3} \end{dmath*}\vspace{-1.2cm}
\subitem \begin{dmath*}P_{245}(x) = {\left(x + 2\right)} {\left(x + 1\right)} {\left(x - 1\right)}^{3} {\left(x - 2\right)} \end{dmath*}\vspace{-1.2cm}
\subitem \begin{dmath*}P_{246}(x) = -{\left(x + 2\right)} {\left(x + 1\right)} {\left(x - 2\right)}^{2} {\left(x - 3\right)}^{2} \end{dmath*}\vspace{-1.2cm}
\subitem \begin{dmath*}P_{247}(x) = -{\left(x + 2\right)}^{2} {\left(x - 1\right)} {\left(x - 2\right)}^{2} {\left(x - 3\right)} \end{dmath*}\vspace{-1.2cm}
\subitem \begin{dmath*}P_{248}(x) = {\left(3 \, x - 1\right)} {\left(x + 2\right)} {\left(x + 1\right)} {\left(x - 1\right)} {\left(x - 3\right)}^{2} \end{dmath*}\vspace{-1.2cm}
\subitem \begin{dmath*}P_{249}(x) = {\left(2 \, x - 1\right)} {\left(x + 2\right)} {\left(x + 1\right)}^{2} {\left(x - 1\right)} {\left(x - 2\right)} \end{dmath*}\vspace{-1.2cm}
\subitem \begin{dmath*}P_{250}(x) = {\left(3 \, x - 2\right)} {\left(x + 2\right)}^{2} {\left(x - 1\right)}^{2} {\left(x - 3\right)} \end{dmath*}\vspace{-1.2cm}


\end{multicols}

\end{document}
