\documentclass[palatino,nosec]{Docencia}

\usepackage{sagetex}

\title{Análisis de Funciones}
%\newcommand{\subtitle}[1]{\def\thesubtitle{#1}\@subsettrue}
%\subtitle{4º ESO Matemáticas Académicas}
\author{Departamento de Matemáticas}
\date{17/18}



\begin{abstract}

Ejercicios de polinomios generados automáticamente.


\textit{Nota: Este documento se ha generado automáticamente utilizando \href{www.sagemath.org/es/}{Sage}, \LaTeX\xspace y \href{https://github.com/sagemath/sagetex}{sagetex} para la integración de las 2 herramientas mencionadas.}

\end{abstract}

% Paquetes adicionales

\usepackage[author={Víctor de Juan, 2018}]{pdfcomment}

\makeatletter
\newcommand{\annotate}[2][]{%
	\pdfstringdef\x@title{#1}%
	\edef\r{\string\r}%
	\pdfstringdef\x@contents{#2}%
	\pdfannot
	width 2\baselineskip
	height 2\baselineskip
	depth 0pt
	{
		/Subtype /Text
		/T (\x@title)
		/Contents (\x@contents)
	}%
}
\makeatother



% --------------------
\newcommand{\cimplies}{\text{\hl{$\implies$}}}
\renewcommand{\vx}{\overset{\rightarrow}{x}}
\renewcommand{\vy}{\overset{\rightarrow}{y}}
\renewcommand{\vz}{\overset{\rightarrow}{z}}
\newcommand{\vi}{\overset{\rightarrow}{i}}
\newcommand{\vj}{\overset{\rightarrow}{j}}
\renewcommand{\vec}[1]{\overset{\rightarrow}{#1}}




\begin{document}

\begin{itemize}

\paragraph{[4 puntos] 1) Dados $P(x) = 3 \, x^{3} - 14 \, x^{2} + 17 \, x - 6 $ y $Q(x) = 3 \, x^{2} + 5 \, x - 2 $, realiza las siguientes operaciones:} \begin{itemize}\item\textit{1 pto}\;\; $P(x) - Q(x)$\item \textit{1pto}\;\; $\left(-2x^2\right) \cdot P(x)$\item\textit{2ptos}\;\;$P(x)\cdot Q(x)$  \end{itemize} \paragraph{[6 puntos] 2) Resuelve las siguientes operaciones:} \begin{itemize} \item $ \left( \frac{3}{7} \, x y + \frac{8}{7} \right) \left( \frac{3}{7} \, x y - \frac{8}{7} \right) $ \item $ \left( 5 \, x + 4 \right) \left( 5 \, x - 4 \right) $ \item $ \left( 3 \, x y + \frac{3}{5} \right)^2 $ \item $ \left( 2 \, x y - \frac{7}{12} \right)^2 $ \item $ \left( 4 \, x y + 7 \right)^2 $ \item $ \left( 3 \, x y - 3 \right)^2 $ \end{itemize}

\end{itemize}

\end{document}