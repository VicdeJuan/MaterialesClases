\documentclass[palatino,nosec,nochap,nobuilddate]{Docencia}
\usetikzlibrary[patterns]


\title{Corrección P1 1ª Ev. 2ºBach B}
\author{Departamento de Matemáticas}
\date{19/20}


% Paquetes adicionales

\usepackage[author={Víctor de Juan, 2019}]{pdfcomment}

\makeatletter
\newcommand{\annotate}[2][]{%
\pdfstringdef\x@title{#1}%
\edef\r{\string\r}%
\pdfstringdef\x@contents{#2}%
\pdfannot
width 2\baselineskip
height 2\baselineskip
depth 0pt
{
/Subtype /Text
/T (\x@title)
/Contents (\x@contents)
}%
}
\makeatother



\usepackage{eso-pic}
\newcommand\BackgroundPic{%
\put(0,0){%
\parbox[b][\paperheight]{\paperwidth}{%
\vfill
\centering
\includegraphics[width=\paperwidth,height=\paperheight,%
keepaspectratio]{../../../../BWLogo.jpeg}%
\vfill
}}}





\begin{abstract}
Corrección del examen parcial de la segunda evaluación del curso 2019-2020.

\nota{No está exento de erratas. En caso de descubrir alguna, por favor, comunicarlas al autor.}
\end{abstract}

% --------------------
\newcommand{\cimplies}{\text{\hl{$\implies$}}}
\renewcommand{\vec}[1]{\overrightarrow{#1}}

\begin{document}
\pagestyle{plain}
\maketitle

%\AddToShipoutPicture{\BackgroundPic}

\newpage
\begin{problem}(2 puntos)

Dado el siguiente sistema, dis
\ppart Calcular la probabilidad de que llegue puntual al trabajo.
\ppart Calcular la probabilidad de que no haya sonado el despertador, si se sabe que ha llegado puntual al trabajo
\solution

$S$ = "Suena el despertador". \;\;\;\; $T$ = "Se despierta y llega puntual al trabajo". 

Datos:
\begin{itemize}
	\item $P(S) = 0'9$
	\item $P(T/S) = 0'7 \to P(\overline{T}/S) = 0'3$ 
	\item $P(T/\overline{S}) = 0'3 \to P(\overline{T}/\overline{S}) = 0'7$
\end{itemize}
\spart
Utilizando el teorema de la probabilidad total:

\[P(T) = P(T\cap S) + P(T\cap \overline{S}) = P(S)·P(T/S) + P(\overline{S})·P(T\overline{S}) = 0'9·0'7 + 0'1·0'3 = 0'66\]

\spart

\[P(\overline{S}/T) = \frac{P(\overline{S}\cap T)}{P(T)} = \frac{P(\overline{S})·P(T\overline{S})}{P(T)} =\frac{0'3·0'1}{0'66} = 0'05\]

\end{problem}

\begin{problem}
Sabiendo que $P(A) = 0'4$, $P(B) = 0'5$ y $P(A\cup B) = 0'7$, calcular las probabilidades de:
\ppart $\overline{A}\cap B$
\ppart $\overline{A}\cap\overline{B}$
\ppart ¿Son independientes?¿Son incompatibles? Razonar las respuestas.

\solution

$P(A\cap B) = P(A) + P(B) - P(A\cup B) = 0'4+0'5-0'7 = 0'2$

\spart \[P(\overline{A}\cap B) = P(B) - P(A\cap B) = 0'5-0'2 = 0'3\]

\spart \[P(\overline{A}\cap\overline{B}) \overset{(1)}{=} P\left(\overline{A\cup B}\right) = 1 - P(A\cup B) = 1-0'7 = 0'3 \]
(1): ley de de Morgan.

\spart 
No son incompatibles porque $P(A\cap B) \neq 0$.

Dos sucesos son independientes si $P(A\cap B) = P(A)·P(B)$. En este caso, tenemos $P(A\cap B) = 0'2$. Comprobamos $P(A\cap B) = P(A)·P(B) = 0'5·0'4 = 0'2$, por lo que los sucesos sí son independientes.

\end{problem}

\begin{problem}
Sabiendo que el peso de los estudiantes varones de segundo de bachillerato se puede aproximar por una variable aleatoria con distribución normal, de media 74kg y desviación típica 6kg, se pide:
\ppart Determinar el porcentaje de estudiantes varones cuyo peso está comprendido entre los 68 y 80kg.
\ppart Estimar cuántos de los 1500 estudiantes varones, que se han presentado a las pruebas de la EvAU en una cierta universidad, pesan más de 80 kg.
\ppart Completa la frase: "El 90\% de los estudiantes pesa más de ... kg"
\ppart \textit{(Apartado original: )} Si se sabe que uno de estos estudiantes pesa más de 76kg, ¿cuál es la probabilidad de que pese más de 86kg?

\solution

$X$: "Peso de los estudiantes varones de 2º de Bachillerato"

$X\equiv N(74,6)$

\spart 
\[
P(68 < X < 80) =  P(X<80) - P(X<68) = 
P\left(\frac{X-74}{6}<\frac{80-74}{6}\right) - 
P\left(\frac{X-74}{6}<\frac{68-74}{6}\right) =\]
\[
P(Z < 1) - P(Z<-1) \overset{(1)}{=} P(Z<1) - (1-P(Z<1)) = 2·P(Z<1) -1 = 2·0'8413 -1 = 0'6826  
\]
(1): aplicamos suceso complementario y simetría en $P(Z<-1)$

El porcentaje de estudiantes cuyo peso está comprendido entre 68 y 80 kg es del 68'26\%.

\spart $1500·P(X>80) = 0,1587·1500 = 238'05 \sim 238$

\[P(X>80) = P\left(\frac{X-74}{6}>\frac{80-74}{6}\right) = P(Z>1) = 1-P(Z<1) = 1-0'8413 = 0,1587\]

\spart El dato del 90\% podemos traducirlo por "la probabilidad de pesar más de $a$ kg es 0'9"

\[P(X>a) = 0'9 \dimplies P\left(\frac{X-74}{6}>\frac{a-74}{6}\right) = 0'9 \dimplies P\left(Z<-\frac{a-74}{6}\right) = 0'9\]

Buscando en la tabla qué valor tiene probabilidad 0'9, tomamos el valor más cercano como aproximación (ya que en las probabilidades de la tabla no encontramos 0'9). $\displaystyle P\left(Z<-\frac{a-74}{6}\right) = 0'8997$

\[-\frac{a-74}{6} = 1'28 \dimplies a-74 = -6·1'28 = -7'68 \dimplies a = 74-7'68 = 66'32\text{kg}\]

El 90\% de los estudiantes pesan más de 66'32kg.

\spart 

\[
P(X>86 /  X>76) = \frac{P(X>86 \cap X>76)}{P(X>76} = \frac{P(X>86)}{P(X>76)} \overset{(1)(2)}{=}\frac{0'0228}{0'3707} = 0,0615
\]

\[(1): P(X>86) = P\left(\frac{X-74}{6}>\frac{86-74}{6}\right) = P(Z>2) = 1-P(Z<2) = 1 - 0'9772 = 0'0228\]
\[(2): P(X>86) = P\left(\frac{X-74}{6}>\frac{76-74}{6}\right) = P\left(Z>\rfrac{1}{3}\right) = 1- P\left(Z<\rfrac{1}{3}\right) = 1-0'6293 = 0'3707\]
\end{problem}

\begin{problem}
A un examen se de verdadero/falso de 6 preguntas se presenta un estudiante con un historial de aciertos en pruebas similares de 7 de cada 10 preguntas. ¿Cuál es la probabilidad de que conteste correctamente más de 2? ¿Y de tener un solo fallo?
\solution

$X:$ número de respuestas contestadas correctamente.

$X\equiv B(6,0'7)$

Probabilidad de acertar más de 2: $P(X>2) = 1 - \left(P(X=0) + P(X=1) + P(X=2)\right) \overset{(1)}{=} 0,93$

(1):
Utilizando $P(X=k) = \displaystyle\comb{n}{k}·p^k·(1-p)^{n-k}$, tenemos:
\[P(X = 0) = \comb{n}{0}·p^0·(1-p)^{n-0} = (1-0'7)^6 = 0'3^{6} = 7,29·10^{-4}   \]
\[P(X = 1) = \comb{n}{1}·p^1·(1-p)^{n-1} = \comb{6}{1}·p^1·(1-p)^{6-1} = 6·0'7·0'3^{5} = 0'01 \]
\[P(X = 2) = \comb{n}{2}·p^2·(1-p)^{n-2} = \comb{6}{2}·p^2·(1-p)^{6-2} = 6·0'7^2·0'3^{4} = 0'06 \]

Probabilidad de tener solo un fallo, es lo mismo que acertar todas menos 1, por lo que $P(X=5) = \comb{6}{5}·0'7^5·0'3^{1} = 0'3$

\end{problem}

\begin{problem}
Lanzamos un dado de 12 caras 30 veces. ¿Cuál es la probabilidad de obtener más de 10 veces un múltiplo de 3?
\solution

$X:$ número de múltiplos de 3 obtenidos en 30 tiradas.

Como hay 4 múltiplos de 3 (\{3,6,9,12\}), entre 12 posibilidades, $p=\rfrac{4}{12} = \rfrac{1}{3}$

$X\equiv B\left(30,\rfrac{1}{3}\right)$

$P(X>10)$ requiere demasiados cálculos, por lo que estudiamos si se puede aproximar por la normal.

\[\left\{\begin{array}{c} np = 30·\rfrac{1}{3} = 10>5\\n·(1-p) = 30·\rfrac{2}{3} = 20>5\end{array}\right\}\overset{(1)}{\to}\left\{\begin{array}{c}
\mu = np = 10\\
\sigma = \sqrt{np(1-p)} = \sqrt{30·\rfrac{1}{3}\rfrac{2}{3}} =  \rfrac{10}{3}
\end{array}\right.\]
(1): es razonable aproximar por una normal estos parámetros.

$X_n \equiv N(10,\rfrac{10}{3})$

\[P(X_n>10) = P\left(\frac{X_n-10}{\rfrac{10}{3}} > \frac{10-10}{\rfrac{10}{3}}\right) = P(Z>0) = 1-P(Z<0) = 0'5\]
\end{problem}

\end{document}