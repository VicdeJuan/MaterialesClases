\documentclass[palatino,nosec]{Docencia}


\title{Cuaderno de clase}
\author{Víctor de Juan}
\date{17/18}

\begin{abstract}
Cuaderno de clase de Matemáticas I, con el desarrollo continuado (sin estar separado por sesiones).
\end{abstract}

% Paquetes adicionales

\usepackage[author={Víctor de Juan, 2017}]{pdfcomment}

\makeatletter
\newcommand{\annotate}[2][]{%
\pdfstringdef\x@title{#1}%
\edef\r{\string\r}%
\pdfstringdef\x@contents{#2}%
\pdfannot
width 2\baselineskip
height 2\baselineskip
depth 0pt
{
/Subtype /Text
/T (\x@title)
/Contents (\x@contents)
}%
}
\makeatother

% --------------------

\begin{document}
\pagestyle{plain}
\maketitle
\tableofcontents


%% Contenido.

\chapter{Primera evaluación}

\section{Introducción a la asignatura}


Hola, soy Víctor y voy a ser vuestro profesor de matemáticas este año. Algunos me conoceréis, tal vez de campamento o de verme en misa. Para otros tal vez soy totalmente nuevo. ¡Genial! \hl{Soy matemático, entre otras cosas}, y espero poder transmitiros la belleza que encierran las matemáticas.

Reitero que me llamo Víctor y así quiero que me llaméis, ¿bien?

\subsection{Las matemáticas son para siempre}

\href{https://www.ted.com/talks/eduardo_saenz_de_cabezon_math_is_forever}{Vídeo de Eduardo Saenz}

\paragraph{Comentarios al vídeo:}

\begin{itemize}
\item Un teorema es para siempre. ¿Os lo había dicho alguien alguna vez? ¿Qué teoremas conocéis? (Resto, factor, Pitágoras, Tales,  Altura, Euler (V+C=A+2),  ) Pero hay que demostrarlo.
\item Este año está lleno de para siempres. Os voy a enseñar unos buenos teoremas con sus demostraciones a lo largo del curso y vamos a darle importancia a las demostraciones de los contenidos. Cuando la demostración utilice matemáticas que sabéis, las desarrollaremos. 
\subitem \hl{Espero, al final de la clase}, poder enseñaros concretamente a qué me refiero.
\item ¿Alguna pregunta del vídeo? 
\item Weaire-Phelan:
\subitem En la naturaleza, aparece en estructuras químicas (un tipo de cristal).
\subitem \href{https://www.e-architect.co.uk/images/jpgs/beijing/watercube_ptw051208_7.jpg}{El estadio de Pekín de Phelps}
\end{itemize}

\hl{Sed “críticos” en general}. Lo que os cuenten razonadlo. No os lo creáis sin más. Y de esto, nos vamos a hartar en Matemáticas. ¿Alguna pregunta?

\subsubsection{Temario de la asignatura}

El \hl{orden} que vamos a seguir difiere un poco del del libro, porque así nos parece a MariNieves y a mi. 

Vamos a dar un paso más en la resolución de \hl{ecuaciones y sistemas}. 
%
El año pasado resolvíais ecuaciones del tipo: $81^{2x} = \frac{1}{3};\quad x-\sqrt{x-3} = \sqrt{x-2}$, sistemas de ecuaciones 2x2... Este año vamos a ir más allá, vamos a resolver sistemas de 3x3. 

El año pasado tuvisteis una introducción a la geometría. Este año vamos a trabajar con \hl{problemas serios el plano}: distancia punto a recta, distancia entre rectas, cónicas... Por cierto, \ul{¿sabéis porqué se llaman cónicas?}

Y ahora que ya sabéis algo de trigonometría, vamos a ir más allá. Vamos a \hl{exprimir la trigonometría} hasta dominarla. Es algo fundamental en matemáticas. 

Y vamos a dar uno de los temas más bonitos de todo el bachillerato. ¡\hl{Los números complejos}! ¿A alguien le dice algo? ¿A alguien le suena? 
%
Ya veréis que, en realidad, no difíciles. Simplemente arrastramos $\sqrt{-1}$ en las operaciones, como una expresión algebraica, llamando $i=\sqrt{-1}$. 
%
Precioso. 

Y uno de los contenidos centrales del curso, del que hablaréis cuando terminéis el colegio y habréis oído hablar... Las míticas \hl{derivadas}. Os suena, ¿no? 
%
Pues este año nos meteremos en ese berenjenal, a aprender qué es una derivada y para que sirve.
%
También veremos las \hl{Integrales}.
%
Terminaremos el curso con probabilidad y estadística.

Todo esto es una visión general del temario del curso. 
%
Pero hay otro aspecto fundamental de la asignatura. 
%
En otros años lo fundamental puede ser coger habilidad y soltura en los cálculos, saber resolver los problemas... 
%
Este curso va de \hl{razonar y justificar}. Me importa relativamente poco si el resultado está bien o no, en comparación con lo que me importa si el procedimiento está razonado y justificado. 
%
Las matemáticas no son magia, todo tiene una razón de ser. Todo tiene una explicación. 
%
Por ejemplo, ya no nos vamos a centrar tanto en "resuelve esto o lo otro", sino en "demuestra, comprueba, discute esto o lo otro".
%
¿Se entiende?

Por ejemplo, ¿porqué el \hl{teorema de Pitágoras} funciona? 
%
Claro, porque probando con unos números funciona. 
%
¿Quién te asegura que para absolutamente cualquier triángulo rectángulo va a funcionar? Eso hace 4 años te lo creíste. 
%
Este año vamos a demostrar el teorema.

En general, va a haber pocos actos de fe. 
%
En el curso habrá demostraciones y los contenidos no vendrán dados por arte de magia. 
%
Ojo, el teorema fundamental del álgebra no lo podremos demostrar porque hacen falta matemáticas más complejas. 
%
Todo lo demás, lo podremos demostrar.

\subsection{Criterios de evaluación}

Esta asignatura tiene evaluación continua. Eso significa que no hay una recuperación exclusiva para quienes suspendan una evaluación, sino que la primera evaluación es contenido evaluable de la segunda, y la segunda de la tercera.
%
(La primera de la tercera no).

Hay 2 exámenes por evaluación, el primero vale \hl{$\rfrac{1}{3}$} y el segundo \hl{$\rfrac{2}{3}$}. 
%
En ese primer examen, la mitad corresponderá a la evaluación pasada (en caso de que haya evaluación pasada).
%
Si tienes la evaluación suspensa, tienes 5 puntos en el primer examen para recuperar.
%
Si tienes la evaluación aprobada, tienes 5 puntos en el primer examen para repasar.
%
En el segundo examen, en principio, no pondremos contenido de la evaluación anterior.


\section{Números reales}

\subsection{Introducción}

\subsubsection{Conjuntos numéricos}

¿Qué conjuntos de números conocéis? $ℕ,ℤ,ℚ,I,ℝ,ℂ$ \hl{Diagrama de contenidos}.

Cómo distinguir que un número dado pertenece a un conjunto o no.
%
Por ejemplo, $\sqrt{2}$. ¿A qué conjunto pertenece y porqué?
%
\hl{¿Por qué $\sqrt{2}$ es irracional?} $\rfrac{4}{3} = 1.\hat{3}$ también tiene infinitos decimales.
%
En este curso ya no vale creerse lo que los profesores os contamos sin más. Es fundamental entender las demostraciones y ser capaz de razonar el porqué de los asuntos.


\begin{proof}[\textbf{Reducción al absurdo}]
Suponemos $\sqrt{2}\in ℚ$.

Entonces 
\[
	\exists a,b\inℤ \tq \left(\sqrt{2} = \rfrac{a}{b}\right) \wedge \left( \mcd{a,b} = 1 \right)
\]


\[
	\sqrt{2} = \frac{a}{b} \implies 2·b^2 = a^2 \implies a^2 \text{ es par} \overset{1}{\implies} a \text{ es par}
\]

Si $a^2$ es par, necesariamente $a$ es par también. Por reducción al absurdo, si $a$ no fuera par, $a·a$, producto de impares no sería par. 
%
Como $a$ es par, podemos escribirlo como $a=2·k$, para algún $k\in ℤ$. 

\[
	2b^2 = (2k)^2 \implies b^2 = 2k^2 \overset{1}{\implies} b \text{ par}.
\]

Conclusión: $\mcd{a,b} = 2 ≠ 1$. Hemos llegado a una contradicción partiendo de un supuesto ($\sqrt{2}\in ℚ$) "inventado", por lo que ese supuesto debe ser falso. 
\end{proof}

\paragraph{\hl{No os creáis cualquier demostración:}}
Partiendo de $x=y$

\[
x^2 = xy \dimplies x^2-y^2 = xy-y^2 \dimplies (x+y)(x-y) = y(x-y) \dimplies x+y = y \dimplies 2y = y \implies \text{\hl{2=1}}
\]

\nota{Si sobrara tiempo: belleza intrínseca de las matemáticas: ¿Qué hay más, enteros o pares?}


\todo{Sesión 1}

\subsubsection{Grupo y Cuerpo}

\begin{defn}[Grupo]
Sea un conjunto $G\neq \emptyset$ y $\ast$ una operación (ley de composición interna), diremos que $(G,\ast)$ es un grupo si cumple las siguientes propiedades:
\begin{enumerate}
	\item \textbf{Cerrado por la operación}. $\forall x, y \in G, \; x \ast y \in G$
	\item \textbf{Asociatividad}. $\forall x, y, z \in G, \; (x \ast y) \ast z = x \ast (y \ast z)$
	\item \textbf{Existencia del elemento neutro}. $\exists  e \in G \tq ,\; \forall\, x\in G\; x\ast e=x$.
	\item \textbf{Existencia del inverso}. $\forall x \in G ,\; \exists x' \in G \tq x\ast x'=x'\ast x=e$.
\end{enumerate}
\end{defn}

\begin{prop}[Unicidad\IS del neutro e inverso]
  En todo grupo se cumplen las propiedades de unicidad del elemento neutro y del inverso.
\end{prop}

\begin{proof}[\textbf{Reducción al absurdo}]
Primero demostramos que el neutro es único. \\
Sean $e$, y $e'$ dos elementos neutros de $G$, se cumple que $e\ast e'\overset{1}{=}e'$, pero también se cumple que $e'\ast e\overset{2}{=}e$. Esto implica que $e'=e$.\\
1: $e$ elemento neutro.\\
2: $e'$ elemento neutro.

Por otro lado, si suponemos la existencia de dos elementos inversos $a',a''\in G$, entonces $e=a\ast a'=a\ast a''$. \\

\[
	aa' = aa'' \dimplies a'(aa') = a'(aa'') \overset{asoc.}{\dimplies} (a'a)a'=(a'a)a'' \overset{e.n.}{\dimplies} a'=a'' 
\]
\end{proof}

\begin{example}
	\begin{itemize}
		\item Sí: $(ℤ,+),(ℚ,+),(ℝ,+),(\{0,1\},+)$
		\item No: $(ℕ,+),(ℤ,·),(?,·)$
		\item No: $(I,+)$ porque $(1+\sqrt{2}) - (\sqrt{2}\in ℤ)$
	\end{itemize}
\end{example}

\nota{Si el grupo es \hl{conmutativo}, se llama grupo conmutativo o abeliano por el matemático noruego Abel del siglo XIX.}

El \hl{producto tiene problemas} para ser ley de composición interna de grupos. 
%
Sin embargo, podemos considerar $(ℚ,+,·)$. 
%
Ahora hay 2 operaciones y por lo tanto no podemos considerarlo grupo.


\begin{defn}[Cuerpo] 
Sea $K$ un conjunto y sean $(+)$ y $(·)$ dos operaciones binarias cerradas, llamadas suma y producto. 
%
$(K,+,·)$ es un cuerpo si se cumple:
\begin{enumerate}
\item $(A, +)$ es un grupo abeliano.
\item El producto es asociativo.
\item Se cumplen las leyes distributivas: 
	\subitem $\forall a,b,c \in A \; a\cdot(b+c)=a\cdot b + a\cdot c$ 
	\subitem $(a+b)\cdot c= a \cdot c + b \cdot c$
\item El producto es conmutativo.
\item Existe elemento neutro para el producto.
\item $\left(A\backslash \{0\},·\right)$ es un grupo. (Siendo 0 el elemento neutro de la suma.)
\end{enumerate}
\end{defn}

\nota{Si sólo se cumplen a,b,c se llama anillo.}

\begin{example}
	\begin{itemize}
		\item Sí: $(ℝ,+,·),(ℚ,+,·),\left[(ℂ,+,·)\right]$
		\item No: $(ℤ,+,·),\left[(\mathcal{M},+,·)\right]$
		\item $\left(I,+,·\right)$ no, porque $(I,+)$ no es un grupo.
	\end{itemize}
\end{example}


\subsubsection{Entornos (libro):} Dado un intervalo $(3,7)$, ¿cuál es su centro? ¿cuál es su amplitud? (Calcular). A veces nos interesará escribir el mismo intervalo de puntos haciendo énfasis en el centro y el radio. 
%
Entonces, estaremos escribiendo \textit{entornos} en lugar de intervalos. 
%
Y como los intervalos, hay varios tipos de entornos.

\begin{defn}[Entorno\IS abierto]
Llamamos entorno abierto de centro $a$ y radio $r$, y se denota por $E_r(a)$, al conjunto de puntos que distan de $a$ a una distancia menor que $r$:

\[
	E_r(a) = \{x\in\real, |x-a| < r\} = (a-r, a+r) 
\]
\end{defn}

\begin{defn}[Entorno\IS cerrado]
Llamamos entorno cerrado de centro $a$ y radio $r$, y se denota por $E_r(a)$, al conjunto de puntos que distan de $a$ a una distancia menor o igual que $r$::
\[
	E_r[a]  = \{x\in\real, |x-a| \leq r\} = [a-r, a+r]
\]
\end{defn}

\begin{defn}[Entorno\IS reducido]
Llamamos entorno reducido de centro $a$ y radio $r$, y se denota por $E_r(a)$ a:
\[
	E^{\ast}_r(a) = \{x\in\real, |x-a| < r\} - \{a\} = (a-r,a) \cup (a, a+r)
\]
\end{defn}

\nota{El libro utiliza otra notación que no nos gusta. Apuntad la buena.}

\hl{Seguimos pasando páginas del libro y comentando}

\subsection{Logaritmos}

\paragraph{\annotate{En la sesión vimos el vídeo.}{Introducción}}

Vamos a aprender una nueva manera de multiplicar. En realidad ya sabéis, aunque no seáis conscientes.\footnote{Fuente: \href{https://www.youtube.com/watch?v=FB3\_BeukBBk\&t=99s}{Mark Foskey, youtube.com}}





\begin{itemize}
\item Caso 1: $1000000·10000000 = 10^6·10^7 = 10^{13}$. ¿Y podremos hacer esto con otros números que no sean el 10?

\item Caso 2: $64·128 = 2^6·2^7 = 2^{13} = 8192$
\item ¿Caso 3?: Me construyo la tabla del 3.
\begin{center}
\begin{tabular}{cccccccccccc}
1& 3& 9& 27& 81& 243& 729& 2187& 6561& 19683& 59049& 177147\\
\textcolor{red}{0} & \textcolor{red}{1} & \textcolor{red}{2} & \textcolor{red}{3} & \textcolor{red}{4} & \textcolor{red}{5} & \textcolor{red}{6} & \textcolor{red}{7} & \textcolor{red}{8} & \textcolor{red}{9} & \textcolor{red}{10} & \textcolor{red}{11}
\end{tabular}
\end{center}
\item Caso 4: ¿Y para números que no son potencias enteras? Por ejemplo, $64*40$. Pues si $32=2^5$ y $64=2^6$, $40=2^{5,...}$ ¿Tiene sentido?
\item Caso 5: Lo que hicieron, Yost y Napier, fue coger la tabla del 1,0001 en lugar de la tabla del 3 y dividir por mil los números rojos, dando lugar a la tabla de logaritmos:

\begin{center}
	\begin{tabular}{cl}
		0.0 & 1.0\\
		0.001 & 1.001\\
		0.002 & 1.002\\
		0.003 & 1.003\\
		0.004 & 1.00401\\
		0.005 & 1.00501\\
		0.006 & 1.00602\\
		0.007 & 1.00702\\
		0.008 & 1.00803\\
		0.009 & 1.00904\\
		$\vdots$ & \quad\quad$\vdots$\\
		0.991 & 2.69259\\
		0.992 & 2.69529\\
		0.993 & 2.69798\\
		0.994 & 2.70068\\
		0.995 & 2.70338\\
		0.996 & 2.70608\\
		0.997 & 2.70879\\
		0.998 & 2.7115\\
		0.999 & 2.71421\\
		1.0 & \hl{2.71692} (Una aproximación de $e$)\\
	\end{tabular}
\end{center}

\end{itemize}

\begin{defn}[Logaritmo]
Sean $a\in ℝ>0,a≠1$ y $N\in\real$.

Se llama logaritmo en base $a$ de $N$ al exponente $x$ que cumple: $a^x = N$ y se escribe:
\[
	\log_aN=x\dimplies a^x=N
\]
\end{defn}

\nota{De la propia definición se entiende:}
\begin{enumerate}
	\item $y\in\real, y<0 \implies ∀a,\nexists\log_a y$
	\item $\log_a 1 = 0 \dimplies a^0 = 1$
	\item $\log_a a = 1 \dimplies a^1 = a$
	\item $\log_a a^q = q \dimplies a^q = a^q$ por definición.
	\item $a^{\log_a N} = N$
\todo{Sesión 2}
	\begin{proof}
		Sea $b = \log_a N \dimplies a^b = N$.

		Entonces, $a^{\log_a N} = N \dimplies a^n = N$
	\end{proof}
\end{enumerate}



\paragraph{Propiedades:}
\begin{itemize}
	\item $\log_a(AB) = \log_a(A) + \log_a(B)$
		\begin{proof}
			\[\left.\begin{array}{c}
				\log_a(A) = p \dimplies a^p = A\\
				\log_a(B) = p \dimplies a^q = B
			\end{array}\right\}\implies A·B = a^{p+q} \implies \log_a(AB) = \log_aa^{p+q}\]
			\[
				\log_a(AB) = \log_aa^{p+q} \overset{?}{\implies} \log_a(AB) = p+q \overset{?}{=} \log_aA + \log_aB
			\]
		\end{proof}
	\item $\log_a\left(\rfrac{A}{B}\right) = \log_a(A) - \log_a(B)$
	\begin{proof}
		Análoga. Para completar por vosotros. Si no lo conseguís, intentadlo.
	\end{proof}
	\item $\log_a(A)^n = n·\log_a(A)$ \hl{\textbf{Ojo:} $\left[\log_a(A+B)\right]^n \neq n·\log_a(A)$}
	\begin{proof}
		Tomamos $\log_aA = p \dimplies a^p = A$

		\[
			a^p = A \dimplies a^{np}=A^n \dimplies \log_aa^{np} = \log_aA^n \dimplies np = \log_aA^n
		\]
		\[ 
			\dimplies n\log_aA = \log_aA^n
		\]
	\end{proof}
	\item $\displaystyle\log_aA = \frac{\log_bA}{\log_ba}$ \textbf{(Cambio de base})
		\begin{proof}
			Tomamos $\log_aA = p \dimplies a^p = A$
			\[
				\log_ba^p = \log_bA \dimplies p\log_ba = \log_bA \dimplies p=\log_aA=\frac{\log_bA}{\log_ba}
			\]

		\end{proof}
\end{itemize}

\paragraph{Tomar logaritmos:}

\[
A = B \overset{1}{\dimplies} a^{\log_a A} = a^{\log_a B} \dimplies \log_aA=\log_aB
\]

\subsubsection{Ejemplos con logaritmos}

\begin{itemize}
	\item $\log \sqrt{0,0001}$
	\item Demuestra $\log_a b · \log_b a = 1$
\end{itemize}

\subsubsection{Ejercicios con logaritmos}
Pág 21, ejers 50,52.

Pág 21, ejers 53.

\subsubsection{Aplicación de los logaritmos}

\paragraph{pH}

Para medir el nivel de acidez, pH, que mide la concentración de iones hidronio ($H_3O^+$)

\[pH=-\log[H_30^+]\]

\paragraph*{Despejar el tiempo} En el crecimiento exponencial de bacterias:
$P = p_0·k^t$ donde $p_0$ es la cantidad inicial, $t$ el tiempo y $k$ la tasa de reproducción.

Pag 23, ejer 57.

\paragraph{Interés compuesto}

Sea $C$ el capital inicial. Tras 1 periodo de tiempo, obtengo el r\% de intereses. Entonces tendré: $C_1 = C_0 + \rfrac{r}{100}C_0 = C_0·\left(1+\frac{r}{100}\right)$

Pasado otro periodo de tiempo tendré $C_2 = C_1·\left(1+\frac{r}{100}\right) = C_0·\left(1+\frac{r}{100}\right)·\left(1+\frac{r}{100}\right) = C_0 ·\left(1+\frac{r}{100}\right)^2$

En general, para $t$ periodos de tiempo tendremos: $C_n = C_0·\left(1+\frac{r}{100}\right)^t$.

La fórmula del interés compuesto capitalizado en $p$ periodos, \hl{para un interés nominal $r$ es}: 

\[C = C_0·\left(1+\frac{r}{100p}\right)^{pt}\]


\paragraph{3 preguntas}
\begin{itemize}
	\ite[a)] Pero... ¿es lo mismo un interés anual del 5\% capitalizado cada mes, que un interés anual del 5\% capitalizado cada año?

	\item[b)] ¿Y si capitalizo cada semana? ¿Y cada día? ¿Y si capitalizo \textit{instantáneamente}? 

	\item[c)] ¿Cuánto tiempo tengo que dejar el dinero si quiero ganar una cantidad concreta?
\end{itemize}

\paragraph{a)}
Pero, ¿es lo mismo un interés mensual del 5\% durante 12 meses que un interés anual del 5\% durante 12 meses? Sí. ¿A quién le importa que sean meses o años?

Pero... ¿es lo mismo un interés anual del 5\% capitalizado cada mes, que un interés anual del 5\% capitalizado cada año? ¿Hay alguno que sea mayor? 

\begin{example}

Tengo $C= 1000\texteuro$ a un $r=5\%$. Tengo 2 opciones para elegir y no se cual es mejor. 

\begin{itemize}
	\item[a)] Un interés anual del 5\% capitalizado cada mes durante 5 años.

	\[
		C = C_0·(1+\rfrac{r}{100p})^{t·p} = 1000·(1+\rfrac{5}{1200})^{12*5} = 1283.36
	\]
	\item[b)] Un interés anual del 5\% capitalizado cada año durante 5 años.

	\[
		C = C_0·(1+\rfrac{r}{100p})^{t·p} = 1000·(1+\rfrac{5}{100})^{5} = 1276.28
	\]

	Es más rentable la opción a. 

	\nota{\hl{De hecho}, ¿cuál es el interés equivalente que hay que aplicar al año en el caso a)?}

	\[
		C = C_0·(1+\rfrac{r}{100})^{t} \dimplies \frac{C}{C_0} = (1+\rfrac{r}{100})^t \implies \sqrt[t]{\frac{C}{C_0}} = \sqrt[t]{(1+\rfrac{r}{100})^t} \dimplies
	\]
	\[
 		\sqrt[t]{\frac{C}{C_0}} -1 = \rfrac{r}{100} \dimplies r = 100·\left(\sqrt[t]{\frac{C}{C_0}} -1 \right)\implies r = 100·\left(\sqrt[5]{\frac{1283.36}{1000}} - 1\right) = 5.1162...
	\]

	Conclusión, un 5\% mensual es lo mismo que un 5.1162...\% anual. Por esto, y para que no nos puedan timar los bancos existe la \concept{TAE} (Tasa Anual Equivalente.)


\end{itemize}
\end{example}

\paragraph{b)}
¿Y si capitalizo cada semana? ¿Y cada día? ¿Y si capitalizo \textit{instantáneamente}? 

\paragraph{c)}
Tengo entendido que durante el mes de julio, Miguel ha trabajado de hamaquero en la playa y cobrado 150\texteuro al mes. ¿Puede ser? Su banco le ofrece un depósito con un interés del 10\% y Miguel querría pagarse su carrera, que será algo así como 10000\texteuro, sino sigue subiendo.
%
¿Cuánto tiempo tiene que dejar su dinero en el banco para pagarse la carrera?

\[
	C = C_0·(1+\rfrac{r}{100})^{t} \dimplies \log\frac{C}{C_0} = t\log(1+\rfrac{r}{100}) \dimplies t = \frac{\log\frac{C}{C_0}}{\log(1+\rfrac{r}{100})} = \frac{\log\frac{10000}{150}}{\log\left(1+\rfrac{10}{100}\right)} 
\]

\[
	t = 44.06\text{ años}
\]

De momento lo dejamos aquí. El próximo tema que tiene ecuaciones volveremos a trabajar con el interés, las bacterias, etc.

\section{Álgebra}

1 - Binomio de Newton

2 - Teorema de raíces enteras y fraccionarias.

3 - Ecuaciones racionales, una con fracciones algebraicas complicadas.

4 - Ecuaciones con radicales -> cuidado con las doble o simple implicaciones. (Demostración de 4=5). 
%
Despejar el interés en problemas de interés compuesto.
%
Ecuaciones logarítmicas y exponenciales. Sistemas
%
Despejar el tiempo en problemas de interés compuesto.


5 - Gauss. Transformaciones elementales, sistemas equivalentes, etc.

\printindex
\end{document}
\grid
