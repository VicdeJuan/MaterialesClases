\documentclass[palatino,nosec]{Docencia}


\title{Ejemplos resueltos del método de Gauss}
\author{}
\date{17/18}



% Paquetes adicionales

\usepackage[author={Víctor de Juan, 2017}]{pdfcomment}

\makeatletter
\newcommand{\annotate}[2][]{%
\pdfstringdef\x@title{#1}%
\edef\r{\string\r}%
\pdfstringdef\x@contents{#2}%
\pdfannot
width 2\baselineskip
height 2\baselineskip
depth 0pt
{
/Subtype /Text
/T (\x@title)
/Contents (\x@contents)
}%
}
\makeatother

% --------------------
\newcommand{\cimplies}{\text{\hl{$\implies$}}}

\begin{document}
\pagestyle{plain}
\newpage
\begin{problem}
Resuelve el siguiente sistema de ecuaciones:

\[
\left\{\begin{array}{lccccc}
e_1: &x&+3y&-2z&=&6\\
e_2: &2x&+3y&-2z&=&8\\
e_3: &4x&+2y&-6z&=&6
\end{array}\right\}
\]

\solution

Aplicamos el método de Gauss. Lo primero, reordenar o simplificar ecuaciones. Vemos que las incógnitas $y$ y $z$ van a ser más fáciles de eliminar, al menos en las 2 primeras ecuaciones. Además, la tercera se puede simplificar.


\[
\left\{\begin{array}{lccccc}
e_1: 3y&-2z&+&x&=&6\\
e_2: 3y&-2z&+&2x&=&8\\
e_3: y&-3z&+&2x&=&3
\end{array}\right\}
\]


\paragraph{1)}
$e_2 = e_2-e_1$

\[
\begin{array}{lccccc}
e_2: 3y&-2z&+&2x&=&8\\
e_1: 3y&-2z&+&x&=&6\\
\hline
e_2: &&&x&=&2\\
\end{array}
\]	

El nuevo sistema equivalente es:

\[
\left\{\begin{array}{lccccc}
e_1: 3y&-2z&+&x&=&6\\
e_2: &&&x&=&2\\
e_3: y&-3z&+&2x&=&3
\end{array}\right\}
\]


\paragraph{2)} Reordenamos

\[
\left\{\begin{array}{lccccc}
e_1: 3y&-2z&+&x&=&6\\
e_2: y&-3z&+&2x&=&3\\
e_3: &&&x&=&2
\end{array}\right\}
\]

Aplicamos otra transformación de Gauss para eliminar la $y$ de la segunda ecuación. Para ello, $e_2 = e_1-3e_2$

\[
\begin{array}{lccccc}
e_1: 3y&-2z&+&x&=&6\\
e_2: -3y&+9z&-&6x&=&-9\\
\hline
e_2: &+7z&-&5x&=&-3\\
\end{array}
\]	

El nuevo sistema equivalente es:

\[
\left\{\begin{array}{lccccc}
e_1: 3y&-2z&+&x&=&6\\
e_2: &+7z&-&5x&=&-3\\
e_3: &&&x&=&2\\
\end{array}\right\}
\]

\paragraph{3)}

Ahora que ya lo tenemos escalonado, sólo hay que discutir y resolver. 

\subparagraph*{Discusión:} El sistema es compatible determinado, porque es un sistema escalonado con ecuaciones compatibles.

\[
\left\{\begin{array}{lccccc}
e_1: 3y&-2z&+&x&=&6\\
e_2: &+7z&-&5x&=&-3\\
e_3: &&&x&=&2\\
\end{array}\right\}
\]


\paragraph{4)} Resolvemos:

\[e_3: x=2\]

Utilizando $x=2$ en $e_2$:


\[e_2: 7z-5x = -3 \dimplies 7z = -3+5x = -3+5·2 \dimplies 7z=7 \dimplies z=1\]

Utilizando $z=1$, $x=2$ en la ecuación que nos falta ($e_1$):

\[
e_1: 3y-2z+x=6 \dimplies y=\frac{6+2z-x}{3} = \frac{6+2-2}{3} = 2
\]


Solución: $(x,y,z) = (2,2,1)$.

\paragraph*{Comprobación}

Sustituimos los valores obtenidos en el sistema inicial:


\[
\left\{\begin{array}{lccccl}
e_1: &x&+3y&-2z&=&6 \to 2+2·3-2 = 6\\
e_2: &2x&+3y&-2z&=&8 \to 2·2 + 2·3-2·1 = 4+6-2 = 8\\
e_3: &4x&+2y&-6z&=&6 \to 4·2+2·2-6·1 = 8+4-6 = 6  
\end{array}\right\} \text{cqc}
\]
\end{problem}







\newpage
\begin{problem}
Resuelve el siguiente sistema de ecuaciones:

\[
\left\{\begin{array}{lccccc}
e_1: &x&+y&+z&=&0\\
e_2: &x&+y&-z&=&2\\
e_3: &2x&+3y&+4z&=&-2
\end{array}\right\}
\]

\solution

Aplicamos el método de Gauss. 

\paragraph{1)}
$e_2 = e_2-e_1$

\[
\begin{array}{lccccc}
e_2: &x&+y&-z&=&2\\
e_1: &x&+y&+z&=&0\\
\hline
e_2: &&&-2z&=&2\\
\end{array}
\]	

El nuevo sistema equivalente es:

\[
\left\{\begin{array}{lccccc}
e_1: &x&+y&+z&=&0\\
e_2: &&&-2z&=&2\\
e_3: &2x&+3y&+4z&=&-2
\end{array}\right\}
\]


\paragraph{2)}


$e_3 = e_3-2*e_1$

\[
\begin{array}{lccccc}
e_3: &2x&+3y&+4z&=&-2\\
2·e_1: &2·x&+2·y&+2·z&=&\textcolor{red}{2}·0\\
\hline
e_3: &&y&+2z&=&-2\\
\end{array}
\]	

El nuevo sistema equivalente es:

\[
\left\{\begin{array}{lccccc}
e_1: &x&+y&+z&=&0\\
e_2: &&&-2z&=&2\\
e_3: &&y&+2z&=&-2\\
\end{array}\right\}
\]

\paragraph{3)}

En este caso ya lo tenemos escalonado, sólo hay que reordenar:

\[
\left\{\begin{array}{lccccc}
e_1: &x&+y&+z&=&0\\
e_3: &&y&+2z&=&-2\\
e_2: &&&-2z&=&2\\
\end{array}\right\}
\]


\paragraph{4)} Resolvemos:

\[e_2: -2z=2 \to z=-1\]


Utilizando $z=-1$ en $e_3$:
\[e_3: y+2z=-2 \to y+2·(-1) = -2 \to y-2=-2 \to y=0\]

Utilizando $z=-1$, $y=0$ en la ecuación que nos falta ($e_1$):

\[
e_1: x+y+z=0 \to x+0-1=0 \to x=1
\]


Solución: $(x,y,z) = (1,0,-1)$.

\paragraph*{Comprobación}

Sustituimos los valores obtenidos en el sistema inicial:

\[
\left\{\begin{array}{lccccl}
e_1: &x&+y&+z&=&0 \to 1+0-1=0 \to 0=0\\
e_2: &x&+y&-z&=&2 \to 1+0-(-1) = 2 \to 2=2 \\
e_3: &2x&+3y&+4z&=&-2 \to 2·(1) + 3·(0) + 4·(-1) = -2 \to  2-4=-2 \to -2=-2
\end{array}\right\}
\]

\end{problem}





\newpage
\begin{problem}
Resuelve el siguiente sistema de ecuaciones:

\[
\left\{\begin{array}{lllll}
e_1: & 3x &+ 2y &+z &=10\\
e_2: & -x&-y&+z&=0\\
e_3: & 2x&+y&+3z&=13
\end{array}\right\}
\]

\solution

Aplicamos el método de Gauss. Lo primero, reordenar o simplificar ecuaciones. Vemos que la incógnita $z$ va a ser más fácil de eliminar, al menos en las 2 primeras ecuaciones. 

\[
\left\{\begin{array}{lcccc}
e_1: & z &+ 2y &+3x &=10\\
e_2: & z&-y&-x&=0\\
e_3: & 3z&+y&+2x&=13
\end{array}\right\} 
\]

\paragraph{1)}
$e_2 = e_2-e_1$

\[
\begin{array}{lcccl}
e_2: & z &+ 2y &+3x &=10\\
e_1: & z&-y&-x&=0\\
\hline
e_2: & &3y+&4x&=10\\
\end{array}
\]	

El nuevo sistema equivalente es:

\[
\left\{\begin{array}{lccccc}
e_1: & z &+ 2y &+3x &=10\\
e_2: & &3y&+4x&=10\\
e_3: &3z&+y&+2x&=13
\end{array}\right\}
\]


\paragraph{2)} 

Aplicamos otra transformación de Gauss para eliminar la $z$ de la tercera ecuación. Para ello, $e_3 = e_3-3e_1$

\[
\begin{array}{lcccl}
e_3: & 3z&+y  &+2x&=13\\
e_1: & 3z&+6y &+9x&=30\\
\hline
e_3: &   &-5y &-7x&=-17\\
\end{array}
\]	

El nuevo sistema equivalente es:

\[
\left\{\begin{array}{lcccl}
e_1: & z&+ 2y&+3x &=10\\
e_2: & &3y&+4x&=10\\
e_3: & &-5y&-7x&=-17\\
\end{array}\right\}
\]


\paragraph{3)} Aplicamos una transformación de Gauss para eliminar la $y$ de la tercera ecuación. 

Podemos hacer 2 transformaciones: $e_3 = e_2+\frac{3}{5}e_1$, que permitiría eliminar la $y$ de la tercera ecuación, aunque aparecerían fracciones.

La otra posibilidad, que es la que vamos a realizar, sería $e_3=5·e_2+3·e_3$. 

\[
\begin{array}{lcccl}
5e_2: & &15y&+20x&=50\\
3e_3: & &-15y&-21x&=-51\\
\hline
e_3: &&&-x&=-1\\
\end{array}
\]	

\paragraph{4)}

Ahora que ya lo tenemos escalonado, sólo hay que discutir y resolver. 

\[
\left\{\begin{array}{lccccc}
e_1: & z&+ 2y&+3x &=10\\
e_2: & &3y&+4x&=10\\
e_3: &&&-x&=&-1\\
\end{array}\right\}
\]

\subparagraph*{Discusión:} El sistema es compatible determinado, porque es un sistema escalonado con ecuaciones compatibles.


\subparagraph*{Resolución:}

\[e_3: -x=-1 \dimplies x=1\]

Utilizando $x=1$ en $e_2$:


\[
	e_2: 3y+4x=10 \dimplies y=\frac{10-4x}{3} = \frac{10-4}{3} = 2
\]

Utilizando $y=2$, $x=1$ en la ecuación que nos falta ($e_1$):

\[
e_1: z+2y+3x = 10 \dimplies z = 10-3x-2y = 10-3·1-2·2 = 10-7 = 3
\]


Solución: $(x,y,z) = (1,2,3)$.

\paragraph*{Comprobación}

Sustituimos los valores obtenidos en el sistema inicial:


\[
\left\{\begin{array}{lccccl}
e_1: & 3x &+ 2y &+z &=10 \to 3·1+2·2+3 = 3+4+3 = 10  \\
e_2: & -x&-y&+z&=0 \to -1-2+3 = 0  \\
e_3: & 2x&+y&+3z&=13 \to 2·1+2+3·3 = 2+2+9 = 13  \\
\end{array}\right\} \text{cqc}
\]
\end{problem}







\end{document}