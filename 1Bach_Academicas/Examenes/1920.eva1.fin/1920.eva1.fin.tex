\documentclass[palatino,nosec,nochap,nobuilddate]{Docencia}
\usetikzlibrary[patterns]


\title{Corrección 2º parcial - 1ª evaluación}
\author{Departamento de Matemáticas}
\date{19/20}


% Paquetes adicionales

\usepackage[author={Víctor de Juan, 2017}]{pdfcomment}

\makeatletter
\newcommand{\annotate}[2][]{%
\pdfstringdef\x@title{#1}%
\edef\r{\string\r}%
\pdfstringdef\x@contents{#2}%
\pdfannot
width 2\baselineskip
height 2\baselineskip
depth 0pt
{
/Subtype /Text
/T (\x@title)
/Contents (\x@contents)
}%
}
\makeatother



\usepackage{eso-pic}
\newcommand\BackgroundPic{%
\put(0,0){%
\parbox[b][\paperheight]{\paperwidth}{%
\vfill
\centering
%\includegraphics[width=\paperwidth,height=\paperheight,%
%keepaspectratio]{../../../../BWLogo.jpeg}%
\vfill
}}}





\begin{abstract}
Corrección del 2º examen de la 1ª evaluación de Matemáticas I, curso 2019-2020.

\nota{No está exento de erratas. En caso de descubrir alguna, por favor, comunicarlas al autor.}
\end{abstract}

% --------------------
\newcommand{\cimplies}{\text{\hl{$\implies$}}}
\renewcommand{\vec}[1]{\overrightarrow{#1}}

\begin{document}
\pagestyle{plain}
\maketitle

\AddToShipoutPicture{\BackgroundPic}

\newpage
\begin{problem}
Resuelve el siguiente sistema de ecuaciones:
\[
	\left\{
		\begin{array}{l}
			4^x-3^{y-1}=-2\\
			2^{x+1}-3^y=-7
		\end{array}
	\right\}
\]
\solution

\[
	\left\{
		\begin{array}{l}
			4^x-3^{y-1}=-2\\
			2^{x+1}-3^y=-7
		\end{array}
	\right\} \dimplies 
	\left\{
		\begin{array}{l}
			\left(2^x\right)^2-\frac{3^{y}}{3}=-2\\
			2·2^x-3^y=-7
		\end{array}
	\right\} 
\]

Hacemos el cambio de variable: $p=2^x$ y $q=3^y$, quedando el sistema:


\[
	\left\{
		\begin{array}{l}
			p^2-\frac{q}{3}=-2\\
			2·p-q=-7
		\end{array}
	\right\} \dimplies 
	\left\{
		\begin{array}{l}
			p^2-\frac{q}{3}=-2\\
			2·p+7=q
		\end{array}
	\right\} \implies
	p^2-\frac{2p+7}{3}=-2 \dimplies 3p^2-2p-7+6=0 
\]

Resolvemos: $3p^2-2p-1=0 \dimplies p_1 = 1 \;\; p_2 = \rfrac{-1}{3}$. 

Para $p_1 = 1 \implies q_1 = 2p_1+7 = 2+7 = 9$

Para $p_2 = -\rfrac{1}{3} \implies q_2 = 2p_2+7 = -\rfrac{2}{3}+7 = \rfrac{19}{3}$

Deshacemos el cambio de variable:

\[
	\left.
	\begin{array}{l}
		\left\{
			\begin{array}{l}
				p_1 = 1\\
				p = 2^x
			\end{array}
		\right\} \implies 2^x = 1 \dimplies x_1 = 0\\
		\left\{
			\begin{array}{l}
				q_1 = 9\\
				q = 3^y
			\end{array}
		\right\} \implies 3^y = 9 \dimplies y = 2 
	\end{array}\right\} \implies (x,y) = (0,2)
\]

\[
	\left\{
			\begin{array}{l}
				p_2 = \rfrac{-1}{3}\\
				p = 2^x
			\end{array}
	\right\} \implies 2^x = \rfrac{-1}{3} \implies \nexists x_2 \implies \nexists y_2 
\]

\paragraph{Comprobación} de la única solución válida:

\[
	\left\{
		\begin{array}{l}
			4^x-3^{y-1}=-2\\
			2^{x+1}-3^y=-7
		\end{array}
	\right\} \overset{(x,y)=(0,2)}{\implies} 
	\left\{
		\begin{array}{l}
			4^0-3^{2-1}= 1-3 = -2\\
			2^{0+1}-3^2= 2-9 = -7 
		\end{array}
	\right\} \implies\text{ Sí es solución.}
\]

\end{problem}

\begin{problem}
El inventor del ajedrez pidió como pago que se llenase cada escaque (cuadrito del tablero) con el doble de trigo que el escaque anterior. Si se comienza con 1 grano de trigo. ¿Cuántos granos habrá que poner en el último cuadrito? ¿En qué escaque habrá que colocar 4194304 granos de trigo?
\solution

Llamando $n$ al número de la casilla, $n\in\{1,2,3,...,63,64\}$.
%
El número de granos de trigo de una casilla viene determinado por $g(n) = 2^{n-1}$.
%
El último escaque es $n=64$, por lo que tendrá $g(64) = 2^{64-1}$ granos de trigo.

El escaque en el que habrá que colocar $4194304$ de trigo será: 

\[
	2^{n-1} = 4194304 \dimplies n=\log_2 (4194304) + 1 = 22
\]

\end{problem}

\begin{problem}
Calcula el área de un rectángulo cuyo perímetro mide 98m y su diagonal 41m.

\solution

Llamando $x,y$ a los lados del rectángulo, tendremos:

\[
\left\{
	\begin{array}{l}
		2x+2y = 98\\
		x^2+y^2 = 41^2 \text{(Pitágoras)}
	\end{array}
\right\} \dimplies 
\left\{
	\begin{array}{c}
		y = 49-x\\
		x^2+y^2 = 41^2 
	\end{array}
\right\} \implies x^2 + (49-x)^2  = 41^2
\]

Resolviendo la ecuación, obtenemos 2 resultados: $x_1 = 9\; x_2=40$

\[
	\begin{array}{c}
		x_1 = 9 \implies y=49-x = 49-9 = 40\\
		x_2 = 40 \implies y=49-x = 49-40 = 9
	\end{array}
\]

\paragraph{Conclusión: } El lado menor medirá $9m$ y el lado mayor $40m$.

\end{problem}

\begin{problem}
Dado el siguiente sistema lineal:
\[
	\left\{
	\begin{array}{ccccl}
		2x&-y&+2z &= &5\\
		ax&+y&-z& = &1\\
		5x&-y&+3z& = &11
	\end{array}
	\right\}
\]

\ppart Disctue el siguiente sistema lineal según los valores del parámetro $a$.
\ppart Resuelve para $a=1$

\solution


Para $a\neq 1$ el sistema es Compatible Determinado.

Para $a=1$ el sistema es Compatible Ineterminado. Las soluciones, según se parametrice una incógnita u otra son:
\begin{itemize}
	\item $(x,y,z) = (\lambda,-4\lambda+7,-3\lambda+6)$,  con $\lambda\in\real$
	\item $(x,y,z) = \left(-\frac{1}{4}\lambda +\frac{7}{4},\lambda, \frac{3}{4}\lambda + \frac{3}{4}\right)$, con $\lambda\in\real$
	\item $(x,y,z) = \left(-\frac{1}{3}\lambda + 2, \frac{4}{3}\lambda - 1 ,\lambda\right)$, con $\lambda\in\real$
\end{itemize}



\end{problem}

\begin{problem}
Resuelve el siguiente sistema de inecuaciones:

\[
	\left\{
		\begin{array}{c}
			x\geq 0\\
			-2x^2+4x-2<0\\
			\frac{3}{x+2}\leq 1
		\end{array}
	\right\}
\]

\solution

Resolvemos cada inecuación por separado:

$$
-2x^2+4x-2<0 \dimplies -2(x-1)^2 < 0 \dimplies x\in \real-\{1\}
$$

$$
	\frac{3}{x+2}\leq 1 \dimplies \frac{3}{x+2}- 1 \leq 0 \dimplies \frac{3-x-2}{x+2}\leq 0 \dimplies \frac{1-x}{x+2}\leq 0
$$

Las raíces del numerados y del denominador son: $-2,1$. Así,
\begin{itemize} 
	\item $\left( -\infty , -2 \right)$: Tomamos, por ejemplo, $x=-4$, obteniendo $-2.5$  y miramos su signo: Negativo 
	\item $\left( -2 , 1 \right)$: Tomamos, por ejemplo, $x=0$, obteniendo $0.5$  y miramos su signo: Positivo 
	\item $\left( 1 , +\infty \right)$: Tomamos, por ejemplo, $x=3$, obteniendo $-0.4$  y miramos su signo: Negativo
\end{itemize}

$$\frac{3}{x+2}\leq 1 \dimplies (-\infty,-2) \cup [1,\infty] $$

\obs El $-2$ se excluye porque anula un denominador.


\paragraph{Solución: }

\[
	\left\{
		\begin{array}{c}
			x\geq 0\\
			-2x^2+4x-2<0\\
			\frac{3}{x+2}\leq 1
		\end{array} 
	\right\}
	\dimplies
	\left\{
		\begin{array}{c}
			[0,\infty)\\
			(-\infty,1) \cup (1,\infty)\\
			(-\infty,-2) \cup [1,\infty] 
		\end{array} 
	\right\}
	\implies \boxed{x\in(1,\infty]}
\]

\end{problem}

\end{document}