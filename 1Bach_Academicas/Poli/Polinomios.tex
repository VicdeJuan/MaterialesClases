\documentclass[palatino,nosec]{Docencia}


\title{Ejercicios de polinomios}
\author{Víctor de Juan}
\date{17/18}



% Paquetes adicionales

\usepackage[author={Víctor de Juan, 2017}]{pdfcomment}

\makeatletter
\newcommand{\annotate}[2][]{%
\pdfstringdef\x@title{#1}%
\edef\r{\string\r}%
\pdfstringdef\x@contents{#2}%
\pdfannot
width 2\baselineskip
height 2\baselineskip
depth 0pt
{
/Subtype /Text
/T (\x@title)
/Contents (\x@contents)
}%
}
\makeatother

% --------------------

\begin{document}


\begin{enumerate}

\item Sea $P(x) = 3x^3-3x^2-3x+3 [\text{sol: } = 3(x-1)(x+1)^2]$
\begin{itemize}
	\item ¿Es divisible por $(x-1)$? Comprobamos $P(1) = 3-3-3+3 = 0 \overset{T.F}{\implies}$ Sí.
\end{itemize}

\hrule

\item Sea $P(x) = 6x^3-10x^2+4x [\text{sol: } = 6x(x-1)(x-\rfrac{2}{3})]$
\begin{itemize}
	\item Factoriza.
	\subitem $P(0) = 0$. Por el teorema del factor sabemos que $x-0$ es un factor.
	\subitem Posibles raíces $α=\rfrac{m}{n}$ con $n=\pm1,\pm2,\pm4$ y $m=\pm1,\pm2,\pm3,\pm6$.	
	\subitem Por el teorema de la factorización, $Q(x) = 3x^3-5x+2x$ tendrá las mismas raíces que $P(x) = 6x^3-10x^2+4x$.\footnote{\textit{Ojo, no podemos simplificar, pero las raíces son las mismas.}}. Ahora las posibles raíces son $\rfrac{n}{m}$ donde $n\in\{\pm1,\pm2\}$ y $m\in\{\pm1,\pm3\}$
	\subitem $P(1) = 0$. Por el teorema del factor sabemos que $0$ es una raíz. ¿Es esto más fácil que Ruffini? ¿Y ahora?
\end{itemize}

\hrule

\item Sea $P(x) = 2x^3-2x^2+kx+4$.
\begin{itemize}
	\item Halla el valor de $k$ para que $P(x)$ sea divisible por $x-2$.
	\subitem Por el teorema del factor, buscamos $P(2) = 0$. Entonces:
	\[
		P(2) = 0 \dimplies 2^4-2^3+2k+4 = 0 \dimplies 16-12+2k = 0 \dimplies k = -2
	\]
\end{itemize}


\hrule

\item Sea $P(x) = x^4+4x^3+6x^2+4x+1$
\begin{itemize}
	\item Factoriza.
	\subitem Por el binomio de Newton $P(x) = (x+1)^4$
\end{itemize}

\hrule

\item Sea $P(x) = x^3+2·k·x^2+4·k·x+8$
\begin{itemize}
	\item ¿Qué valor puede tomar $k$ para que el polinomio tenga una raíz de multiplicidad 3?
	\subitem $P(x) = x^3+2·k·x^2+4·k·x+8 = x^32^0+k·x^22^1 + k·x^12^2 + x^02^3$. 

	Buscamos $(x-a)^3 = P(x)$. La solución es tomar $a=-2$ y $k=3$, para poder aplicar el binomio de Newton.
\end{itemize}

\hrule

\item Sea $P(x) = 6x^3+ax^2+bx-1$, con $a,b\inℤ$
\begin{itemize}
	\item Halla el valor de $a,b$ para que $P(x)$ sea divisible por $(x-\rfrac{1}{3})$ y por $(x-\rfrac{1}{5})$.
	\subitem Por el teorema de las raíces fraccionarias, $5$ no divide al coeficiente principal, por lo que $P(x)$ no puede ser divisible por $(x-\rfrac{1}{5})$.
	\item Halla el valor de $a,b$ para que $P(x)$ sea divisible por $(x-\rfrac{1}{3})$ y por $(x-\rfrac{1}{2})$.
	\subitem Por el teorema del factor, buscamos:
	\[
	\left\{
		\begin{array}{c}
			P(\rfrac{1}{2}) = 0 \dimplies \frac{6}{8} + \frac{a}{4} + \frac{b}{2} - 1 = 0\\
			P(\rfrac{1}{3}) = 0 \dimplies \frac{6}{27} + \frac{a}{9} + \frac{b}{3} - 1 = 0
		\end{array}\right\}\dimplies ... \quad (a,b) = (-1,-4)
	\]
\end{itemize}

\hrule

\item Sea $P(x) = 4x^2+bx+1$, $b\inℤ$. 
\begin{itemize}
	\item Sabemos que sus raíces $α_1,α_2$ son fraccionarias y positivas. ¿Cuáles son? ¿Cuánto vale $b$?
	\subitem Por el teorema de las raíces fraccionarias, $α_1 = \rfrac{n_1}{m_1}$, sabemos que $n_1$ divide a $1$. Análogo para $α_2$.

	Por otro lado, sabemos que $m_2$ divide a 4. Las posibilidades son $2,4$, con lo que $α_1,α_2 \in \{\rfrac{1}{2},\rfrac{1}{4}\}$

	Por el teorema del resto, $P(\rfrac{1}{2}) = 1+b\rfrac{1}{2}+1 = 0 \implies b=-4$. 
	Por el teorema del resto, $P(\rfrac{1}{4}) = \rfrac{1}{4}+b\rfrac{1}{4}+1 = 0 \implies b=-2$.

	Si queremos que sea divisible por los 2 factores, b tiene que valer a la vez $4$ y $-2$. Entonces, necesariamente $P(x) = 4(x-\rfrac{1}{2})^2$ o $P(x) = 4(x-\rfrac{1}{4})^2$. 

	Desarrollando la segunda opción, obtenemos como término independiente $\rfrac{1}{4}$, por lo que no es posible.

\end{itemize}


\hrule

\item Sea $P(x) = 21x^2+10x-2 [\text{sol: } =  21(x+1/3)(x+1/7)]$.

\end{enumerate}

\end{document}
\grid