
\chapter{Geometría}

\section{Puntos y vectores en $\real^2$}

Vamos a comenzar definiendo formal y matemáticamente el plano:

El plano es el conjunto de los puntos. La manera formal y matemática de definir el conjunto de puntos es el producto cartesiano.

\begin{defn}[Producto cartesiano]
Sean $A$ y $B$ 2 conjuntos cualesquiera. Se define el producto cartesiano como

\[
	A\times B = \left\{ (a,b) \tq a,b\in B \right\}
\]
\end{defn}

\obs $A\times B \neq B\times A$

\begin{example}
\[\{2,4,6\}\times \{1,3,5\} = \{(2,1),(2,3),(2,5),(4,1),...,(6,5)\}\]
\[\{1,3,5\}\times \{2,4,6\} = \{(1,2),(1,4),(1,6),(3,2),...,(5,6)\}\]
\end{example}

Vamos a definir $\real^2$, con el que vamos a trabajar a partir de ahora. 

\[\real^2 = \real\times\real = \{ (x,y)\in\real^2 \tq x,y\in\real\}\]

\begin{example}
	\begin{itemize}
		\item $(2,3)\in\real^2$
		\item $(\sqrt{-1},\pi)\not\in\real^2$
		\item $(e,\sqrt{2})\in\real^2$
	\end{itemize}
\end{example}


\subsection{Espacio vectorial sobre $\real^2$}

$\real^2$ es el conjunto de todos los puntos sobre el que podemos construir el conjunto donde viven los vectores.

\begin{defn}[Espacio vectorial sobre $\real^2$]
Un espacio vectorial es una estructura algebraica, $V^2=(\real^2,+,·)$, donde:
\begin{itemize}
	\item $\appl{+}{\real^2\times\real^2}{\real^2}$  y se define como $(a,b)+(c,d) = (a+c,b+d)$
	\item $\appl{·}{\real^2\times\real}{\real^2}$  y se define como $\lambda·(a,b) = (\lambda·a,\lambda·b)$ con $\lambda\in\real$
\end{itemize}


Las operaciones $+$ y $·$ deben cumplir las siguientes propiedades:
\begin{itemize}
    \item $\vec{u} + (\vec{v} + \vec{w}) = (\vec{u} + \vec{v}) + \vec{w}, \qquad \forall \vec{u}, \vec{v}, \vec{w} \in V $  (asociativa)
    \item $\vec{u} + \vec{v} = \vec{v} + \vec{u}, \qquad \forall \vec{u}, \vec{v} \in V$ (conmutativa)
\item $ \exists{}\vec{0} \in{} V : $  $ \vec{u} + \vec{0} = \vec{u} , \forall{} \vec{u} \in{} V
$ (elemento neutro)

\item $
   \forall{} \vec{u} \in{} V , \quad
   \exists{} \vec{-u} \in{} V : $  $
    \vec{u} + (\vec{-u}) = \vec{0}
$ (elemento opuesto)

\item $
   \mathit{a} \cdot (\mathit{b} \cdot \vec{u})=(\mathit{a} \cdot \mathit{b}) \cdot \vec{u} ,$  $
   \forall{} \mathit{a} ,\mathit{b} \in{}K , $  $
   \forall{} \vec{u} \in{} V
$

\item$
   \exists{e} \in{K}: $  $
   e \cdot \vec{u}   = \vec{u} , 
   \forall{} \vec{u} \in{} V
$ (elemento neutro del producto)
\item $
   \mathit{a} \cdot (\vec{u}+ \vec{v}) =
   \mathit{a} \cdot \vec{u}+ \mathit{a} \cdot \vec{v} , $  $
   \forall{} \mathit{a}\in{}K , $  $
   \forall{} \vec{u}, \vec{v} \in{} V
$ (propiedad distributiva)
\item $
   (\mathit{a} + \mathit{b}) \cdot \vec{u} =
   \mathit{a} \cdot \vec{u} + \mathit{b} \cdot \vec{u} , $  $
   \forall{} \mathit{a}, \mathit{b} \in{} K , $  $
   \forall{} \vec{u} \in{} V
$ (propiedad distributiva)
\end{itemize}

\end{defn}

\obs $\vec{a} - \vec{b} = \vec{a}+ (-1)·\vec{b}$
\obs $\nexists \vec{a}·\vec{b}$ (todavía).


\begin{example}
$(2,3) + (-1,4) \overset{(1)}{=} (2+(-1),3+4)  \overset{(2)}{=} (1,7)$

Donde en $(1)$ aplicamos la suma en $ V^2$ y en $(2)$ aplicamos la suma en $\real$.

%\textbf{Representación gráfica de la suma}

\end{example}



%¿Podemos decir que $\real^2$ es un cuerpo? Para ello tendríamos que definir $(a,b)·(c,d),\;\forall(a,b),(c,d)\in\real^2$, cosa que no hemos hecho. A la estructura algebraica $(\real^2,+,·)$ tal y como la hemos definido la llamamos \concept[Espacio\IS Vectorial]{Espacio vectorial}.

Igual que hemos definido $\real^2$, podríamos haber definido $\real^3$, estructura con la que trabajaréis el año que viene. De hecho, podríamos definir $\real^{25}$ de una manera muy similar.
%
Como curiosidad matemática, hasta podríamos definir $\real\times...\times\real$ infinitas veces y estudiar qué pasa con ello. 

Vamos a aterrizar un poco.


\begin{defn}[Combinación lineal]
Combinación lineal de $\vec{x},\vec{y},\vec{z}\in V^2$ es cualquier expresión algebraica de la forma $α\vec{x}+β\vec{y}+λ\vec{z}$ con $α,β,λ\in\real$
\end{defn}

\begin{example}
	\begin{itemize}
		\item $2·(1,1) + \sqrt{2}·(0,1) = (2,2+\sqrt{2})$
		\item $1·(1,0) + (-\rfrac{1}{2})(2,0) = (0,0)$ (\textit{Combinación lineal nula})
		\item Completa con dos ejemplos de tu propia cosecha.
	\end{itemize}
\end{example}

\obs Es la misma idea de fondo que la de combinaciones lineales de ecuaciones en el método de Gauss.

\begin{defn}[Dependencia lineal]
Sean $\vec{x},\vec{y},\vec{z}\in V^2$. 

\[\vz \text{ depende linealmente de } \vx,\vy \dimplies \existsα,λ\in\real \tq \vz = α\vx+λ\vy\]
\end{defn}

También podríamos definirlo para 2 vectores. $\vx = λ\vy$ con $\lambda\in\real$

\begin{defn}[Independencia lineal]
Sean $\vec{x},\vec{y}\in V^2$. 

\[\vx,\vy \text{ linealmente independientes } \dimplies \nexists \lambda\in\real \tlq \lambda\vec{x}=\vec{y}\]
\end{defn}

\begin{problem} Demostrar que $\vec{a}=(2,2)$ y $\vec{b}=(3,5)$ son linealmente independientes.
\solution

Buscamos $α$ tales que $α\vec{a} = \vec{b}$. 

\[
	α(2,2) = β(3,5) \dimplies
\left\{
	\begin{array}{c}
		2α=3\\
		2α=5
	\end{array}
\right\}\implies \text{Sistema incompatible} \implies \nexists\alpha\tlq \alpha\vec{a}=\vec{b}
\]
\end{problem}

\begin{problem} Halla el valor o valores de $m$ para que  $\vec{a}=(1,3)$ y $\vec{b}=(2,m)$ sean linealmente dependientes.
\solution

Por definición de dependencia lineal, buscamos $α$ tal que $α\vec{a} = \vec{b}$.

\[
α(1,3) = (2,m) \dimplies 
\]
\[
\left\{
	\begin{array}{c}
		α=2\\
		3α=m
	\end{array}
\right\}\dimplies m=6
\]

\textbf{Conclusión:} para $m=6$ los vectores son linealmente dependientes. 

\end{problem}

\obs $(1,3),(2,6)$ son linealmente dependientes porque son proporcionales. Esto es cierto en general:

\begin{prop}
Los vectores $\vx=(x_1,x_2),\vy=(y_1,y_2)\in V^2$ son linealmente dependientes $\dimplies \frac{x_1}{y_1}=\frac{x_2}{y_2} \text{ (es decir, son proporcionales)}$ 
\end{prop}


\begin{problem} Halla el valor o valores de $m$ para que  $a=(1,3)$ y $b=(2,m)$ sean linealmente dependientes.
\solution

Utilizamos: "2 vectores son linealmente dependientes si y sólo si son proporcionales".

Buscamos $m$ tal que $\rfrac{1}{2} = \rfrac{3}{m}$, lo cual es cierto si $m=6$.
\end{problem}

\begin{defn}[Rango de un conjunto de vectores]
El rango de un conjunto de vectores es el número de vectores linealmente independientes que contiene.
\end{defn}

\begin{example}
	\begin{itemize}
		\item $\text{Rg}\{(0,1),(1,0),(0,3)\} = 2$
	\end{itemize}
\end{example}

\begin{problem}
Halla los valores de $m$ para los que el conjunto de vectores $C=\{(1,3),(3,m)\}$ tenga rango 2. ¿Hay algún valor para que el rango sea 0?

\solution

\end{problem}


\subsubsection{Base de un espacio vectorial}

\begin{defn}[Sistema de generadores]
$B = \{\vx_1,\vx_2,...,\vx_n\}$ con $i=1,...,n\;\;\vx_i,\in V^2$ es un sistema de generadores si y sólo si \[\forall \vz\in V^2,\;\; \existsα_1,α_2,...,α_n\in\real \tq \vz=α_1\vx_1+α_2\vx_2 + ... + α_n\vx_n \left( = \sum_{i=1}^n α_i\vx_i\right)\]
\end{defn}

\begin{defn}[Base]
Un conjunto $B=\{\vx,\vy\}$ con $\vx,\vy\in V^2$ es una base de $ V^2$ si y sólo si cumple:

	\begin{itemize}
		\item $\vx,\vy$ linealmente independientes.
		\item $\{\vx,\vy\}$ es un sistema de generadores.
	\end{itemize}
\end{defn}


\begin{defn}[Coordenada de un vector] 
	Dados $\mathcal{B}=\{\vec{v},\vec{w}\}$ una base de $V^2$. Sea $\vec{z}\in V^2$ con $\vec{z} = \alpha \vec{v} + \beta\vec{w}$. 

	A esos coeficientes $\alpha,\beta$ los llamamos \textit{coordenadas del vector en la base $\mathcal{B}$}. 
	%
	Podemos escribir $\vec{z} = (\alpha,\beta)$
\end{defn}


\begin{example}
		Sea $B_1=\{(0,1),(1,0)\}$ una base de $ V^2$. Por ejemplo, el vector $\vec{z}$ de coordenadas $(3,6)$ en la base $B_1$ se puede construir como combinación lineal de los vectores de la base: $(3,6) = 3·(1,0) + 6·(0,1)$.

		\obs $B_1$ se denomina \concept{base canónica} y se suele representar por $B=\{\vi,\vj\}$.
\end{example}


\obs Las coordenadas de un vector en una base son únicas. Pero un mismo vector puede tener coordenadas diferentes en basees diferentes.

\begin{example}
	Sea $B_2=\{(1,1),(0,1)\}$ una base de $ V^2$. Por ejemplo, el vector anterior $\vec{z}=(3,6)$ se puede construir como combinación lineal de los vectores de la base: $(3,6) = \alpha·(1,1) + \beta·(0,1)$


	Haciendo las cuentas obtendríamos $\alpha=3, \beta=3$, por lo que $\vec{z}=(3,3)$.
\end{example}

\begin{problem}
			\ppart ¿Es $B_2 = \{(1,1),(1,-1)\}$ una base de $ V^2$?
			\ppart ¿Cuáles son las coordenadas del vector $(7,5)$ en esta base?

			\solution
			\spart Buscamos demostrar que los vectores son linealmente independientes y que las coordenadas de un vector cualquiera $\vz=(z_1,z_2)$ son únicas. 

			\paragraph{Independencia lineal:} 

			\[
				α(1,1) = (1,-1) \dimplies 
				\left\{
					\begin{array}{c}
						α=β\\
						α=-β
					\end{array}
				\right\}\implies \text{Sistema incompatible} \dimplies \text{Linealmente independientes}
			\]

			\paragraph{Generación:} Las coordenadas de un vector cualquiera $\vz=(z_1,z_2)$ deberían ser únicas.

			\[
				(z_1,z_2) = α(1,1) + β(1,-1) \dimplies (z_1,z_2) = (α+β,α-β) \dimplies 
			\]
			\[
				\left\{
					\begin{array}{c}
						α+β=z_1\\
						α-β=z_2
					\end{array}
				\right\}\dimplies
				\left\{
					\begin{array}{c}
						α+β=z_1\\
						2α=z_2
					\end{array}
				\right\}\implies \textit{Sistema compatible determinado} \implies \textit{Solución única}
			\]

			
			Entonces, el conjunto genera $ V^2$ por lo que es base de $V^2$

			\spart 
			\[
				(7,5) = α(1,1) + β(1,-1) \dimplies (7,5) = (α+β,α-β) \dimplies 
			\]
			\[
				\left\{
					\begin{array}{c}
						α+β=7\\
						α-β=5
					\end{array}
				\right\}\dimplies
				\left\{
					\begin{array}{c}
						α+β=7\\
						2α=12
					\end{array}
				\right\}\implies (α,β) = (6,1)
			\]
			\end{problem}




\begin{problem}
Sea $B=\{(3,0),(1,-1)\}$. 
\ppart ¿Es una base de $ V^2$?
\ppart En caso de que sea una base, calcula las coordenadas del vector $(6,6)$ en la base $B$.
\ppart ¿Es una base ortogonal?
\solution
\spart
\textbf{Linealmente independientes:}
\[
α(3,0) + β(1,-1) = (0,0) \dimplies \left\{\begin{array}{c}3α+β=0\\-β=0 \end{array}\right\} \dimplies  (α,β) = (0,0)
\]
\textbf{Generadores:} 
\[
α(3,0) + β(1,-1) = (x,y) \dimplies\underbrace{\left\{\begin{array}{c}3α+β=x\\-β=y \end{array}\right\}}_{SCD} \dimplies  (α,β) = (\rfrac{x+y}{3},-y) \implies \exists! (a,b)\in V^2
\]

\spart 
\[
	(6,6) = α(3,0) + β(1,-1) \dimplies (6,6) = (3α+β,-β) \dimplies (α,β) = (4,-6)
\]
\end{problem}



\begin{figure}[h]
\centering
\definecolor{xdxdff}{rgb}{0.49,0.49,1}
\definecolor{ttqqff}{rgb}{0.2,0,1}
\begin{tikzpicture}[line cap=round,line join=round,>=triangle 45,x=1.0cm,y=1.0cm]
\draw[->,color=black] (-1.8,0) -- (4.14,0);
\foreach \x in {-1,1,2,3,4}
\draw[shift={(\x,0)},color=black] (0pt,2pt) -- (0pt,-2pt) node[below] {\footnotesize $\x$};
\draw[color=black] (4.04,0.02) node [anchor=south west] { x};
\draw[->,color=black] (0,-0.33) -- (0,3.31);
\foreach \y in {,1,2,3}
\draw[shift={(0,\y)},color=black] (2pt,0pt) -- (-2pt,0pt) node[left] {\footnotesize $\y$};
\draw[color=black] (0.03,3.19) node [anchor=west] { y};
\draw[color=black] (0pt,-10pt) node[right] {\footnotesize $0$};
\clip(-1.8,-0.33) rectangle (4.14,3.31);
\draw [->,line width=1pt] (0,0) -- (1,0);
\draw [->,line width=1pt] (0,0) -- (0,1);
\draw [->,line width=1pt,color=ttqqff] (0,0) -- (1,0.5);
\draw [->] (0,0) -- (1,2);
\draw [->,line width=1pt,color=ttqqff] (0,0) -- (-0.2,1);
\draw [->] (0,0) -- (-0.27,1.36);
\draw [->] (0,0) -- (1.27,0.64);
\draw [dotted,domain=-1.8:4.14] plot(\x,{(-1.91--1.36*\x)/-0.27});
\draw [dotted,domain=-1.8:4.14] plot(\x,{(--1.91--0.64*\x)/1.27});
\begin{scriptsize}
\draw[color=black] (0.47,-0.1) node {$i$};
\draw[color=black] (-0.06,0.48) node {$j$};
\draw[color=ttqqff] (0.56,0.21) node {$v$};
\fill [color=xdxdff] (0,0) circle (1.5pt);
\draw[color=xdxdff] (0.05,0.08) node {$A$};
\draw[color=ttqqff] (-0.13,0.41) node {$u$};
\draw[color=black] (-0.21,1.15) node {αu};
\draw[color=black] (1.12,0.61) node {βv};
\draw[color=black] (-1.75,0.73) node {$c$};
\end{scriptsize}
\end{tikzpicture}
\caption{Cambio de base de unos vectores gráficamente.}
\end{figure}



\section{Plano afín}

\paragraph{Concreción}

¿Qué necesitamos para orientarnos en el plano? ¿Cuál puede ser un sistema de referencia? 

Imagínate un punto cualquiera dibujado en una pizarra. ¿Cómo ponernos de acuerdo de cómo llegar a él? Necesitamos un \concept{sistema de referencia}, es decir, un origen y una escala (base). Normalmente tomamos como origen el $(0,0)$ y como base la canónica $\{(1,0),(0,1)\}$.

\begin{defn}[Plano afín]
Llamamos plano afín a la estructura algebraica $\{\real^2,V^2,\rho\}$, donde $\real^2$ es el conjunto de los puntos del  planos, $\vec{V^2}$ es el espacio vectorial de los vectores \textbf{libres} del plano y $\rho$ es la \concept{relación de equipolencia} que nos permite construir vectores a partir de los puntos.
\end{defn}


\begin{defn}[Vector\IS fijo]
Dados $A,B\in\real^2$, llamamos vector fijo $\vec{AB}$ al segmento que une los puntos $A$ y $B$, orientado desde $A$ hasta $B$.
\end{defn}

Así, de esta definición se deduce que los vectores tienen un origen y un destino, una dirección, un sentido y una longitud (\concept[Módulo\IS de un vector]{módulo}). 


\begin{defn}[Vector\IS libre]
Representante canónico del conjunto de vectores con mismo módulo, dirección y sentido.
\end{defn}

Para entender el concepto de vector libre vamos a recurrir a otro ejemplo: las fracciones equivalentes.

$[\rfrac{1}{2}] = \{\rfrac{1}{2},\rfrac{2}{4},\rfrac{4}{8},...\}$. Tenemos distintas fracciones  de expresar la misma realidad. Así, llamamos \concept{representante canónico} a la fracción $\rfrac{1}{2}$. ¿Qué relación hay entre $\rfrac{1}{2},\rfrac{2}{4}$? 

De la misma manera, el vector libre es el representante canónico de los vectores con el mismo módulo, dirección y sentido. ¿Qué relación hay entre ellos? Se llama relación de equipolencia y, por ello, se dice que 2 vectores son \textit{equipolentes} si tienen el mismo módulo, misma dirección y mismo sentido.
 



Como no se puede operar con puntos y nuestras operaciones las hemos definido para vectores, necesitamos conocer cómo obtener vectores a partir de los puntos. Para ello, definimos

\begin{defn}[Vector\IS de posición] Llamamos $[\vec{OA}]$ al vector de posición del punto $A=(a_1,a_2)$, cuyas coordenadas son $[\vec{OA}] = (a_1-0,a_2-0)=(a_1,a_2)$

\obs Si tuviéramos un sistema de referencia con otro origen, cambiaría ligeramente la expresión.
\end{defn}


\subparagraph{Coordenadas de un vector entre los puntos $A=(a_1,a_2)$ y $B=(b_1,b_2)$:} 

$[\vec{OA}] + [\vec{AB}] =  [\vec{OB}] \dimplies [\vec{AB}] =  [\vec{OB}] - [\vec{OA}] = (b_1,b_2) - (a_1,a_2) = (b_1-a_1,b_2-a_2)$


\obs Escribiremos $\vec{AB}$ para referirnos al vector fijo que une los puntos $A$ y $B$ y $[\vec{AB}]$  para referirnos al vector libre.

\begin{example}
Hallar las coordenadas de $[\vec{AB}]$ con $A=(1,2)$ y $B=(3,5)$.

$[\vec{AB}] = (1-3,2-5)$
\end{example}

\begin{problem}
Sean 3 puntos $A=(1,-1), B=(0,2), C=(-1,m)$.

\ppart Halla, si es posible, el valor de $m$ para que $[\vec{AB}] = [\vec{BC}]$

\ppart Halla, si es posible, el valor de $m$ para que $[\vec{AB}] = [\vec{AC}]$

\solution

\spart 
\[
\left.\begin{array}{l}
	\left[\vec{AB}\right] = (-1,3)\\\\
	\left[\vec{BC}\right] = (-1,m-2)
\end{array}\right\} \implies m-2=3 \dimplies m=5
\]

\spart 
\[
\left.\begin{array}{l}
	\left[\vec{AB}\right] = (-1,3)\\\\
	\left[\vec{AC}\right] = (-2,m+1)
\end{array}\right\} \implies \text{ Imposible } [\vec{AB}] = [\vec{AC}]
\]

\end{problem}

\obs ¿Es importante la base en la que estén definidos los vectores? No, porque los 2 están en la misma base.

\subparagraph{Punto medio de un segmento} Dados $A=(a_1,a_2)$ y $B=(b_1,b_2)$. Por definición, el punto medio $M=(m_1,m_2)$ cumplirá: $[\vec{AM}] = [\vec{MB}]$

\[
[\vec{AB}] = [\vec{AM}] + [\vec{MB}]\dimplies
[\vec{AB}] = 2[\vec{AM}] \dimplies 
2(m_1-a_1,m_2-a_2) = (b_1-a_1,b_2-a_1) \dimplies
\]
\[
\left\{
	\begin{array}{c}
	2m_1-2a_1 = b_1-a_1\\
	2m_2-2a_2 = b_2-a_2
	\end{array}
\right\}\dimplies
\left\{
	\begin{array}{c}
	m_1 = \displaystyle\frac{b_1+a_1}{2}\\
	m_2 = \displaystyle\frac{b_2+a_2}{2}
	\end{array}
\right\}
\dimplies (m_1,m_2) = \left(\frac{b_1+a_1}{2},\frac{b_2+a_2}{2}\right)
\]


\begin{prop}[Cálculo del módulo de un vector en la base canónica] 
Sea $\vec{x} = (x_1,x_2)$.

$|\vec{x}| = \sqrt{x_1^2+x_2^2}$

\obs ¿Y si fuera otra base? ¿Se mantendría? No

\end{prop}


\begin{defn}[Vector\IS unitario] $\vec{v}\in V^2$ es unitario si $|\vec{v}|=1$
\end{defn}

\begin{problem}
Dado $\vec{v}=(3,4)$ haya un vector unitario con la misma dirección y sentido.
\solution

\end{problem}

\subsection{Producto escalar}

\begin{defn}[Producto escalar]
Sean $\vx=(x_1,x_2),\vy=(y_1,y_2)\in V^2$. 

El producto escalar es una operación: $\appl{·}{ V^2\times V^2}{ V}$ que opera de la siguiente manera: 

\[\vx·\vy = |\vx|·|\vy|·\cos{\widehat{(\vx,\vy)}}\]
\end{defn}

\paragraph{Propiedades:} $\forall\vx,\vx,\vz\in V^2$

\begin{itemize}
	\item \textbf{Conmutativo: } $\vx·\vy = \vy·\vx$
	\item \textbf{Bilineal:}
	\subitem  $(λ\vx)·\vy=λ(\vx·\vy)$
	\subitem  $\vx·(\vy+\vz) = \vx·\vy + \vx·\vz$
	\item \textbf{No negatividad:} $\vx·\vx ≥0$
\end{itemize}


\begin{example}
Dados $\vec{v}=(1,1),\vec{w}=(-1,5) \in V^2$:

\[\vv·\vec{w} = |\vv|·|\vec{w}|·\cos(\widehat{\vec{v},\vec{w}})\]

¿Cómo podemos calcular el ángulo que forman los vectores?
\end{example}

Existe otra manera de calcular el producto escalar de dos vectores.


\begin{prop}[Expresión analítica del producto escalar]

Dados $\vec{v}=(v_1,v_2),\vec{w}=(w_1,w_2)\in V^2$ respecto de una base ortonormal $\mathcal{B} = \vec{a}=(a_1,a_2),\vec{b}=(b_1,b_2)$, entonces $\vec{v}·\vec{w} = v_1·w_1 + v_2·w_2$
\end{prop}

\begin{proof}

\[
\vec{v}·\vec{w} = (v_1·\vec{a} + v_2\vec{b})(w_1·\vec{a} + w_2\vec{b}) =\]
\[  
v_1w_1·(\vec{a}·\vec{b}) + v_1w_2(\vec{a}·\vec{b}) + v_2w_1\vec{b}·\vec{a} + v_2w_2\vec{b}\vec{b} = \]
\[
v_1w_1·1 + v_1w_2·0 + v_2w_1·0 + v_2w_2·1 = \]
\[
v_1·w_1 + v_2·w_2
\]

\end{proof}


\begin{example}
Dados $\vec{v}=(1,1),\vec{w}=(-1,5) \in V^2$:

En lugar de utilizar la definición, vamos a utilizar la expresión analítica:

\[\vv·\vec{w} = (1,1)·(-1,5) = 1-5= -4\]
\end{example}


\subsubsection{Ángulo entre vectores}

Ahora que tenemos otra manera de calcular el producto escalar, podemos calcular el ángulo que forman dos vectores. Despejando el coseno de la definición de producto escalar obtenemos:

\[\vv·\vec{w} = |\vv|·|\vec{w}|·\cos(\widehat{\vec{v},\vec{w}})\dimplies\cos\left(\widehat{\vec{v},\vec{w}}\right)=\frac{\vv·\vec{w}}{|\vv|·|\vec{w}|} = \frac{v_1w_1+v_2w_2}{|\vv|·|\vec{w}|}\]



\begin{example}
Dados $\vec{v}=(1,1),\vec{w}=(-1,5) \in V^2$, calcula el ángulo que forman.

Calculamos: 
\begin{itemize}
	\item $\vv·\vec{w} = (1,1)·(-1,5) = 1-5= -4$
	\item $|\vv| = \sqrt{2}$
	\item $|\vec{w}| = \sqrt{(-1)^2+5^2} = \sqrt{26}$
\end{itemize}

\[\cos\left(\widehat{\vec{v},\vec{w}}\right) = \frac{-4}{\sqrt{2}\sqrt{26}} = \frac{-2\sqrt{13}}{13}\]
\end{example}


\begin{defn}[Ortogonalidad]
$\vx,\vy\in V^2$ son ortogonales $\dimplies \vx·\vy=0$.
\end{defn}

\begin{defn}[Perpendicular]
$\vx,\vy\in V^2$ son perpendiculares $\dimplies$ forman un ángulo recto.
\end{defn}

¿Es lo mismo perpendicular que ortogonal? Perpendicular $\implies$ ortogonal, pero no al revés. El vector $\vec{0} = (0,0)$ es ortogonal a todos los demás vectores pero perpendicular a ninguno.


\begin{defn}[Base\IS ortogonal]
$\mathcal{B} = \{\vec{a}=(a_1,a_2),\vec{b}=(b_1,b_2)\}$ es ortogonal si $\vec{a},\vec{b}$ son ortogonales, es decir, $\vec{a}\cdot\vec{b} = 0$ 
\end{defn}

\begin{defn}[Base\IS ortonormal]
$\mathcal{B} = \vec{a}=(a_1,a_2),\vec{b}=(b_1,b_2)$ es ortonormal si $\vec{a},\vec{b}$ son vectores unitarios ortogonales.
\end{defn}



\subsection{Ejercicios recomendados}

Todos los del libro.

\subsection{Ejercicios variados resueltos}
\begin{problem}
Sea $B=\{(-2,2),(1,-1)\}$. 
\ppart ¿Es una base de $ V^2$?
\ppart En caso de que sea una base, calcula las coordenadas del vector $(6,6)$ en la base $B$.
\solution
\spart
\textbf{Linealmente independientes:}
\[
α(-2,2) + β(1,-1) = (0,0) \dimplies \left\{\begin{array}{c}-2α+β=0\\2α-β=0 \end{array}\right\} \dimplies \underbrace{\left\{\begin{array}{c}-2α+β=0\end{array}\right\} }_{SCI}
\]
Los vectores no son linealmente independientes. Tomando $α=1,β=2$ por ejemplo.

\textbf{Generadores:} 
Ni me lo planteo.


\spart 
Menos todavía.

\end{problem}

\begin{problem}
Sean $A=(0,0),B=(-1,3),C=(2,5)$. ¿Están alineados?
\solution

Que estén alineados es lo mismo que decir que 2 de ellos sean linealmente dependientes, o proporcionales y que tengan algún punto en común.

$[\vec{AB}] = (-1,3)$, $[\vec{AC}] = (2,5)$. Como vectores fijos tienen en común el punto $A$. ¿Son L.D.? 

\[
	\frac{-1}{2} ≠ \frac{3}{5} \implies \text{ No L.D.} \dimplies \text{ No alineados.}
\]

\end{problem}

\begin{problem}
Sean $A=(0,0),B=(-1,3),C=(m,m+8)$. Halla $m$ para que estén alineados.
\solution
$[\vec{AB}] = (-1,3)$, $[\vec{AC}] = (m,m+8)$

\[
	\frac{-1}{m} = \frac{3}{m+8} \implies -m-8 = 3m \dimplies -8=4m \dimplies m=-2
\]

Para $m=-2$ son linealmente dependendientes y, puesto que comparten el punto $A$, podemos decir que los 3 puntos están alineados.

\obs También podríamos haber construido el vector $[\vec{BC}]$, pero hemos elegido $[\vec{AC}]$ por simplificar los cálculos.
\end{problem}

\begin{problem}
Sean $A=(0,0),B=(-1,3),C=(2,m)$. Halla $m$ para que $|[\vec{AB}]| = λ|[\vec{AC}]|$
\solution

$[\vec{AB}] = (-1,2\sqrt{6}) \to |[\vec{AB}]| = \sqrt{1+24}=5$

$[\vec{AC}] = (3,m) \to |[\vec{AB}]| = \sqrt{9+m^2}$

Necesitamos $\sqrt{9+m^2} = 5 \implies 9+m^2 = 25 \dimplies m^2=16 \dimplies m=\pm4$.

\end{problem}


\section{Rectas}
% \section{Geometría Analítica}

% \subsection{}

% \subsection{}


% \subsection{Lugares geométricos}

% Definición, fórmula, desarrollo fórmula (por el libro), ejemplo.

% \begin{itemize}
% 	\item Circunferencia.
% 	\subitem Posición relativa de recta y circunferencia. Intersección y distancia.
% 	\subitem Halla $k$ para que la recta $Ax+By+k=0$ sea tangente a la circunferencia.
% 	\item Mediatriz.
% 	\subitem 2 maneras de calcular. Ejercicio 53 (ellos).
% 	\item Bisectriz. 
% 	\subitem Ejercicio 55 (yo).
% 	\subitem Ejercicio 56 (ellos).
% 	\subitem Deberes: 127.
% 	\subitem Lugar geométrico de los puntos del plano que equidistan de 2 rectas paralelas.
% 	\item Elipse: suma de distancias a los focos.
% 	\subitem Fórmula (deducción del libro), eje mayor, eje menor.
% 	\subitem Focos eje $X$, focos eje $Y$.
% 	\subitem Deducir la fórmula dado un dibujo.
% 	\subitem Ejercicio 18.
% 	\subitem Si $a<b$, la ecuación $\rfrac{x^2}{a} + \rfrac{y^2}{b} = 1$ representa una elipse cuyo eje mayor está contenido en el eje Y.
% 	\item Hipérbola: diferencia de distancias a los focos.
% 	\subitem Fórmula, eje mayor, eje menor.
% 	\subitem Focos eje $X$, focos eje $Y$.
% 	\subitem Forma ejercicio 22.
% 	\subitem Actividad resuelta 161.34.
% 	\item Parábola: equidista foco y directriz.
% 	\subitem Ejercicio resuelto 24 (yo). ¿Cuántas soluciones?
% 	\subitem Parábola de directriz x=3. ¿Forma?
% 	\subitem El punto medio entre la directriz y el foco siempre pertenece a la parábola. ¿V/F?
% \end{itemize}


% \begin{itemize}
% 	\item Trabajo cooperativo. 
% 	\item Los 3 lugares geométricos que faltan. Imponer la restricción de distancias en los 3. Ver en el libro la deducción de las fórmulas y a lo que llegamos.
% 	\item Ejercicio 18.
% 	\item Ejercicio 22 cambiando enunciado (forma de la hipérbola).
% 	\item Ejercicio 25 cambiando enunciado (forma de la parábola).
% 	\item Sin geogebra:
% 		\subitem Esboza la hipérbola 59ad parábolas del 62abhi.
% 	\item Con geogebra: 
% 		\subitem 55d (salvo excentricidad). ¿Qué pasa si intercambias el 6 y el 10? 
% 		\subitem 59a (salvo excentricidad). ¿Qué pasa si intercambias el 144 y el 25?

% \end{itemize}

% \subsubsection{Cónicas}
% A saber:
% \begin{itemize}
% 	\item Reconocer y diferenciar los 4 tipos de cónicas, tanto gráfica como algebraicamente.
% 	\item Conocer la definición como lugar geométrico de los puntos de los 3 tipos de cónicas.
% 	\subitem Aplicar la definición para contestar cuestiones breves.
% 	\item Dibujar a mano alzada una cónica dados los datos necesarios (una directriz y un foco, etc).
% 	\item Resolver posición relativa de una recta y una cónica (resolver el sistema).
% 	\item Halla $k$ para que la recta $Ax+By+k=0$ sea tangente a la circunferencia.

% \end{itemize}
