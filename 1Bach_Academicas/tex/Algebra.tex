% -*- root: ../Cuaderno.tex -*-
\newpage
\section{Números reales}

\subsection{Introducción}



% \subsubsection{Grupo y Cuerpo}

% \begin{defn}[Grupo]
% Sea un conjunto $G\neq \emptyset$ y $\ast$ una operación (ley de composición interna), diremos que $(G,\ast)$ es un grupo si cumple las siguientes propiedades:
% \begin{enumerate}
% 	\item \textbf{Cerrado por la operación}. $\forall x, y \in G, \; x \ast y \in G$
% 	\item \textbf{Asociatividad}. $\forall x, y, z \in G, \; (x \ast y) \ast z = x \ast (y \ast z)$
% 	\item \textbf{Existencia del elemento neutro}. $\exists  e \in G \tq ,\; \forall\, x\in G\; x\ast e=x$.
% 	\item \textbf{Existencia del inverso}. $\forall x \in G ,\; \exists x' \in G \tq x\ast x'=x'\ast x=e$.
% \end{enumerate}
% \end{defn}

% \begin{prop}[Unicidad\IS del neutro e inverso]
%   En todo grupo se cumplen las propiedades de unicidad del elemento neutro y del inverso.
% \end{prop}

% \begin{proof}[\textbf{Reducción al absurdo}]
% Primero demostramos que el neutro es único. \\
% Sean $e$, y $e'$ dos elementos neutros de $G$, se cumple que $e\ast e'\overset{1}{=}e'$, pero también se cumple que $e'\ast e\overset{2}{=}e$. Esto implica que $e'=e$.\\
% 1: $e$ elemento neutro.\\
% 2: $e'$ elemento neutro.

% Por otro lado, si suponemos la existencia de dos elementos inversos $a',a''\in G$, entonces $e=a\ast a'=a\ast a''$. \\

% \[
% 	aa' = aa'' \dimplies a'(aa') = a'(aa'') \overset{asoc.}{\dimplies} (a'a)a'=(a'a)a'' \overset{e.n.}{\dimplies} a'=a'' 
% \]
% \end{proof}

% \begin{example}
% 	\begin{itemize}
% 		\item Sí: $(ℤ,+),(ℚ,+),(ℝ,+),(\{0,1\},+)$
% 		\item No: $(ℕ,+),(ℤ,·),(?,·)$
% 		\item No: $(I,+)$ porque $(1+\sqrt{2}) - (\sqrt{2}\in ℤ)$
% 	\end{itemize}
% \end{example}

% \nota{Si el grupo es \hl{conmutativo}, se llama grupo conmutativo o abeliano por el matemático noruego Abel del siglo XIX.}

% El \hl{producto tiene problemas} para ser ley de composición interna de grupos. 
% %
% Sin embargo, podemos considerar $(ℚ,+,·)$. 
% %
% Ahora hay 2 operaciones y por lo tanto no podemos considerarlo grupo.


% \begin{defn}[Cuerpo] 
% Sea $K$ un conjunto y sean $(+)$ y $(·)$ dos operaciones binarias cerradas, llamadas suma y producto. 
% %
% $(K,+,·)$ es un cuerpo si se cumple:
% \begin{enumerate}
% \item $(A, +)$ es un grupo abeliano.
% \item El producto es asociativo.
% \item Se cumplen las leyes distributivas: 
% 	\subitem $\forall a,b,c \in A \; a\cdot(b+c)=a\cdot b + a\cdot c$ 
% 	\subitem $(a+b)\cdot c= a \cdot c + b \cdot c$
% \item El producto es conmutativo.
% \item Existe elemento neutro para el producto.
% \item $\left(A\backslash \{0\},·\right)$ es un grupo. (Siendo 0 el elemento neutro de la suma.)
% \end{enumerate}
% \end{defn}

% \nota{Si sólo se cumplen a,b,c se llama anillo.}

% \begin{example}
% 	\begin{itemize}
% 		\item Sí: $(ℝ,+,·),(ℚ,+,·),\left[(ℂ,+,·)\right]$
% 		\item No: $(ℤ,+,·),\left[(\mathcal{M},+,·)\right]$
% 		\item $\left(I,+,·\right)$ no, porque $(I,+)$ no es un grupo.
%  	\end{itemize}
%  \end{example}


\subsubsection{Entornos (libro, pero dando las definiciones algebraicas):} Dado un intervalo $(3,7)$, ¿cuál es su centro? ¿cuál es su amplitud? (Calcular). A veces nos interesará escribir el mismo intervalo de puntos haciendo énfasis en el centro y el radio. 
%
Entonces, estaremos escribiendo \textit{entornos} en lugar de intervalos. 
%
Y como los intervalos, hay varios tipos de entornos.

\begin{defn}[Entorno\IS abierto]
Llamamos entorno abierto de centro $a$ y radio $r$, y se denota por $E_r(a)$, al conjunto de puntos que distan de $a$ a una distancia menor que $r$:

\[
	E_r(a) = \{x\in\real, |x-a| < r\} = (a-r, a+r) 
\]
\end{defn}

\begin{defn}[Entorno\IS cerrado]
Llamamos entorno cerrado de centro $a$ y radio $r$, y se denota por $E_r(a)$, al conjunto de puntos que distan de $a$ a una distancia menor o igual que $r$::
\[
	E_r[a]  = \{x\in\real, |x-a| \leq r\} = [a-r, a+r]
\]
\end{defn}

\begin{defn}[Entorno\IS reducido]
Llamamos entorno reducido de centro $a$ y radio $r$, y se denota por $E_r(a)$ a:
\[
	E^{\ast}_r(a) = \{x\in\real, |x-a| < r\} - \{a\} = (a-r,a) \cup (a, a+r)
\]
\end{defn}

\hl{El libro utiliza otra notación que no nos gusta. Apuntad la buena.}

\hl{Seguimos pasando páginas del libro y comentando}

\subsection{Logaritmos}

\textit{En el curso COVID (20-21) se hizo el repaso en sus casas con vídeos}

\paragraph{Introducción}

Vamos a aprender una nueva manera de multiplicar. En realidad ya sabéis, aunque no seáis conscientes.\footnote{Fuente: \href{https://www.youtube.com/watch?v=FB3\_BeukBBk\&t=99s}{Mark Foskey, youtube.com}}



\begin{itemize}
\item Caso 1: $1000000·10000000 = 10^6·10^7 = 10^{13}$. ¿Y podremos hacer esto con otros números que no sean el 10?

\item Caso 2: $64·128 = 2^6·2^7 = 2^{13} = 8192$
\item ¿Caso 3?: Me construyo la tabla del 3.
\begin{center}
\begin{tabular}{cccccccccccc}
1& 3& 9& 27& 81& 243& 729& 2187& 6561& 19683& 59049& 177147\\
\textcolor{red}{0} & \textcolor{red}{1} & \textcolor{red}{2} & \textcolor{red}{3} & \textcolor{red}{4} & \textcolor{red}{5} & \textcolor{red}{6} & \textcolor{red}{7} & \textcolor{red}{8} & \textcolor{red}{9} & \textcolor{red}{10} & \textcolor{red}{11}
\end{tabular}
\end{center}
\item Caso 4: ¿Y para números que no son potencias enteras? Por ejemplo, $64*40$. Pues si $32=2^5$ y $64=2^6$, $40=2^{5,...}$ ¿Tiene sentido?
\item Caso 5: Lo que hicieron, Yost y Napier, fue coger la tabla del 1,0001 en lugar de la tabla del 3 y dividir por mil los números rojos, dando lugar a la tabla de logaritmos:

\begin{center}
	\begin{tabular}{cl}
		0.0 & 1.0\\
		0.001 & 1.001\\
		0.002 & 1.002\\
		0.003 & 1.003\\
		0.004 & 1.00401\\
		0.005 & 1.00501\\
		0.006 & 1.00602\\
		0.007 & 1.00702\\
		0.008 & 1.00803\\
		0.009 & 1.00904\\
		$\vdots$ & \quad\quad$\vdots$\\
		0.991 & 2.69259\\
		0.992 & 2.69529\\
		0.993 & 2.69798\\
		0.994 & 2.70068\\
		0.995 & 2.70338\\
		0.996 & 2.70608\\
		0.997 & 2.70879\\
		0.998 & 2.7115\\
		0.999 & 2.71421\\
		1.0 & \hl{2.71692} (Una aproximación de $e$)\\
	\end{tabular}
\end{center}

\end{itemize}

\begin{defn}[Logaritmo]
Sean $a\in ℝ>0,a≠1$ y $N\in\real$.

Se llama logaritmo en base $a$ de $N$ al exponente $x$ que cumple: $a^x = N$ y se escribe:
\[
	\log_aN=x\dimplies a^x=N
\]
\end{defn}

\nota{De la propia definición se entiende:}
\begin{enumerate}
	\item $y\in\real, y<0 \implies ∀a,\nexists\log_a y$
	\item $\log_a 1 = 0 \dimplies a^0 = 1$
	\item $\log_a a = 1 \dimplies a^1 = a$
	\item $\log_a a^q = q \dimplies a^q = a^q$ por definición.
	\item $a^{\log_a N} = N$
	\begin{proof}
		Sea $b = \log_a N \dimplies a^b = N$.
		Entonces, $a^{\log_a N} = N \dimplies a^n = N$
	\end{proof}
\end{enumerate}

\paragraph{Propiedades:}
\begin{itemize}
	\item $\log_a(AB) = \log_a(A) + \log_a(B)$
		\begin{proof}
			\[\left.\begin{array}{c}
				\log_a(A) = p \dimplies a^p = A\\
				\log_a(B) = p \dimplies a^q = B
			\end{array}\right\}\implies A·B = a^{p+q} \implies \log_a(AB) = \log_aa^{p+q}\]
			\[
				\log_a(AB) = \log_aa^{p+q} \overset{?}{\implies} \log_a(AB) = p+q \overset{?}{=} \log_aA + \log_aB
			\]
		\end{proof}
		\subitem "El exponente de un producto es la suma de los exponentes".
	\item $\log_a\left(\rfrac{A}{B}\right) = \log_a(A) - \log_a(B)$
	\begin{proof}
		Análoga. Para completar por vosotros. Si no lo conseguís, intentadlo.
	\end{proof}
	\subitem "El exponente de un cociente es la resta de los exponentes".
	\item $\log_a(A)^n = n·\log_a(A)$
	\begin{proof} 
		\[\log_aA^n = \log_a(A·A·A·A...) = \log_aA+\log_aA+... = n \log_aA\]
	\end{proof}
	\hl{¿Os convence esta demostración?} No debería... ¿Y si $n\not\in\mathbb{N}$? La demostración deja de valer. ¡Sed críticos!
	\begin{proof}
		Tomamos $\log_aA = p \dimplies a^p = A$

		\[
			a^p = A \dimplies a^{np}=A^n \dimplies \log_aa^{np} = \log_aA^n \dimplies np = \log_aA^n
		\]
		\[ 
			\dimplies n\log_aA = \log_aA^n
		\]
	\end{proof}
	\item $\displaystyle\log_aA = \frac{\log_bA}{\log_ba}$ \textbf{(Cambio de base})
		\begin{proof}
			Tomamos $\log_aA = p \dimplies a^p = A$
			\[
				\log_ba^p = \log_bA \dimplies p\log_ba = \log_bA \dimplies p=\log_aA=\frac{\log_bA}{\log_ba}
			\]

		\end{proof}
\end{itemize}

\paragraph{Tomar logaritmos:}

\[
A = B \overset{1}{\dimplies} a^{\log_a A} = a^{\log_a B} \dimplies \log_aA=\log_aB
\]

¿Implicaciones o equivalencias?


\[
A = B \implies a^{\log_a A} = a^{\log_a B} \dimplies \log_aA=\log_aB
\]


\subsubsection{Ejemplos con logaritmos}

\begin{itemize}
	\item $\log \sqrt{0,0001}$
	\item Ejercicio 21.49\_a,c,e,g; 52\_a,c;104\_bd
	\item Demuestra $\log_a b · \log_b a = 1$ 
\end{itemize}

\subsubsection{Aplicación de los logaritmos}

\paragraph{pH}

Para medir el nivel de acidez, pH, que mide la concentración de iones hidronio ($H_3O^+$)

\[pH=-\log[H_30^+]\]

\paragraph*{Despejar el tiempo} En el crecimiento exponencial de bacterias:
$P = p_0·k^t$ donde $p_0$ es la cantidad inicial, $t$ son los periodos de tiempo y $k$ la tasa de multiplicación.

Pag 23, ejer 57.

\paragraph{Interés compuesto}

\nota{\hl{Que no copien todo esto, sino que lo entiendan.}}

Sea $C$ el capital inicial. Tras 1 periodo de tiempo, obtengo el r\% de intereses. Entonces tendré: $C_1 = C_0 + \rfrac{r}{100}C_0 = C_0·\left(1+\frac{r}{100}\right)$

Pasado otro periodo de tiempo tendré $C_2 = C_1·\left(1+\frac{r}{100}\right) = C_0·\left(1+\frac{r}{100}\right)·\left(1+\frac{r}{100}\right) = C_0 ·\left(1+\frac{r}{100}\right)^2$

En general, para $t$ periodos de tiempo tendremos: $C_t = C_0·\left(1+\frac{r}{100}\right)^t$.

La fórmula del interés compuesto capitalizado en $p$ periodos, \hl{para un interés nominal $r$ es}: 

\[C = C_0·\left(1+\frac{r}{100p}\right)^{pt}\]


\paragraph{3 preguntas?} Contestadas más adelante:
\begin{itemize}
	\item[a] Pero... ¿es lo mismo un interés anual del 5\% capitalizado cada mes, que un interés anual del 5\% capitalizado cada año? \hl{No.}

	\item[b] ¿Y si capitalizo cada semana? ¿Y cada día? ¿Y si capitalizo \textit{instantáneamente}? \hl{Tampoco}

	\item[c] ¿Cuánto tiempo tengo que dejar el dinero si quiero ganar una cantidad concreta? Aquí sirven los logaritmos.
\end{itemize}

\paragraph{a)}
Pero, ¿es lo mismo un interés mensual del 5\% durante 12 meses que un interés anual del 5\% durante 12 años? Sí. ¿A quién le importa que sean meses o años?

Pero... ¿es lo mismo un interés anual del 5\% capitalizado cada mes, que un interés anual del 5\% capitalizado cada año? ¿Hay alguno que sea mayor? 

\begin{example}

Tengo $C= 1000$ a un $r=5\%$. Tengo 2 opciones para elegir y no se cual es mejor. 

\begin{itemize}
	\item[a)] Un interés anual del 5\% capitalizado cada mes durante 5 años.

	\[
		C = C_0·(1+\rfrac{r}{100p})^{t·p} = 1000·(1+\rfrac{5}{1200})^{12*5} = 1283.36
	\]
	\item[b)] Un interés anual del 5\% capitalizado cada año durante 5 años.

	\[
		C = C_0·(1+\rfrac{r}{100p})^{t·p} = 1000·(1+\rfrac{5}{100})^{5} = 1276.28
	\]

	Es más rentable la opción a. 

	\nota{\hl{De hecho}, ¿cuál es el interés equivalente que hay que aplicar al año en el caso a)?}

	\[
		C = C_0·(1+\rfrac{r}{100})^{t} \dimplies \frac{C}{C_0} = (1+\rfrac{r}{100})^t \implies \sqrt[t]{\frac{C}{C_0}} = \sqrt[t]{(1+\rfrac{r}{100})^t} \dimplies
	\]
	\[
 		\sqrt[t]{\frac{C}{C_0}} -1 = \rfrac{r}{100} \dimplies r = 100·\left(\sqrt[t]{\frac{C}{C_0}} -1 \right)\implies r = 100·\left(\sqrt[5]{\frac{1283.36}{1000}} - 1\right) = 5.1162...
	\]

	Conclusión, un 5\% mensual es lo mismo que un 5.1162...\% anual. Por esto, y para que no nos puedan timar los bancos existe la \concept{TAE} (Tasa Anual Equivalente.)


\end{itemize}
\end{example}

\paragraph{b)}
¿Y si capitalizo cada semana? ¿Y cada día? ¿Y si capitalizo \textit{instantáneamente}? 

\paragraph{c)}
Tengo entendido que durante el mes de julio, ??? ha trabajado de hamaquero en la playa y cobrado 150 al mes. ¿Puede ser? Su banco le ofrece un depósito con un interés del 10\% y Miguel querría pagarse su carrera, que será algo así como 10000, sino sigue subiendo.
%
¿Cuánto tiempo tiene que dejar su dinero en el banco para pagarse la carrera?

\[
	C = C_0·(1+\rfrac{r}{100})^{t} \dimplies \log\frac{C}{C_0} = t\log(1+\rfrac{r}{100}) \dimplies t = \frac{\log\frac{C}{C_0}}{\log(1+\rfrac{r}{100})} = \frac{\log\frac{10000}{150}}{\log\left(1+\rfrac{10}{100}\right)} 
\]

\[
	t = 44.06\text{ años}
\]

Ejercicio 31.122

\section{Tema 2: álgebra}

\subsection{Binomio de Newton}

$
\displaystyle (a\text{\hl{+}}b)^n = \sum_{i=0}^n \binom{n}{i} a^ib^{n-i}
$

Vamos a ir desgranando y entendiendo la fórmula. Empezamos con $\binom{n}{i}$

\paragraph{Combinatoria básica} ¿Cuántas parejas puedo hacer con 5 cartas? $\binom{5}{2}$

\begin{defn}[Número combinatorio] Dados 2 números naturales no nulos $m$ y $n$, siendo $m≥n$, se denomina número combinatorio $\binom{m}{n}$, y se lee "m" sobre "n" a $\binom{m}{n} = \frac{m!}{n!(m-n)!}$
\end{defn}

\paragraph{Propiedades}
\begin{itemize}
	\item $\binom{m}{0}=\binom{m}{m}=1$
	\begin{proof}
		\[\binom{m}{m} = \frac{m!}{m!(m-m)!} = 1\]
	\end{proof}
	\item $\binom{m}{n}=\binom{m}{m-n}$ 
	\begin{proof}
		\[\binom{m}{n} = \frac{m!}{n!(m-n)!} = \frac{m!}{(m-(m-n))!(m-n)!} = \binom{m}{m-n} \]
	\end{proof}
	\item $\binom{m}{n} + \binom{m}{n+1}=\binom{m+1}{n+1}$
\end{itemize}

\nota{No copiar}

En cursos anteriores hemos visto $(a+b)^2 = a^2+2ab+b^2$. Si tenemos muy buena memoria igual recordamos $(a+b)^3 = a^3+3ab^2+3a^2b+a^3$. Pero, ¿cómo calcular, sin desarrollar $(a+b)^6$? 

$(a+b)^6 = (a+b)(a+b)(a+b)(a+b)(a+b)(a+b)(a+b)$ ¿Cuántos productos $abbbbb$ va a haber? $babbbb,bbabbb,bbbabb,bbbbab,bbbbba$. Esto es lo mismo que preguntarnos, ¿de cuántas maneras diferentes puedo ordenar la secuencia de letras $abbbbb$? O, lo que es lo mismo, ¿cuántas combinaciones hay de colocar la $a$ en las 6 posiciones?
%
¿Y para el producto $aabbbb$? Hay 6 huecos y queremos rellenar 2 de ellos con $a$ y dejar 4 libres (fijémonos que es lo mismo que querer rellenar 4 con $b$ y dejar 2 libres. Debemos obtener las mismas posibilidades). ¿De cuántas maneras lo puedo hacer? ¿Cuántas parejas de casilleros puedo hacer con 6 casilleros?

Razonando de esta manera, llegamos a que el coeficiente de $aabbbb = a^2b^4 = \binom{6}{2}=\binom{6}{4}$. Repitiendo el razonamiento, llegaríamos a que el coeficiente sería $a^{m-n}b^m = \binom{m}{n}$. 

De esta manera podemos escribir:

\[
	(a+b)^6 = \binom{6}{0}a^0b^6 + \binom{6}{1}a^1b^5 + \binom{6}{2}a^2b^4 + ... = \sum_{i=0}^6 \binom{6}{i}a^ib^{6-i}
\]

Generalizando esta idea para cualquier exponente $n$ obtenemos:

\concept{Binomio de Newton}:
$
\displaystyle (a\text{\hl{+}}b)^n = \sum_{i=0}^n \binom{n}{i} a^ib^{n-i}
$

\paragraph{Curiosidad matemática:} Triángulo de Pascal (vídeo brutal de \href{https://www.youtube.com/watch?v=0iMtlus-afo}{Numberphile})

De aquí construimos el triángulo y jugamos un rato con él. Potencias de 2 sumando los números, potencias de 11 y Fibonnaci.

\begin{figure}[hbpt]
	\centering
	\includegraphics[scale=0.17]{img/Pascal.png}
	\label{img::TrianguloPascal}
	\caption{Triángulo de Pascal y su relación con la serie de Fibonacci}
\end{figure}




Ejemplo de uso de la fórmula para $(3x-2)^5$. Así, utilizamos la calculadora para números combinatorios. 
Ejercicios del libro. (Yo 15. Ellos 20. Yo 16, ellos 21).


\subsection{Teoremas de factorización}

Ejemplo. La factorización del polinomio. ¿Qué raíces puede tener? Ni $\pm1,\pm3$, ¿entonces? Teorema de las raíces racionales.

Factoriza (yo con ellos): $P(x) = 3x^3-x^2+9x-3 = 3(x^2+3)\left(x-\rfrac{1}{3}\right)$

\begin{theorem}[Teorema\IS del factor]
Sea $P(x) = a_nx^n+a_{n-1}x^{n-1}+...+a_1x+a_0$ con $a_n≠0$ y $a_n,a_{n-1},...,a_1,a_0\in\real$. 
Sea $α\in\real$.

\[
	P(α) = 0 \dimplies \frac{P(x)}{(x-α)} = Q(x)
\]
\end{theorem}

De hecho este teorema es un caso particular del teorema del resto:
\begin{theorem}[Teorema\IS del resto]
Sea $P(x) = a_nx^n+a_{n-1}x^{n-1}+...+a_1x+a_0$ con $a_n≠0$ y $a_n,a_{n-1},...,a_1,a_0\in\real$.

Entonces, el resto de $\frac{P(x)}{x-α} = P(α)$
\end{theorem}


\begin{theorem}[Teorema\IS de la factorización]
Sea $P(x) = a_nx^n+a_{n-1}x^{n-1}+...+a_1x+a_0$ con $a_n≠0$ y $a_n,a_{n-1},...,a_1,a_0\in\real$ y $α_1,α_2,...,α_n\in\real$ las raíces o ceros de $P(x)$. 

Entonces,\[P(x) = a_n(x-α_1)(x-α_2)...(x-α_n)\]
\end{theorem}


\begin{theorem}[Teorema\IS de las raíces enteras]
Sea $P(x) = a_nx^n+a_{n-1}x^{n-1}+...+a_1x+a_0$, con $a_n≠0$, una raíz entera $r$ de $P(x)$ tiene que ser divisor del término independiente.
\end{theorem}



\begin{theorem}[Teorema\IS de las raíces racionales]
Sea $P(x) = a_nx^n+a_{n-1}x^{n-1}+...+a_1x+a_0$, con $a_n≠0$,$a_i\inℤ$ una raíz fraccionaria $\rfrac{n}{m}$ del polinomio $P(x)$ tiene que cumplir $n|a_0$ y $m|a_n$.
\end{theorem}


\subsubsection{Ejercicios:}

\begin{enumerate}
\item (YO) Sea $P(x) = 3x^3-3x^2-3x+3 = 3(x-1)(x+1)^2$
\begin{itemize}
	\item ¿Es divisible por $(x-1)$? Comprobamos $P(1) = 3-3-3+3 = 0 \overset{T.F}{\implies}$ Sí.
\end{itemize}

\item (YO) Sea $P(x) = 6x^3-10x^2+4x = 6x(x-1)(x-\rfrac{2}{3})$ 
\begin{itemize}
	\item Factoriza.
	\subitem $P(0) = 0$. Por el teorema del factor sabemos que $x-0$ es un factor.
	\subitem Posibles raíces: $n=\pm1,\pm2,\pm4$ y $m=\pm1,\pm2,\pm3,\pm6$	
	\subitem Por el teorema de la factorización, $Q(x) = 3x^3-5x+2x$ tendrá las mismas raíces que $P(x) = 6x^3-10x^2+4x$. \hl{(Ojo, no podemos simplificar, pero las raíces son las mismas)}. Ahora las posibles raíces son $\rfrac{n}{m}$ donde $n\in\{\pm1,\pm2\}$ y $m\in\{\pm1,\pm3\}$
	\subitem $P(1) = 0$. Por el teorema del factor sabemos que $0$ es una raíz. ¿Es esto más fácil que Ruffini? ¿Y ahora?
\end{itemize}


\item Sea $P(x) = 2x^3-2x^2+kx+4$.
\begin{itemize}
	\item Halla el valor de $k$ para que $P(x)$ sea divisible por $x-2$.
	\subitem Por el teorema del factor, buscamos $P(2) = 0$. Entonces:
	\[
		P(2) = 0 \dimplies 2^4-2^3+2k+4 = 0 \dimplies 16-12+2k = 0 \dimplies k = -2
	\]
\end{itemize}


\item Sea $P(x) = x^4+4x^3+6x^2+4x+1$
\begin{itemize}
	\item Factoriza (Newton)
\end{itemize}


\item Sea $P(x) = x^3+2·k·x^2+4·k·x+8$
\begin{itemize}
	\item ¿Qué valor puede tomar $k$ para que el polinomio tenga una raíz de multiplicidad 3? (Newton)
\end{itemize}


\item Sea $P(x) = 4x^2+kx+1$.
\begin{itemize}
	\item Halla el valor de $k$ para que sea divisible por $\left(x-\rfrac{1}{3}\right)$. $k=\frac{13}{3}$.
	\item Pero, $3$ no divide a $4$. ¿Cómo podría ser una raíz $\rfrac{1}{3}$?
\end{itemize}


\item Sea $P(x) = 6x^3+ax^2+bx-1$, con $a,b\inℤ$
\begin{itemize}
	\item Halla el valor de $a,b$ para que $P(x)$ sea divisible por $(x-\rfrac{1}{3})$ y por $(x-\rfrac{1}{5})$.
	\subitem Por el teorema de las raíces racionales, $5$ no divide al coeficiente principal, por lo que $P(x)$ no puede ser divisible por $(x-\rfrac{1}{5})$.
	\item Halla el valor de $a,b$ para que $P(x)$ sea divisible por $(x-\rfrac{1}{3})$ y por $(x-\rfrac{1}{2})$.
	\subitem Por el teorema del factor, buscamos:
	\[
	\left\{
		\begin{array}{c}
			P(\rfrac{1}{2}) = 0 \dimplies \frac{6}{8} + \frac{a}{4} + \frac{b}{2} - 1 = 0\\
			P(\rfrac{1}{3}) = 0 \dimplies \frac{6}{27} + \frac{a}{9} + \frac{b}{3} - 1 = 0
		\end{array}\right\}\dimplies ... \quad (a,b) = (-1,-4)
	\]
\end{itemize}

\item\textbf{Ampliación, puesto pero sin corregir} Sea $P(x) = 4x^2+bx+1$, con $b∈ℤ$. 
\begin{itemize}
	\item Sabemos que sus raíces $α_1,α_2$ son fraccionarias y negativas. ¿Cuáles son? ¿Cuánto vale $b$?
	\subitem Por el teorema de las raíces racionales, $α_1 = \rfrac{n_1}{m_1}$, sabemos que $n_1$ divide a $1$. Análogo para $α_2$.

	Por otro lado, sabemos que $m_2$ divide a 4. Las posibilidades son $2,4$, con lo que $α_1,α_2 \in \{\rfrac{1}{2},\rfrac{1}{4}\}$

	Por el teorema del factor, $P(\rfrac{1}{2}) = 1+b\rfrac{1}{2}+1 = 0 \implies b=-4$. 

	Por el teorema del factor, $P(\rfrac{1}{4}) = \rfrac{1}{4}+b\rfrac{1}{4}+1 = 0 \implies b=-2$.

	Si queremos que sea divisible por los 2 factores, b tiene que valer a la vez $4$ y $-2$. Entonces, necesariamente $P(x) = 4(x-\rfrac{1}{2})^2$ o $P(x) = 4(x-\rfrac{1}{4})^2$. 

	Desarrollando la segunda opción, obtenemos como término independiente $\rfrac{1}{4}≠1$, por lo que no es posible. 
	%
	Por otro lado, desarrollando la primera opción obtenemos algo con sentido.

	\[
		4\left(x+\rfrac{1}{2}\right)^2 = 4\left(x^2+x+\rfrac{1}{4}\right) = 4x^2+4x+1 \implies b=4
	\]

\end{itemize}


\item Sea $P(x) = 21x^2+10x-2$. $P(x) + 3 = 21(x+1/3)(x+1/7)$.

\item Factorizar $P(x) = 9x^3-\frac{27}{2}x^2+\frac{13}{2}x-1 = 9·(x-1/2)(x-2/3)(x-1/3)$. Pista (para ahorraros pruebas innecesarias con Ruffini), todas las raíces son fraccionarias y positivas.

\item Factorizar $P(x) = x^7+2x^4+x = x(x^3+1)^2$

\end{enumerate}

\subsection{Ecuaciones}

\subsubsection{Teoría sobre ecuaciones}

\begin{defn}
2 ecuaciones son equivalentes si y solamente si tienen las mismas soluciones.
\end{defn}

\obs Dividir por 0 no mantiene la equivalencia (ver demostración de $0=1$).

\obs
\[
	-20 = -20 \dimplies 25-45 = 16-36 \dimplies 5^2-5·9 = 4^2-4·9 \dimplies 5^2-5·9+\left(\rfrac{9}{2}\right)^2 = 4^2-4·9+\left(\rfrac{9}{2}\right)^2 \dimplies
\]
\[
	\left(5-\rfrac{9}{2}\right)^2 = \left(4-\rfrac{9}{2}\right)^2 \text{\hl{\;\;;\;\;}} 5-\rfrac{9}{2} = 4-\rfrac{9}{2} \dimplies 5=4
\]
Es decir, en general, tomar una raíz no mantiene equivalencia entre ecuaciones (tampoco elevar a una potencia).


\paragraph{Clasificación de ecuaciones}

Las ecuaciones según sus soluciones pueden ser:
\begin{itemize}
	\item Incompatible: no tiene ninguna solución. Ejemplo: $5x=5x+2$
	\item Compatible determinada: tiene un número finito de soluciones. Ejemplo: $3x=6$.
	\item Compatible indeterminada: tiene infinitas soluciones. Ejemplo $2x-\frac{3x-1}{3} = x+\frac{1}{3}$. Solución: $x=λ, ∀λ∈ℝ$.
\end{itemize}


\subsubsection{Ecuación de segundo grado}

$ax^2+bx+c=0$. ¿Alguien sabe de dónde viene la fórmula?

\[
	x^2+\frac{b}{a}x+\frac{c}{a}=0 \dimplies x^2+2\frac{b}{2a}x+\frac{c}{a}=0\dimplies x^2+2\frac{b}{2a}+\left(\frac{b}{2a}\right)^2-\left(\frac{b}{2a}\right)^2+\frac{c}{a} = 0
\]
\[
	\left(x+\frac{b}{2a}\right)^2 = \frac{b^2}{4a^2}-\frac{c}{a} \implies x+\frac{b}{2a} = \sqrt{\frac{b^2-4ac}{4a^2}} \dimplies x = \frac{\sqrt{b^2-4ac}}{2a} -\frac{b}{2a}
\]

\subsubsection{Racionales}

Ecuaciones racionales.

\paragraph{Ejemplos}
\subparagraph{a)}

Este ejemplo no está bien:

\[
	\frac{2x-5}{x-2} = \frac{1}{2x-x^2}+1 \dimplies \frac{2x-5}{x-2} = \frac{-x^2+2x+1}{-x(x-2)} \text{\hl{$\implies$}} -2x^2+5x = -x^2+2x+1\dimplies \]
	\[ \dimplies -x^2+3x-1 = 0 \dimplies x^2-3x+1 = 0 \dimplies x_1=1\wedge x_2=2
\]
¿Son soluciones las 2? El 2 no. La hemos "ganado" en la equivalencia.

\subparagraph{b)}
\[
	\frac{2x}{x-2} + \frac{3x}{x+2} = \frac{7x^2}{x^2-4} \dimplies \frac{2x(x+2)}{(x-2)(x+2)} + \frac{3x(x-2)}{(x+2)(x-2)} = \frac{7x^2}{x^2-4} \dimplies 
\]
\[
	\frac{2x(x+2)+3x(x-2)}{x^2-4} = \frac{7x^2}{x^2-4} \text{\hl{$\implies$}} 2x^2+4x+3x^2-6x=7x^2 \dimplies 5x^2-7x^2-2x = 0 \dimplies 
\]
\[
	x(-x-1) = 0 \dimplies x_1 = 0 \wedge x_2 = -1
\]

\hl{¿Son soluciones las 2?}

Para trabajar: Pág 46, ejer 36.

\paragraph{36,a}
\[
	\frac{1}{1-\frac{1}{x+1}} = \frac{x+1}{x} \dimplies \frac{1}{\frac{x+1-1}{x+1}}=\frac{x+1}{x} \dimplies \frac{x+1}{x} = \frac{x+1}{x} \text{\hl{$\implies$}} x+1=x+1 \dimplies x=λ, ∀λ∈ℝ - \{0,-1\}
\]


\begin{itemize}
	\item Corregir la otra ecuación (36,b). Mientras alguien la hace en la pizarra, trabajamos con: $2x-\frac{3x-1}{3} = x+\frac{1}{4}$. Solución: ecuación incompatible.
\end{itemize}

\paragraph{36,b}
\[
	\frac{1+\displaystyle\frac{x+1}{x-1}}{2-\displaystyle\frac{x-1}{x+1}}=2 \dimplies \frac{\displaystyle\frac{x-1+x+1}{x-1}}{\displaystyle\frac{2x+2-x+1}{x+1}} = 2 \dimplies
\]
\[	
	\frac{\displaystyle\frac{2x}{x-1}}{\displaystyle\frac{x+3}{x+1}}=2 \dimplies 2x^2+2x=2x^2+4x-6 \dimplies 2x=6 \dimplies x=3
\]


Mientras corregimos los 2, para los siguientes \todo{Sin hacer}:

\[
	\frac{3}{x} - \frac{x}{x+2} = \frac{5x-1}{x^2+x-2}
\]

Lo corrigen ellos. El 33.

\subsubsection{Radicales}


\paragraph{Ejemplo:}
\[
	\sqrt{x+1} - \sqrt{x^2-5}=0 \text{\hl{$\implies$}} x+1 = x^2-5 \dimplies (x_0,x_1) = (3,-2)
\]

Comprobamos: $\sqrt{-2+1} = \sqrt{(-2)^2-5} \dimplies \sqrt{-1} = \sqrt{-1}$; -2 no es una solución en los reales.

Por otro lado: $\sqrt{3+1} = \sqrt{3^2-5} \dimplies \sqrt{2}=\sqrt{2}$

\paragraph{Ejercicio:} (creo que está sin corregir)
\[
	\sqrt{x+4}+\sqrt{x-1} = 5 \text{\hl{$\implies$}} (x+4)+(x-1) + 2\sqrt{(x+4)(x-1)} = 25 \text{\hl{$\implies$}} (22-2x)^2 = 4(x^2+3x-4) \dimplies 
\]
\[
	4x^2-88x + 484 = 4x^2+12x-16 \dimplies -100x + 500 = 0 \dimplies x=5
\]

Comprobamos:
\[
	\sqrt{5+4}+\sqrt{5-1} = 3+2 = 5
\]


\paragraph{Ejercicios} 

40 b Truco: aplicar el binomio de Newton.

40 a

	\[
		\sqrt{x^3+2\sqrt{x^3}+1} = 0 \dimplies \sqrt{(\sqrt{x^3}+1)^2} = 0 \text{\hl{$\implies$}} \sqrt{x^3}=-1 \dimplies x=-1
	\]
	Comprobamos: $\sqrt{(-1)^2+2\sqrt{-1^3}+1} = \sqrt{2+2\sqrt{-1}} \notin ℝ$

	El otro camino:
	\[
		\sqrt{x^3+2\sqrt{x^3}+1} = 0 \implies x^3+2\sqrt{x^3}+1 = 0 \dimplies 2\sqrt{x^3} = -x^3-1 \implies 4x^3=x^6+2x^3+1 \dimplies 
	\]
	\[
		x^6-2x^3+1=0 \dimplies t^2-2t+1=0 \dimplies t_1=-1 \implies x^3=t=-1\implies x=\sqrt[3]{-1}=-1
	\]
	Comprobamos: $\sqrt{(-1)^2+2\sqrt{-1^3}+1} = \sqrt{2+2\sqrt{-1}} \notin ℝ$


\subsubsection{Logarítmicas}

Los logaritmos tampoco conservan las equivalencias:

Versión corta:
\[
	(-2)^2 = (2)^2 \text{\hl{$\implies$}} 2\log(-2) = 2\log(2) \dimplies \log(-2)^2 = \log(2^2) \dimplies \log(-2) = \log(2) \dimplies -2=2
\]

Ecuación de ejemplo:

\[
	5\log x=3\log x+2\log 6 \dimplies 2\log x=2\log6 \text{\hl{$\implies$}} x=6
\]


El libro lo hace de otra manera:

\[
	5\log x=3\log x+2\log 6 \dimplies \log x^5=\log36x^3 \text{\hl{$\implies$}} x^5=36x^3 \dimplies
\]
\[
	x^5-36x^3 = 0 \dimplies x^3(x^2-36) = 0 \dimplies x^3(x+6)(x-6) = 0
\]

Aquí aparecen 2 soluciones más que habíamos perdido en la equivalencia, pero no hay problema por las condiciones del logaritmo.

Incluso, habría una tercera manera:

\[
	\log x=t \implies 5t=3t+\log6^2 \dimplies 2t=2\log6 \dimplies t=\log6 \dimplies \log x=\log6 \text{\hl{$\implies$}} x=6
\]

Trabajamos: página 48, ejercicio 45 b,d.

\paragraph{45b}
\[
	\log\frac{2x-2}{x} = 2\log(x-1)-\log x \dimplies \log \frac{2x-2}{x}=\log\frac{(x-1)^2}{x} \cimplies \frac{2(x-1)}{x} = \frac{(x-1)^2}{x} \overset{1}{\cimplies}
	\]
	\[ 
	x-1=2 \dimplies x=3
\]
En 1 hemos simplificado 2 factores. $x$ y $(x-1)$. En esta simplificación podríamos haber perdido soluciones, en concreto, si $0,1$ fueran soluciones no lo obtendríamos. 

En este caso no son solución porque $\log 0$ no existe.

\paragraph{45d}
\[
\frac{\log (4-x)}{\log(x+2)}=2 \cimplies \log(4-x) = \log(x+2)^2 \cimplies 4-x=(x+2)^2 \dimplies 4-x=x^2+4x+4 \dimplies
	\]
	\[ x^2+5x=0 \dimplies x_1=0 \wedge x_2=-5
\]
$x_2=-5$ no es solución porque $\nexists\log(-5+2)=\log(-3)$. Por otro lado, $\frac{\log4}{\log2} = \log_24=2$ cqc.

\paragraph{Sesión 3/10 (en 2017)}
\begin{itemize}
	\item Empezar re-explicando la pérdida de equivalencias. Cuando simplificamos, marcamos en la equivalencia $x\neq-1$. Al final, comprobamos si los valores intermedios son soluciones o no.
	\item Corregimos el 45bd.
	\item Mientras, trabajar el 46 b.
	\item Comentario al 46a (cambio de base), 45c (rango inexistente).
	\item Problema.
	\item Equivalencia exponenciales. Trabajamos 52 abc, mientras corrigen el d.
\end{itemize}

\paragraph{46a} mientras corrigen

\[
	\log_x 3 = \ln \sqrt{3} \dimplies \frac{\ln3}{\ln x}=\ln\sqrt{3} \dimplies \frac{\ln3}{\ln x}=\frac{\ln3}{2} \dimplies \frac{1}{\ln x}=\frac{1}{2} \implies \ln x = 2 \implies e^2=x
\]

\paragraph{46b}
\[
	\log_332+\log_{\rfrac{1}{3}}(6-x) = \log_{\sqrt{3}}x \dimplies \log_332+\frac{\log_3(6-x)}{\log_3\rfrac{1}{3}} = \frac{\log_3x}{\log_3{\sqrt{3}}} \dimplies 
\]
\[
	\log_332-\log(6-x)=2\log_3x\dimplies \log_3\left(\frac{32}{6-x}\right)=\log_3x^2 \cimplies 32=x^2(6-x) \dimplies -x^3+6x^2-32 = 0
\]
\[
	-(-2)^3 + 6(-2)^2-32 = 8+24-32 = 0\implies x_1=-2 \wedge x_2=x_3=4
\]
 
Comprobamos:

\[
	\log_332+\log_{\rfrac{1}{3}}(6-4) = \log_{\sqrt{3}}4 \dimplies \log_332-\log_32=\log_34^2 \dimplies \log_3\frac{32}{2}=\log_316 \;\;\text{   cqc.}
\]




\subsubsection{Exponenciales}

Pregunta: $a^x=a^y \overset{?}{\dimplies} x=y$

$x=y\implies a^x=a^y$ Sí.

$a^x=a^y\implies x=y$ No. Contraejemplo: $1^2=1^3$.  Basicamente, si los logaritmos no mantenían la equivalencia, tampoco lo iban a hacer estas.

Siempre que la base no sea $0,±1$ sí serán equivalentes. ¿Y si tenemos un polinomio como base? Pues como puede ser uno de esos valores, no mantenemos la equivalencia.

Trabajamos 52 entero.

\section{Sistemas de ecuaciones}

\subsection{Sistemas lineales: Gauss}


Clase 1 (9/10/2017 ; 21/10/2019): Plantear un Gauss a ver si lo sacan ellos.

Clase 2: Entrega hojas de teoría impresas y formalización de la idea intuitiva que han tenido + ejemplo de un Gauss CD bien hecho. Deberes: un CD bien hecho.

\paragraph{Discusión de un sistema}

Una vez llegado al \hl{sistema escalonado} pueden darse 3 situaciones:

\begin{itemize}
	\item La ecuación con una única incógnita es incompatible $\implies$ Sistema Incompatible.
	\item Número de incógnitas > número de ecuaciones $\implies$ Compatible indeterminado.
	\item Número de incógnitas = número de ecuaciones, siendo la última ecuación compatible determinada $\implies$ Compatible determinado.
\end{itemize}

Ejercicios.

\begin{itemize}
	\item Corregimos (si quieren) 113 incompatible y CI.
	\item ¿En qué consiste discutir un sistema? Escribir completo los 3 casos
	\item Sistemas con parámetros
\end{itemize}

\subsubsection{Gauss con parámetro (viernes 25)}

Se considera el siguiente sistema lineal:
\[
\left\{\begin{array}{ccccc}
x&+y&+z&=&6\\
x&-y&+2z&=&4\\
2x&-y&+az&=&0
\end{array}\right\}
\]
Se pide:
\begin{itemize}
	\item Discutir el sistema según los valores del parámetro a.
	\item Resolverlo para $a=2$.
\end{itemize}


\hrule

\hl{Escribir:} Para encontrar los casos posibles para discutir, se iguala a 0 el coeficiente de la "z" (o de la incógnita correspondiente).

\begin{itemize}

\item Se considera el siguiente sistema lineal: \[
\left\{\begin{array}{ccccc}
x&+y&-z&=&0\\
2x&-ay&+z&=&0\\
-x&+2y&-2z&=&0
\end{array}\right\}
\]
Se pide:
\begin{itemize}
	\item Discutir el sistema según los valores del parámetro a.
	\item Resolverlo para $a=0$.
\end{itemize}


\item Se considera el siguiente sistema lineal:\[
\left\{\begin{array}{ccccc}
x&+y&+2z&=&2\\
-3x&+2y&+3z&=&-2\\
2x&-ay&-5z&=&-4
\end{array}\right\}
\]
Se pide:
\begin{itemize}
	\item Discutir el sistema según los valores del parámetro a.
	\item Resolverlo para $a=2$.
\end{itemize}


\paragraph{Sesión 24/10/2017}
\begin{itemize}
	\item Corregir el anterior.
	\item Trabajar los 2 siguientes.
\end{itemize}


\item Se considera el siguiente sistema lineal:\[
\left\{\begin{array}{ccccc}
5x&-2y&-az&=&3\\
&-y&+z&=&1\\
x&-y&+z&=&a
\end{array}\right\}
\]
Se pide:
\begin{itemize}
	\item Discutir el sistema según los valores del parámetro a.
	\item Resolverlo para $a=2$.
\end{itemize}

\item Se considera el siguiente sistema lineal:\[
\left\{\begin{array}{ccccc}
ax&+2y&+z&=&0\\
ax&-y&+2z&=&0\\
x&-ay&+2z&=&0
\end{array}\right\}
\]
Se pide:
\begin{itemize}
	\item Discutir el sistema según los valores del parámetro a.
	\item Resolverlo para $a=1$.
\end{itemize}

\item Se considera el siguiente sistema lineal:
\[\left\{\begin{array}{rcrcrcl}
x&+&2y&+&z&=&0\\
x&+&(a+2)y&+&2z&=&0\\
x&+&(2-a)y&+&(a-2)z&=&0
\end{array}\right\}
\]
Se pide:
\begin{itemize}
	\item Discutir el sistema según los valores del parámetro a.
	\item Resolverlo para $a=2$.
\end{itemize}

\end{itemize}

\subsection{Sistemas no lineales}

Ejercicios: 114cf (f es interesante),115acdf

\[
\left\{
	\begin{array}{c}
		x^2-2xy+y^2 = 1\\
		x^2-y^2 = 12\\
	\end{array}
\right\}
\]