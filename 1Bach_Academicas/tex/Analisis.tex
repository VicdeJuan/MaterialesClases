
\chapter{Análisis}


\section{Funciones}

El objetivo de esta función es ser capaz de representar gráficas de funciones dada su expresión algebraica. 
%
Para ello vamos a ir desarrollando diferentes herramientas que nos van a resultar tremendamente útiles para este propósito.

\subsection{Concepto}

Lo primero es tener claro qué es una función.

\begin{defn}[Función]
Una función asigna a cada elemento de un conjunto $X$, un único elemento de otro, $Y$. Escribimos $f: X \to Y; x\in X\to f(x)\in Y$
\end{defn}

Ejemplos hablados. Añade 2 de tu propia cosecha.
\begin{itemize}
	\item A cada uno le asigno su edad.
	\item A cada uno su estatura.
	\item A cada uno su brazo.
	\item A un número su doble.
	\item A un número su raíz cuadrada.
\end{itemize}

\begin{itemize}
	\item A un número complejo su conjugado.
	\subitem Sí, pero sólo vamos a hablar de \concept{funciones reales de variable real}. 
\end{itemize}

Escribimos:  $f: D(f)\subset\real \to I\subset\real; x\to f(x)$

\textit{Como curiosidad hay funciones de variable ``funcional'', es decir, que asocia funciones con otras cosas. Este tipo de cosas se estudian en matemáticas.}


\subsubsection{Dominio y recorrido} Condiciones generales de cada función:

$f:\real\to\real\quad\quad D(f) = {x / \exists f(x)}$

Ejemplos:
\begin{itemize}
	\item $f_1(x) = 3x^2+2$
	\item Racionales $f_2(x) = \frac{x+2}{x-1}$
	\item Radicales $f_3(x) = \sqrt{x^2-9}$
	\item Radicales $f_4(x) = \sqrt[3]{x^2-16}$
	\item Exponenciales $f_5(x) = e^x$
	\item Logaritmos $f_6(x) = \log(x+2)$
	\item Trigonométricas $f_7(x) = \cos(x)$
	\item Trigonométricas $f_8(x) = \sin(x)$
	\item Combo: $f_9(x) = \frac{\log(x-5)}{x+7}$
	\item Combo: $f_{10}(x) = \frac{\sqrt{5-x}}{x+7}$
\end{itemize}



\subsubsection{Clasificacion de funciones}

\begin{defn}[Inyectividad]
$f(x) : D(f) \to \real$ inyectiva $\dimplies\forall a,b\in D(f) a\neq b \implies f(a) \neq f(b)$
\end{defn} 

\begin{defn}[Sobreyectividad]
	$f(x) : D(f) \to \real$ sobreyectiva $\dimplies \forall b\in\real \exists a\in D(f) \tlq f(a) = b$

	Es decir, una función es sobreyectiva si su imagen son todos los números reales.
\end{defn}


\begin{defn}[Biyectividad] 
	Inyectividad y sobreyectividad
\end{defn}

\paragraph{Ejemplos:} Razonando gráficamente.

\begin{itemize}
	\item $f(x) = x^3$ (biyectiva)
	\item $f(x) = x^2$ (nada)
	\item Dibujo función cúbica sí sobreyectica, no inyectiva.
	\item $f(x) = e^x$ (inyectiva)
	\item $f(x) = \log(x)$ (inyectiva)
\end{itemize}

\concept{Gráfica de una función:} $G(f) = {\left(a,f(a)\right)}$

\subsection{Operaciones con funciones}
\begin{itemize}
	\item Suma, resta $(f\pm g)(x) = f(x) \pm g(x)$
	\subitem $D(f\pm g) = D(f)\cap D(g)$.
	\item Producto
	\subitem $D(f·g) = D(f)\cap D(g)$.
	\item Cociente 
	\subitem $D\left(\rfrac{f(x)}{g(x)}\right) = D(f)\cap D(g) \setminus \{x/g(x) = 0\}$
	\item Composición $(g\circ f)(x) = g\left(f(x)\right)$ y se lee ``f compuesta con g''.
	\subitem El dominio se recalcula.
\end{itemize}

\concept{Función inversa} $f^{-1}(x):Y\to X$ y cumple $(f\circ f^{-1})(x) = x$

\paragraph{Ejemplo:} Comprueba si son inversas
\begin{itemize}
	\item $f(x) = x^3-1$; $f^{-1}(x) = \sqrt[3]{x+1}$ (sí)
	\item $f(x) = x^2+1$; $f^{-1}(x) = \sqrt{x}$ (no)
\end{itemize}

\paragraph{Ejemplo:} Calcula la función inversa de $f(x) = \frac{2x+1}{3}$ y de $g(x) = \frac{-1}{3x}$ 

\paragraph{Construir funciones inversas.}

\obs Una función no inyectiva no puede tener inversa.

\section{Continuidad}
\subsection{Límites}

\paragraph{Introducción}

Con el excel y geogebra. Definición intuitiva de límite. \textit{Reto: escribe tu propia definición de límite y vete hablándola conmigo.}

\begin{itemize}
	\item Excel: valor del límite: me voy acercando sin llegar a tocar.
	\item Geogebra: mismo gráficamente. Hacer zoom hasta la saciedad.
\end{itemize}

\begin{defn}[Límite en un punto finito]
Sea $f(x):\real\to\real$.

\[\lim_{x\to a} f(x) = b\]

El límite de la función $f(x)$ en un punto $x=a$ es el valor $b$ al que se aproxima la función cuando la variable se aproxima al punto, sin \hl{nunca} alcanzarlo.
\end{defn}

Ejemplos:
\begin{itemize}
	\item $\lim_{x\to 3} x^2 + 1 = 10$
	\item $\lim_{x\to 0} \frac{\sin x}{x} = 1$ (cálculo con calculadora) ¿Existe la función en el punto? Nooooo ¿Existe el límite? Siiiii.
	\item $\lim_{x\to 0} \frac{1}{x^2}$ (cálculo con calculadora). ¿Existe la función en el punto? Nooooo ¿Existe el límite? Puede existir. Decimos que es $\pm\infty$.

\obs Puede ser que la función no se acerque a ningún valor, sino que cada vez se haga más grande. En ese caso el resultado del límite será $+\infty$.

\obs Puede ser que la función no se acerque a ningún valor, sino que cada vez se haga más pequeño. En ese caso el resultado del límite será $-\infty$.
\end{itemize}


\paragraph{Infinito, ¿concepto o valor?}

\subparagraph{Aritmética del infinito}

\begin{defn}[Límite en el infinito]
Sea $f(x):\real\to\real$.

\[\lim_{x\to \pm\infty} f(x) = b\]

El límite de una función $f(x):\real\to\real$ en el infinito ($\pm\infty$) es el valor $b$ al que se aproxima la función cuando la variable toma valores \hl{arbitrariamente} grandes (o pequeños).
\end{defn}

\obs Se cumplen las mismas observaciones de antes.

\begin{example}
	\[\lim_{x\to\infty}x=\infty\]
	\[\lim_{x\to\infty}\frac{1}{x}=0\]
\end{example}

\begin{defn}[Exsitencia del límite en un punto finito] 
	El límite existe si existen y son iguales los límites laterales. 
\end{defn}


\paragraph{Propiedades de los límites} 

Sea $a\in\real\cup\{\infty,-\infty\}$, $f,g:\real\to\real$ y que dado $a$, $\exists\lim_{x\to a}f(x) \;;\; \exists\lim_{x\to a}g(x)$
\begin{itemize}
	\item $\lim_{x\to a} \left(f(x) \pm g(x)\right) = \lim_{x\to a} f(x) \pm \lim_{x\to a} g(x)$
	\item $\lim_{x\to a} \left(f(x) · g(x)\right) = \lim_{x\to a} f(x) · \lim_{x\to a} g(x)$
	\item $\lim_{x\to a} \left(f(x) \pm g(x)\right) = \lim_{x\to a} f(x) \pm g(x)$
	\item $\lim_{x\to a} \left(\frac{f(x)}{g(x)}\right) = \lim_{x\to a} f(x) \pm g(x)$
	\item $\lim_{x\to a} \left(f(x)\right)^{g(x)} = \left(\lim_{x\to a} f(x)\right)^{\lim_{x\to a} g(x)}$	
	\item $\lim_{x\to a} (f\circ g)(x) = ?$	
\end{itemize}

\begin{example}
\[\lim_{x\to 0}\frac{1}{x}\]
 \[\lim_{x\to 4}\left(\frac{x+1}{x}\right)^{\frac{1}{x-4}}\]
\end{example}

\paragraph{Indeterminaciones}

\begin{itemize}
	\item $k/0$
	\subitem Vistas (la $h$). Son indeterminaciones a medias, ya que sólo tienen 2 posibilidades: $\pm∞$
	\item $\infty/\infty$ (potencia más alta del denominador)
	\subitem $\displaystyle\lim_{x\to∞}\frac{x^2+5x+6}{x+\sqrt{x}} = \lim_{x\to∞}\frac{\frac{x^2}{x}+5\frac{x}{x}+\frac{6}{x}}{\frac{x}{x}+\sqrt{\frac{x}{x^2}}} = \lim_{x\to∞}\frac{x+5+\rfrac{6}{x}}{1+\frac{1}{\sqrt{x}}} = ∞$
	\item $0/0$ (simplificar)
	\subitem $\displaystyle\lim_{x\to1}\frac{x^2-5x+4}{x-1}$
	\subitem $\displaystyle\lim_{x\to1}\frac{x-1}{\sqrt{x-1}}$
	\item $\infty-\infty$
	\subitem $\displaystyle\lim_{x\to-∞}x^2+x = \lim_{x\to-∞} x(x+1) = (-∞)·(-∞) = ∞$
	\subitem El problema viene con 

	$\displaystyle\lim_{x\to∞}\sqrt{x^2+1}-x = \lim_{x\to∞}\frac{(\sqrt{x^2+1}-x)(\sqrt{x^2+1}+x)}{\sqrt{x^2+1}+x} = \lim_{x\to∞} \frac{x^2+1-x^2}{\sqrt{x^2+1}+x} = 0$
	\item $1^{+\infty}$
		\subitem Estas indeterminaciones aparecen cuando $\displaystyle\lim_{x\to a}f(x)^{g(x)}\;\lim_{x\to a}f(x) = 1\;\lim_{x\to a}g(x) = ∞$ y se resuelven utilizando:
		\[
			\displaystyle\lim_{x\to a}f(x)^{g(x)} = e^λ, \text{ donde } λ = \lim_{x\to a} (f(x)-1)g(x)
		\]
		\subitem $\displaystyle\lim_{x\to \infty}\left(\frac{x-1}{x}\right)^x = e^λ$ donde $λ=\displaystyle\lim_{x\to∞}\left(\frac{x-1}{x}-1\right)·x = -1$
\end{itemize}

\subsection{Continuidad}

\begin{defn}[Continuidad\IS en un punto]

Una función $f(x):ℝ\toℝ$ es continua en un punto $a$ si y sólo si se cumplen:
\begin{itemize}
	\item $\displaystyle∃\lim_{x\to a}f(x)$
	\item $\displaystyle∃f(a)$
	\item $\displaystyle\lim_{x\to a}f(x) = f(a)$
\end{itemize}

\obs Normalmente diremos: si existen y son iguales el límite y el valor en el punto.
\end{defn}

\paragraph{Tipos de discontinuidades}

Explicar gráficamente los casos.s

\begin{example}
\begin{itemize}
	\item ¿Es continua la función $f(x)= \frac{x^2-1}{x-1}$ en $x=2$?
	\item ¿Es continua la función $f(x)$ en $x=1$? 
	\item ¿Es continua la función $f(x) = \sqrt{x}$ en $x=0$?
	\item Funciones definidas a trozos (hasta aquí).
\end{itemize}
\end{example}

\begin{defn}[Continuidad\IS en un intervalo abierto]
Una función $f(x):ℝ\toℝ$ es continua en $(a,b)\inℝ$ si es continua $∀c\in(a,b)$.
\end{defn}

\begin{itemize}
	\item ¿Es continua la función $f(x) = \frac{x^2-1}{x-1}$ en $(2,∞)$?
	\item ¿Es continua la función $f(x) = \sqrt{x}$ en $(0,∞)$?
\end{itemize}



\begin{example}
Estudia la continuidad de la función $f(x) = \frac{\sqrt{x}}{x-4}$

\obs Una función sólo puede ser continua en su dominio. En este caso: $D(f) = \{x\inℝ\tq x≥0\}-\{x\in\real\tq x-4=0\} = [0,∞)-\{4\}$

La función es continua en $(0,4) ∪ (4,∞)$

\end{example}

\section{Derivabilidad}

\begin{defn}[Derivada\IS en un punto]
Sea $\appl{f}{ℝ}{ℝ}$ una función continua. Se define la derivada de $f$ en el punto $a$ como:

\[
	f'(a) = \lim_{x\to a}\frac{f(x)-f(a)}{x-a}
\]
\end{defn}

\obs $f$ \concept[Función\IS derivable]{derivable} en $x=a \dimplies \exists f'(a)$
\obs A la función inicial la llamaremos primitiva.

\begin{example}
Cálculo de la derivada de $f(x) = x^2+2x-1$ en $x=3$.

\[
	f'(3) = \lim_{\x\to3}\frac{x^2+2x-1 - (3^2+2·3-1)}{x-3} = \lim_{\x\to3}\frac{x^2+2x-15}{x-3} = \lim_{\x\to3}\frac{(x-3)(x+5)}{x-3} = \lim_{\x\to3}(x+5) = 8
\]

\end{example}

\begin{example}
Cálculo de la derivada de $f(x) = |x|$ en $x=0$.

\[
	f'(0) = \lim_{x\to0}\frac{|x| - |0|}{x-0} = \lim_{x\to0}\frac{|x|}{x} = \left\{\begin{array}{l}\displaystyle\lim_{x\to0^+}\frac{x}{x} = 1 \\ \displaystyle\lim_{x\to0^-}\frac{-x}{x}=-1\end{array}\right\}\implies \nexists\lim_{x\to0}f(x)
\]

Conclusión: La función $f(x)$ no es derivable en $x=0$. ¿Es continua? Sí.
\end{example}

\begin{example}
Cálculo de la derivada de $f(x) = \frac{1}{x}$ en $x=0$.

\[
	f'(0) = \nexists\lim_{x\to0}\frac{\rfrac{1}{x}-\rfrac{1}{0}}{x-0}
\]

No es derivable. ¿Es continua? No.
\end{example}

\obs \textbf{Derivable} $\implies$ \textbf{Continua}. \textit{Para que una función sea derivable en un punto es necesario que sea continua en ese punto}\footnote{Está en su tabla de derivadas}.

¿En física habéis derivado polinomios? ¿Cuál sería la "derivada" del polinomio? $f'(x) = 2x+2$. ¿Cuánto vale $f'(3)$? $f'(3) = 2·3+2=8$. ¿Casualidad?

Pero... ¿porqué esto es cierto? ¿De dónde sale ese $3x^2+2$? Esto es a lo que llamamos la función derivada:

\begin{defn}[Función\IS derivada]
Sea $\appl{f}{ℝ}{ℝ}$ una función continua. Se define la función derivada de $f$ como la función que a cada punto le asigna el valor de su derivada.

\[
	f'(x) = \lim_{h\to 0}\frac{f(x+h)-f(x)}{h}
\]
\end{defn}

\begin{example}
Cálculo de la función derivada $f'(x)$ de un polinomio, por ejemplo $f(x)=x^2+2x-1$

\[
	f'(x) = \lim_{h\to0}\frac{f(x+h)-f(x)}{h} = \lim_{h\to0}\frac{(x+h)^2+2(x+h)-1 - x^2+2x-1}{h} =
\]
\[
	\lim_{h\to0}\frac{(x^2+2xh+h^2+2x+2h-1 - x^2-2x+1}{h} = \lim_{h\to0}\frac{2xh+h^2+2h}{h} =
\]
\[
	\lim_{h\to0}\frac{h(2x+h+2)}{h} = \lim_{h\to0} 2x+h+2) = 2x+2
\]

Conclusión: la función derivada de $f(x) = x^2+2x-1$ es $f'(x) = 2x+2$.
\end{example}

\paragraph{Propiedades de la derivada:}
\begin{prop}[Cálculo operativo]
	Sean $f, g$ derivables en $a$. Entonces
	\begin{itemize}
		\item $(k·f(x))' = k·f'(x) ∀k\in\real$
		\item $(f\pm g)'(a)=f'(a)\pm g'(a)$
		\item $(fg)'(a)=f'(a)g(a)+f(a)g'(a)$
		\item Si $g(a)\neq 0 $, $\left(\frac{1}{g}\right)'(a)=\frac{-g'(a)}{(g(a))^2}$
		\item Si $g(a)\neq 0$, $\left(\frac{f}{g}\right)'(a)=\frac{f'(a)g(a)-f(a)g'(a)}{(g(a))^2}$
		\item $(g\circ f)'(a)= g'(f(a))f'(a)$
	\end{itemize}
\end{prop}

\hl{Explicación de la tabla de derivadas}

\paragraph{Regla de la cadena}

\paragraph*{A practicar:}
\begin{itemize}
	\item $\displaystyle f(x) = x\sqrt[3]{x^2}$
	\item $\displaystyle f(x) = x·e^x$
	\item $\displaystyle f(x) = x·\sen(x)$
	\item $\displaystyle f(x) = x·\ln(x)$
	\item $\displaystyle f(x) = \sen(x)·\cos(x)$
	\item $\displaystyle f(x) = \frac{x^2+1}{x}$
	\item $\displaystyle f(x) = (x^2+2x)·\sen(x)$
	\item $\displaystyle f(x) = (e^x - x)·\ln{x}$
	\item $\displaystyle f(x) = \sqrt{x^2+x}$
	\item $\displaystyle f(x) = (\arcsen{x})^3$
	\item $\displaystyle f(x) = \ln(4x)$
	\item $\displaystyle f(x) = (\cos{x})^2 = \cos^2{x}$
	\item $\displaystyle f(x) = \sen{(3x^2)}$
	\item $\displaystyle f(x) = \cos{(x^2+1)} $
	\item $\displaystyle f(x) = \tg{(x^2-3x)}$
	\item $\displaystyle f(x) = \sen{\sqrt{x^2+3x}} $
	\item $\displaystyle f(x) =  \cos{\frac{x-1}{x}}$
	\item $\displaystyle f(x) = \tg{\sqrt{x-1}} $
	\item $\displaystyle f(x) = -\sen{\frac{x}{-x^4+x-1}} $
	\item $\displaystyle f(x) = \tg{\frac{2}{\sqrt{1-x}}} $
\end{itemize}

\subsection{Interpretación geométrica de la derivada}



\section{Integrales inmediatas}



\section{Estudio sistemático de una función}

\begin{itemize}
	\item Dominio, siempre.
	\item Puntos de corte con los ejes.
	\item Simetría.
	\item Asíntotas, repaso de 4º.
	\item Monotonía.
	\item Curvatura.
\end{itemize}

\subsection{Monotonía}

Diferenciar los intervalos en los que la función crece de aquellos en los que la función decrece. También, ¿dónde están los máximos y los mínimos?

\begin{itemize}
	\item Ejemplo: dada la función $f(x) = 3x^2+2x+5$, encuentra sus extremos relativos.
	\item Los extremos relativos se encuentran en los puntos donde la derivada se hace 0. ¿Por qué? Interpretación gráfica.
	\item ¿Cómo distinguir máximo de mínimo? En este caso, la parábola va hacia arriba, por lo que debe ser mínimo. 
	\subitem Por un lado negativa (función decrece), por otro positiva (función creciente).
	\subitem Segunda derivada positiva, lo que marca que es un mínimo.
	\item Concavidad: segunda derivada positiva -> mínimo, convexa.
\end{itemize}

\begin{itemize}
	\item $f'(x) > 0 \implies f(x)$ crece. 
	\item $f'(x) < 0 \implies f(x)$ decrece. 
	\item $f'(x) = 0 \implies $ extremo relativo. ¿Máximo, mínimo?
	\item $f''(x) > 0 \implies $ mínimo y convexa-contenta (desde el semieje negativo de $y$).
	\item $f''(x) < 0 \implies $ máximo y cóncava desde el semieje negativo de $y$.
	\item 
\end{itemize}

Clase del jueves:

\begin{itemize}
	\item Corregir: demuestra que el vértice de la parábola es $\rfrac{-b}{2a}$
	\item Halla las rectas tangentes a la función $f(x)$ en los máximos (no es necesario intervalos de crecimiento y decrecimiento).
	\[
		f(x) = 3x^4-4x^3-36x^2
	\]
	\item Estudia la monotonía (intervalos de crecimiento, decrecimiento y extremos relativos y absolutos) de $f(x) = x^4-2x^2$
	
	\item Estudia sistemáticamente la función: $f(x) = \frac{2(x-2)+1}{(x-2)^2} = \frac{2x-1}{x^2-4x+4}$
	\subitem Punto de corte eje x: $(1.5,0)$, eje y: $(0,-0.75)$, mínimo absoluto: $(1,-1)$
	
	\item Estudia las asíntotas de $f(x) =\frac{x^3-4x^2+x+6}{2x^3-14x^2+32x-24}$
	
	\item Deriva $f(x) = \tan(x^2-3x)$
\end{itemize}
