

\chapter{Segunda Evaluación}
\section{Derivadas}
\subsection*{Clase 1} Función derivada y ¿De dónde sale la tabla de las derivadas?

\begin{itemize}
\item  ¿De dónde sale la tabla de las derivadas? Ejemplo $f(x) = x^2+x$
\[
\lim_{h\to 0} \frac{f(x+h)-f(x)}{h} = \lim_{h\to 0} \frac{(x+h)^2+(x+h) - x^2-x}{h} = \lim_{h\to 0}\frac{x^2+2xh+h^2+x+h-x^2-x}{h} =\]
\[ \lim_{h\to 0}{2xh+h+h^2}{h} = \lim_{h\to 0}\frac{h·(2x+1+h)}{h} = 2x+1
\]

\item  ¿De dónde sale la tabla de las derivadas? Ejemplo $f(x) = \sen(x)$
\[
\lim_{h\to 0} \frac{f(x+h)-f(x)}{h} = \lim_{h\to 0} \frac{\sen(x+h) - \sen(x)}{h} = ??? = \cos(x)
\]

Demostración gráfica con Geogebra.

\end{itemize}

\subsection*{Clase 3} Dominio de derivabilidad, funciones no derivables.

\[
\left[f(x) = \frac{x^2-5x+6}{x-2}\right] == \left[g(x) = x-3\right]
\]

No son exactamente la misma función. Sí tienen el mismo valor de los límites, pero una función es continua en $\real$ y la otra en $\real-\{2\}$, con lo que no son iguales.
\\
Algo parecido pasa con las derivadas.
