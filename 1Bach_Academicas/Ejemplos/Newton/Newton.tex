\documentclass[palatino,nosec]{Docencia}


\title{Ejemplos Binomio de Newton}
\author{}
\date{17/18}


% Paquetes adicionales

\usepackage[author={Víctor de Juan, 2017}]{pdfcomment}

\makeatletter
\newcommand{\annotate}[2][]{%
\pdfstringdef\x@title{#1}%
\edef\r{\string\r}%
\pdfstringdef\x@contents{#2}%
\pdfannot
width 2\baselineskip
height 2\baselineskip
depth 0pt
{
/Subtype /Text
/T (\x@title)
/Contents (\x@contents)
}%
}
\makeatother



\usepackage{eso-pic}
\newcommand\BackgroundPic{%
\put(0,0){%
\parbox[b][\paperheight]{\paperwidth}{%
\vfill
\centering
\includegraphics[width=\paperwidth,height=\paperheight,%
keepaspectratio]{../../../BWLogo.jpeg}%
\vfill
}}}





\begin{abstract}
Este documento contiene algún ejemplo de utilización del binomio de Newton.

\nota{Estos ejemplos no están exentos de erratas. En caso de descubrir alguna, por favor, comunicarlas al autor.}
\end{abstract}

% --------------------
\newcommand{\cimplies}{\text{\hl{$\implies$}}}

\begin{document}
\pagestyle{plain}
\maketitle

\AddToShipoutPicture{\BackgroundPic}

\begin{problem}

Sea \[P(x) = \left( 3x^2 - \frac{2}{x} \right)^9\]

\ppart Calcula el término independiente de $P(x)$.
\ppart ¿Tiene el polinomio algún término de grado $2$?
\ppart (Para profundizar) ¿Cuál es el coeficiente principal?

\solution

\spart

Buscamos el término del desarrollo del binomio que tenga $x^0$.

Por el binomio de Newton, sabemos que $P(x)$, en este caso, estará formado por sumandos de la forma:

\[
	\binom{9}{k} (3x^2)^k·\left(\text{\hl{\;--\;}}\frac{2}{x}\right)^{9-k} \overset{(1)}{=} \text{\hl{\;--\;}}\binom{9}{k}·3^k·2^{9-k} · \frac{x^{2k}}{x^{9-k}}
\]

\textit{(1): Operando.}

Buscamos $k$ para que el exponente de la $x$ sea 0, es decir:

\[-\binom{9}{k}·3^k·2^{9-k} · \frac{x^{2k}}{x^{9-k}} = T·x^0\]

donde $T$ es el término independiente que buscamos.

\[
	\frac{x^{2k}}{x^{9-k}} = x^0 \implies 2k-(9-k) = 0 \dimplies 2k+k=9 \dimplies k=3
\]

Si $k=3$, calculamos el término independiente

\[
	-\binom{9}{3}·3^3·2^{9-3} · \frac{x^{2·3}}{x^{9-3}} = -\frac{9!}{6!3!}·3^3·2^6·\frac{x^6}{x^6} = 145152
\]

\spart Buscamos el desarrollo del binomio que tenga $x^2$, si puede ser.

Por el binomio de Newton, sabemos que $P(x)$, en este caso, estará formado por sumandos de la forma:

\[
	\binom{9}{k} (3x^2)^k·\left(-\frac{2}{x}\right)^{9-k} \overset{(1)}{=} -\binom{9}{k}·3^k·2^{9-k} · \frac{x^{2k}}{x^{9-k}}
\]


\textit{(1): Operando.}


Buscamos $k$ para que el exponente de la $x$ sea 2, es decir:

\[-\binom{9}{k}·3^k·2^{9-k} · \frac{x^{2k}}{x^{9-k}} = T·x^0\]

donde $T$ es el término independiente que buscamos. 

\[
	\frac{x^{2k}}{x^{9-k}} = x^2 \implies 2k-(9-k) = 2 \dimplies 2k+k=11 \dimplies k=\frac{11}{3}
\]

El desarrollo de este binomio no tiene un término de grado 2, ya que $k$ sólo puede ser un número entero.

\spart El coeficiente principal es el coeficiente que acompaña a la potencia de mayor grado. En este caso, la potencia de mayor grado se obtendrá cuando:

\[
\binom{9}{9}(3x^2)^9·\left(-\frac{2}{x}\right)^0 = 1·3^9·x^{18}
\]

Entonces, el coeficiente principal será $3^9 = 19683$.

\end{problem}




\end{document}