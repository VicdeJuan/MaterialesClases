\documentclass[palatino,nosec]{Docencia}


\title{Ejemplos resueltos del método de Gauss}
\author{}
\date{17/18}


% Paquetes adicionales

\usepackage[author={Víctor de Juan, 2017}]{pdfcomment}

\makeatletter
\newcommand{\annotate}[2][]{%
\pdfstringdef\x@title{#1}%
\edef\r{\string\r}%
\pdfstringdef\x@contents{#2}%
\pdfannot
width 2\baselineskip
height 2\baselineskip
depth 0pt
{
/Subtype /Text
/T (\x@title)
/Contents (\x@contents)
}%
}
\makeatother



\usepackage{eso-pic}
\newcommand\BackgroundPic{%
\put(0,0){%
\parbox[b][\paperheight]{\paperwidth}{%
\vfill
\centering
\includegraphics[width=\paperwidth,height=\paperheight,%
keepaspectratio]{../../../BWLogo.jpeg}%
\vfill
}}}





\begin{abstract}
Este documento contiene algunas resoluciones de sistemas lineales de 3 incógnitas con 3 ecuaciones utilizando el método de Gauss (sin utilizar matrices).

\nota{Estos ejemplos no están exentos de erratas. En caso de descubrir alguna, por favor, comunicarlas al autor.}
\end{abstract}

% --------------------
\newcommand{\cimplies}{\text{\hl{$\implies$}}}

\begin{document}
\pagestyle{plain}
\maketitle

\AddToShipoutPicture{\BackgroundPic}

\newpage
\begin{problem}
\textbf{Libro ejercicio 112, a}

Resuelve el siguiente sistema de ecuaciones:

\[
\left\{\begin{array}{lccccc}
e_1: &x&+3y&-2z&=&6\\
e_2: &2x&+3y&-2z&=&8\\
e_3: &4x&+2y&-6z&=&6
\end{array}\right\}
\]

\solution

Lo primero, reordenar o simplificar ecuaciones. Vemos que las incógnitas $y$ y $z$ van a ser más fáciles de eliminar, al menos en las 2 primeras ecuaciones. Además, la tercera se puede simplificar.


\[
\left\{\begin{array}{lccccc}
e_1: &x&+3y&-2z&=&6\\
e_2: &2x&+3y&-2z&=&8\\
e_3: &4x&+2y&-6z&=&6
\end{array}\right\}
\dimplies
\left\{\begin{array}{lccccc}
e_1: 3y&-2z&+&x&=&6\\
e_2: 3y&-2z&+&2x&=&8\\
e_3: y&-3z&+&2x&=&3
\end{array}\right\}
%%
%%
\overset{(1)}{\dimplies}
\]
\[
%%
%%
\left\{\begin{array}{lccccc}
e_1: 3y&-2z&+&x&=&6\\
e_2: &&&x&=&2\\
e_3: y&-3z&+&2x&=&3
\end{array}\right\}
%%
%%
\dimplies
%%
%%
\left\{\begin{array}{lccccc}
e_1: 3y&-2z&+&x&=&6\\
e_2: y&-3z&+&2x&=&3\\
e_3: &&&x&=&2
\end{array}\right\}
%%
%%
\overset{(2)}{\dimplies}\]
\[
%%
%%
\left\{\begin{array}{lccccc}
e_1: 3y&-2z&+&x&=&6\\
e_2: &+7z&-&5x&=&-3\\
e_3: &&&x&=&2\\
\end{array}\right\}
\]


\paragraph{1)}
$e_2 = e_2-e_1$

\[
\begin{array}{rccccc}
e_2: 3y&-2z&+&2x&=&8\\
e_1: 3y&-2z&+&x&=&6\\
\hline
e_2: &&&x&=&2\\
\end{array}
\]	


\paragraph{2)} $e_2 = e_1-3e_2$

\[
\begin{array}{rccccc}
e_1: 3y&-2z&+&x&=&6\\
e_2: -3y&+9z&-&6x&=&-9\\
\hline
e_2: &+7z&-&5x&=&-3\\
\end{array}
\]	



\subparagraph{Discusión:} El sistema es compatible determinado, porque es un sistema escalonado con ecuaciones compatibles.

\[
\left\{\begin{array}{lccccc}
e_1: 3y&-2z&+&x&=&6\\
e_2: &+7z&-&5x&=&-3\\
e_3: &&&x&=&2\\
\end{array}\right\}
\]


\subparagraph{Resolución} 

\[e_3: x=2\]

Utilizando $x=2$ en $e_2$:

\[e_2: 7z-5x = -3 \dimplies 7z = -3+5x = -3+5·2 \dimplies 7z=7 \dimplies z=1\]

Utilizando $z=1$, $x=2$ en la ecuación que nos falta ($e_1$):

\[
e_1: 3y-2z+x=6 \dimplies y=\frac{6+2z-x}{3} = \frac{6+2-2}{3} = 2
\]


Solución: $(x,y,z) = (2,2,1)$.

\paragraph{Comprobación}

Sustituimos los valores obtenidos en el sistema inicial:


\[
\left\{\begin{array}{lccccl}
e_1: &x&+3y&-2z&=&6 \to 2+2·3-2 = 6\\
e_2: &2x&+3y&-2z&=&8 \to 2·2 + 2·3-2·1 = 4+6-2 = 8\\
e_3: &4x&+2y&-6z&=&6 \to 4·2+2·2-6·1 = 8+4-6 = 6  
\end{array}\right\} \begin{array}{c}\\\\\\\\\text{cqc}\end{array}
\]
\end{problem}


\newpage
\begin{problem}
\textbf{Libro ejercicio 112, c}

Resuelve el siguiente sistema de ecuaciones:

\[
\left\{\begin{array}{lccccc}
e_1: &	2x &	+	y &		- 2z &	= & 8 \\
e_2: &	2x &	-	4y &	+ 3z &	= & -2 \\
e_3: &	4x &	-	y &		+ 6z &	= & -4 
\end{array}\right\}
\]

\solution


\[
\left\{\begin{array}{lccccc}
e_1: &	2x &	+	y &		- 2z &	= & 8 \\
e_2: &	2x &	-	4y &	+ 3z &	= & -2 \\
e_3: &	4x &	-	y &		+ 6z &	= & -4 
\end{array}\right\}
%%
%%
\overset{(1)}{\dimplies}
%%
%%
\left\{\begin{array}{lccccc}
e_1: &	2x &	+	y &		- 2z &	= & 8 \\
e_2: &	   &	-	5y &	+ 5z &	= & -10\\
e_3: &	   &	-	3y &	+ 10z &	= & -20\\
\end{array}\right\}
%%
%%
\dimplies\]
\[
%%
%%
\left\{\begin{array}{lccccc}
e_1: &	2x &	+	y &		- 2z &	= & 8 \\
e_2: &	   &	-	y &		+  z &	= & -2\\
e_3: &	   &	-	3y &	+ 10z &	= & -20\\
\end{array}\right\}
\overset{(2)}{\dimplies}
\left\{\begin{array}{lccccc}
e_1: &	2x &	+	y &		- 2z &	= & 8 \\
e_2: &	   &	-	y &		+  z &	= & -2\\
e_3: &	   &	-	3y &	+ 10z &	= & -20\\
\end{array}\right\}
\]


\paragraph{1)} $e_2 = e_2-e_1$ y $e_3 = e_3-2·e_1$

\[
\begin{array}{rccccc}
e_2: &	2x &	-	4y &	+ 3z &	= & -2 \\
e_1: &	2x &	+	y  &	- 2z &	= & 8 \\
\hline
e_2: &	   &	-	5y &	+ 5z &	= & -10\\
\end{array}
\]	


\[
\begin{array}{rccccc}
e_3:   &	4x &	-	y   &	+ 6z &	= & -4 \\
2·e_1: &	4x &	+	2y  &	- 4z &	= & 16 \\
\hline
e_3:   &	   &	-	3y &	+ 10z &	= & -20\\
\end{array}
\]	


\paragraph{2)} $e_3 = e_3-2·e_2$

\[
\begin{array}{rccccc}
e_3: &	   &	-	3y &	+ 10z &	= & -20\\
3·e_2: &   &	-	3y &	+  3z &	= & -6\\
\hline
e_3: &	   &		   &	+  7z & = &-14
\end{array}
\]	


\paragraph{Discusión y resolución}

\[e_3: 7z=-14 \dimplies z=-2\]


Utilizando $z=-2$ en $e_2$:
\[e_2: -y+z=-2 \dimplies -y + (-2) = -2 \dimplies -y-2=-2 \dimplies y=0\]

Utilizando $z=-2$, $y=0$ en la ecuación que nos falta ($e_1$):

\[
e_1: 2x	+ y - 2z = 8 \dimplies 2x + 0 - 2·(-2) = 8 \dimplies 2x=4 \dimplies x=2
\]


Solución: $(x,y,z) = (2,0,-2)$.

\paragraph{Comprobación}

Sustituimos los valores obtenidos en el sistema inicial:

\[
\left\{\begin{array}{lccccl}
e_1: &	2x &	+	y &		- 2z &	= & 8 \to 2·2 + 0 - 2·(-2) = 4+4 = 8\\
e_2: &	2x &	-	4y &	+ 3z &	= & -2 \to 2·2 - 4·0 + 3·(-2) = 4-6 = -2\\
e_3: &	4x &	-	y &		+ 6z &	= & -4 \to 4·2 - 0 + 6·(-2) = 8-12 = -4
\end{array}\right\}\begin{array}{c}\\\\\\\\\text{cqc}\end{array}
\]

\end{problem}





\newpage
\begin{problem}
Resuelve el siguiente sistema de ecuaciones:

\[
\left\{\begin{array}{lccccc}
e_1: &x&+y&+z&=&0\\
e_2: &x&+y&-z&=&2\\
e_3: &2x&+3y&+4z&=&-2
\end{array}\right\}
\]

\solution


\[
\left\{\begin{array}{lccccc}
e_1: &x&+y&+z&=&0\\
e_2: &x&+y&-z&=&2\\
e_3: &2x&+3y&+4z&=&-2
\end{array}\right\}
%%
%%
\overset{(1)}{\dimplies}
%%
%%
\left\{\begin{array}{lccccc}
e_1: &x&+y&+z&=&0\\
e_2: &&&-2z&=&2\\
e_3: &2x&+3y&+4z&=&-2
\end{array}\right\}
%%
%%
\overset{(2)}{\dimplies}\]
\[
%%
%%
\left\{\begin{array}{lccccc}
e_1: &x&+y&+z&=&0\\
e_2: &&&-2z&=&2\\
e_3: &&y&+2z&=&-2\\
\end{array}\right\}
\dimplies
\left\{\begin{array}{lccccc}
e_1: &x&+y&+z&=&0\\
e_3: &&y&+2z&=&-2\\
e_2: &&&-2z&=&2\\
\end{array}\right\}
\]



\paragraph{1)} $e_2 = e_2-e_1$

\[
\begin{array}{rccccc}
e_2: &x&+y&-z&=&2\\
e_1: &x&+y&+z&=&0\\
\hline
e_2: &&&-2z&=&2\\
\end{array}
\]	

\paragraph{2)} $e_3 = e_3-2·e_1$

\[
\begin{array}{rccccc}
e_3: &2x&+3y&+4z&=&-2\\
2·e_1: &2·x&+2·y&+2·z&=&\textcolor{red}{2}·0\\
\hline
e_3: &&y&+2z&=&-2\\
\end{array}
\]	


\paragraph{Discusión y resolución}

En este caso ya lo tenemos escalonado, sólo hay que reordenar:

\[
\left\{\begin{array}{lccccc}
e_1: &x&+y&+z&=&0\\
e_3: &&y&+2z&=&-2\\
e_2: &&&-2z&=&2\\
\end{array}\right\}
\]

\subparagraph{Resolución}

\[e_3: -2z=2 \dimplies z=-1\]


Utilizando $z=-1$ en $e_2$:
\[e_2: y+2z=-2 \dimplies y+2·(-1) = -2 \dimplies y-2=-2 \dimplies y=0\]

Utilizando $z=-1$, $y=0$ en la ecuación que nos falta ($e_1$):

\[
e_1: x+y+z=0 \dimplies x+0-1=0 \dimplies x=1
\]


Solución: $(x,y,z) = (1,0,-1)$.

\paragraph{Comprobación}

Sustituimos los valores obtenidos en el sistema inicial:

\[
\left\{\begin{array}{lccccl}
e_1: &x&+y&+z&=&0 \to 1+0-1=0\\
e_2: &x&+y&-z&=&2 \to 1+0-(-1) = 2 \\
e_3: &2x&+3y&+4z&=&-2 \to 2·(1) + 3·(0) + 4·(-1) = 2-4=-2 
\end{array}\right\}\begin{array}{c}\\\\\\\\\text{cqc}\end{array}
\]

\end{problem}







\newpage
\begin{problem}
Resuelve el siguiente sistema de ecuaciones:

\[
\left\{\begin{array}{lllll}
e_1: & 3x &+ 2y &+z &=10\\
e_2: & -x&-y&+z&=0\\
e_3: & 2x&+y&+3z&=13
\end{array}\right\}
\]

\solution

Lo primero, reordenar o simplificar ecuaciones. Vemos que la incógnita $z$ va a ser más fácil de eliminar, al menos en las 2 primeras ecuaciones. 

\[
\left\{\begin{array}{lcccc}
e_1: & z &+ 2y &+3x &=10\\
e_2: & z&-y&-x&=0\\
e_3: & 3z&+y&+2x&=13
\end{array}\right\} 
%%
%%
\overset{(1)}{\dimplies}
%%
%%
\left\{\begin{array}{lccccc}
e_1: & z &+ 2y &+3x &=10\\
e_2: & &3y&+4x&=10\\
e_3: &3z&+y&+2x&=13
\end{array}\right\}
%%
%%
\overset{(2)}{\dimplies}\]
\[
%%
%%
\left\{\begin{array}{lcccl}
e_1: & z&+ 2y&+3x &=10\\
e_2: & &3y&+4x&=10\\
e_3: & &-5y&-7x&=-17\\
\end{array}\right\}
%%
%%
\overset{(3)}{\dimplies}
%%
%%
\left\{\begin{array}{lccccc}
e_1: & z&+ 2y&+3x &=10\\
e_2: & &3y&+4x&=10\\
e_3: &&&-x&=-1\\
\end{array}\right\}
\]


\paragraph{1)}
$e_2 = e_2-e_1$

\[
\begin{array}{rcccl}
e_2: & z &+ 2y &+3x &=10\\
e_1: & z&-y&-x&=0\\
\hline
e_2: & &3y+&4x&=10\\
\end{array}
\]	

\paragraph{2)} $e_3 = e_3-3e_1$

\[
\begin{array}{rcccl}
e_3: & 3z&+y  &+2x&=13\\
e_1: & 3z&+6y &+9x&=30\\
\hline
e_3: &   &-5y &-7x&=-17\\
\end{array}
\]	



\paragraph{3)} $e_3=5·e_2+3·e_3$

\[
\begin{array}{rcccl}
5e_2: & &15y&+20x&=50\\
3e_3: & &-15y&-21x&=-51\\
\hline
e_3: &&&-x&=-1\\
\end{array}
\]	
\nota{También podríamos hacer $e_3 = e_2+\frac{3}{5}e_1$.}


\paragraph{Discusión y resolución:}

\[
\left\{\begin{array}{lccccc}
e_1: & z&+ 2y&+3x &=10\\
e_2: & &3y&+4x&=10\\
e_3: &&&-x&=-1\\
\end{array}\right\}
\]

\subparagraph{Discusión:} El sistema es compatible determinado, porque es un sistema escalonado con ecuaciones compatibles.


\subparagraph{Resolución:}

$e_3: -x=-1 \dimplies x=1$

Utilizando $x=1$ en $e_2$:


\[
	e_2: 3y+4x=10 \dimplies y=\frac{10-4x}{3} = \frac{10-4}{3} = 2
\]

Utilizando $y=2$, $x=1$ en la ecuación que nos falta ($e_1$):

\[
e_1: z+2y+3x = 10 \dimplies z = 10-3x-2y = 10-3·1-2·2 = 10-7 = 3
\]


Solución: $(x,y,z) = (1,2,3)$.

\paragraph{Comprobación}

Sustituimos los valores obtenidos en el sistema inicial:


\[
\left\{\begin{array}{lcccl}
e_1: & 3x &+ 2y &+z &=10 \to 3·1+2·2+3 = 3+4+3 = 10  \\
e_2: & -x&-y&+z&=0 \to -1-2+3 = 0  \\
e_3: & 2x&+y&+3z&=13 \to 2·1+2+3·3 = 2+2+9 = 13  \\
\end{array}\right\} \begin{array}{c}\\\\\\\\\text{cqc}\end{array}
\]
\end{problem}


\newpage
\begin{problem}
\textbf{Libro ejercicio 113,a}

Discute y resuelve el siguiente sistema:

\[
\left\{\begin{array}{lcccl}
x&+2y&-2z&=&4\\
2x&+5y&-2z&=&10\\
4x&+9y&-6z&=&18
\end{array}\right\}
\]

\solution

\[
\left\{\begin{array}{lcccl}
x&+2y&-2z&=&4\\
2x&+5y&-2z&=&10\\
4x&+9y&-6z&=&18
\end{array}\right\}
\overset{(1)}{\dimplies}
\left\{\begin{array}{lcccl}
x&+2y&-2z&=&4\\
 &y&+2z&=&2 \\
4x&+9y&-6z&=&18
\end{array}\right\}
\overset{(2)}{\dimplies}\]
\[
\left\{\begin{array}{lcccl}
x&+2y&-2z&=&4\\
 &y&+2z&=&2 \\
 &y&+2z&=&2 \\
\end{array}\right\}
\dimplies
\underbrace{\left\{\begin{array}{lcccl}
x&+2y&-2z&=&4\\
 &y&+2z&=&2 
\end{array}\right\}}_{\text{Discusión: C.I (*)}}
\]

(*): Es un sistema compatible indeterminado porque es un sistema escalonado con más incógnitas que ecuaciones.

\paragraph{Resolución:} Parametrizamos. Tomamos $y=λ$ y sustituimos en $E_2$.

\[y+2z=2 \dimplies λ+2z=2 \dimplies z=\frac{2-λ}{2}\]

Sustituimos $y=λ,z=\frac{2-λ}{2}$ en $E_1$:

\[x+2y-2z = 4 \dimplies x= 4+2z-2y = 4+2\left(\frac{2-λ}{2}\right)-2λ = 4+2-λ-2λ = 6-3λ = 3(2-λ)\]

\textbf{Solución:} $(x,y,z) = \left(3(2-λ),λ,\frac{2-λ}{2}\right)$

\paragraph{1)} $E_2=E_2-2E_1$

\[
\left\{\begin{array}{lcccl}
2x&+4y&-4z&=&8\\
2x&+5y&-2z&=&10\\
\hline
&-y&-2z&=&-2 
\end{array}\right\}
\]

\paragraph{2)} $E_3=E_2-4E_1$

\[
\left\{\begin{array}{lcccl}
4x&+9y&-6z&=&18\\
4x&+10y&-4z&=&20\\
\hline
&-y&-2z&=&-2 
\end{array}\right\}
\]


\paragraph*{Comprobación:} Sustituimos $(x,y,z) = \left(3(2-λ),λ,\frac{2-λ}{2}\right)$ en el sistema inicial:


\[
\left\{\begin{array}{lcccll}
x&+2y&-2z&=&4 &\to 6-3λ + 2λ - 2\displaystyle\left(\frac{2-λ}{2}\right) = 6-λ-2+λ = 4\\
2x&+5y&-2z&=&10 &\to 12-6λ +5λ - 2\displaystyle\left(\frac{2-λ}{2}\right) = 12-λ-2+λ = 10\\
4x&+9y&-6z&=&18 &\to 24-12λ + 9λ - 6\displaystyle\left(\frac{2-λ}{2}\right) = 24-3λ-6+3λ = 18
\end{array}\right\}\begin{array}{c}\\\\\\\\\text{cqc}\end{array}
\]

\end{problem}


\newpage
\begin{problem}
\textbf{Libro ejercicio 113,b}

Discute y resuelve el siguiente sistema:

\[
\left\{\begin{array}{lcccl}
x&+2y&-z&=&-5\\
5x&-y&+2z&=&11\\
6x&+y&+z&=&5
\end{array}\right\}
\]

\solution

\[
\left\{\begin{array}{lcccl}
x&+2y&-z&=&-5\\
5x&-y&+2z&=&11\\
6x&+y&+z&=&5
\end{array}\right\}
\overset{(1)}{\dimplies}
\left\{\begin{array}{lcccl}
x&+2y&-z&=&-5\\
&-11y&+7z&=&36\\
6x&+y&+z&=&5
\end{array}\right\}
\overset{(2)}{\dimplies}\]
\[
\left\{\begin{array}{lcccl}
x&+2y&-z&=&-5\\
&-11y&+7z&=&36\\
& -11y&+7z&=&35
\end{array}\right\}
\overset{(3)}{\dimplies}
\underbrace{\left\{\begin{array}{lcccl}
x&+2y&-z&=&-5\\
&-11y&+7z&=&36\\
& &0&=&1
\end{array}\right\}}_{\text{\textbf{Discusión:} Sistema Incompatible (*)}}
\]

(*): El sistema es incompatible porque tiene una ecuación incompatible.


\paragraph{1)} $e_2 = e_2-5e_1$

\[
\begin{array}{lcccl}
e_2: 5x&-y&+2z&=&11\\
e_1: 5x&+10y&-5z&=&-25\\
\hline
e_2:   & -11y&+7z&=&36\\
\end{array}
\]	

\paragraph*{2)} $e_3 = e_3-6e_1$
\[
\begin{array}{lcccl}
e_3: 6x&+y&+z&=&5\\
e_1: 6x&+12y&-6z&=&-30\\
\hline
e_2:   & -11y&+7z&=&35\\
\end{array}
\]	

\paragraph{3)} $e_3=e_3-e_2$
\[
\begin{array}{lcccl}
&-11y&+7z&=&36\\
&-11y&+7z&=&35\\
\hline
e_2:   & 0y&+0z&=&1\\
\end{array}
\]	



\end{problem}




\end{document}