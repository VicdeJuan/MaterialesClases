\documentclass[palatino,nosec]{Docencia}


\title{Ejercicios funciones}
\author{Departamento de Matemáticas}
\date{17/18}

\usepackage{sagetex}

\begin{abstract}
\end{abstract}

% Paquetes adicionales

\usepackage[author={Víctor de Juan, 2017}]{pdfcomment}

\makeatletter
\newcommand{\annotate}[2][]{%
	\pdfstringdef\x@title{#1}%
	\edef\r{\string\r}%
	\pdfstringdef\x@contents{#2}%
	\pdfannot
	width 2\baselineskip
	height 2\baselineskip
	depth 0pt
	{
		/Subtype /Text
		/T (\x@title)
		/Contents (\x@contents)
	}%
}
\makeatother

% --------------------
\newcommand{\cimplies}{\text{\hl{$\implies$}}}
\renewcommand{\vx}{\overset{\rightarrow}{x}}
\renewcommand{\vy}{\overset{\rightarrow}{y}}
\renewcommand{\vz}{\overset{\rightarrow}{z}}
\newcommand{\vi}{\overset{\rightarrow}{i}}
\newcommand{\vj}{\overset{\rightarrow}{j}}
\renewcommand{\vec}[1]{\overset{\rightarrow}{#1}}

\usepackage{pgf,tikz}
\usetikzlibrary{arrows}

\begin{document}
\pagestyle{plain}

\newif\ifverbose
\verbosetrue

\ifverbose
\begin{sagesilent}
	Verbose=1
\end{sagesilent}
\else
\begin{sagesilent}
	Verbose=0
\end{sagesilent}
\fi
\begin{sagesilent}
def random_between(j,k):
 if ((j==-Infinity) and (k == Infinity)):
  return 0

 if (j==-Infinity):
  return int(k-2)

 if (k==Infinity):
  return int(j+2)

 if (j < 0 and k>0):
  return 0

 if (j < 1 and k>1):
  return 1
 
 if (k-j>=2):
  return int(random()*(k-j-1))+j+1
 else:
  return round((k+j)/2.0,2)


def _getRealroots(f):
 __roots = solve(f(x)==0,x,solution_dict=True)
 _roots = [a[x].full_simplify() for a in __roots if imag(a[x])==0]
 return _roots

def intervalos(recta):
 r = ""
 for i in xrange(len(recta)-1):
  r += "$\\left("+latex(recta[i])+","+latex(recta[i+1])+"\\right)$;\\quad"  
 return r

def _f_sign_monot(recta,g):
 min=[]
 max=[]
 prev=0
 actual=0
 retval = "\\begin{itemize}"
 for i in xrange(len(recta)-1):
  rval=random_between(recta[i],recta[i+1])
  retval += "\\item $\\left("+latex(recta[i])+","+latex(recta[i+1])+"\\right)$:"
  if (Verbose == 1):
   retval += " Tomamos, por ejemplo, f'($"+latex(rval)+") = "+latex(round(g(rval),3))+"...$  y miramos su signo:" 
  else:
   retval += "f'($"+latex(rval)+") = "+latex(round(g(rval),3)) +"$"
  prev=actual
  actual=sign(g(rval))
  if actual==1: 
   if (Verbose == 1):
    retval += "Positivo, por lo que la función \\textbf{crece} en este intervalo"  
   else:
    retval += "Positivo $\\Rightarrow$ crece"
  else:
   if (Verbose == 1):
    retval += "Negativo, por lo que la función \\textbf{decrece} en este intervalo"
   else:
    retval += "Negativo $\\Rightarrow$ decrece"


 return [retval+"\\end{itemize}",min,max]

def estudiarSigno(f):
 recta=[-Infinity,Infinity]
 for a in _getRealroots(f):
  recta.append(a)
 recta=sorted(recta) 
 retval = "Los intervalos a estudiar son: "+latex(intervalos(recta))
 
 [_retval,min,max] = _f_sign_monot(recta,f)
 return retval + _retval


## Function str ismin(f,x0).
## f: second derivate of the function to study.
## x0: value to check if it is maximum.
## return 	string with the answer.
def ismin(f,x0):
 val = f(x0)
 retval = "f("+str(x0)+") = "+str(f(x0))
 if val<0:
  retval += "<0 por lo que hay un máximo relativo en "+str(x0)
 elif val>0:
  retval += ">0 por lo que hay un mínimo relativo en "+str(x0)
 else:
  retval += "No podemos determinar si se trata de un máximo o un mínimo de esta manera" 
 return retval
 
\end{sagesilent}

\begin{problem} Demuestra que el vértice de la parábola $f(x) = ax^2+bx+c$ tiene de coordenada horizontal $x=\rfrac{-b}{2a}$
\solution

El vértice será extremo relativo de la parábola (máximo o mínimo según el signo de $a$) y se encontrará en $f'(x) = 0$.

\[f'(x) = 0 \dimplies 2ax+b=0 \dimplies x=\frac{-b}{2a}\]

Las coordenadas del vértice serán \[\left(\frac{-b}{2a},f\left(\frac{-b}{2a}\right) \right)\]


\end{problem}


\begin{problem} Halla las rectas tangentes a la función $f(x)$ en los máximos (no es necesario intervalos de crecimiento y decrecimiento).
\[
f(x) = 3x^4-4x^3-36x^2
\]
¿Hay algún máximo absoluto?

\solution

Primero necesitamos encontrar los máximos. 
%
Para ello resolvemos $f'(x) = 0$


\[
f'(x) = 12x^3-12x^2-72x = 12x(x^2-x-6) = 0 \]
\[\dimplies \left\{\begin{array}{l}
x_1=0\\
x^2-x-7 = 0 \dimplies x_2 = 2 \;\;\; x_3= -3
\end{array}\right\}
\]

¿Cuáles de estos extremos son máximos? Aquellos que cumplan $f''(x_i) < 0$

\[f''(x) = 36x^2-24x-72\]
\begin{itemize}
	\item $f''(0) = -72 <0 \to $ el extremo de $x_1=0$ es un máximo. 
	\item $f''(2) = 24 > 0 \to $ el extremo de $x_2=2$ es un mínimo. 
	\item $f''(3) = 180 > 0 \to$ el extremo ce $x_3=3$ es un mínimo.
\end{itemize}

\textbf{Recta tangente}

\[
	y-f(x_0) = f'(x_0)·(x-x_0) \dimplies y-0 = 0·(x-0) \implies y=0
\]

\textbf{Máximo absoluto} El máximo no puede ser absoluto porque $\lim_{x\to\infty} f(x) = \infty$, por lo que la función no tiene puede tener máximo absoluto.

\textit{Como curiosidad, ¿y el mínimo? $\lim_{x\to-\infty}f(x) = \infty$. En ambos extremos la función va a $+\infty$, por lo que alguno de los 2 mínimos encontrados debe ser absoluto. 
%
Para distinguir cuál es necesitamos comparar $f(2)$ y $f(3)$. 
%
Si $f(2)<f(3)$, entonces $(x,f(2))$ será el mínimo absoluto (ya que es más mínimo que le otro) y viceversa.}

\begin{sagesilent}
  f(x) = 3*x**4-4*x**3-36*x**2
\end{sagesilent}

$f(2) = \sagestr{latex(f(2))}$

$f(3) = \sagestr{latex(f(3))}$

\textit{El mínimo absoluto está en $x=3$ ya que $f(3)<f(2)$. Gráficamente, un mínimo se encuentra \textit{más abajo} que el otro.   }                                                                                   


\end{problem}


\begin{problem} Estudia la monotonía (intervalos de crecimiento, decrecimiento y extremos relativos y absolutos) de $f(x) = x^4-2x^2$
	\begin{sagesilent}
		f(x)=x^4-2*x^2
	\end{sagesilent}

	\solution
	
	1) Resolvemos $f'(x) = 0$. Los extremos relativos son:

	\sagestr{estudiarSigno(f(x))}
	
	\[
	f'(x) = 4x^3-4x =x(4x^2-4)   \left\{\begin{array}{l}
	x_1=0\\
	4x^2-4 = 0 \dimplies x_2 = 1 \;\;\; x_3= -1
	\end{array}\right\}
	\]
	
	2) Estudiamos el signo de la derivada en los intervalos que nos marcan los puntos críticos. 
	
	
	\begin{itemize}
		\item $(-\infty,-1) \to f'(a) < 0$
		\item $(-1,0) \to f'(x) > 0$
		\item $(0,1) \to f'(x) < 0$
		\item $(1,\infty) \to f'(x) > 0$
	\end{itemize}
	
\end{problem}

\begin{problem} Estudia la monotonía (intervalos de crecimiento, decrecimiento y extremos relativos y absolutos) de $f(x) = x^3$
	\solution
	
	1) Resolvemos $f'(x) = 0$
	
	\[
	f'(x) = 3x^2 \dimplies x=0
	\]
	
	2) Estudiamos el signo de la derivada en los intervalos que nos marcan los puntos críticos.
	
	$f'(x) > 0 \forall x \implies$ la función \textbf{siempre} es creciente.
	
\end{problem}


\newcommand{\newexample}[1]{
	\input{#1}
	%%% DEFINIR TieneLog=1 si la función tiene logaritmo.
\begin{sagesilent}
TieneLog=0
\end{sagesilent}

%\paragraph{Dominio}


\begin{sagesilent}
_den0=solve(f.denominator(normalize=False) == 0,x,solution_dict=false)
den0 = [a[x] for a in _den0 if imag(a[x])==0]
\end{sagesilent}
\[
	D(f) = \{x\in\real \tq \sagestr{f.denominator()} \neq 0\} = \real-\{\sagestr{den0}\}
\]
\subparagraph{Simetría}

\ifverbose
    Para estudiar la simetría de una función calculamos $f(-x)$ y comparamos con $f(x)$. 
    %
    En este caso:
\fi

\[f(-x) = \sage{latex(f(-x))}\]

¿Es igual a f(x) o a -f(x)? \sagestr{simetria(f)}


\subparagraph{Puntos de corte con los ejes}


\textit{\textbf{Eje X}
}\ifverbose
Para calcular los puntos de corte de la función con el eje x resolvemos la ecuación $f(x) = 0$ cuya solución es: 
\\

\else 
$f(x) = 0\rightarrow $
\fi
\sagestr{puntosEjeX(func = f)} 

\textit{\textbf{Eje Y}}
\ifverbose
Para calcular los puntos de corte de la función con el eje y calculamos $f(0)$. 
%
En este caso:
\fi
\sagestr{puntosEjeY(func = f)}

\paragraph{Asíntotas}
\subparagraph{Asíntotas verticales}

\begin{sagesilent}
den=f.denominator(normalize=False)
denIs0=ptsDiscontinuidad(f = f,isLog = TieneLog)
\end{sagesilent}


\ifverbose
Los posibles puntos en los que la función puede tener una asíntota vertical son aquellos en los que se anula el denominador. 
%
Por ello calculamos:
%
\[\sagestr{latex(den)} = 0 \]
\fi
\begin{sagesilent}
[strAV,AV] = asintotesV(f = f,den = den, isLog = TieneLog)
\end{sagesilent}

\sagestr{strAV}

\subparagraph{Asíntotas horizontales u oblícuas}

\ifverbose
Las asíntotas horizontales y oblicuas nos dan la información acerca de la tendencia de la función en $-\infty$ y en $+\infty$.

Para calcular las asíntotas, necesitamos calcular el límite de la función tanto en $+\infty$ como en $-\infty$:
\fi

\[\lim_{x\mapsto \pm\infty} \sagestr{latex(f(x))} \]

\begin{sagesilent}
[strAHO,AH,AO] = asintotesHO(f = f)
\end{sagesilent}

\sagestr{strAHO}


\subparagraph{Estudio de la monotonía}


\ifverbose
La monotonía responde a los intervalos de crecimiento y decrecimiento de la función, así como a la existencia de extremos relativos.
%
Para ello, la mejor herramienta es utilizar la derivada de la función.

Calculamos la derivada de la función:
\fi

\[f'(x) = \sagestr{latex(diff(f,x,1).full_simplify()(x))} \]

\ifverbose
    Para calcular los puntos críticos\footnote{Aquellos puntos que pueden marcar un cambio en la tendencia de la función.}, resolvemos f'(x) = 0 cuyas soluciones son:
    \\

\else
    Resolvemos $f'(x) = 0\rightarrow $. 
\fi
\begin{sagesilent}
[ptscrit,ptscritStr] = solveDerivadaNula(f,true)
[puntosFrontera,ptsFronteraStr] = ptsFrontera(f,ptscrit,denIs0)
\end{sagesilent}

\ifverbose
\hl{E incluimos los puntos de discontinuidad de la función primitiva} (en caso de haberlos) ya que se puede producir un cambio en la tendencia de la función. 
%
Por lo tanto, los puntos críticos son : 
\sagestr{ptsFronteraStr}

\else
Añadidmos discontinuidades. 

Puntos críticos = \{\sagestr{ptsFronteraStr}\}
\fi



\ifverbose
Con estos resultados ya podemos estudiar los intervalos de crecimiento y decrecimiento. 
%
Para ello tomamos los siguientes intervalos en los que los puntos críticos dividen la recta real:
\fi

\sagestr{intervalos(puntosFrontera)}

\ifverbose
Y ahora, estudiamos la monotonía el signo de la derivada en esos intervalos:
\fi
\begin{sagesilent}
[max,min,diffsign_str] = diffsign(f,puntosFrontera)

\end{sagesilent}

\sagestr{diffsign_str}

\textbf{\textit{Máximos relativos}}
\ifverbose
Los puntos en los que a la izquiera la función crece (derivada positiva) y a la derecha decrece (derivada negativa) son los máximos relativos. En este caso: 
\fi

\[\text{Máximos relativos: } \sagestr{printlist(max)}\]

\textbf{\textit{Mínimos relativos}}
\ifverbose
Los puntos en los que a la izquierda la función crece (derivada positiva) y a la derecha decrece (derivada negativa) son los máximos relativos. En este caso son: 
\fi

\[\text{Mínimos relativos: }\sagestr{printlist(min)}\]





	\newpage
}


\begin{sagesilent}
	TieneLog=0
	f=(-3*x+6)/(x^2-6*x+8)
\end{sagesilent}


\begin{problem} 

	Estudia sistemáticamente la función:
	
	\[f(x) =\sagestr{latex(f(x))}\]
	
	\solution
%\[\sage{latex(f(x))}\]

%\sage{latex(f(x))}	

	\newexample{tex/fun}
\end{problem}


\begin{problem} Estudia las asíntotas de \[f(x) =\frac{x^3-4x^2+x+6}{2x^3-14x^2+32x-24}\]
\solution
\end{problem}
\begin{problem} Deriva $f(x) = \tan(x^2-3x)$
\solution
\end{problem}

\end{document}