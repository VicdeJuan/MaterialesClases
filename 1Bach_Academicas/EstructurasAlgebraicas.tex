
 
 \section{Relaciones}
\begin{defn}[Relación de orden]
Una relación $\rel$ es de orden si cumple las siguientes propiedades para dos elementos cualesquiera $a,b \in X$.
	\begin{description}
		\item[Reflexiva] $a\rel a$.
		\item[Antisimétrica] $a\rel b \y b\rel a \implies b=a$.
		\item[Transitiva] $a\rel b \y b\rel c \implies a\rel c$.
	\end{description}
\end{defn}

\begin{defn}[Orden total] Una relación de orden es lineal o total si $\forall \; a ,b \in X (a\rel b) \Or (b\rel a)$.
\end{defn}

\begin{example}
	\begin{itemize}
		\item $\geq,\leq$ dentro de $ℝ$ es una relación de orden. Además, es total.
		\item $>$ no es una ralación de orden, ya que $a\not> a$.
		\item $<$ en $ℝ^2$ no es total. ¿Cuál es menor, $(0,0) $ o $ (3,3)$? ¿Y entre $(1,2)$ y $(2,1)$?
	\end{itemize}
\end{example}






\section{Valor absoluto:}  Lo siguiente que necesitamos ver es el valor absoluto.

\begin{defn}[Valor absoluto]
\[
	|x| = \left\{\begin{array}{cc}x & \text{ si } x>0\\-x & \text{ si } x<0 \end{array}\right.
\]
\end{defn}

¿Y el igual? No importa, ya que tanto $x=0$ y $-x = 0$.

\paragraph{Propiedades:}
\begin{itemize}
	\item $∀a\inℝ, |a| \geq 0$
	\item $∀a,b\in\real, |ab| = |a||b|$
	\item $∀a,b\in\real, |a+b|\leq|a|+|b|$ \hl{(\textbf{Desigualdad triangular})}
\end{itemize}

Es fundamental manejar con soltura el valor absoluto. Vamos a hacer un par de ejercicios a ver cómo lo lleváis.

\begin{example}
	\begin{itemize}
		\item Sea $b\in\real: |x-b|$
		\item $x+|x-7|$
		\item $|x^2-4|$
		\item $|x+2|+|x-3|$
	\end{itemize}
\end{example}

\nota{En el libro, p.14 vienen ejercicios resueltos que pueden ser de ayuda}-