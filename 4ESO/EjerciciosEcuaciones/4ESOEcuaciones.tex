\documentclass[palatino,nosec,nochap]{Docencia}


\title{Matemáticas 4º ESO}
\author{Víctor de Juan}
\date{17/18}


% Paquetes adicionales

\usepackage[author={Víctor de Juan, 2017}]{pdfcomment}

\makeatletter
\newcommand{\annotate}[2][]{%
\pdfstringdef\x@title{#1}%
\edef\r{\string\r}%
\pdfstringdef\x@contents{#2}%
\pdfannot
width 2\baselineskip
height 2\baselineskip
depth 0pt
{
/Subtype /Text
/T (\x@title)
/Contents (\x@contents)
}%
}
\makeatother

%%% He tardado 40 minutos en preparar este solucionario.

% --------------------
\newcommand{\cimplies}{\text{\hl{$\implies$}}}

\begin{document}
\pagestyle{plain}

\paragraph{Resuelve las siguientes ecuaciones} (incluir comprobación)

\begin{enumerate}
	\item $\displaystyle 2\log_9(x+1) - 2\log_9\sqrt{x^2-1} = \frac{1}{2}$
	\vspace{1cm}
	\item $\displaystyle\frac{\log(35-x^3)}{\log(5-x)} = 3$
	\vspace{1cm}
	\item $\displaystyle \ln(x^2+3x+2) - \ln (x^2-1) = \ln 2$
	\vspace{1cm}
	\item $\displaystyle \log x=\frac{2-\log x}{\log x}$ (¿cambio de variable?)
	\vspace{1cm}
	\item $\displaystyle \log_4 x - \frac{\log_464}{\log_4x} = 2$
	\vspace{1cm}
	\item $\displaystyle \log_9\sqrt[5]{27} = 2x-1$
	\vspace{1cm}
	\item $\displaystyle \log_2\sqrt{x} = \log_4{(x^2-1)}$
	\vspace{1cm}
	\item $\displaystyle \log_2\sqrt{x+1} = \log_8\sqrt{x}$
	\vspace{1cm}
	\item $\displaystyle 2^x = 10$
	\vspace{1cm}
	\item $\displaystyle \frac{5}{2^x} = 10·2^x$
\end{enumerate}

\printindex
\end{document}
\grid
