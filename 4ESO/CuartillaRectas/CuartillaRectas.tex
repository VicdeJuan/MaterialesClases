\documentclass[palatino,noprobframes]{CuartillaSafa}


\title{Cuartilla de Geometría Plana}
\date{24-02-2017}

% Paquetes adicionales

% --------------------

\renewcommand{\vec}[1]{\overrightarrow{#1}}

\begin{document}

\cabecera

\begin{problem}\textbf{(0.8 puntos)}
Señala todas las opciones correctas, siendo $\vec{u} = (u_1,u_2)$ el vector director de la recta $r: Ax+By+C = 0$ y siendo $m$ su pendiente:

a) $\displaystyle m=-\frac{u_1}{u_2}$\;\;\;\;\;\;\;\;\;
b) $\displaystyle m=\frac{u_2}{u_1}$\;\;\;\;\;\;\;\;\;
c) $\displaystyle m=-\frac{A}{B}$\;\;\;\;\;\;\;\;\;
d) $\displaystyle m=\frac{B}{A}$\;\;\;\;\;\;\;\;\;
\end{problem}


\begin{problem}\textbf{(0.6 puntos)} 
Une cada ecuación de la recta con su denominación:

\begin{tabular}{L{2.25cm}L{3cm}L{8cm}}
General 		&o& o\;\;$Ax+By+C=0$ \\\\
Explícita 		&o& o\;\;$y=mx+n$ \\\\
Punto-pendiente &o& o\;\;$y-a_2 = m·(x-a_1)$ \\\\
Continua 		&o& o\;\;$\displaystyle\frac{y-a_2}{u_2} = \frac{x-a_1}{u_1}$ \\\\
Paramétrica 	&o& o\;\;$\left\{\begin{array}{c} x=a_1+u_1·t\\y=a_2+u_2·t\end{array}\right.$ \\\\
Vectorial 		&o& o\;\;$(x,y) = (a_1,a_2) + t·(u_1,u_2)$ \\\\
\end{tabular}
\end{problem}
\vspace{-0.7cm}

\begin{problem}\textbf{ Posición relativa (3 puntos)}
\begin{enumerate}[a)]
	\vspace{-0.3cm}\item Determina la ecuación explícita de la recta $\vec{r}$ que pasa por $A(0,0)$ y $B(5,4)$.
	\vspace{-0.3cm}\item Determina la recta $\vec{s}$ que pasa por $C(10,8)$ y tiene pendiente $m=0.8$
	\vspace{-0.3cm}\item Estudia la posición relativa de las rectas $\vec{r}$ y $\vec{s}$ de los apartados anteriores.
	\vspace{-0.3cm}\item Justifica si son perpendiculares.
\end{enumerate}

\vspace{12cm}

\end{problem}


\begin{problem}\textbf{(2.5 puntos)}

Halla el ángulo que forman las rectas $\displaystyle \vec{r}: \frac{x-3}{2} = \frac{y+1}{-2}$ y $\vec{s}: \left\{\begin{array}{c} x=3\\y=1+2t\end{array}\right.$. 

\end{problem}

\vspace{7cm}

\begin{problem}\textbf{(1.5 puntos)}
¿Están los puntos $A(2,2)$, $B(0,1)$ y $C(-2,0)$ alineados? Justifica tu respuesta.
\end{problem}

\vspace{8cm}

\begin{problem}\textbf{(0.6 puntos)}
Evalúa \textbf{sinceramente} al profesor de prácticas. ¿Han sido claras las explicaciones? ¿Qué aspectos consideras que debería mantener y qué aspectos mejorar? \textit{Por si te preocupa la calificación de esta pregunta, simplemente contestando a la pregunta seriamente obtendrás la máxima puntuación.}

\end{problem}

%% Apendices (ejercicios, examenes)


\end{document}
