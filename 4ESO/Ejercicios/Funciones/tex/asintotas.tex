\paragraph{Asíntotas}
\subparagraph{Asíntotas verticales}

\begin{sagesilent}
den=f.denominator(normalize=False)
denIs0=ptsDiscontinuidad(f = f,isLog = TieneLog)
\end{sagesilent}


\ifverbose
Los posibles puntos en los que la función puede tener una asíntota vertical son aquellos en los que se anula el denominador. 
%
Por ello calculamos:
%
\[\sagestr{latex(den)} = 0 \]
\fi
\begin{sagesilent}
[strAV,AV] = asintotesV(f = f,den = den, isLog = TieneLog)
\end{sagesilent}

\sagestr{strAV}

\subparagraph{Asíntotas horizontales u oblícuas}

\ifverbose
Las asíntotas horizontales y oblicuas nos dan la información acerca de la tendencia de la función en $-\infty$ y en $+\infty$.

Para calcular las asíntotas, necesitamos calcular el límite de la función tanto en $+\infty$ como en $-\infty$:
\fi

\[\lim_{x\mapsto \pm\infty} \sagestr{latex(f(x))} \]

\begin{sagesilent}
[strAHO,AH,AO] = asintotesHO(f = f)
\end{sagesilent}

\sagestr{strAHO}
