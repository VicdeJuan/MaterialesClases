\documentclass[palatino,nosec,nochap]{Docencia}


\title{Ecuaciones logarítmicas V2}
\author{Víctor de Juan}
\date{17/18}


\begin{abstract}
Algunas ecuaciones logarítmicas del nivel de 4º ESO resueltas.
\end{abstract}

% Paquetes adicionales

\usepackage[author={Víctor de Juan, 2017}]{pdfcomment}

\makeatletter
\newcommand{\annotate}[2][]{%
\pdfstringdef\x@title{#1}%
\edef\r{\string\r}%
\pdfstringdef\x@contents{#2}%
\pdfannot
width 2\baselineskip
height 2\baselineskip
depth 0pt
{
/Subtype /Text
/T (\x@title)
/Contents (\x@contents)
}%
}
\makeatother

\usepackage{eso-pic}
\newcommand\BackgroundPic{%
\put(0,0){%
\parbox[b][\paperheight]{\paperwidth}{%
\vfill
\centering
\includegraphics[width=\paperwidth,height=\paperheight,%
keepaspectratio]{../../BWLogo.jpeg}%
\vfill
}}}



% --------------------
\newcommand{\cimplies}{\text{\hl{$\implies$}}}

\begin{document}
\pagestyle{plain}
\maketitle
\AddToShipoutPicture{\BackgroundPic}



Propiedades:
\begin{itemize}
	\item $(1): \log_aA^n = n\log_aA$
	\item $(2): \log_aA+\log_aB = \log_aAB$\;\; y\;\; $\log_aA-\log_aB = \log_a\frac{A}{B}$
	\item $(3): \rfrac{\log_bA}{\log_ba} = \log_aA$
\end{itemize}

\textbf{Resuelve las siguientes ecuaciones} (incluir comprobación)

\begin{enumerate}
	\hrule
	\item $\displaystyle 2\log_9(x+1) - 2\log_9\sqrt{x^2-1} = \frac{1}{2}$
	
	\[
	 	2\log_9(x+1) - 2\log_9\sqrt{x^2-1} = \frac{1}{2} \dimplies \log_9(x+1)^2-\log_9\left(\sqrt{x^2-1}\right)^2 = \frac{1}{2}
	\]
	\[
		\log_9\frac{ (x+1)^2}{(x+1)(x-1)} = \frac{1}{2} \implies \log_9 \frac{(x+1)}{(x-1)} = \frac{1}{2} \implies 9^{\rfrac{1}{2}} = \frac{x+1}{x-1}
	\]

	Como $9^{\rfrac{1}{2}} = \sqrt{9} = \pm3$
	\subitem $+3 = \frac{x+1}{x-1} \implies 3x-3=x+1 \dimplies 2x=4 \dimplies x_1=2$
	\subitem $-3 = \frac{x+1}{x-1} \implies -3x+3=x+1 \dimplies -4x=-2 \dimplies x_2=\frac{1}{2}$

	\textbf{Comprobación:}
	\subitem $x_1=2 \to 2\log_9(2+1) - 2\log_9\sqrt{2^2-1} = 1-\frac{1}{2} = \frac{1}{2} \implies $ sí es solución.
	\subitem $x_1=\frac{1}{2} \to 2\log_9\left(\frac{1}{2}+1\right) - 2\log_9\sqrt{\left(\frac{1}{2}\right)^2-1} \not\in\real \implies $ no es solución.

	\hrule
	\item $\displaystyle\frac{\log(35-x^3)}{\log(5-x)} = 3$
	\[
		\frac{\log(35-x^3)}{\log(5-x)} = 3 \dimplies \log(35-x^3)=3\log(5-x) \overset{(1)}{\dimplies} \log(35-x^3)=\log(5-x)^3 \implies 
	\]
	\[35-x^3 = (5-x)^3 \dimplies 35-x^3 = (5-x)(5-x)(5-x) \dimplies \]
	\[ \dimplies 35-x^3 = -x^3+15x^2-75x+125 \dimplies 15x^2-75x+90 = 0
	\]

	$15x^2-75x+90 = 0$ cuyas soluciones son $x_1=2, x_2=3$.

	\textbf{Comprobación:}
	\subitem $\displaystyle x_1=2 \to \frac{\log(35-2^3)}{\log(5-2)} = \frac{\log27}{\log3}\overset{(3)}{=}\log_327=3 \implies $ sí es solución.
	\subitem $\displaystyle x_2=3 \to \frac{\log(35-3^3)}{\log(5-3)} = \frac{\log8}{\log2} \overset{(3)}{=} \log_28=3\implies $ sí es solución.

	\vspace{0.5cm}
	\hrule
	\item $\displaystyle \ln(x^2+3x+2) - \ln (x^2-1) = \ln 2$
	
	\[
		\ln(x^2+3x+2) - \ln (x^2-1) = \ln 2 \overset{(2)}{\dimplies} \ln\frac{x^2+3x+2}{x^2-1} = \ln2 \implies \frac{(x+1)(x+2)}{(x+1)(x-1)} = 2 \]
		\[\implies \frac{(x+2)}{(x-1)} = 2 \implies x+2=2x-2 \dimplies x=4
	\]

	\textbf{Comprobación:}
	\subitem $x=4 \to \ln(4^2+3·4+2) - \ln (4^2-1) = \ln30 - \ln15 \overset{(2)}{=} \ln\rfrac{30}{15} = \ln 2\implies$ sí es solución.
	
	\vspace{0.5cm}\hrule 
	\item $\displaystyle \log_9\sqrt[5]{27} = 2x-1$
	\[
		\displaystyle \log_9\sqrt[5]{27} = 2x-1 \dimplies \log_93^{\rfrac{3}{5}} = 2x-1 \overset{(1)}{\dimplies} \frac{3}{5}\log_93=2x-1 \dimplies \frac{3}{10} = 2x-1
	\]
	\[
 		\frac{3}{10} = 2x-1 \dimplies 3=20x-10 \implies x=\frac{13}{20}
	\]
	\textbf{Comprobación:}

	\subitem $x=\displaystyle\rfrac{13}{20} \to \log_9\sqrt[5]{27} = 2x-1 \dimplies \frac{3}{10} = \frac{26-20}{20} \dimplies \frac{3}{10} = \frac{6}{20} \implies$ sí es solución.

	\vspace{0.5cm}\hrule 
	\item $\displaystyle \log_2\sqrt{x} = \log_4{(x^2-1)}$
	\vspace{1cm}
	\item $\displaystyle \log_2\sqrt{x+1} = \log_8\sqrt{x}$
	\vspace{1cm}
	\item $\displaystyle \log x=\frac{2-\log x}{\log x}$ (¿cambio de variable?)
	\vspace{1cm}
	\item $\displaystyle \log_4 x - \frac{\log_464}{\log_4x} = 2$
	\vspace{1cm}
	\item $\displaystyle 2^x = 10$
	\vspace{1cm}
	\item $\displaystyle \frac{5}{2^x} = 10·2^x$
\end{enumerate}

\end{document}
\grid
