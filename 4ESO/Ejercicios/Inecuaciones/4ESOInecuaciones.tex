\documentclass[palatino,nosec,nochap]{Docencia}


\title{Inecuaciones}
\author{Víctor de Juan}
\date{17/18}


\begin{abstract}

\end{abstract}

% Paquetes adicionales

\usepackage[author={Víctor de Juan, 2017}]{pdfcomment}

\makeatletter
\newcommand{\annotate}[2][]{%
\pdfstringdef\x@title{#1}%
\edef\r{\string\r}%
\pdfstringdef\x@contents{#2}%
\pdfannot
width 2\baselineskip
height 2\baselineskip
depth 0pt
{
/Subtype /Text
/T (\x@title)
/Contents (\x@contents)
}%
}
\makeatother

\usepackage{eso-pic}
\newcommand\BackgroundPic{%
\put(0,0){%
\parbox[b][\paperheight]{\paperwidth}{%
\vfill
\centering
\includegraphics[width=\paperwidth,height=\paperheight,%
keepaspectratio]{../../BWLogo.jpeg}%
\vfill
}}}



% --------------------
\newcommand{\cimplies}{\text{\hl{$\implies$}}}

\begin{document}
\pagestyle{plain}
\maketitle
\AddToShipoutPicture{\BackgroundPic}
\Large

Vamos a retomar la clase anterior. Para Naiara, Nacho y el resto, repasamos inecuaciones.

Sería bueno que no os estuvérais mandando callar continuamente, ¿vale?

Scad la tabla que tenéis. La de similitudes de inecuaciones y ecuaciones. Nacho, cámbiate de sitio con Claudia. Así Herrero puede ir poniéndote al día y Claudia a Naiara.

Tened la tabla a mano. Preguntas de uno en uno, por favor. 

\textbf{Resolver la siguiente inecuación:}

\[
	3x -1 < x+2
\]

Para resolver la inecuación, despejamos como si fueran ecuaciones. Tratamos el '<' como si fuera el "=" de la ecuación.

¿Qué te da Nerea? ¿Qué has hecho?

La solución de una inecuación es un intervalo. ¿Os acordáis?

Pasamos las $x$ a un lado, y todo lo que no tiene $x$ al otro.

\[
	3x -1 < x+2 \dimplies 3x-x<2+1 \dimplies 2x<3 \dimplies x<\frac{3}{2}
\]



Nacho y Herrero, el otro día decíais que podáis trabajar juntos. Demostrarlo


\textbf{Detalle más importante de la clase} y lo único que quería contaros antes de que empezarais a trabajar. 5 minutos de absoluto silencio por favor.

\textbf{Lo más importante del tema:} transforma en intervalo. Escribe cada inecuación como un intervalo.

\[
\begin{array}{cc}
	-x<0 &\dimplies <
	\\
	x<0 &\dimplies (-∞,0)
\end{array}
\]



\end{document}
\grid
