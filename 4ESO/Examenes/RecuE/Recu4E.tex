\documentclass[palatino,nosec]{Docencia}


\title{Examen Recuperación 4ºE}
\author{Departamento de Matemáticas}
\date{17/18}


% Paquetes adicionales

\usepackage[author={Víctor de Juan, 2017}]{pdfcomment}

\usepackage{pgf,tikz}
\usetikzlibrary{arrows}

\definecolor{qqttcc}{rgb}{0,0.2,0.8}
\definecolor{qqqqff}{rgb}{0,0,1}


\makeatletter
\newcommand{\annotate}[2][]{%
\pdfstringdef\x@title{#1}%
\edef\r{\string\r}%
\pdfstringdef\x@contents{#2}%
\pdfannot
width 2\baselineskip
height 2\baselineskip
depth 0pt
{
/Subtype /Text
/T (\x@title)
/Contents (\x@contents)
}%
}
\makeatother



\usepackage{eso-pic}
\newcommand\BackgroundPic{%
\put(0,0){%
\parbox[b][\paperheight]{\paperwidth}{%
\vfill
\centering
\includegraphics[width=\paperwidth,height=\paperheight,%
keepaspectratio]{../../../BWLogo.jpeg}%
\vfill
}}}




% --------------------
\newcommand{\cimplies}{\text{\hl{$\implies$}}}

\begin{document}
\pagestyle{plain}
%\maketitle


\AddToShipoutPicture{\BackgroundPic}

\begin{problem}(2 puntos)

Un gurpo de estudiantes decide hacer una fiesta y para ello planean una compra de 90 euros que pagan entre todos a partes iguales. El día de la compra, una pareja decide no ir a la fiesta y cada uno de los demás tiene que pagar 0,5  euros más. ¿Cuántos estudiantes había en el grupo originariamente?

\solution

Las incógnitas del problema son:
\begin{itemize}
	\item \textbf{x:} El número de estudiantes.
	\item \textbf{y:} El dinero que paga cada uno. 
\end{itemize}

Antes de que la pareja decida no ir, tenemos la ecuación: $x·y=90$.

Cuando la pareja decide no ir, tenemos $(x-2)$ estudiantes que tienen que pagar $(y+0.5)$ y el precio a pagar siguen siendo 90 euros.

Por lo tanto:

\[
	\left\{
		\begin{array}{c}
			xy=90\\
			(x-2)(y+\rfrac{1}{2})=90
		\end{array}
	\right\} \dimplies 
	\left\{
		\begin{array}{c}
			x=\displaystyle\frac{90}{y}\\
			(x-2)(y+\rfrac{1}{2})=90
		\end{array}
	\right\} 
\]

Sustituyendo:

\[
	\left(\frac{90}{y}-2\right)·\left(y+\frac{1}{2}\right)=90 \dimplies 90+\frac{90}{2y}-2y-1=90 \dimplies \frac{90}{2y}-2y-1=0 \dimplies
\]
\[
	 \frac{90-4y^2-2y}{2y}=0 \implies 4y^2+2y-90=0 \dimplies 2y^2+y-45=0
\]

Resolviendo esta ecuación obtenemos 2 soluciones: $y_1=-5$ (que no tiene sentido porque el dinero no puede ser negativo en este problema) y $y_2=\frac{9}{2}$.

Con este valor, calculamos $x$:
\[
\left\{
	\begin{array}{c}
		x·y=90\\
		y=\rfrac{9}{2}
	\end{array}
\right\}\implies x=\frac{90·2}{9} = 20
\]

\textbf{Originariamente había 20 estudiantes en el grupo.}

\end{problem}


\begin{problem} (4.5 puntos)
Resuelve con comprobación

\ppart (1,5 puntos)
\[
	\frac{5}{x^2-x-6} = \frac{3}{x^2-4}+\frac{3}{2x^2-10x+12}
\]

\ppart (1,5 puntos)
\[
	\sqrt{x^2-3} + 1 = \sqrt{x^2+4}
\]

\ppart (1,5 puntos)
\[
\left\{
	\begin{array}{c}
		2·3^{2x+1}+2^{y-1} = 8\\
		3^{x+1}+3·2^y = 15
	\end{array}	
\right\}
\]


\solution

\spart

\begin{itemize}
	\item $x^2-x-6 = (x-3)(x+2)$
	\item $x^2-4 = (x+2)(x-2)$
	\item $2x^2-10x+12 = \text{\hl{2}}(x^2-5x+6) = 2(x-2)(x-3)$
\end{itemize}

$\text{mcm}[(x-3)(x+2);(x+2)(x-2);2(x-2)(x-3)] = 2(x-3)(x+2)(x-2)$

\[
	\frac{5}{x^2-x-6} = \frac{3}{x^2-4}+\frac{3}{2x^2-10x+12} \dimplies
	\frac{5}{(x-3)(x+2)} = \frac{3}{(x+2)(x-2)} + \frac{3}{2(x-2)(x-3)}
\]
\[
\dimplies \frac{10(x-2)}{2(x-3)(x+2)(x-2)} = \frac{6(x-3)}{2(x+2)(x-2)(x-3)} + \frac{3(x+2)}{2(x+2)(x-2)(x-3)} \implies
\]

\[
{10(x-2)} = {6(x-3)} + {3(x+2)} \dimplies 10x-20=6x-18+3x+6 \dimplies x=8
\]

\textbf{Comprobación:}

\[
	\frac{5}{8^2-8-6} = \frac{3}{8^2-4}+\frac{3}{2·8^2-10·8+12} \dimplies
	\frac{5}{50} = \frac{3}{60}+\frac{3}{60} \dimplies \frac{1}{10}=\frac{1}{10}
\]

\spart 

\[
	\sqrt{x^2-3} + 1 = \sqrt{x^2+4} \implies (x^2-3) + 1 + 2·\sqrt{x^2-3} = x^2+4 \dimplies\]
\[
	2\sqrt{x^2-3} = 6 \implies x^2-3 = 9 \dimplies x^2=12 \dimplies x=\pm\sqrt{12}
\]


\textbf{Comprobación}

$x=+\sqrt{12}$

\[
	\sqrt{\sqrt{12}^2-3} + 1 = \sqrt{\sqrt{12}^2+4} \dimplies \sqrt{9}+1=\sqrt{16} \dimplies 4=4
\]


$x=-\sqrt{12}$

\[
	\sqrt{\left(-\sqrt{12}\right)^2-3} + 1 = \sqrt{\left(-\sqrt{12}\right)^2+4} \dimplies \sqrt{9}+1=\sqrt{16} \dimplies 4=4
\]


\spart 
El cambio que buscamos será $t=2^x$ y $q=3^y$. Para ello hay que reescribir el sistema:

\[
	\left\{
		\begin{array}{c}
			2·3^{2x+1}+2^{y-1} = 8\\
			3^{x+1}+3·2^y = 15
		\end{array}	
	\right\}\dimplies
	\left\{
		\begin{array}{c}
			2·\left(3^{x}\right)^2·3+\frac{2^{y}}{2} = 8\\
			3^{x}·3+3·2^y = 15
		\end{array}	
	\right\}
\]
Hacemos el cambio $t=2^x$ y $q=3^y$
\[
	\left\{
		\begin{array}{c}
			6·q^2+\frac{t}{2} = 8\\
			3q+3t = 15 \to q+t=5
		\end{array}	
	\right\}\dimplies
	\left\{
		\begin{array}{c}
			6·q^2+\frac{t}{2} = 8\\
			t=5-q
		\end{array}	
	\right\}
\]

Sustituyendo: 

\[
	6·q^2+\frac{5-q}{2} = 8 \dimplies \frac{12q^2 +5-q}{2} = 8 \dimplies 12q^2 - q +5=16 \dimplies 12q^2-q-11=0
\]

2 soluciones: $q_1 = \frac{-11}{12}$ y $q_2 = 1$

Obtenemos los valores de $t$
\[
	\left\{
		\begin{array}{c}
			t=5-q\\
			t_1=\frac{-11}{12}
		\end{array}	
	\right\}\to q_1=5-\frac{-11}{12} = \frac{71}{12}
\]\[
	\left\{
		\begin{array}{c}
			t=5-q\\
			t_2=1
		\end{array}	
	\right\}\to q_2= 5-1=4
\]

Una vez obtenidos los valores de $t$ y $q$ hallamos los valores de $x$ e $y$.

\begin{align*}
	&\left\{
		\begin{array}{c}
			t_1=\frac{-11}{12}\\
			t=2^x
		\end{array}	
	\right\}\to 2^x=\frac{-11}{12} \implies \nexists x
\\
	&\left\{
		\begin{array}{c}
			q_1=\frac{71}{12}\\
			q=3^y
		\end{array}	
	\right\}\to \text{Como el valor de $x$ no existe, no importa el valor de $y$ asociado.}
\\
	&\left\{
		\begin{array}{c}
			t_1=4\\
			t=2^x
		\end{array}	
	\right\}\to 2^x=4 \dimplies x=2
\\
	&\left\{
		\begin{array}{c}
			q_1=1\\
			q=3^y
		\end{array}	
	\right\}\to 3^y=1 \dimplies y=0
\end{align*}

\textbf{Solución: $(x,y) = (2,0)$}

\textbf{Comprobación:}


\[
	\left\{
		\begin{array}{c}
			2·3^{0+1}+2^{2-1} = 8\\
			3^{0+1}+3·2^2 = 15
		\end{array}	
	\right\}\dimplies
	\left\{
		\begin{array}{c}
			6+2=8 \\
			3+12=15
		\end{array}	
	\right\}
\]


\end{problem}


\begin{problem}(1,5 puntos)
Despeja el valor de la incógnita en las siguientes expresiones. Indica las propiedades que utilizas.

\ppart $\log_3x=4$

\ppart $7^{\log_7x}=5$

\ppart $3\log_3x+\log_3x^2-\log_39=3$

\ppart $\log_{\sqrt{8}}\sqrt{128} = x$

\ppart $\log_2x+\log_4x^2=\log_88$

\solution

\spart $\log_3x=4 \dimplies 3^4=x$, aplicando la definición de logaritmo.

\spart $7^{\log_7x}=5 \dimplies x=5$, utilizando $a^{\log_ax} = x$.

\spart $3\log_3x+\log_3x^2-\log_39=3 \overset{(1)}{\dimplies} 3\log_3x+2\log_3x-2=3 \dimplies 5\log_3x =5 \dimplies \log_3x=1 \dimplies x=3$ 

Donde $(1)$ utiliza $\log_aA^n = n\log_aA$

\spart $\log_{\sqrt{8}}\sqrt{128} = x \dimplies \sqrt{8}^x = \sqrt{128} \dimplies \left(2^{\frac{3}{2}}\right)^x = 2^{\rfrac{7}{2}} \implies \frac{3}{2}x=\frac{7}{2} \dimplies x=\frac{7}{2}$

\spart $\log_2x+\log_4x^2=\log_88 \overset{(1)}{\dimplies} \log_2x+2\log_4x=1 \overset{(2)}{\dimplies} \log_2x+2\frac{\log_2x}{\log_24} = 1  \dimplies \log_2x+\log_2x=1 \dimplies 2\log_2x=1\dimplies \log_2x=\frac{1}{2} \dimplies 2^{\rfrac{1}{2}} = \sqrt{2} = x$

Donde $(1)$ utiliza $\log_aA^n = n\log_aA$, $(2)$ utiliza $\log_aA=\displaystyle\frac{\log_bA}{\log_ba}$


\end{problem}

\begin{problem} (2 puntos)
Elige una de las siguientes preguntas.

\ppart
Sabemos que $a$ es entero ($a\inℤ$), pero desconocemos su valor. ¿Puede ser $(x-\rfrac{1}{2})$ factor del polinomio $P(x) = 7x^3+2x^4-ax+6$? Enuncia el teorema que utilices.

\ppart

Halla todas las soluciones reales de la ecuación 
\[
	\left(x^4-\frac{97}{2}\right)^2 = \frac{4225}{2}
\]

\solution

\spart 

Por el \textbf{teorema del factor}, $(x-\rfrac{1}{2})$ será factor si $x=\rfrac{1}{2}$ es raíz del polinomio.

Por el \textbf{teorema de las raíces fraccionarias} $x=\rfrac{1}{2}$ podría ser raíz del polinomio si $1$ divide a $6$ y $2$ divide a $2$. 

\textbf{Respuesta: } si podría ser raíz.

\textbf{Teorema del factor:} $a$ es raíz del polinomio $P(x)$ si y sólo $(x-a)$ es factor del mismo.

\textbf{Teorema de las raíces fraccionarias} Sea $P(x)$ un polinomio con coeficientes enteros. $a=\rfrac{m}{n}$ puede ser raíz del polinomio si $m$ divide al término independiente de $P(x)$ y $n$ divide al coeficiente principal de $P(x)$.

\spart 

2 caminos:

\[
	\left(x^4-\frac{97}{2}\right)^2 = \frac{4225}{2} \implies
	\left(x^4-\frac{97}{2}\right) = \sqrt{\frac{4225}{2}}
\]

Como la raíz tiene 2 soluciones, tenemos que resolver 2 ecuaciones:

\[
\left(x^4-\frac{97}{2}\right) = +\sqrt{\frac{4225}{2}} \;\;\;\;\;\;\;\; \left(x^4-\frac{97}{2}\right) = -\sqrt{\frac{4225}{2}}
\]

La otra alternativa es desarrollar el cuadrado.

\[
	\left(x^4-\frac{97}{2}\right)^2 = \frac{4225}{2} \dimplies
	x^8-97x^4+\frac{9409}{4}-\frac{4225}{2} = 0\dimplies 
	x^8-97x^4+1296 = 0
\]

Hacemos el cambio $t=x^4$

\[
x^8-97x^4+1296 = 0 \dimplies t^2-97t+1296=0 \implies 
\]
\[\left\{
\begin{array}{c}
	\left\{\begin{array}{l}
		t_1=81\\
		t=x^4
	\end{array}\right\} \implies x^4=81\implies x=\sqrt[4]{81} \implies \left\{\begin{array}{l} x_1=+3\\x_2=-3\end{array}\right. \\
	\left\{\begin{array}{l}
		t_2=16\\
		t=x^4
	\end{array}\right\} \implies x^4=16\implies x=\sqrt[4]{16} \implies \left\{\begin{array}{l} x_3=+2\\x_4=-2\end{array}\right. \\
\end{array}
\right.
\]

Las 4 soluciones reales son $x_1=3\;;\;x_2=-3\;;\;x_3=2\;;\;x_4=-2$

\end{problem}


\end{document}
