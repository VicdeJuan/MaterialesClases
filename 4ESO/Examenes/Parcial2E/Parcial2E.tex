\documentclass[palatino,nosec]{Docencia}


\title{Segundo parcial 4ºE}
\author{Departamento de Matemáticas}
\date{17/18}


% Paquetes adicionales

\usepackage[author={Víctor de Juan, 2017}]{pdfcomment}

\usepackage{pgf,tikz}
\usetikzlibrary{arrows}

\definecolor{qqttcc}{rgb}{0,0.2,0.8}
\definecolor{qqqqff}{rgb}{0,0,1}


\makeatletter
\newcommand{\annotate}[2][]{%
\pdfstringdef\x@title{#1}%
\edef\r{\string\r}%
\pdfstringdef\x@contents{#2}%
\pdfannot
width 2\baselineskip
height 2\baselineskip
depth 0pt
{
/Subtype /Text
/T (\x@title)
/Contents (\x@contents)
}%
}
\makeatother



\usepackage{eso-pic}
\newcommand\BackgroundPic{%
\put(0,0){%
\parbox[b][\paperheight]{\paperwidth}{%
\vfill
\centering
\includegraphics[width=\paperwidth,height=\paperheight,%
keepaspectratio]{../../../BWLogo.jpeg}%
\vfill
}}}




% --------------------
\newcommand{\cimplies}{\text{\hl{$\implies$}}}

\begin{document}
\pagestyle{plain}
%\maketitle


\AddToShipoutPicture{\BackgroundPic}

\begin{problem} (2.5 puntos)
Resuelve:

\ppart (2 puntos)
\[
	\sqrt{5-2x}+\sqrt{8x}-5 = 0
\]

\ppart (1 punto)
\[
3^{x+2}-9^x = 0
\]

\ppart (1,5 puntos)

\[
\left\{
	\begin{array}{c}
		2x-y=8\\
		4x^2-2xy=32
	\end{array}	
\right\}
\]

\ppart (2 puntos)
\[
	4\log_3x-\log_3\left(x^2-\frac{20}{9}\right) = 2
\]

\solution

\spart
\[
	\sqrt{5-2x}+\sqrt{8x}-5 = 0 \dimplies \sqrt{5-2x} = -\sqrt{8x}+5 \cimplies 
	\left(\sqrt{5-2x}\right)^2 = \left(-\sqrt{8x}+5\right)^2\]
\[ 
	5-2x=8x+25-10\sqrt{8x} \dimplies -10x\sqrt{8x} = -20-10x \cimplies 8x = (x+2)^2 \dimplies 8x = x^2+4x+4
\]
\[
	x^2+4x+4-8x=0 \dimplies x^2-4x+4=0 \dimplies x_1=x_2=2
\]

\textbf{Comprobación:}

\[
	\sqrt{5-2·2}+\sqrt{8·2}-5 = 1+4-5=0 \implies\text{ sí es solución.}
\]

\spart 
\[
	3^{x+2}-9^x = 0 \dimplies 3^{x+2} = 3^{2x} \implies x+2=2x \dimplies x=2
\]

\textbf{Comprobación:}

\[
	3^{2+2} - 9^2 = 3^4-9^2 = 81-81 = 0
\]

\spart

\[
\left\{
	\begin{array}{c}
		2x-y=8\\
		4x^2-2xy=32
	\end{array}	
\right\} \dimplies 
\left\{
	\begin{array}{c}
		2x-y=8\\
		2x(2x-y)=32
	\end{array}	
\right\}\dimplies 
\left\{
	\begin{array}{c}
		2x-y=8\\
		16x=32
	\end{array}	
\right\}
\]
\[
\left\{
	\begin{array}{l}
		x=2\\
		2·2-y = 8 \dimplies -y=8-4 \dimplies y=-4		
	\end{array}	
\right.
\]

\spart 
$(1): \log_aA^n = n\log_aA$

$(2): \log_a\rfrac{A}{B} = \log_aA-\log_aB$

\[
4\log_3x-\log_3\left(x^2-\frac{20}{9}\right) = 2 \overset{(1)}{\dimplies} \log_3x^4-\log_3\left(x^2-\frac{20}{9}\right) = 2 \overset{(2)}{\dimplies}
\]
\[
\log_3\frac{x^4}{x^2-\frac{20}{9}} = 2 \implies 3^2 = \frac{x^4}{x^2-\frac{20}{9}} \dimplies  3^2 = \frac{x^4}{\frac{9x^2-20}{9}} \]
\[
	9 = \frac{9x^4}{9x^2-20} \dimplies  1 = \frac{x^4}{9x^2-20} \dimplies 9x^2-20 = x^4 \dimplies x^4-9x^2+20=0
\]

Hacemos el cambio $t=x^2$

\[
x^4-9x^2+20=0 \implies t^2-9t+20 \implies \left\{
	\begin{array}{l}
		\left\{\begin{array}{l}
			t_1 = \pm5\\
			t_1 = x^2
			\end{array}
		\right.\to x=\pm\sqrt{5}\\
		\left\{\begin{array}{l}
			t_2 = \pm4\\
			t_2 = x^2
			\end{array}
		\right.\to x=\pm2\\
	\end{array}
\right.
\]

\textbf{Comprobación}

$x=-2;x=-\sqrt{5}$ no pueden ser solución porque no existen $\log$ de números negativos.

$x=2$

\[
	4\log_32-\log_3\left(2^2-\frac{20}{9}\right) = 4\log_32 - \log_3\left(\frac{36-20}{9}\right) \overset{(1)}{=} \log_32^4 - \log_3\frac{16}{9} \overset{(2)}{=}
\]
\[
	\log_3(16) - \log_3(16) + \log_39 = 2
\]

$x=5$

\end{problem}


\begin{problem}(1,5 puntos)
Cuando el polinomio $P(x) = x^3-2x^2+5kx-2$ se divide por $x+2$ da resto $k$. Halla el valor de $k$. ¿Qué teorema se puede utilizar? Enúncialo.

\solution
Teorema del resto: El resto de dividir $P(x)$ entre $x-a$ coincide con el valor $P(a)$

En este caso, calculamos $P(-2)=k$

\[
	P(-2) = k \dimplies -2^3-2·(-2)^2+5k(-2)-2= k \dimplies -16-10k-2=k\dimplies
\]
\[
	-18=11k \dimplies k=\frac{-18}{11}
\]


\end{problem}

\begin{problem} (2 puntos)
Simplificar las fracciones:

\ppart
\[
	\frac{x^4-2x^2-8}{x^3-4x} 
\]

\ppart

\[
	\frac{x^3-3x+2}{2x^4-2x^2}
\]

\solution

\spart 
\[\frac{x^4-2x^2-8}{x^3-4x}\]

\textbf{Factorizamos el numerador:}

$x^4-2x^2-8 = 0 \implies t^2-2t-8 = 0 \dimplies t_1=4\;;\;t_2=-2 $ 

Factorizamos: $t^2-2t-8 = (t-4)(t+2)$ y deshacemos el cambio

\[
	x^4-2x^2-8 = (x^2-4)(x^2+2) = (x+2)(x-2)(x^2+2)	
\]

\textbf{Factorizamos el denominador:}

$x^3-4x = x(x^2-4) = x(x-2)(x+2)$

Simplificamos:
\[
\frac{x^4-2x^2-8}{x^3-4x} = \frac{(x+2)(x-2)(x^2+2)	}{x(x-2)(x+2)} = \frac{x^2+2}{x}
\]

\spart 

\[
\frac{x^3-3x+2}{2x^4-2x^2}\]

Factorizamos \textbf{numerador}: $x^3-3x+2 = (x-1)^2(x+2)$

Factorizamos \textbf{denominador}: $2x^4-2x^2 = 2x^2(x^2-1) = 2x^2(x+1)(x-1)$

Simplificamos:
\[\frac{(x-1)(x+2)(x-1)}{2x^2(x^2-1)} = \frac{(x-1)(x+2)(x-1)}{2x^2(x-1)(x+1)} = \frac{(x+2)(x-1)}{2x^2(x+1)}
\]

\end{problem}


\end{document}
