\documentclass[palatino,nosec]{Docencia}


\title{Examen modelo (4ºA)}
\author{Departamento de Matemáticas}
\date{17/18}


% Paquetes adicionales

\usepackage[author={Víctor de Juan, 2017}]{pdfcomment}

\usepackage{pgf,tikz}
\usetikzlibrary{arrows}

\definecolor{qqttcc}{rgb}{0,0.2,0.8}
\definecolor{qqqqff}{rgb}{0,0,1}


\makeatletter
\newcommand{\annotate}[2][]{%
\pdfstringdef\x@title{#1}%
\edef\r{\string\r}%
\pdfstringdef\x@contents{#2}%
\pdfannot
width 2\baselineskip
height 2\baselineskip
depth 0pt
{
/Subtype /Text
/T (\x@title)
/Contents (\x@contents)
}%
}
\makeatother



\usepackage{eso-pic}
\newcommand\BackgroundPic{%
\put(0,0){%
\parbox[b][\paperheight]{\paperwidth}{%
\vfill
\centering
\includegraphics[width=\paperwidth,height=\paperheight,%
keepaspectratio]{../../../BWLogo.jpeg}%
\vfill
}}}





\begin{abstract}
Examen modelo de ecuaciones con el temario visto en 4ºA. 

\end{abstract}

% --------------------
\newcommand{\cimplies}{\text{\hl{$\implies$}}}

\begin{document}
\pagestyle{plain}


\AddToShipoutPicture{\BackgroundPic}

\begin{problem} (1.5 puntos)

Resuelve la siguiente ecuación:

\[
	\frac{2x}{x^2-4} - \frac{2}{x+2} = \frac{-1}{x+1}
\]

\solution

\end{problem}

\begin{problem}(1.5 puntos)
Resuelve la siguiente ecuación:

\[ 2\log_2(x-1) - 2\log_2\sqrt{x^2-1} = 1\]
\solution

\end{problem}

\begin{problem} (2 puntos)
Resuelve la siguiente ecuación:

\[
	2^{4x}-4·2^{2x-1} + 2^0 = 0
\]

\solution
\end{problem}

\begin{problem}(1,5 puntos)

Elige una de las 2 ecuaciones para resolver:
\ppart
\[
	\sqrt{x} - \sqrt{x+1} = \frac{\sqrt{x}}{\sqrt{x}+\sqrt{x+1}}
\]
\ppart 
\[
	1+\sqrt{x-1} = \frac{2}{\sqrt{x-1}}
\]
\solution

\end{problem}

\begin{problem}(1,5 puntos) 
Resuelve el siguiente sistema 

\[
\left\{
	\begin{array}{ccc}
		6x&-y&=6\\
		5x&+2y&=-12
	\end{array}
\right\}
\]

\solution


\end{problem}

\begin{problem} (2 puntos)

Resuelve la siguiente ecuación logarítmica:

\[\log_2\sqrt{x+1} = \log_8\sqrt{x^2+4x+3}\]

\solution


\end{problem}


\end{document}
