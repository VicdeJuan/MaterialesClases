\documentclass[palatino,nosec]{Docencia}


\title{Examen 4º A, preevaluación}
\author{Víctor de Juan}
\date{17/18}


% Paquetes adicionales

\usepackage[author={Víctor de Juan, 2017}]{pdfcomment}

\usepackage{pgf,tikz}
\usetikzlibrary{arrows}

\definecolor{qqttcc}{rgb}{0,0.2,0.8}
\definecolor{qqqqff}{rgb}{0,0,1}


\makeatletter
\newcommand{\annotate}[2][]{%
\pdfstringdef\x@title{#1}%
\edef\r{\string\r}%
\pdfstringdef\x@contents{#2}%
\pdfannot
width 2\baselineskip
height 2\baselineskip
depth 0pt
{
/Subtype /Text
/T (\x@title)
/Contents (\x@contents)
}%
}
\makeatother



\usepackage{eso-pic}
\newcommand\BackgroundPic{%
\put(0,0){%
\parbox[b][\paperheight]{\paperwidth}{%
\vfill
\centering
\includegraphics[width=\paperwidth,height=\paperheight,%
keepaspectratio]{../../../BWLogo.jpeg}%
\vfill
}}}





\begin{abstract}
Solución del examen de preevaluación.

\nota{Estos ejemplos no están exentos de erratas. En caso de descubrir alguna, por favor, comunicarlas al autor.}
\end{abstract}

% --------------------
\newcommand{\cimplies}{\text{\hl{$\implies$}}}

\begin{document}
\pagestyle{plain}
\maketitle

\AddToShipoutPicture{\BackgroundPic}

\begin{problem}
(1p.) Simplifica utilizando propiedades de potencias o radicales:

\[
\frac{b\sqrt[3]{8ab}\sqrt{9b^5a}}{6\sqrt[3]{a}}
\]

\solution

\[
\frac{b\sqrt[3]{8ab}\sqrt{9b^5a}}{6\sqrt[3]{a}} = \frac{b\sqrt[3]{2^3ab}\sqrt{3^2b^5a}}{6\sqrt[3]{a}} = \frac{b2\sqrt[3]{ab}3b^2\sqrt{ba}}{6\sqrt[3]{a}} = \frac{b^3\sqrt[3]{ab}\sqrt{ba}}{\sqrt[3]{a}}
\]

Ahora se puede hacer radical común en el numerador, racionalizar o simplificar:

\[
\frac{b^3\sqrt[3]{a}\sqrt[3]{b}\sqrt{ba}}{\sqrt[3]{a}} = b^3\sqrt[3]{b}\sqrt{ba} = b^3\sqrt[6]{b^2·(ab)^3} = b^3\sqrt[6]{b^5a^3}
\]

\end{problem}

\begin{problem}(1,5 p.) Halla el valor de $A$ en la expresión:

\[
	\log A = 2\log 2 + 3\log3 - 2 - \log 18
\]

\solution

\[
	\log A = 2\log 2 + \log3 - 2 - \log 18 \overset{(1)}{\dimplies} \log A = \log 2^2 + \log3 - 2 - \log 18 \overset{(2)}{\dimplies}
\]
\[ 
	\log A = \log\left(\frac{2^2·3}{18}\right) - \log 100 \overset{(2)}{\dimplies} \log A = \log\frac{2}{3·100} \implies A=\frac{1}{150}
\]

donde $(1) \log_aA^n = n\log_aA$ y $(2): \log_a(AB) = \log_aA + \log_aB\;;\;\log_a\frac{A}{B} = \log_aA - \log_aB$
\end{problem}

\begin{problem}(1 p.)
Demuestra $\log_{a^2}x^2 = \log_ax$
\solution

Utilizando la propiedad cambio de base: $(1): \log_aA = \frac{\log_bA}{\log_ba}$

\[
\log_{a^2}x^2 \overset{(1)}{=}  \frac{\log_ax^2}{\log_aa^2} \overset{(2)}{=} \frac{2\log_ax}{2\log_aa} = \log_ax
\]

donde $(2): \log_aA^n = n\log_aA$
\end{problem}

\begin{problem}
Sabiendo $\log 4 = 0,602$, calcula:
\ppart $\log64 - \log16$
\ppart $\log_4 \sqrt{1000}$
\ppart $\log_16 100$
\solution

$(1): \log_a\frac{A}{B} = \log_aA - \log_aB$ y $(2): \log_aA = \frac{\log_bA}{\log_ba} $

\spart $\log64 - \log16 \overset{(1)}{=} \log{\frac{64}{16}} = \log 4 = 0,602$

\spart $\log_4\sqrt{1000} = \log_410^{\rfrac{3}{2}} \overset{(2)}{=} \displaystyle\frac{\log 10^{\rfrac{3}{2}}}{\log 4} = \frac{\rfrac{3}{2}}{0,602} = \frac{3}{2·0,602}$

\spart Utilizando el ejercicio 3, $\log_{a^2}x^2 = \log_ax$; 

\[
\log_{16} 100 = \log_{4^2}10^2 = \log_410=0,602
\]
\end{problem}



\begin{problem} (1 p.)
Halla el valor o valores de $m$ para que $P(x) = 5x^4 - (mx)^2 + 20$ sea divisible por $(x+1)$. \textbf{Enuncia} el teorema o teoremas que utilices.

\solution

Teorema del factor: Sea $a\in\real$ y $P(x)$ un polinomio.Entonces, $(x-a)$ es un factor de $P(x)$, es decir, $P(x)$ es divisible por $(x-a)$ si, y sólo si, $a$ es una raíz, es decir, $P(a) = 0$.

Utilizando el teorema del factor, buscamos $P(-1) = 0$

\[
	P(-1) = 0 \dimplies 5(-1)^4 - (m·(-1))^2 + 20 = 0 \dimplies 5-m^2(-1)^2+20 = 0 
\]
\[
	\dimplies 5-m^2+20=0 \dimplies m^2 = \sqrt{25} \implies m=\pm 5
\]

$m$ puede tomar 2 posibles valores, $+5$ y $-5$

\end{problem}


\begin{problem} (1,5 p.) 
Factoriza el polinomio $P(x) = x^3+3x^2+3x+1$
\solution

Buscamos la posibles raíces:

$P(1) = 1+3+3+1 ≠ 0$, por lo que $1$ no es raíz del polinomio.

$P(-1) = (-1)^3 + 3(-1)^2+3(-1)+1 = -1+3-3+1 = 0$, por lo que $-1$ es raíz del polinomio. Por el teorema del factor, $(x+1)$ será un factor del polinomio.

Dividiendo (por Ruffini), obtenemos una identidad notable: 

\[
	\frac{x^3+3x^2+3x+1}{x+1} = x^2+2x+1 = (x+1)^2
\]

Por lo tanto, $P(x) = x^3+3x^2+3x+1 = (x+1)^3$
\end{problem}



\begin{problem} (1,5 p.)

Pau Gasol quiere construir un estadio de baloncesto con sus ahorros. Dispone de $1.000.000$ euros y su banco de confianza, que le conoce desde hace muchos años, le ofrece un interés del 100\%. 

Si quiere llegar a $128.000.000$ euros para el estadio, ¿cuántos años debe esperar?
\solution

Utilizamos la fórmula del interés compuesto: $C_F = C_I·\left(1+\frac{r}{100}\right)^t$, con los datos del enunciado:
\begin{itemize}
	\item $C_F = 128·10^6$
	\item $C_I = 10^6$
	\item $r=100\%$
	\item $t=?$
\end{itemize}

Despejamos el tiempo:

\[
C_F = C_I·\left(1+\frac{r}{100}\right)^t \dimplies \frac{C_F}{C_I} = \left(1+\frac{r}{100}\right)^t \implies \log \frac{C_F}{C_I} = \log \left(1+\frac{r}{100}\right)^t 
\]

\[\overset{(1)}{\dimplies} \log \frac{C_F}{C_I} = t·\log \left(1+\frac{r}{100}\right) \dimplies  t= \frac{\log\frac{C_F}{C_I}}{\log \left(1+\frac{r}{100}\right)}
\]

donde $(1): \log_aA^n = n\log_aA$

Sustituyendo los datos del problema, obtenemos:

\[
  t= \frac{\log\frac{C_F}{C_I}}{\log \left(1+\frac{r}{100}\right)} = \frac{\log{\frac{128·10^6}{10^6}}}{\log\left(1+\frac{100}{100}\right)} = \frac{\log 128}{\log 2} \overset{\ast}{=} \frac{\log 2^7}{\log 2} \overset{(1)}{=} \frac{7·\log2}{\log2} = 7
\]

En $\ast$ también podríamos haber aplicado un cambio de base, obteniendo $\log_2128 = 7$.

\begin{center}
\textcolor{red}{Error común: } $\displaystyle\frac{\log 128}{\log 2} = \log 64 = 8$
\end{center}
\end{problem}



\begin{problem}
Expresa en forma de entorno e intervalo el siguiente valor absoluto: $\{|x-1| < 2\}$
\solution

Definición de entorno: $E_c(r) = \{ x\in\real: |x-c|<r\}$, por lo que 
\[
	\{x\in\real: |x-1| < 2\} = E_1(2)
\]

Transformación de entorno en intervalo $E_c(r) = (c-r,c+r)$. En este caso, $E_1(2) = (1-2,1+2) = (-1,3)$.

Representación en la recta real.

\begin{center}
\begin{tikzpicture}[line cap=round,line join=round,>=triangle 45,x=1.0cm,y=1.0cm]
\draw[->,color=black] (-4.06,0) -- (8.5,0);
\foreach \x in {-4,-3,-2,-1,1,2,3,4,5,6,7,8}
\draw[shift={(\x,0)},color=black] (0pt,2pt) -- (0pt,-2pt) node[below] {\footnotesize $\x$};
\draw[color=black] (8.27,0.06) node [anchor=south west] { x};
\draw[color=black] (0pt,-10pt) node[right] {\footnotesize $0$};
\clip(-4.06,-3.8) rectangle (8.5,5.02);
\draw [line width=3.2pt,color=qqttcc] (-1,0)-- (3,0);
\begin{scriptsize}
\draw [color=qqqqff] (-1,0) arc (0:45:-0.7);
\draw [color=qqqqff] (-1,0) arc (0:-45:-0.7);
\draw [color=qqqqff] (3,0) arc (0:45:0.7);
\draw [color=qqqqff] (3,0) arc (0:-45:0.7);
\end{scriptsize}
\end{tikzpicture}
\end{center}

\end{problem}

\end{document}
