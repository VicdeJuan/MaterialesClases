\documentclass[palatino,nosec]{Docencia}


\title{Segundo parcial 4ºA}
\author{Departamento de Matemáticas}
\date{17/18}


% Paquetes adicionales

\usepackage[author={Víctor de Juan, 2017}]{pdfcomment}

\usepackage{pgf,tikz}
\usetikzlibrary{arrows}

\definecolor{qqttcc}{rgb}{0,0.2,0.8}
\definecolor{qqqqff}{rgb}{0,0,1}


\makeatletter
\newcommand{\annotate}[2][]{%
\pdfstringdef\x@title{#1}%
\edef\r{\string\r}%
\pdfstringdef\x@contents{#2}%
\pdfannot
width 2\baselineskip
height 2\baselineskip
depth 0pt
{
/Subtype /Text
/T (\x@title)
/Contents (\x@contents)
}%
}
\makeatother



\usepackage{eso-pic}
\newcommand\BackgroundPic{%
\put(0,0){%
\parbox[b][\paperheight]{\paperwidth}{%
\vfill
\centering
\includegraphics[width=\paperwidth,height=\paperheight,%
keepaspectratio]{../../../BWLogo.jpeg}%
\vfill
}}}




% --------------------
\newcommand{\cimplies}{\text{\hl{$\implies$}}}

\begin{document}
\pagestyle{plain}
%\maketitle


\AddToShipoutPicture{\BackgroundPic}

\begin{problem} (2.5 puntos)

Resuelve la siguiente ecuación:

\[
	\sqrt{x+9}-1+\sqrt{x+2} = 0
\]

\solution

\[
	\sqrt{x+9}-1+\sqrt{x+2} = 0 \dimplies \sqrt{x+9} = 1-\sqrt{x+2} \implies \left(\sqrt{x+9}\right)^2 = \left(1-\sqrt{x+2}\right)^2\dimplies\]
\[
	\left(\sqrt{x+9}\right)^2 = 1+\left(\sqrt{x+2}\right)^2 - 2\sqrt{x+2}\dimplies  
	x+9-1-x-2 = -2\sqrt{x+2} \dimplies\]
\[ 6=-2\sqrt{x+2} \implies (-3)^2=\left(\sqrt{x+2}\right)^2 \dimplies 9 = x+2 \dimplies x=7
\]

\textbf{Comprobación:}

\[
	\sqrt{7+9} - 1 + \sqrt{7+2} = 4-1+3 = 6 ≠ 0 \implies \text{No es solución.}
\]
\end{problem}


\begin{problem}(1 puntos)
Despeja $t$ en la siguiente ecuación
\[A=B·C^t\]
\solution

\[A=B·C^t \dimplies \frac{A}{B} = C^t \implies \log\frac{A}{B} = \log C^t \overset{(1)}{\dimplies} \log\frac{A}{B} = t\log C \implies t=\frac{\log\frac{A}{B}}{\log C}\]

Donde $(1): \log_aA^n=n\log_aA$

\end{problem}

\begin{problem} (2 puntos)
Resuelve la siguiente ecuación:

\[
	\frac{2}{x+1}-\frac{3-3x}{x^2-1} =\frac{2}{x-1} + \frac{7}{x+1}
\]

\solution

\[
\frac{2(x-1)}{(x+1)(x-1)}-\frac{3-3x}{(x+1)(x-1)} =\frac{2(x+1)}{(x+1)(x-1)} + \frac{7(x-1)}{(x+1)(x-1)} \implies 2x-2-3+3x = 2x+2+7x-7\]
\[
	5x-5=9x-5 \dimplies 4x=0 \dimplies x=0
\]

\textbf{Comprobación:}

\[
	\frac{2}{0+1}-\frac{3-3·0}{0^2-1} =\frac{2}{0-1} + \frac{7}{0+1} \dimplies 2+3=-2+7 \dimplies 5=5\implies\text{ Sí es solución.}
\]

\end{problem}



\begin{problem}(1.5 puntos) 
Indica el número de soluciones que tiene el siguiente sistema. En caso de que tenga una única solución, resuélvelo con comprobación.

\[
\left\{
	\begin{array}{ccc}
		x&-\rfrac{3}{7}y&=2\\
		-14x&+6y&=2
	\end{array}
\right\}
\]

\solution


\[
\left\{
	\begin{array}{ccc}
		x&-\rfrac{3}{7}y&=2\\
		-14x&+6y&=2
	\end{array}
\right\} \dimplies 
\left\{
	\begin{array}{ccc}
		14x&-6y&=28\\
		-14x&+6y&=2
	\end{array}
\right\} \implies 
\left.
	\begin{array}{lccc}
		 &14x&-6y&=28\\
		+&-14x&+6y&=2\\\hline
		 &0x&+0y&=30
	\end{array}
\right.
\]

El sistema es incompatible. No tiene solución.

\end{problem}


\begin{problem}(3 puntos)

Elige una de las 2 ecuaciones para resolver:
\ppart
\[
	(\log x)^4 + (2\log x)^2 = -3
\]
\ppart 
\[
	152·\log\left(\sqrt[152]{x+1}\right) - 2\log(\sqrt{6}x)-1=-1
\]
\ppart 
\[
	\log_4x = \log_{16}(4x)
\]

\solution

\spart
\[
	(\log x)^4 + (2\log x)^2 = -3 \dimplies (\log x)^4 + 4(\log x)^2 + 3 =0
\]

Hacemos el cambio $(\log x)^2 = t$ y obtenemos: $t^2+4t+3 = 0 \dimplies t_1=-1\;;\; t_2=-3$.

Deshaciendo el cambio:

\[
\left\{
\begin{array}{cc}
	t=-1\\
	t=(\log	x)^2
\end{array}
\right\} \implies (\log x)^2 = -1 \implies \log x=\sqrt{-1}\not\in\real
\]

\[
\left\{
\begin{array}{cc}
	t=-3\\
	t=(\log	x)^2
\end{array}
\right\} \implies (\log x)^2 = -3 \implies \log x=\sqrt{-3}\not\in\real
\]

\spart 
\[
	152·\log\left(\sqrt[152]{x+1}\right) - 2\log(\sqrt{6}x)-1=-1 \overset{(1)}{\dimplies} \log\left(\sqrt[152]{x+1}\right)^{152} -  \log\left(\sqrt{6}x\right)^2 = -1+1
\]
Utilizando $\log_a\rfrac{A}{B} = \log_a A - \log_a B$
\[
	\log\frac{x+1}{6x^2}=0 \implies 10^0 = \frac{x+1}{6x^2} \implies 6x^2-x-1 = 0 \dimplies x_1=\rfrac{1}{2} \;\;;\;\;x_2=\rfrac{-1}{3}
\]

$(1): \log_aA^n = n\log_aA$

\textbf{Comprobación:}

$x=\rfrac{1}{2}:$

\[
	152·\log\left(\sqrt[152]{\rfrac{1}{2}+1}\right) - 2\log(\sqrt{6}\rfrac{1}{2})-1=-1 \overset{(1)}{\dimplies} \log\left(\sqrt[152]{\rfrac{1}{2}+1}\right)^{152} -  \log\left(\sqrt{6}\rfrac{1}{2}\right)^2 = -1+1
\]
\[
	\log\frac{3}{2} - \log\frac{\sqrt{6^2}}{4} = 0 \dimplies \log\frac{3}{2} - \log\frac{3}{2} = 0
\]

$x=\rfrac{-1}{3}:$

\[
	152·\log\left(\sqrt[152]{\rfrac{-1}{3}+1}\right) - 2\log(\sqrt{6}\rfrac{-1}{3})\not\in\real\implies \text{ No es solución}
\]

\spart 

Cambio de base: $\log_aA = \frac{\log_bA}{\log_ba}$

\[
	\log_4x = \log_{16}(4x) \dimplies \log_4x = \frac{\log_44x}{\log_416} \dimplies 2\log_4x = \log_44x \dimplies \log_4x^2=\log_4x\implies\]
\[
	x^2=4x \dimplies x_1=4\;\;;\;\;x_2=0
\]

\textbf{Comprobación:}

$x=0:$ No es solución porque $\log_40\not\in\real$

$x=4:$ 

\[
	\log_44 = \log_16(4·4) \dimplies \log_44 = \log_{16}{16} \dimplies 1=1
\]

\end{problem}



\end{document}
