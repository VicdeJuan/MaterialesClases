\documentclass[palatino,nosec]{Docencia}


\title{Examen 4º E, preevaluación}
\author{Víctor de Juan}
\date{17/18}


% Paquetes adicionales

\usepackage[author={Víctor de Juan, 2017}]{pdfcomment}

\usepackage{pgf,tikz}
\usetikzlibrary{arrows}

\definecolor{qqttcc}{rgb}{0,0.2,0.8}
\definecolor{qqqqff}{rgb}{0,0,1}


\makeatletter
\newcommand{\annotate}[2][]{%
\pdfstringdef\x@title{#1}%
\edef\r{\string\r}%
\pdfstringdef\x@contents{#2}%
\pdfannot
width 2\baselineskip
height 2\baselineskip
depth 0pt
{
/Subtype /Text
/T (\x@title)
/Contents (\x@contents)
}%
}
\makeatother



\usepackage{eso-pic}
\newcommand\BackgroundPic{%
\put(0,0){%
\parbox[b][\paperheight]{\paperwidth}{%
\vfill
\centering
\includegraphics[width=\paperwidth,height=\paperheight,%
keepaspectratio]{../../../BWLogo.jpeg}%
\vfill
}}}





\begin{abstract}
Solución del examen de preevaluación.

\nota{Estos ejemplos no están exentos de erratas. En caso de descubrir alguna, por favor, comunicarlas al autor.}
\end{abstract}

% --------------------
\newcommand{\cimplies}{\text{\hl{$\implies$}}}

\begin{document}
\pagestyle{plain}
\maketitle

\AddToShipoutPicture{\BackgroundPic}

\begin{problem}
(1,5 p.) Expresa en forma de entorno e intervalo el siguiente valor absoluto $\{|x+0,75| < 2,5\}$. Represéntalo.

\solution

Definición de entorno: $E_c(r) = \{ x\in\real: |x-c|<r\}$, por lo que 
\[
	\{x\in\real: |x+0.75| < 2.5\} = E_{-0.75}(2.5)
\]

Transformación de entorno en intervalo $E_c(r) = (c-r,c+r)$. En este caso, $E_{-0.75}(2.5) = (-0.75-2.5 , -0.75+2.5) = (-3.25,1.75)$.

Representación en la recta real:

\begin{center}
\vspace{-4cm}
\begin{tikzpicture}[line cap=round,line join=round,>=triangle 45,x=1.0cm,y=1.0cm]
\draw[->,color=black] (-4.06,0) -- (8.5,0);
\foreach \x in {-4,-3,-2,-1,1,2,3,4,5,6,7,8}
\draw[shift={(\x,0)},color=black] (0pt,2pt) -- (0pt,-2pt) node[below] {\footnotesize $\x$};
\draw[color=black] (8.27,0.06) node [anchor=south west] {};
\draw[color=black] (-3.25,0) node [anchor=south] {$-3.25$};
\draw[color=black] (1.75,0) node [anchor=south] {$1.75$};
\draw[color=black] (0pt,-10pt) node[right] {\footnotesize $0$};
\clip(-4.06,-3.8) rectangle (8.5,5.02);
\draw [line width=3.2pt,color=qqttcc] (-3.25,0)-- (1.75,0);
\begin{scriptsize}
\draw [color=qqqqff] (-3.25,0) arc (0:45:-0.7);
\draw [color=qqqqff] (-3.25,0) arc (0:-45:-0.7);
\draw [color=qqqqff] (1.75,0) arc (0:45:0.7);
\draw [color=qqqqff] (1.75,0) arc (0:-45:0.7);
\end{scriptsize}
\end{tikzpicture}
\vspace{-3cm}
\end{center}



\end{problem}

\begin{problem}(1,25 p.) Simplifica, utilizando propiedades de potencias

\[
\frac{4b\sqrt[3]{2ab}\sqrt{b^5a}}{\sqrt[3]{8a}}
\]

\solution

\[
\frac{4b\sqrt[3]{2ab}\sqrt{b^5a}}{\sqrt[3]{2^3a}} = \frac{4b\sqrt[3]{2ab}\sqrt{b^5a}}{2\sqrt[3]{a}} = \frac{2b\sqrt[3]{2ab}·b^2·\sqrt{ba}}{\sqrt[3]{a}} = \frac{2b^3\sqrt[3]{2ab}\sqrt{ba}}{\sqrt[3]{a}}
\]

Ahora se puede hacer radical común en el numerador, racionalizar o simplificar. En este caso, $\sqrt[n]{ab} = \sqrt[n]{a}\sqrt[n]{b}$

\[
\frac{2b^3\sqrt[3]{2b}\sqrt[3]{a}\sqrt{ba}}{\sqrt[3]{a}} = 2b^3\sqrt[3]{2b}\sqrt{ba} = 2b^3\sqrt[6]{(2b)^2(ba)^3} = 2b^3\sqrt[6]{4b^2b^3a^3} = 2b^3\sqrt[6]{2^2b^5a^3}
\]
Tres maneras diferentes de expresar el resultado final.
\[
	\left[2b^3\sqrt[6]{2^2b^5a^3}\right] \dimplies \sqrt[6]{2^62^2b^{18}b^5a^3} = \left[\sqrt[6]{2^8b^{23}a^3}\right] \dimplies 2^{\rfrac{8}{6}}b^{\rfrac{23}{6}}a^{\rfrac{3}{6}} = \left[2^{\rfrac{4}{3}}b^{\rfrac{23}{6}}a^{\rfrac{1}{2}}\right]
\]


\end{problem}

\begin{problem}(1,5 p.) Halla el valor de $A$ en la expresión:
\[
	\log A = 2\log 2 + \log3 - 2 - \log 18
\]

\solution

\[
	\log A = 2\log 2 + \log3 - 2 - \log 18 \overset{(1)}{\dimplies} \log A = \log 2^2 + \log3 - 2 - \log 18 \overset{(2)}{\dimplies}
\]
\[ 
	\log A = \log\left(\frac{2^2·3}{18}\right) - \log 100 \overset{(2)}{\dimplies} \log A = \log\frac{2}{3·100} \implies A=\frac{1}{150}
\]

donde $(1) \log_aA^n = n\log_aA$ y $(2): \log_a(AB) = \log_aA + \log_aB\;;\;\log_a\frac{A}{B} = \log_aA - \log_aB$

\end{problem}

\begin{problem}(1 p.)Demuestra $\log_ab·\log_ba = 1$

\solution

Utilizando $(2): \log_aA = \frac{\log_bA}{\log_ba} $, escribimos:

\[
\log_ab·\log_ba = \log_ab ·\frac{\log_aa}{\log_ab} \overset{(1)}{=}  \frac{\log_ab}{\log_ab} = 1
\]

$(1): \log_aa=1$ por definición de logaritmo.
\end{problem}



\begin{problem} (1,25 p.)
Racionaliza y simplifica: $\displaystyle \frac{\sqrt{2}+\sqrt{3}}{\sqrt{2}-\sqrt{3}}$
\solution
\[
	\frac{\sqrt{2}+\sqrt{3}}{\sqrt{2}-\sqrt{3}} = \frac{\sqrt{2}+\sqrt{3}}{\sqrt{2}-\sqrt{3}} · \frac{\sqrt{2}+\sqrt{3}}{\sqrt{2}+\sqrt{3}} = \frac{(\sqrt{2}+\sqrt{3})^2}{(\sqrt{2}-\sqrt{3})(\sqrt{2}+\sqrt{3})} = \frac{\sqrt{2}^2+\sqrt{3}^2+2\sqrt{2}\sqrt{3}}{\sqrt{2}^2-\sqrt{3}^2} =
\]
\[
	 \frac{5+2\sqrt{6}}{-1} = -5-2\sqrt{6}
\]

\end{problem}


\begin{problem} (2 p.) 
Halla las raíces del polinomio $P(x) = 4x^3-8x^2-x+2$. Factorízalo.

\solution

Buscamos la posibles raíces:

$P(1) = 4-8-1+2 ≠ 0$, por lo que $1$ no es raíz del polinomio.

$P(-1) = 4·(-1)^3 + 8(-1)^2-(-1)+2 = -4+8+1+2 ≠ 0$,por lo que $-1$ no es raíz del polinomio.

$P(2) = 4·2^3-8·2^2-2+2 = 64-64-2+2 = 0$, por lo que $x_1=2$ será una raíz del polinomio, y por el teorema del factor, $(x-2)$ será un factor del polinomio.



Dividiendo (por Ruffini), obtenemos una identidad notable: 
\[
	\frac{4x^3-8x^2-x+2}{x-2} = 4x^2-1 = (2x+1)(2x-1)
\]


¿Por lo tanto $P(x) = 4x^3-8x^2-x+2 = (x-2)(2x+1)(2x-1)$?

\hl{Esto NO es una factorización.} Una factorización de un polinomio es un producto de factores de la forma $(x-a)$. 

Por otro lado, el ejercicio nos pide hallar las raíces. En este caso, las otras 2 raíces del polinomio son:

$2x+1 = 0 \dimplies x_2=-\rfrac{1}{2} \to\text{ factor: } \left(x+\rfrac{1}{2}\right)$

$2x-1 = 0 \dimplies x_3=+\rfrac{1}{2} \to\text{ factor: } \left(x-\rfrac{1}{2}\right)$

\[
P(x) = 4x^3-8x^2-x+2 = (x-2)(2x+1)(2x-1) = \left[4(x-2)\left(x-\rfrac{1}{2}\right)\left(x+\rfrac{1}{2}\right)\right]
\]



\end{problem}



\begin{problem} (1,5 p.)
Calcula la incógnita aplicando la definición de logaritmo:

\ppart $\log_a 36 = -2$

\ppart $\log_3\frac{\sqrt{3}}{9} = x$

\ppart $\log_{\frac{1}{10}}N = 4$

\solution

\spart 

\[\log_a 36 = -2 \dimplies a^{-2} = 36 \dimplies \frac{1}{a^2}=36 \dimplies a^2=\frac{1}{36} \implies a=\pm\frac{1}{6}\]

Como la base del logaritmo no puede ser negativa, $a=\rfrac{1}{6}$.

\spart 

Forma 1 de resolver, utilizando $\log_a\frac{A}{B} = \log_aA-\log_aB$:
\[
	\log_3\frac{\sqrt{3}}{9} = \log_3\sqrt{3} - \log_39 = \frac{1}{2}-2=\frac{1-4}{2} = \frac{-3}{4}
\]

Forma 2 de resolver, utilizando propiedades de potencias:
\[
	\log_3\frac{\sqrt{3}}{9} = \log_3\frac{3^{\rfrac{1}{2}}}{3^2} =\log_33^{\rfrac{1}{2}-2} = \frac{1}{2}-2 = \frac{-3}{4} 
\]

\spart 

\[	\log_{\frac{1}{10}}N = 4 \dimplies \left(\frac{1}{10}\right)^4 = N \dimplies N=10^{-4}\]

\end{problem}



\end{document}
