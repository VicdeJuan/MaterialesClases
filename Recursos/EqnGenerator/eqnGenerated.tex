\documentclass[palatino,nosec]{Docencia}

\usepackage{sagetex}

\title{Análisis de Funciones}
%\newcommand{\subtitle}[1]{\def\thesubtitle{#1}\@subsettrue}
%\subtitle{4º ESO Matemáticas Académicas}
\author{Departamento de Matemáticas}
\date{17/18}



\begin{abstract}

Ejemplos resueltos del análisis completo de una función. Los ejercicios han sido generados automáticamente


\textit{Nota: Este documento se ha generado automáticamente a partir de las funciones utilizando \href{www.sagemath.org/es/}{Sage} para los cálculos matemáticos, \LaTeX\xspace para la generación del pdf y \href{https://github.com/sagemath/sagetex}{sagetex} para la integración de las 2 herramientas mencionadas.}

\end{abstract}

% Paquetes adicionales

\usepackage[author={Víctor de Juan, 2017}]{pdfcomment}

\makeatletter
\newcommand{\annotate}[2][]{%
	\pdfstringdef\x@title{#1}%
	\edef\r{\string\r}%
	\pdfstringdef\x@contents{#2}%
	\pdfannot
	width 2\baselineskip
	height 2\baselineskip
	depth 0pt
	{
		/Subtype /Text
		/T (\x@title)
		/Contents (\x@contents)
	}%
}
\makeatother



% --------------------
\newcommand{\cimplies}{\text{\hl{$\implies$}}}
\renewcommand{\vx}{\overset{\rightarrow}{x}}
\renewcommand{\vy}{\overset{\rightarrow}{y}}
\renewcommand{\vz}{\overset{\rightarrow}{z}}
\newcommand{\vi}{\overset{\rightarrow}{i}}
\newcommand{\vj}{\overset{\rightarrow}{j}}
\renewcommand{\vec}[1]{\overset{\rightarrow}{#1}}



\begin{document}

Sol: -10:\[-\frac{2687}{336}x + 17-\frac{65}{882}x + \frac{5}{126}+\frac{160}{3}x - 15=-8x + \frac{2851}{168}-\frac{2}{27}x + \frac{95}{2646}+\frac{140}{3}x - \frac{245}{3}\]


Sol: -9:\[\frac{27}{4}x - 6+40x - 110+\frac{4}{5}x - 15=6x - \frac{51}{4}-470-\frac{111}{5}\]


Sol: -8:\[\frac{13}{4}x - \frac{3}{2}-1125x - 1875-91x=3x - \frac{7}{2}-1500x - 4875-109x - 144\]


Sol: -7:\[\frac{143}{45}x+12x + 4-\frac{632}{7}x=3x - \frac{56}{45}-3x - 101-\frac{1208}{7}x - 576\]


Sol: -6:\[-\frac{119}{30}x + \frac{1}{2}+27x - 206-\frac{2}{3}x - \frac{20}{3}=-4x + \frac{3}{10}-368-\frac{3}{2}x - \frac{35}{3}\]


Sol: -5:\[52x - 725+\frac{11}{210}x + \frac{62}{15}-2x - 12=36x - 805+\frac{1}{21}x + \frac{863}{210}-4x - 22\]


Sol: -4:\[-\frac{145}{12}x + \frac{15}{8}-\frac{6803}{486}x + \frac{185}{27}+\frac{8}{21}x=-\frac{25}{2}x + \frac{5}{24}-14x + \frac{1663}{243}-\frac{32}{21}\]


Sol: -3:\[\frac{149}{2}x - 6-\frac{9}{4}x - \frac{1}{4}+\frac{357}{2}x - 9=62x - \frac{87}{2}-\frac{7}{2}x - 4+171x - \frac{63}{2}\]


Sol: -2:\[\frac{25}{3}x - 200+\frac{8}{35}x - 172\frac{8}{35}+\frac{1}{4}x + \frac{33}{4}=\frac{50}{7}x - \frac{4250}{21}-\frac{1744}{35}+\frac{31}{4}\]


Sol: -1:\[\frac{1}{1080}x + \frac{7}{60}+552x - 360+\frac{19}{12}x - \frac{1}{12}=\frac{25}{216}+492x - 420+\frac{37}{24}x - \frac{1}{8}\]


Sol: 0:\[5x + \frac{79}{12}-\frac{595}{12}x - \frac{235}{6}+91x - 60=\frac{79}{12}-50x - \frac{235}{6}+81x - 60\]


Sol: 1:\[\frac{337}{36}x - \frac{169}{18}+\frac{31}{2}x + \frac{3}{4}-\frac{19}{18}x=\frac{28}{3}x - \frac{337}{36}+\frac{185}{12}x + \frac{5}{6}-\frac{685}{648}x + \frac{1}{648}\]


Sol: 2:\[-\frac{17}{30}x - 1+\frac{14}{3}x + \frac{10}{9}+\frac{17}{18}x + \frac{11}{8}=-\frac{2}{3}x - \frac{4}{5}+\frac{83}{18}x + \frac{11}{9}+\frac{67}{72}x + \frac{101}{72}\]


Sol: 3:\[120x + \frac{40}{3}-\frac{135}{64}x+\frac{10}{27}x - \frac{20}{9}=\frac{1120}{3}-\frac{17}{8}x + \frac{3}{64}+\frac{5}{27}x - \frac{5}{3}\]


Sol: 4:\[-\frac{175}{648}x + \frac{155}{648}+960x + 28+\frac{5}{84}x - \frac{124}{21}=-\frac{5}{18}x + \frac{175}{648}+840x + 508+\frac{1}{21}x - \frac{41}{7}\]


Sol: 5:\[-\frac{419}{540}x - \frac{1}{3}+114x - 432+22x - 8=-\frac{7}{9}x - \frac{35}{108}+90x - 312-98x + 592\]


Sol: 6:\[\frac{489}{7}x + 48+\frac{147}{4}x + \frac{7}{2}-\frac{292}{21}x + \frac{51}{2}=63x + \frac{624}{7}+\frac{279}{8}x + \frac{59}{4}-14x + \frac{365}{14}\]


Sol: 7:\[9x + \frac{75}{8}-\frac{13}{128}x + \frac{25}{64}-\frac{193}{16}x + \frac{195}{16}=\frac{579}{8}-\frac{7}{64}x + \frac{57}{128}-\frac{197}{16}x + \frac{223}{16}\]


Sol: 8:\[\frac{185}{288}x + \frac{221}{288}+\frac{5258}{735}x+\frac{53}{96}x - 7=\frac{23}{36}x + \frac{229}{288}+\frac{1752}{245}x + \frac{16}{735}+\frac{1}{2}x - \frac{79}{12}\]


Sol: 9:\[\frac{3}{100}x - \frac{1807}{300}-\frac{17}{2}x - 9+\frac{8}{3}x - \frac{7}{3}=\frac{2}{75}x - \frac{899}{150}-10x + \frac{9}{2}+\frac{5}{3}x + \frac{20}{3}\]


Sol: 10:\[-\frac{23}{15}x + \frac{16}{15}+2x - \frac{55}{7}-2275x - 83=-\frac{8}{5}x + \frac{26}{15}+\frac{41}{21}x - \frac{155}{21}-2650x + 3667\]


Sol: 11:\[\frac{145}{48}x - \frac{35}{4}+\frac{1}{18}x - \frac{5}{18}-\frac{25}{6}x - \frac{3}{2}=3x - \frac{409}{48}+\frac{1}{3}-\frac{15}{2}x + \frac{211}{6}\]


Sol: 12:\[-\frac{9}{7}x-23x + 3+\frac{112}{3}x=-\frac{12}{7}x + \frac{36}{7}-25x + 27+36x + 16\]


Sol: 13:\[-\frac{40}{3}x + 60+51x + 13+\frac{63}{10}x - 15=-\frac{50}{3}x + \frac{310}{3}+39x + 169+\frac{151}{24}x - \frac{1787}{120}\]


Sol: 14:\[-x + 63+\frac{196}{3}x-\frac{35}{6}x + \frac{53}{6}=-5x + 119+\frac{124}{3}x + 336-6x + \frac{67}{6}\]


Sol: 15:\[\frac{13}{294}x + \frac{8}{49}-\frac{11975}{96}x - \frac{55}{2}+158x + 8=\frac{1}{42}x + \frac{23}{49}-125x - \frac{755}{32}+8x + 2258\]


Sol: 16:\[-\frac{41}{9}x - \frac{20}{3}+\frac{173}{24}x - 1-\frac{2335}{576}x + \frac{1}{4}=-5x + \frac{4}{9}+7x + \frac{7}{3}-\frac{65}{16}x + \frac{7}{18}\]


Sol: 17:\[\frac{43}{120}x + 15-\frac{148}{3}x+464x - 224=\frac{7}{20}x + \frac{1817}{120}-60x + \frac{544}{3}+368x + 1408\]


Sol: 18:\[\frac{100}{567}x + 5+\frac{863}{7}x + 8+\frac{23}{27}x + \frac{8}{3}=\frac{515}{63}+9x + \frac{14456}{7}+\frac{8}{27}x + \frac{38}{3}\]


Sol: 19:\[-\frac{68}{9}x + 8+\frac{1}{5}x - 3\frac{1}{5}+\frac{1403}{10080}x - \frac{1}{36}=-8x + \frac{148}{9}-\frac{12}{5}+\frac{5}{36}x - \frac{223}{10080}\]




\end{document}