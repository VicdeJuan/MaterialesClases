\paragraph{[3 puntos] 1) Estudia las asíntotas de la siguiente función (sin olvidarte de decir algo sobre las asíntotas oblicuas)} \[f(x) = \frac{ 2 \, x^{2} - 5 \, x + 2 }{ 3 \, x^{2} - 7 \, x + 2 }\] \paragraph{[3 puntos] 2) Resuelve los siguientes límites:}\begin{itemize}\item $\displaystyle\lim_{x\to-\infty} \frac{ -2 \, x^{2} + 4 \, x - 2 }{ x^{2} - 2 \, x + 1 }$ \item $\displaystyle\lim_{x\to-\infty} \frac{ x - 1 }{ -2 \, x^{2} - 6 \, x - 4 }$ \item $\displaystyle\lim_{x\to+\infty} \frac{ x^{3} - 3 \, x^{2} - x + 3 }{ x^{2} - 6 \, x + 9 }$\end{itemize} \paragraph{[2 puntos] 3) Halla, si es posible, el valor de k para que las siguientes funciones sean continuas}\begin{itemize} \item\[ f(x)=\left\{\begin{array}{ccc} x - 2 & si & x<4 \\ \text{\texttt{{-}kx{ }+{ }1}} & si & x>=4 \\ \end{array}\right. \] \end{itemize} \paragraph{[2 puntos] 4) Halla el dominio de las siguientes funciones:}\begin{itemize} \item \[f(x) = \log{ -3 \, x^{3} + 3 \, x^{2} - 12 \, x + 12 }\] \item \[f(x) = \sqrt[3]{ 4 \, x^{2} - 7 \, x + 3 }\] \end{itemize}