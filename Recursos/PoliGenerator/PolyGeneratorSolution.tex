\documentclass[palatino,nochap]{Docencia}


\usepackage{sagetex}
\usepackage{breqn}
%\geometry{reset,margin=0.2in,rmargin=0.5in}


\title{Polinomios para factorizar.}
%\newcommand{\subtitle}[1]{\def\thesubtitle{#1}\@subsettrue}
%\subtitle{4º ESO Matemáticas Académicas}
\author{Departamento de Matemáticas}
\date{18/19}



\begin{abstract}

Ejercicios de polinomios generados automáticamente.

$x_2 = 3 [2]$, significa que la segunda raíz obtenida es $3$ y tiene multiplicidad $2$, es decir, el polinomio tiene 2 factores $(x-2)$.
%
Si entre corchetes no aparece ningún número, significa que su multiplicidad es 1.


\textit{Nota: Este documento se ha generado automáticamente utilizando \href{www.sagemath.org/es/}{Sage}, \LaTeX\xspace y \href{https://github.com/sagemath/sagetex}{sagetex} para la integración de las 2 herramientas mencionadas.}

\end{abstract}

% Paquetes adicionales

\usepackage[author={Víctor de Juan, 2018}]{pdfcomment}

\makeatletter
\newcommand{\annotate}[2][]{%
	\pdfstringdef\x@title{#1}%
	\edef\r{\string\r}%
	\pdfstringdef\x@contents{#2}%
	\pdfannot
	width 2\baselineskip
	height 2\baselineskip
	depth 0pt
	{
		/Subtype /Text
		/T (\x@title)
		/Contents (\x@contents)
	}%
}
\makeatother



% --------------------
\newcommand{\cimplies}{\text{\hl{$\implies$}}}
\renewcommand{\vx}{\overset{\rightarrow}{x}}
\renewcommand{\vy}{\overset{\rightarrow}{y}}
\renewcommand{\vz}{\overset{\rightarrow}{z}}
\newcommand{\vi}{\overset{\rightarrow}{i}}
\newcommand{\vj}{\overset{\rightarrow}{j}}
\renewcommand{\vec}[1]{\overset{\rightarrow}{#1}}




\begin{document}
\pagestyle{plain}
%\maketitle
%\tableofcontents
%\newpage

\section{Evaluación 1ºD - Funciones}

%\begin{multicols}{3}
\begin{itemize}\item[Ejercicio1] Realiza las siguientes sumas de polinomios: \\ \subitem [0]$ x^{3} - 4 \, x^{2} + -2 \, x^{6} - 2 \, x^{4} + 4 \, x^{3} + -4 \, x^{4} - 6 \, x = -2 \, x^{6} - 6 \, x^{4} + 5 \, x^{3} - 4 \, x^{2} - 6 \, x $ \subitem [1]$ 3 \, x^{4} + 4 \, x^{3} - x + -x^{6} + 4 \, x^{5} + x^{3} + -2 \, x^{6} + 4 \, x^{4} - 2 \, x = -3 \, x^{6} + 4 \, x^{5} + 7 \, x^{4} + 5 \, x^{3} - 3 \, x $ \subitem [2]$ 3 \, x^{6} + 3 \, x^{4} + 2 \, x + -3 \, x^{2} + 7 \, x + ( -x^{6} - 2 \, x^{4} + 4 \, x^{2} ) = 2 \, x^{6} + x^{4} + x^{2} + 9 \, x $ \subitem [3]$ 2 \, x^{6} - x^{2} - 4 \, x + -5 \, x^{5} + x^{2} + -x^{6} - 4 \, x^{5} - 3 \, x^{2} = x^{6} - 9 \, x^{5} - 3 \, x^{2} - 4 \, x $ \subitem [4]$ 4 \, x^{4} + 4 \, x^{2} - x + -x^{3} + x^{2} - 4 \, x + 4 \, x^{6} - 4 \, x^{3} - 4 \, x = 4 \, x^{6} + 4 \, x^{4} - 5 \, x^{3} + 5 \, x^{2} - 9 \, x $ \subitem [5]$ -2 \, x^{6} - 3 \, x^{5} - 2 \, x^{4} + ( -7 \, x^{3} + x ) + ( -2 \, x^{6} + 4 \, x^{5} - 2 \, x^{4} ) = -4 \, x^{6} + x^{5} - 4 \, x^{4} - 7 \, x^{3} + x $ \subitem [6]$ x^{5} + 4 \, x^{4} - 3 \, x^{2} + -4 \, x^{6} - x^{3} - 4 \, x^{2} + ( -7 \, x^{5} + 4 \, x ) = -4 \, x^{6} - 6 \, x^{5} + 4 \, x^{4} - x^{3} - 7 \, x^{2} + 4 \, x $ \subitem [7]$ -4 \, x^{5} + 2 \, x^{2} + x + ( -2 \, x^{3} + 3 \, x ) + ( -4 \, x^{4} + x^{2} ) = -4 \, x^{5} - 4 \, x^{4} - 2 \, x^{3} + 3 \, x^{2} + 4 \, x $ \subitem [8]$ x^{6} + 4 \, x^{3} - 2 \, x + -x^{5} - x^{3} + -x^{6} + 3 \, x^{5} - 4 \, x^{2} = 2 \, x^{5} + 3 \, x^{3} - 4 \, x^{2} - 2 \, x $ \subitem [9]$ 3 \, x^{5} - 3 \, x^{3} - 2 \, x^{2} + -8 \, x^{5} + -6 \, x^{3} - 2 \, x = -5 \, x^{5} - 9 \, x^{3} - 2 \, x^{2} - 2 \, x $ \item[Ejercicio2] Realiza las siguientes sumas de polinomios: \\ \subitem [0]$ 0 + 0 + 0 = 0 $ \subitem [1]$ 4 \, x^{2} y^{2} - x^{2} y + 3 \, x y^{2} + -x^{2} y + -4 \, x^{2} y - 2 \, x y = 4 \, x^{2} y^{2} - 6 \, x^{2} y + 3 \, x y^{2} - 2 \, x y $ \subitem [2]$ 10 \, x^{2} y^{2} - 4 \, x y^{2} + 16 \, x^{2} y^{2} - 8 \, x y^{2} - 2 \, x y + -12 \, x^{2} y + 12 \, x y^{2} = 26 \, x^{2} y^{2} - 12 \, x^{2} y - 2 \, x y $ \subitem [3]$ 63 \, x^{2} y^{2} + -12 \, x^{2} y^{2} - 3 \, x^{2} y + 6 \, x y + -36 \, x^{2} y^{2} - 9 \, x^{2} y + 12 \, x y = 15 \, x^{2} y^{2} - 12 \, x^{2} y + 18 \, x y $ \subitem [4]$ 48 \, x^{2} y^{2} + 12 \, x y^{2} + 48 \, x y + -4 \, x^{2} y^{2} + 8 \, x^{2} y - 48 \, x y + -64 \, x^{2} y^{2} + 24 \, x^{2} y = -20 \, x^{2} y^{2} + 32 \, x^{2} y + 12 \, x y^{2} $ \subitem [5]$ 10 \, x^{2} y + 25 \, x y + -25 \, x^{2} y + ( -25 \, x^{2} y^{2} - 35 \, x^{2} y ) = -25 \, x^{2} y^{2} - 50 \, x^{2} y + 25 \, x y $ \subitem [6]$ 102 \, x^{2} y + -12 \, x y^{2} - 6 \, x y + 36 \, x^{2} y^{2} - 36 \, x^{2} y - 72 \, x y = 36 \, x^{2} y^{2} + 66 \, x^{2} y - 12 \, x y^{2} - 78 \, x y $ \subitem [7]$ 28 \, x^{2} y^{2} - 28 \, x y + 224 \, x^{2} y^{2} - 196 \, x^{2} y + -147 \, x^{2} y^{2} + 28 \, x^{2} y - 196 \, x y^{2} = 105 \, x^{2} y^{2} - 168 \, x^{2} y - 196 \, x y^{2} - 28 \, x y $ \subitem [8]$ 24 \, x^{2} y^{2} - 208 \, x^{2} y + -16 \, x^{2} y + 24 \, x y + 24 \, x^{2} y + 32 \, x y^{2} = 24 \, x^{2} y^{2} - 200 \, x^{2} y + 32 \, x y^{2} + 24 \, x y $ \subitem [9]$ 27 \, x^{2} y^{2} - 9 \, x y^{2} + 18 \, x y + 9 \, x^{2} y^{2} + 162 \, x^{2} y - 27 \, x y + 45 \, x y = 36 \, x^{2} y^{2} + 162 \, x^{2} y - 9 \, x y^{2} + 36 \, x y $ \item[Ejerciio 3] Realiza las siguientes sumas y restas de polinomios: \\ \subitem [0]$ 0 + 0 - ( 0 ) = 0 $ \subitem [1]$ -2 \, x^{2} y - x y^{2} + -4 \, x^{2} y^{2} + 2 \, x^{2} y + 3 \, x y^{2} - ( -4 \, x^{2} y^{2} - 2 \, x^{2} y - 2 \, x y ) = 2 \, x^{2} y + 2 \, x y^{2} + 2 \, x y $ \subitem [2]$ 2 \, x^{2} y + 12 \, x y^{2} - 8 \, x y + -2 \, x^{2} y^{2} + 8 \, x^{2} y - ( 18 \, x^{2} y^{2} - 2 \, x y ) = -20 \, x^{2} y^{2} + 10 \, x^{2} y + 12 \, x y^{2} - 6 \, x y $ \subitem [3]$ 27 \, x^{2} y^{2} + 9 \, x^{2} y + 12 \, x^{2} y^{2} + 27 \, x y^{2} - ( 9 \, x^{2} y^{2} - 30 \, x y^{2} ) = 30 \, x^{2} y^{2} + 9 \, x^{2} y + 57 \, x y^{2} $ \subitem [4]$ 12 \, x^{2} y^{2} + 4 \, x^{2} y + -28 \, x^{2} y - 64 \, x y - ( 12 \, x^{2} y^{2} - 16 \, x^{2} y + 32 \, x y ) = -8 \, x^{2} y - 96 \, x y $ \subitem [5]$ 90 \, x^{2} y + 50 \, x y^{2} + -25 \, x^{2} y^{2} + 75 \, x^{2} y - 75 \, x y^{2} - ( -75 \, x^{2} y^{2} + 75 \, x y^{2} - 25 \, x y ) = 50 \, x^{2} y^{2} + 165 \, x^{2} y - 100 \, x y^{2} + 25 \, x y $ \subitem [6]$ -18 \, x^{2} y - 120 \, x y + -132 \, x y^{2} + 108 \, x y - ( -18 \, x^{2} y^{2} - 24 \, x^{2} y + 72 \, x y ) = 18 \, x^{2} y^{2} + 6 \, x^{2} y - 132 \, x y^{2} - 84 \, x y $ \subitem [7]$ 56 \, x^{2} y^{2} + 196 \, x y + -35 \, x y^{2} + 21 \, x y - ( 98 \, x^{2} y^{2} - 98 \, x y^{2} + 14 \, x y ) = -42 \, x^{2} y^{2} + 63 \, x y^{2} + 203 \, x y $ \subitem [8]$ -8 \, x^{2} y + 112 \, x y^{2} + -96 \, x^{2} y - 32 \, x y - ( -256 \, x^{2} y^{2} - 64 \, x y^{2} + 24 \, x y ) = 256 \, x^{2} y^{2} - 104 \, x^{2} y + 176 \, x y^{2} - 56 \, x y $ \subitem [9]$ -18 \, x^{2} y - 18 \, x y^{2} + 243 \, x^{2} y - ( -243 \, x^{2} y + 243 \, x y^{2} + 9 \, x y ) = 468 \, x^{2} y - 261 \, x y^{2} - 9 \, x y $ \item[Ejercicio3] Realiza las siguientes multiplicaciones de monomios: \\ \subitem [0]$ ( 0 ) · ( 0 ) = 0 $ \subitem [1]$ ( 2 \, b^{3} x^{2} y z^{2} ) · ( -b^{3} x y z^{3} ) = -2 \, b^{6} x^{3} y^{2} z^{5} $ \subitem [2]$ ( 12 \, b^{2} x y^{2} z^{2} ) · ( -4 \, b^{3} x^{2} y^{3} z ) = -48 \, b^{5} x^{3} y^{5} z^{3} $ \subitem [3]$ ( 81 \, b x y^{2} z^{3} ) · ( 27 \, b x y z ) = 2187 \, b^{2} x^{2} y^{3} z^{4} $ \subitem [4]$ ( 128 \, b x^{3} y^{2} z ) · ( 8 \, b x^{3} y^{2} z ) = 1024 \, b^{2} x^{6} y^{4} z^{2} $ \subitem [5]$ ( 25 \, b x^{3} y z^{2} ) · ( 75 \, b x y^{3} z^{3} ) = 1875 \, b^{2} x^{4} y^{4} z^{5} $ \subitem [6]$ ( 6 \, b x y^{2} z^{2} ) · ( 72 \, b^{3} x^{2} y^{2} z^{3} ) = 432 \, b^{4} x^{3} y^{4} z^{5} $ \subitem [7]$ ( -686 \, b^{3} x y^{3} z^{3} ) · ( 49 \, b x^{2} y^{2} z^{3} ) = -33614 \, b^{4} x^{3} y^{5} z^{6} $ \subitem [8]$ ( -1024 \, b x y z^{3} ) · ( 24 \, b x^{2} y z^{3} ) = -24576 \, b^{2} x^{3} y^{2} z^{6} $ \subitem [9]$ ( -1458 \, b^{2} x y z^{3} ) · ( 2187 \, b^{2} x y^{3} z^{2} ) = -3188646 \, b^{4} x^{2} y^{4} z^{5} $ \item[Ejercicio4] Realiza las siguientes multiplicaciones de polinomios: \\ \subitem [0]$ ( -3 \, x ) · ( 4 \, x^{2} + 5 \, x ) = -12 \, x^{3} - 15 \, x^{2} $ \subitem [1]$ ( 3 \, x ) · ( -2 \, x^{2} - 2 \, x ) = -6 \, x^{3} - 6 \, x^{2} $ \subitem [2]$ ( 3 \, x^{2} ) · ( 5 \, x^{2} + x ) = 15 \, x^{4} + 3 \, x^{3} $ \subitem [3]$ ( -x^{2} ) · ( -3 \, x^{2} - x ) = 3 \, x^{4} + x^{3} $ \subitem [4]$ ( x ) · ( x^{2} - 4 \, x ) = x^{3} - 4 \, x^{2} $ \subitem [5]$ ( -4 \, x^{2} ) · ( -3 \, x^{2} - 7 \, x ) = 12 \, x^{4} + 28 \, x^{3} $ \subitem [6]$ ( 2 \, x ) · ( -10 \, x^{2} - x ) = -20 \, x^{3} - 2 \, x^{2} $ \subitem [7]$ ( -2 \, x ) · ( x^{2} + 4 \, x ) = -2 \, x^{3} - 8 \, x^{2} $ \subitem [8]$ ( -3 \, x ) · ( 2 \, x^{2} + 8 \, x ) = -6 \, x^{3} - 24 \, x^{2} $ \subitem [9]$ ( x^{2} ) · ( -x^{2} + 5 \, x ) = -x^{4} + 5 \, x^{3} $ \item[Ejercicio5] Realiza las siguientes multiplicaciones de polinomios: \\ \subitem [0]$ ( 2 \, x^{2} ) · ( -2 \, x^{2} + 4 \, x ) = -4 \, x^{4} + 8 \, x^{3} $ \subitem [1]$ ( -x ) · ( -x^{2} + 6 \, x ) = x^{3} - 6 \, x^{2} $ \subitem [2]$ ( 4 \, x^{2} + 2 \, x ) · ( -x^{2} - 3 \, x ) = -4 \, x^{4} - 14 \, x^{3} - 6 \, x^{2} $ \subitem [3]$ ( -3 \, x^{2} ) · ( 7 \, x^{2} - 4 \, x ) = -21 \, x^{4} + 12 \, x^{3} $ \subitem [4]$ ( -4 \, x^{2} + 3 \, x ) · ( -2 \, x^{2} ) = 8 \, x^{4} - 6 \, x^{3} $ \subitem [5]$ ( -4 \, x^{2} ) · ( -3 \, x^{2} ) = 12 \, x^{4} $ \subitem [6]$ ( 3 \, x^{2} - 4 \, x ) · ( 3 \, x ) = 9 \, x^{3} - 12 \, x^{2} $ \subitem [7]$ ( 2 \, x^{2} ) · ( 3 \, x^{2} + 6 \, x ) = 6 \, x^{4} + 12 \, x^{3} $ \subitem [8]$ ( -2 \, x^{2} - 4 \, x ) · ( -x^{2} - x ) = 2 \, x^{4} + 6 \, x^{3} + 4 \, x^{2} $ \subitem [9]$ ( 0 ) · ( 2 \, x^{2} - x ) = 0 $ \item[Ejercicio6] Realiza las siguientes multiplicaciones de polinomios: \\ \subitem [0]$ ( x^{3} - x^{2} - x ) · ( -3 \, x^{2} + 2 \, x ) = -3 \, x^{5} + 5 \, x^{4} + x^{3} - 2 \, x^{2} $ \subitem [1]$ ( 3 \, x^{2} + x ) · ( -x^{3} - 2 \, x^{2} + 2 \, x ) = -3 \, x^{5} - 7 \, x^{4} + 4 \, x^{3} + 2 \, x^{2} $ \subitem [2]$ ( 3 \, x^{3} + 3 \, x^{2} ) · ( -x^{2} - 4 \, x ) = -3 \, x^{5} - 15 \, x^{4} - 12 \, x^{3} $ \subitem [3]$ ( -4 \, x ) · ( 2 \, x ) = -8 \, x^{2} $ \subitem [4]$ ( -2 \, x^{3} - 3 \, x^{2} ) · ( -3 \, x^{3} - 3 \, x^{2} - 4 \, x ) = 6 \, x^{6} + 15 \, x^{5} + 17 \, x^{4} + 12 \, x^{3} $ \subitem [5]$ ( -3 \, x^{2} ) · ( -3 \, x^{3} + 5 \, x^{2} - 4 \, x ) = 9 \, x^{5} - 15 \, x^{4} + 12 \, x^{3} $ \subitem [6]$ ( 4 \, x^{3} - 3 \, x^{2} - 3 \, x ) · ( x ) = 4 \, x^{4} - 3 \, x^{3} - 3 \, x^{2} $ \subitem [7]$ ( 2 \, x^{3} + x ) · ( 4 \, x^{3} + 3 \, x^{2} - 3 \, x ) = 8 \, x^{6} + 6 \, x^{5} - 2 \, x^{4} + 3 \, x^{3} - 3 \, x^{2} $ \subitem [8]$ ( 2 \, x^{2} ) · ( -4 \, x^{3} + x^{2} + 2 \, x ) = -8 \, x^{5} + 2 \, x^{4} + 4 \, x^{3} $ \subitem [9]$ ( -x^{2} + 2 \, x ) · ( 4 \, x^{3} + 7 \, x^{2} ) = -4 \, x^{5} + x^{4} + 14 \, x^{3} $ \item[Ejercicio7] Realiza las siguientes multiplicaciones de polinomios: \\ \subitem [0]$ ( -2 \, x^{2} y^{2} - 4 \, x y^{2} ) · ( 3 \, x^{2} y + 2 \, x y^{2} ) = -6 \, x^{4} y^{3} - 4 \, x^{3} y^{4} - 12 \, x^{3} y^{3} - 8 \, x^{2} y^{4} $ \subitem [1]$ ( -3 \, x^{2} y + 3 \, x y^{2} ) · ( -4 \, x^{2} y - 4 \, x y^{2} + 3 \, x y ) = 12 \, x^{4} y^{2} - 12 \, x^{2} y^{4} - 9 \, x^{3} y^{2} + 9 \, x^{2} y^{3} $ \subitem [2]$ ( x^{2} y^{2} + 2 \, x y^{2} ) · ( -2 \, x^{2} y^{2} + 3 \, x y^{2} - 4 \, x y ) = -2 \, x^{4} y^{4} - x^{3} y^{4} - 4 \, x^{3} y^{3} + 6 \, x^{2} y^{4} - 8 \, x^{2} y^{3} $ \subitem [3]$ ( -5 \, x y^{2} ) · ( -2 \, x^{2} y - x y ) = 10 \, x^{3} y^{3} + 5 \, x^{2} y^{3} $ \subitem [4]$ ( x y^{2} - 4 \, x y ) · ( -2 \, x^{2} y - x y^{2} ) = -2 \, x^{3} y^{3} - x^{2} y^{4} + 8 \, x^{3} y^{2} + 4 \, x^{2} y^{3} $ \subitem [5]$ ( 4 \, x^{2} y + 4 \, x y ) · ( 4 \, x^{2} y + 3 \, x y^{2} ) = 16 \, x^{4} y^{2} + 12 \, x^{3} y^{3} + 16 \, x^{3} y^{2} + 12 \, x^{2} y^{3} $ \subitem [6]$ ( -2 \, x^{2} y^{2} ) · ( 2 \, x^{2} y^{2} - x y^{2} - 4 \, x y ) = -4 \, x^{4} y^{4} + 2 \, x^{3} y^{4} + 8 \, x^{3} y^{3} $ \subitem [7]$ ( -3 \, x^{2} y + x y ) · ( -2 \, x^{2} y^{2} - x y ) = 6 \, x^{4} y^{3} - 2 \, x^{3} y^{3} + 3 \, x^{3} y^{2} - x^{2} y^{2} $ \subitem [8]$ ( x^{2} y^{2} - x y ) · ( -x y^{2} + x y ) = -x^{3} y^{4} + x^{3} y^{3} + x^{2} y^{3} - x^{2} y^{2} $ \subitem [9]$ ( -3 \, x^{2} y^{2} + 4 \, x y ) · ( 3 \, x^{2} y^{2} + x^{2} y + 4 \, x y^{2} ) = -9 \, x^{4} y^{4} - 3 \, x^{4} y^{3} - 12 \, x^{3} y^{4} + 12 \, x^{3} y^{3} + 4 \, x^{3} y^{2} + 16 \, x^{2} y^{3} $ \end{itemize}
%\end{multicols}

\end{document}
